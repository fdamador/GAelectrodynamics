%
% Copyright � 2018 Peeter Joot.  All Rights Reserved.
% Licenced as described in the file LICENSE under the root directory of this GIT repository.
%
%{
%%%\input{../latex/blogpost.tex}
%%%\renewcommand{\basename}{planewavesMultivector}
%%%%\renewcommand{\dirname}{notes/phy1520/}
%%%\renewcommand{\dirname}{notes/ece1228-electromagnetic-theory/}
%%%%\newcommand{\dateintitle}{}
%%%%\newcommand{\keywords}{}
%%%
%%%\input{../latex/peeter_prologue_print2.tex}
%%%
%%%\usepackage{peeters_layout_exercise}
%%%\usepackage{peeters_braket}
%%%\usepackage{peeters_figures}
%%%\usepackage{siunitx}
%%%%\usepackage{mhchem} % \ce{}
%%%%\usepackage{macros_bm} % \bcM
%%%%\usepackage{macros_qed} % \qedmarker
%%%%\usepackage{txfonts} % \ointclockwise
%%%
%%%\beginArtNoToc
%%%
%%%\generatetitle{Multivector plane wave representation}
%\chapter{Multivector plane wave representation}
\label{chap:planewavesMultivector}

With all sources zero,
the free space Maxwell's equation as given by \cref{dfn:isotropicMaxwells:680} for the
electromagnetic field strength reduces to just
\begin{equation}\label{eqn:planewavesMultivector:300}
\stgrad F(\Bx, t) = 0.
\end{equation}

Utilizing a phasor representation of the form \cref{dfn:greensFunctionOverview:300},
we will define the
phasor representation of the field as
\input{Definition_plane_wave.tex}

We will now show that solutions of the electromagnetic field wave equation have the form

\index{\(\omega\)}
\index{\(\Bk\)}
\index{\(\kcap\)}
\input{Theorem_plane_wave_solutions.tex}
\begin{proof}
We wish to act on \( F(\Bk) e^{-j \Bk \cdot \Bx + j \omega t } \) with the spacetime gradient \( \spacegrad + (1/c)\partial_t \), but
must take care of order when applying the gradient to a non-scalar valued function.  In particular, if \( A \) is a multivector, then
\begin{equation}\label{eqn:planewavesMultivector:660}
\begin{aligned}
\spacegrad A e^{-j \Bk \cdot \Bx}
&= \sum_{m = 1}^3 \Be_m \partial_m A e^{-j \Bk \cdot \Bx} \\
&= \sum_{m = 1}^3 \Be_m A \lr{ -j k_m } e^{-j \Bk \cdot \Bx} \\
&= -j \Bk A.
\end{aligned}
\end{equation}
Therefore, insertion of the presumed phasor solution of the field from
\cref{dfn:planewavesMultivector:680} into
\cref{eqn:planewavesMultivector:300} gives
\begin{equation}\label{eqn:planewavesMultivector:60}
0 = -j \lr{ \Bk - \frac{\omega}{c} } F(\Bk).
\end{equation}

If \( F(\Bk) \) has a left multivector factor
\begin{equation}\label{eqn:planewavesMultivector:80}
F(\Bk) = \lr{ \Bk + \frac{\omega}{c} } \tilde{F},
\end{equation}
where \( \tilde{F} \) is a multivector to be determined, then
\begin{equation}\label{eqn:planewavesMultivector:100}
\begin{aligned}
\lr{ \Bk - \frac{\omega}{c} } F(\Bk)
&= \lr{ \Bk - \frac{\omega}{c} } \lr{ \Bk + \frac{\omega}{c} } \tilde{F} \\
&= \lr{ \Bk^2 - \lr{\frac{\omega}{c}}^2 } \tilde{F},
\end{aligned}
\end{equation}
which is zero if \( \Norm{\Bk} = \ifrac{\omega}{c} \).
Let \( \Norm{\Bk} \tilde{F} = F_0 + F_1 + F_2 + F_3 \), where
\( F_0, F_1, F_2, \) and \( F_3 \) respectively have grades 0,1,2,3, so that
\begin{equation}\label{eqn:planewavesMultivector:120}
\begin{aligned}
F(\Bk)
&= \lr{ 1 + \kcap } \lr{ F_0 + F_1 + F_2 + F_3 } \\
&= F_0 + F_1 + F_2 + F_3 + \kcap F_0 + \kcap F_1 + \kcap F_2 + \kcap F_3 \\
&= F_0 + F_1 + F_2 + F_3
+
\kcap F_0 + \kcap \cdot F_1 + \kcap \cdot F_2 + \kcap \cdot F_3 \\
&\quad +
\kcap \wedge F_1 + \kcap \wedge F_2 \\
&=
\lr{
   F_0 + \kcap \cdot F_1
}
+
\lr{
   F_1 + \kcap F_0 + \kcap \cdot F_2
} \\
&\quad +
\lr{
   F_2 + \kcap \cdot F_3 + \kcap \wedge F_1
}
+
\lr{
   F_3 + \kcap \wedge F_2
}.
\end{aligned}
\end{equation}
Since the field \( F \) has only vector and bivector grades, the grades zero and three components of the expansion above must be zero, or
\begin{equation}\label{eqn:planewavesMultivector:140}
\begin{aligned}
   F_0 &= - \kcap \cdot F_1 \\
   F_3 &= - \kcap \wedge F_2,
\end{aligned}
\end{equation}
so
\begin{equation}\label{eqn:planewavesMultivector:160}
\begin{aligned}
F(\Bk)
&=
\lr{ 1 + \kcap } \lr{
   F_1 - \kcap \cdot F_1 +
   F_2 - \kcap \wedge F_2
} \\
&=
\lr{ 1 + \kcap } \lr{
   F_1 - \kcap F_1 + \kcap \wedge F_1 +
   F_2 - \kcap F_2 + \kcap \cdot F_2
}.
\end{aligned}
\end{equation}
The multivector \( 1 + \kcap \) has the projective property of gobbling any leading factors of \( \kcap \)
\begin{equation}\label{eqn:planewavesMultivector:180}
\begin{aligned}
(1 + \kcap)\kcap
&= \kcap + 1 \\
&= 1 + \kcap,
\end{aligned}
\end{equation}
so for \( F_i \in F_1, F_2 \)
\begin{equation}\label{eqn:planewavesMultivector:200}
(1 + \kcap) ( F_i - \kcap F_i )
=
(1 + \kcap) ( F_i - F_i )
= 0,
\end{equation}
leaving
\begin{equation}\label{eqn:planewavesMultivector:220}
F(\Bk)
=
\lr{ 1 + \kcap } \lr{
   \kcap \cdot F_2 +
   \kcap \wedge F_1
}.
\end{equation}

For \( \kcap \cdot F_2 \) to be non-zero \( F_2 \) must be a bivector that lies in a plane containing \( \kcap \), and
\( \kcap \cdot F_2 \) is a vector in that plane that is perpendicular to \( \kcap \).
On the other hand \( \kcap \wedge F_1 \) is non-zero only if \( F_1 \) has a non-zero component that does not lie in along the \( \kcap \) direction, but \( \kcap \wedge F_1 \), like \( F_2 \) describes a plane that containing \( \kcap \).
This means that having both bivector and vector free variables \( F_2 \) and \( F_1 \) provide more degrees of freedom than required.
For example, if \( \BE \) is any vector, and \( F_2 = \kcap \wedge \BE \), then
\begin{equation}\label{eqn:planewavesMultivector:240}
\begin{aligned}
\lr{ 1 + \kcap }
   \kcap \cdot F_2
&=
\lr{ 1 + \kcap }
   \kcap \cdot \lr{ \kcap \wedge \BE } \\
&=
\lr{ 1 + \kcap }
\lr{
   \BE
-
\kcap \lr{ \kcap \cdot \BE }
} \\
&=
\lr{ 1 + \kcap }
\kcap \lr{ \kcap \wedge \BE } \\
&=
\lr{ 1 + \kcap }
\kcap \wedge \BE,
\end{aligned}
\end{equation}
which has the form \( \lr{ 1 + \kcap } \lr{ \kcap \wedge F_1 } \), so the electromagnetic field strength phasor may be generally written
\begin{equation}\label{eqn:planewavesMultivector:280}
F(\Bk)
=
\lr{ 1 + \kcap }
\kcap \wedge \BE \, e^{-j \Bk \cdot \Bx}
,
\end{equation}
%\end{boxed}
Expanding the multivector factor \( \lr{ 1 + \kcap } \kcap \wedge \BE \) we find
\begin{equation}\label{eqn:planewavesMultivector:720}
\begin{aligned}
\lr{ 1 + \kcap }
\kcap \wedge \BE
&=
\kcap \cdot \lr{ \kcap \wedge \BE }
+\cancel{\kcap \wedge \lr{ \kcap \wedge \BE }}
+
\kcap \wedge \BE \\
&=
\BE - \kcap \lr{ \kcap \wedge \BE }
+
\kcap \wedge \BE.
\end{aligned}
\end{equation}
The vector grade has the component of \( \BE \) along the propagation direction removed (i.e. it is the rejection), so there is no loss of generality should a
\( \BE \cdot \Bk = 0 \) constraint be imposed.  Such as constraint let's us write the bivector as a vector product \( \kcap \wedge \BE = \kcap \BE \), and then use the projective property \cref{eqn:planewavesMultivector:180} to gobble the leading \( \kcap \) factor, leaving
\begin{equation}\label{eqn:planewavesMultivector:700}
F(\Bk)
=
\lr{ 1 + \kcap }
\BE \, e^{-j \Bk \cdot \Bx}
=
\lr{ \BE + I \kcap \cross \BE }
\, e^{-j \Bk \cdot \Bx}.
\end{equation}

It is also noteworthy that
the directions \( \kcap, \Ecap, \Hcap \) form a right handed triple, which can be seen by computing their product
\begin{equation}\label{eqn:planewavesMultivector:740}
\begin{aligned}
(\kcap \Ecap) \Hcap
&= (-\Ecap \kcap) (-I \kcap \Ecap) \\
&= +I \Ecap^2 \kcap^2 \\
&= I.
\end{aligned}
\end{equation}
These vectors must all be mutually orthonormal for their product to be a pseudoscalar multiple.
Should there be doubt, explicit dot products may be computed with ease using grade selection operations
\begin{equation}\label{eqn:planewavesMultivector:760}
\begin{aligned}
\kcap \cdot \Hcap &= \gpgradezero{ \kcap (-I \kcap \Ecap) } = -\gpgradezero{ I \Ecap } = 0 \\
\Ecap \cdot \Hcap &= \gpgradezero{ \Ecap (-I \kcap \Ecap) } = -\gpgradezero{ I \kcap } = 0,
\end{aligned}
\end{equation}
where the zeros follow by noting that \( I \Ecap, I \kcap \) are both bivectors.
The conventional representation of the right handed triple relationship between the propagation direction and fields is stated as a cross product,
not as a
pseudoscalar relationship as in
\cref{eqn:planewavesMultivector:740}. These are easily seen to be equivalent
\begin{equation}\label{eqn:planewavesMultivector:340}
\begin{aligned}
\kcap
&= I \Hcap \Ecap \\
&= I (\Hcap \wedge \Ecap) \\
&= I^2 (\Hcap \cross \Ecap) \\
&= \Ecap \cross \Hcap.
\end{aligned}
\end{equation}
\end{proof}
%}
%\EndNoBibArticle
