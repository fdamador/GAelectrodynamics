%
% Copyright © 2017 Peeter Joot.  All Rights Reserved.
% Licenced as described in the file LICENSE under the root directory of this GIT repository.
%
%\makeexample{Circular line charge.}{example:circularlinecharge:circularlinecharge}{
\index{line charge}
In this example we will examine the (electric) field due to a line charge density of \( \lambda \) along a circular arc segment \( \phi' \in [a,b] \), of radius \( r \) in the x-y plane.
The field will be evaluated at the
spherical coordinate point \( (R, \theta, \phi) \), as illustrated in \cref{fig:circularlinecharge:circularlinechargeFig1}.

%\imageFigure{../figures/GAelectrodynamics/circularlinechargeFig1}{Circular line charge.}{fig:circularlinecharge:circularlinechargeFig1}{0.3}
\mathImageFigure{../figures/GAelectrodynamics/lineChargeArcFig1}{Circular line charge.}{fig:circularlinecharge:circularlinechargeFig1}{0.5}{lineChargeArcFigure.nb}

\index{spherical coordinates}
Using the GA spherical parameterization of \cref{eqn:curvilinearspherical:40}, and \cref{eqn:curvilinearspherical:60}, the
%To setup this problem, first consider a GA factorization of the spherical radial point.
%Let
%\( i = \Be_{12} \), so that
%
%\begin{dmath}\label{eqn:circularlinecharge:20}
%\Bx
%= R \lr{ \Be_1 \sin\theta \cos\phi + \Be_2 \sin\theta \sin\phi + \Be_3 \cos\theta }
%= R \Be_1 \sin\theta ( \cos\phi + \Be_{12} \sin\phi ) + \Be_3 \cos\theta
%= R \Be_1 \sin\theta e^{i \phi} + \Be_3 \cos\theta
%= R \Be_3 \lr{ \cos\theta + \Be_{31} e^{i \phi} \sin\theta }.
%\end{dmath}
%
%Let the bivector for the rotational plane between \( R \Be_3 \rightarrow \Bx \) be represented by
%
%\begin{dmath}\label{eqn:circularlinecharge:40}
%j = \Be_{31} e^{i \phi}.
%\end{dmath}
%
%The bivector \( j \) is a function of the azimuthal angle \( \phi \), but encodes all the geometry of the rotation.
observation point now has the simple representation
\begin{dmath}\label{eqn:circularlinecharge:60}
\Bx = R \Be_3 e^{j \theta },
\end{dmath}
and is the product of a polar directed vector with a complex exponential whose argument is the polar rotation angle.
The bivector \( j \) is a function of the azimuthal angle \( \phi \), and encodes all the geometry of the rotation.
To sum the contributions of the charge elements we need the distance between the charge element and the observation point.
That vector difference is
\begin{dmath}\label{eqn:circularlinecharge:80}
\Bx - \Bx'
=
R \Be_3 e^{j \theta } - r \Be_1 e^{i \phi'}.
\end{dmath}

Compare this to the tuple representation
\begin{dmath}\label{eqn:circularlinecharge:100}
\Bx - \Bx'
= ( R \sin\theta \cos\phi - r \cos\phi', R \sin\theta \sin\phi - r \cos\phi', \cos\theta ),
\end{dmath}
for which the prospect of working with is considerably less attractive.
The squared length of \cref{eqn:circularlinecharge:80} is
\begin{dmath}\label{eqn:circularlinecharge:120}
(\Bx - \Bx')^2
=
R^2 + r^2 - 2 R r \lr{\Be_3 e^{j \theta} } \cdot \lr{ \Be_1 e^{i \phi'} }.
\end{dmath}

The dot product of unit vectors in \cref{eqn:circularlinecharge:120} can be reduced using scalar grade selection
\begin{dmath}\label{eqn:circularlinecharge:140}
\lr{\Be_3 e^{j \theta }} \cdot \lr{ \Be_1 e^{i \phi'} }
=
\gpgradezero{
\lr{ \Be_1 \sin\theta e^{i \phi} } \lr{ \Be_1 e^{i \phi'}}
}
=
\sin\theta
\gpgradezero{
e^{-i \phi} e^{i \phi'}
}
=
\sin\theta \cos( \phi' - \phi ),
\end{dmath}
so
\begin{dmath}\label{eqn:circularlinecharge:160}
\Norm{ \Bx - \Bx' }
=
\sqrt{
R^2 + r^2 - 2 R r \sin\theta \cos( \phi' - \phi )
}.
\end{dmath}

The electric field is
\begin{dmath}\label{eqn:circularlinecharge:180}
F = \frac{1}{4 \pi \epsilon_0} \int_a^b \lambda r d\phi' \frac{ R \Be_3 e^{j \theta } - r \Be_1 e^{i \phi'} }{\lr{ R^2 + r^2 - 2 R r \sin\theta \cos( \phi' - \phi ) }^{3/2} }.
\end{dmath}

Non-dimensionalizing \cref{eqn:circularlinecharge:180} with \( u = r/R \), a change of variables \( \alpha = \phi' - \phi \), and noting that \( i \phicap = \Be_1 e^{i \phi} \), the field is
\begin{dmath}\label{eqn:circularlinecharge:200}
F
= \frac{\lambda r}{4 \pi \epsilon_0 R^2} \int_{a-\phi}^{b-\phi} d\alpha \frac{ \Be_3 e^{j \theta } - u \Be_1 e^{i\phi} e^{i \alpha} }{\lr{ 1 + u^2 - 2 u \sin\theta \cos \alpha }^{3/2} }
= \frac{\lambda r}{4 \pi \epsilon_0 R^2} \int_{a-\phi}^{b-\phi} d\alpha \frac{ \rcap + \phicap u i e^{i \alpha } }{\lr{ 1 + u^2 - 2 u \sin\theta \cos \alpha }^{3/2} },
\end{dmath}
or
\begin{equation}\label{eqn:circularlinecharge:220}
\begin{aligned}
F
&= \rcap \lr{ \frac{\lambda r}{4 \pi \epsilon_0 R^2} \int_{a-\phi}^{b-\phi} \frac{ d\alpha }{\lr{ 1 + u^2 - 2 u \sin\theta \cos \alpha }^{3/2} } } \\
&+ \phicap \lr{
\frac{\lambda r u i}{4 \pi \epsilon_0 R^2} \int_{a-\phi}^{b-\phi} \frac{ e^{i \alpha } d\alpha }{\lr{ 1 + u^2 - 2 u \sin\theta \cos \alpha }^{3/2} }
}.
\end{aligned}
\end{equation}

Without CAS support for GA, this pair of integrals has to be evaluated separately.
The first integral
scales the radial component of the electric field.
The second integral scales and rotates \( \phicap \) within the azimuthal plane, producing an electric field component in a \( \phicap' = \phicap e^{i \Phi} \) direction.

%} % example
