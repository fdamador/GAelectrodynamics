%
% Copyright © 2017 Peeter Joot.  All Rights Reserved.
% Licenced as described in the file LICENSE under the root directory of this GIT repository.
%
\makeproblem{Index permutation in vector selection.}{problem:generalizedDot:distributionUnitVectorsa}{
Prove \cref{eqn:generalizedDot_vectorBivector:1020}.
That is,
given \( j \ne k \), and \( i = j \) or \( i = k \), show that
\begin{equation}\label{eqn:distributionDotVectorBivector:1140}
\gpgradeone{ \Be_i \Be_j \Be_k }
= \gpgradeone{ \Be_k \Be_j \Be_i },
\end{equation}
} % problem
\makeanswer{problem:generalizedDot:distributionUnitVectorsa}{
Since \( j \ne k \), \( \Be_j \Be_k = - \Be_k \Be_j \), so for \( i = k \)
\begin{equation}\label{eqn:distributionDotVectorBivector:20}
\gpgradeone{ \Be_i \Be_j \Be_k }
=
-\gpgradeone{ \Be_i \Be_k \Be_j }
=
-\gpgradeone{ \Be_k \Be_i \Be_j }
=
\gpgradeone{ \Be_k \Be_j \Be_i },
\end{equation}
and for \( i = j \)
\begin{equation}\label{eqn:distributionDotVectorBivector:40}
\gpgradeone{ \Be_i \Be_j \Be_k }
=
\gpgradeone{ \Be_j \Be_i \Be_k }
=
-\gpgradeone{ \Be_j \Be_k \Be_i }
=
\gpgradeone{ \Be_k \Be_j \Be_i }.
\end{equation}
} % answer
\makeproblem{Dot product of unit vector with unit bivector.}{problem:generalizedDot:distributionUnitVectorsb}{
Prove \cref{eqn:generalizedDot_vectorBivector:980}.
That is,
given \( j \ne k \), show that
\begin{equation}\label{eqn:distributionDotVectorBivector:1040}
\gpgradeone{ \Be_i \Be_j \Be_k }
=
\Be_k \lr{ \Be_j \cdot \Be_i }
-\Be_j \lr{ \Be_k \cdot \Be_i }.
\end{equation}
} % problem
\makeanswer{problem:generalizedDot:distributionUnitVectorsb}{
We can tackle this first looking at the \( i = j \) case, where
\begin{equation}\label{eqn:distributionDotVectorBivector:1060}
\gpgradeone{ \Be_i \Be_j \Be_k }
=
\gpgradeone{
\lr{ \Be_i \cdot \Be_j } \Be_k
}
=
\lr{ \Be_i \cdot \Be_j } \Be_k.
\end{equation}
For the \( i = k \) case, we have
\begin{equation}\label{eqn:distributionDotVectorBivector:1080}
\gpgradeone{ \Be_i \Be_j \Be_k }
=
-\gpgradeone{ \Be_i \Be_k \Be_j }
=
-\gpgradeone{ \lr{ \Be_i \cdot \Be_k } \Be_j }
=
-\lr{ \Be_i \cdot \Be_k } \Be_j.
\end{equation}
Combining both possibilities we have
\begin{equation}\label{eqn:distributionDotVectorBivector:1100}
\gpgradeone{ \Be_i \Be_j \Be_k }
=
\lr{ \Be_i \cdot \Be_j } \Be_k
-\lr{ \Be_i \cdot \Be_k } \Be_j.
\end{equation}
Incidentally, note that this only holds when \( j \ne k \).  More generally
\begin{equation}\label{eqn:distributionDotVectorBivector:1120}
\gpgradeone{ \Be_i \Be_j \Be_k }
=
\lr{ \Be_i \cdot \Be_j } \Be_k
-\lr{ \Be_i \cdot \Be_k } \Be_j
+\lr{ \Be_j \cdot \Be_k } \Be_i,
\end{equation}
(since there is a term for each permutation of \( i, j, k \) and a sign change when that permutuation is not even.)
} % answer
