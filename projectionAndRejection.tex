%
% Copyright © 2017 Peeter Joot.  All Rights Reserved.
% Licenced as described in the file LICENSE under the root directory of this GIT repository.
%
Let's now look at how the dot plus wedge product
decomposition of the vector product can be applied to compute vector projection and rejection, which are defined as

\index{projection}
\index{rejection}
\makedefinition{Vector projection and rejection.}{dfn:projectionAndRejection:projectionandrejection}{
Given a vector \( \Bx \) and vector \( \Bu \) the projection of \( \Bx \) onto the direction of \( \Bu \) is defined as
\begin{equation*}
\Proj_\Bu(\Bx) = (\Bx \cdot \ucap) \ucap,
\end{equation*}
where \( \ucap = \Bu/\Norm{\Bu} \).
The rejection of \( \Bx \) by \( \Bu \) is defined as the component of \( \Bx \) that is
orthogonal to \( \Bu \)
\begin{equation*}
\Rej{\Bu}{\Bx} = \Bx - \Proj_\Bu(\Bx).
\end{equation*}
} % definition

%Calling \( \Rej{\Bu}{\Bx} \) the rejection of \( \Bx \) by \( \Bu \) is arguably not the clearest, but as this is the language that is used in the literature (\citep
An example of projection and rejection with respect to a direction vector \( \Bu \) is illustrated in
\cref{fig:projectionAndRejection:projectionAndRejectionFig1}.

\mathImageFigure{../figures/GAelectrodynamics/projectionAndRejectionFig1}{Projection and rejection illustrated.}{fig:projectionAndRejection:projectionAndRejectionFig1}{0.3}{projectionAndRejectionDiagram.nb}

Computation of the projective and rejective components of a vector \( \Bx \) relative to a direction \( \ucap \) requires little more than a multiplication by \( 1 = \ucap \ucap \), and some rearrangement
\begin{equation}\label{eqn:projectionAndRejection:680}
\begin{aligned}
\Bx
&= \Bx \ucap \ucap \\
&= \lr{ \Bx \ucap } \ucap \\
&= \Biglr{ \Bx \cdot \ucap + \Bx \wedge \ucap } \ucap \\
&= \lr{ \Bx \cdot \ucap } \ucap + \lr{ \Bx \wedge \ucap } \ucap.
\end{aligned}
\end{equation}

The vector \( \Bx \) is split nicely into its projection and rejective components in a complementary fashion
\begin{subequations}
\label{eqn:projectionAndRejection:1080}
\begin{equation}\label{eqn:projectionAndRejection:900}
\Proj_\Bu(\Bx) = \lr{ \Bx \cdot \ucap } \ucap
\end{equation}
\begin{equation}\label{eqn:projectionAndRejection:910}
\Rej{\Bu}{\Bx} = \lr{ \Bx \wedge \ucap } \ucap.
\end{equation}
\end{subequations}

By construction,
\( \lr{ \Bx \wedge \ucap } \ucap \) must be a vector, despite any appearance of a multivector nature.

The utility of this multivector rejection formula is not for hand or computer algebra calculations, where it will generally be faster and simpler to compute \( \Bx - (\Bx \cdot \ucap) \ucap \), than to use \cref{eqn:projectionAndRejection:910}.
Instead this will come in handy as a new abstract algebraic tool.

When it is desirable to perform this calculation explicitly, it can be done more efficiently using a no-op grade selection operation.
In particular, a vector can be written as its own grade-1 selection
\begin{equation}\label{eqn:projectionAndRejection:920}
\Bx = \gpgradeone{ \Bx },
\end{equation}
so the rejection can be re-expressed
using \cref{dfn:generalizedDot:100}
as a generalized bivector-vector dot product
\begin{equation}\label{eqn:projectionAndRejection:940}
\Rej{\Bu}{\Bx}
= \gpgradeone{ \lr{ \Bx \wedge \ucap } \ucap }
= \lr{ \Bx \wedge \ucap } \cdot \ucap.
\end{equation}

In \R{3}, using \cref{thm:generalizedDot:tripleCross}, the rejection operation can also be expressed as a vector triple product
\begin{equation}\label{eqn:projectionAndRejection:1140}
\Rej{\Bu}{\Bx}
= \ucap \cross \lr{ \Bx \cross \ucap }.
\end{equation}

To help establish some confidence with these new additions to our toolbox, here are a
pair of illustrative examples using
\cref{eqn:projectionAndRejection:910}, and
\cref{eqn:projectionAndRejection:940} respectively.

\makeexample{An \R{2} rejection.}{example:projectionAndRejection:1}{
Let \( \Bx = a \Be_1 + b \Be_2 \) and \( \Bu = \Be_1 \) for which the wedge is \( \Bx \wedge \ucap = b \Be_2 \Be_1 \).
Using \cref{eqn:projectionAndRejection:910} the rejection of \( \Bx \) by \( \Bu \) is
\begin{equation}\label{eqn:projectionAndRejection:1000}
\begin{aligned}
\Rej{\Bu}{\Bx}
&= \lr{ \Bx \wedge \ucap } \ucap \\
&= (b \Be_2 \Be_1 )\Be_1 \\
&= b \Be_2 (\Be_1 \Be_1) \\
&= b \Be_2,
\end{aligned}
\end{equation}
as expected.
} % example

This example provides some guidance about what is happening geometrically in
\cref{eqn:projectionAndRejection:910}.
The wedge operation produces a pseudoscalar for the plane spanned by \( \setlr{\Bx, \Bu} \) that is scaled as \( \sin\theta \) where \( \theta \) is the angle between \( \Bx \) and \( \Bu \).
When that pseudoscalar is multiplied by \( \ucap \), \( \ucap \) is rotated in the plane by \( \pi/2 \) radians towards \( \Bx \), yielding the normal component of the vector \( \Bx \).

Here's a slightly less trivial \R{3} example

\makeexample{An \R{3} rejection.}{example:projectionAndRejection:r3rejection}{
Let \( \Bx = a \Be_2 + b \Be_3 \) and \( \ucap = ( \Be_1 + \Be_2 )/\sqrt{2} \) for which the
wedge product is
\begin{equation}\label{eqn:projectionAndRejection:1040}
\begin{aligned}
\Bx \wedge \ucap
&= \inv{\sqrt{2}}
\begin{vmatrix}
\Be_{23} & \Be_{31} & \Be_{12} \\
0 & a & b \\
1 & 1 & 0
\end{vmatrix} \\
&=
\inv{\sqrt{2}}
\lr{
\Be_{23}(-b) - \Be_{31}(-b) + \Be_{12} (-a)
} \\
&=
\inv{\sqrt{2}}
\lr{
b (\Be_{32} + \Be_{31} ) + a \Be_{21}
}.
\end{aligned}
\end{equation}

Using \cref{eqn:projectionAndRejection:940} the rejection of \( \Bx \) by \( \Bu \) is
\begin{equation}\label{eqn:projectionAndRejection:1060}
(\Bx \wedge \ucap) \cdot \ucap
=
\inv{2}
\lr{ b (\Be_{32} + \Be_{31} ) + a \Be_{21} } \cdot ( \Be_1 + \Be_2 ).
\end{equation}

Each of these bivector-vector dot products has the form \( \Be_{rs} \cdot \Be_t = \gpgradeone{ \Be_{rst} } \) which is zero whenever the indexes \( r,s, t\) are unique, and is a vector whenever one of indexes are repeated (\( r = t \), or \( s = t \)).
This leaves
\begin{equation}\label{eqn:projectionAndRejection:1100}
\begin{aligned}
(\Bx \wedge \ucap) \cdot \ucap
&= \inv{2} \lr{ b \Be_3 + a \Be_2 + b \Be_3 - a \Be_1 } \\
&= b \Be_3 + \frac{a}{2}( \Be_2 - \Be_1 ).
\end{aligned}
\end{equation}
} % example

\makeexample{Velocity and acceleration in polar coordinates.}{example:projectionAndRejection:1160}{
In \cref{eqn:2dRotations:1280}, and \cref{eqn:2dRotations:1300} we found the polar representation of the velocity and acceleration vectors associated with the radial parameterization \( \Br(r, \theta) = r \rcap(\theta) \).

We can alternatively compute the radial and azimuthal components of these vectors in terms of their projective and rejective components
\begin{equation}\label{eqn:projectionAndRejection:1180}
\begin{aligned}
\Bv &= {\Bv \rcap } \rcap = \lr{\Bv \cdot \rcap  + \Bv \wedge \rcap } \rcap \\
\Ba &= {\Ba \rcap } \rcap = \lr{\Ba \cdot \rcap  + \Ba \wedge \rcap } \rcap,
\end{aligned}
\end{equation}
so
\begin{equation}\label{eqn:projectionAndRejection:1200}
\begin{aligned}
\Bv \cdot \rcap &= r' \\
\Bv \wedge \rcap &= r \omega \thetacap \wedge \rcap = \omega \thetacap \wedge \Br \\
\Ba \cdot \rcap &= r'' - r \omega^2 \\
\Ba \wedge \rcap &=
\inv{r} \lr{ r^2 \omega }' \thetacap \wedge \rcap
%=
%\lr{ r^2 \omega }' \thetacap \wedge \inv{\Br}
.
\end{aligned}
\end{equation}

We see that it is natural to introduce angular velocity and acceleration bivectors.
These both lie in the \( \thetacap \wedge \rcap \) plane.
Of course, it is also possible to substitute the cross product for the wedge product, but doing so requires the introduction of a normal direction that may not intrinsically be part of the problem (i.e. two dimensional problems).
} % example

In the GA literature the projection and rejection operations are usually written using the vector inverse.
\index{vector inverse}
\index{\(\Bx^{-1}\)}
\makedefinition{Vector inverse.}{dfn:projectionAndRejection:vectorinverse}{
Define the inverse of a vector \( \Bx \), when it exists, as
\begin{equation*}
\Bx^{-1} = \frac{\Bx}{\Norm{\Bx}^2}.
\end{equation*}

This inverse satisfies \( \Bx^{-1} \Bx = \Bx \Bx^{-1} = 1 \).
} % definition

The vector inverse may not exist in a non-Euclidean vector space where \( \Bx^2 \) can be zero for non-zero vectors \( \Bx \).

In terms of the vector inverse, the projection and rejection operations with respect to \( \Bu \) can be written without any reference to the unit vector \( \ucap = \Bu/\Norm{\Bu} \) that lies along that vector
\boxedEquation{eqn:projectionAndRejection:1120}{
\begin{aligned}
\Proj_\Bu(\Bx) &= \lr{ \Bx \cdot \Bu } \inv{\Bu} \\
\Rej{\Bu}{\Bx} &=
\lr{ \Bx \wedge \Bu } \inv{\Bu}
=
\lr{ \Bx \wedge \Bu } \cdot \inv{\Bu}.
\end{aligned}
}

It was claimed in the definition of rejection that the rejection is orthogonal to the projection.
This can be shown trivially without any use of GA (\cref{problem:projectionAndRejection:rejectionnormality}).
This also follows naturally using the grade selection operator representation of the dot product
\begin{equation}\label{eqn:projectionAndRejection:720}
\begin{aligned}
\Rej{\Bu}{\Bx} \cdot \Proj_\Bu(\Bx)
&= \gpgradezero{ \Rej{\Bu}{\Bx} \Proj_\Bu(\Bx) } \\
&= \gpgradezero{ \lr{ \Bx \wedge \ucap } \ucap \lr{ \Bx \cdot \ucap } \ucap } \\
&= \lr{ \Bx \cdot \ucap } \gpgradezero{ \lr{ \Bx \wedge \ucap } \ucap^2 } \\
&= \lr{ \Bx \cdot \ucap } \gpgradezero{ \Bx \wedge \ucap } \\
&= 0.
\end{aligned}
\end{equation}
This is zero because the scalar grade of a wedge product, a bivector, is zero by definition.
\makeproblem{Rejection orthogonality.}{problem:projectionAndRejection:rejectionnormality}{
Prove, without any use of GA, that \( \Bx - \Proj_\Bu(\Bx) \) is orthogonal to \( \Bu \), as claimed in
\cref{dfn:projectionAndRejection:projectionandrejection}.
} % problem
\makeanswer{problem:projectionAndRejection:rejectionnormality}{
\begin{equation}\label{eqn:projectionAndRejection:1220}
\begin{aligned}
\lr{ \Bx - \Proj_\ucap(\Bx) } \cdot \ucap
&=
\Bx \cdot \ucap - \lr{ \lr{ \Bx \cdot \ucap } \ucap } \cdot \ucap \\
&=
\Bx \cdot \ucap - \lr{ \Bx \cdot \ucap } \lr{ \ucap \cdot \ucap } \\
&=
\Bx \cdot \ucap - \Bx \cdot \ucap \\
&=
0.
\end{aligned}
\end{equation}
} % answer
\makeproblem{Rejection example.}{problem:projectionAndRejection:rejectionExample1}{
\makesubproblem{}{problem:projectionAndRejection:rejectionExample1:a}
Repeat \cref{example:projectionAndRejection:r3rejection} by calculating \( (\Bx \wedge \ucap) \ucap \) and show that all the grade three components of this multivector product vanish.
\makesubproblem{}{problem:projectionAndRejection:rejectionExample1:b}
Compute \( \Bx - (\Bx \cdot \ucap) \ucap \) and show that this matches \cref{eqn:projectionAndRejection:1100}.
} % problem
\makeanswer{problem:projectionAndRejection:rejectionExample1}{
\makesubanswer{}{problem:projectionAndRejection:rejectionExample1:a}
Given \( \Bx = a \Be_2 + b \Be_3 \) and \( \ucap = (\Be_1 + \Be_2)/\sqrt{2} \), we found that
\begin{equation}\label{eqn:projectionAndRejection:1240}
\Bx \wedge \ucap = \inv{\sqrt{2}} \lr{ b \lr{ \Be_{32} + \Be_{31} } + a \Be_{21} }.
\end{equation}
Multiplying once more by \( \ucap \) on the right, we have
\begin{equation}\label{eqn:projectionAndRejection:1260}
\begin{aligned}
\lr{ \Bx \wedge \ucap }
\ucap
&=
\inv{2} \lr{ b \lr{ \Be_{32} + \Be_{31} } + a \Be_{21} } \lr{ \Be_1 + \Be_2 } \\
&=
\inv{2} \lr{
b \lr{ \Be_{321} + \Be_{311} } + a \Be_{211}
+ b \lr{ \Be_{322} + \Be_{312} } + a \Be_{212}
} \\
&=
\inv{2} \lr{
b \lr{ \Be_{321} + \Be_{3} } + a \Be_{2}
+ b \lr{ \Be_{3} + \Be_{312} } - a \Be_{1}
} \\
&=
\inv{2} \lr{
2 b \Be_{3} + a \lr{ \Be_{2} - \Be_1 }
}.
\end{aligned}
\end{equation}
We are left with \( \lr{ \Bx \wedge \ucap } \ucap = \lr{ \Bx \wedge \ucap } \cdot \ucap \), since all the trivector components cancel perfectly.
\makesubanswer{}{problem:projectionAndRejection:rejectionExample1:b}
Now we can compare the above to \( \Bx - \lr{\Bx \cdot \ucap } \ucap \).  First
\begin{equation}\label{eqn:projectionAndRejection:1280}
\Bx \cdot \ucap
=
\lr{ a \Be_2 + b \Be_3 } \cdot \lr{ \Be_1 + \Be_2} \inv{ \sqrt{2} }
=
\frac{a}{\sqrt{2}},
\end{equation}
so
\begin{equation}\label{eqn:projectionAndRejection:1300}
\begin{aligned}
\Bx - \lr{ \Bx \cdot \ucap } \ucap
&=
a \Be_2 + b \Be_3 - \frac{a}{2} \lr{ \Be_1 + \Be_2 } \\
&=
b \Be_3 + \frac{a}{2} \lr{ -\Be_1 + \Be_2 },
\end{aligned}
\end{equation}
as calculated from \( \lr{ \Bx \wedge \ucap } \ucap \).
} % answer
%\makeproblem{Prove \ref{dfn:projectionAndRejection:r3pcommutation}.}{problem:projectionAndRejection:1160}{
%} % problem
