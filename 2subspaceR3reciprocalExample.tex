%
% Copyright © 2016 Peeter Joot.  All Rights Reserved.
% Licenced as described in the file LICENSE under the root directory of this GIT repository.
%

\index{reciprocal frame}
\makeproblem{Two vector reciprocal frame}{problem:2subspaceR3reciprocalExample:2subspaceR3reciprocalExample}{
Calculate the reciprocal frame for the \R{3} subspace spanned by \( \setlr{ \Bx_1, \Bx_2 } \) where
\begin{dmath}\label{eqn:2subspaceR3reciprocalExample:20}
\begin{aligned}
\Bx_1 &= \Be_1 + 2 \Be_2 \\
\Bx_2 &= \Be_2 - \Be_3.
\end{aligned}
\end{dmath}
} % problem

\makeanswer{problem:2subspaceR3reciprocalExample:2subspaceR3reciprocalExample}{
The bivector for the plane spanned by this basis is
\begin{equation}\label{eqn:2subspaceR3reciprocalExample:40}
\begin{aligned}
\Bx_1 \wedge \Bx_2
&= \lr{ \Be_1 + 2 \Be_2 } \wedge \lr{ \Be_2 - \Be_3 } \\
&= \Be_{12} - \Be_{13} - 2 \Be_{23} \\
&= \Be_{12} + \Be_{31} + 2 \Be_{32}.
\end{aligned}
\end{equation}

This has the square
\begin{equation}\label{eqn:2subspaceR3reciprocalExample:60}
\begin{aligned}
\lr{ \Bx_1 \wedge \Bx_2 }^2
&=
\lr{ \Be_{12} + \Be_{31} + 2 \Be_{32} }
\cdot
\lr{ \Be_{12} + \Be_{31} + 2 \Be_{32} } \\
&= -1 -1 -4 \\
&= -6.
\end{aligned}
\end{equation}

Dotting \( -\Bx_1 \) with the bivector is
\begin{equation}\label{eqn:2subspaceR3reciprocalExample:80}
\begin{aligned}
\Bx_1 \cdot \lr{ \Bx_2 \wedge \Bx_1 }
&= -\lr{ \Be_1 + 2 \Be_2 } \cdot \lr{\Be_{12} + \Be_{31} + 2 \Be_{32} } \\
&= -\lr{ \Be_2 - \Be_3 - 2 \Be_1 - 4 \Be_3 } \\
&= 2 \Be_1 - \Be_2 + 5 \Be_3.
\end{aligned}
\end{equation}

For \( \Bx_2 \) the dot product with the bivector is
\begin{equation}\label{eqn:2subspaceR3reciprocalExample:100}
\begin{aligned}
\Bx_2 \cdot \lr{ \Bx_1 \wedge \Bx_2 }
&= \lr{ \Be_2 - \Be_3 } \cdot \lr{\Be_{12} + \Be_{31} + 2 \Be_{32} } \\
&= - \Be_1 - 2 \Be_3 - \Be_1 - 2 \Be_2 \\
&= - 2 \Be_1 - 2 \Be_2 - 2 \Be_3,
\end{aligned}
\end{equation}
so
\begin{equation}\label{eqn:2subspaceR3reciprocalExample:120}
\begin{aligned}
\Bx^1 &= \inv{3} \lr{ \Be_1 + \Be_2 + \Be_3 } \\
\Bx^2 &= \inv{6} \lr{ -2 \Be_1 + \Be_2 - 5 \Be_3 }.
\end{aligned}
\end{equation}
It is easy to verify that this has the desired semantics.
} % answer
