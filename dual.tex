%
% Copyright © 2017 Peeter Joot.  All Rights Reserved.
% Licenced as described in the file LICENSE under the root directory of this GIT repository.
%
Pseudoscalar multiplication maps a subspace to its orthogonal complement, called the dual.

\index{dual}
\index{\(M^\conj\)}
\makedefinition{Dual}{dfn:definitions:dual}{
Given a multivector \( M \) and a unit pseudoscalar \( I \) for the space, the dual is
designated \( M^\conj \), and has the value \( M^\conj = M I \).
} % definition

For example, in \R{2} with \( i = \Be_{12} \), the dual of a vector \( \Bx = x \Be_1 + y \Be_2 \) is

\begin{equation}\label{eqn:dual:n}
\begin{aligned}
\Bx i
&= (x \Be_1 + y \Be_2) i  \\
&= x \Be_2 - y \Be_1,
\end{aligned}
\end{equation}
which is perpendicular to \( \Bx \).
This was also observed in
\cref{eqn:2dRotations:300} and \cref{eqn:2dRotations:3} which showed that multiplying a vector in a plane by a unit pseudoscalar for that plane, maps a vector (say \( \Bx \)) to a vector \( \Bx i \) that is orthogonal to \( \Bx \).  The direction that \( \Bx i \) points depends on the orientation of the chosen pseudoscalar.
%For the x-y plane with \( i = \Be_{12} \), these can be summarized in polar form as
%\begin{equation}\label{eqn:dual:1620}
%\begin{aligned}
%\Br &= r \Be_1 e^{i\theta} \\
%\Br i &= r \Be_2 e^{i\theta} \\
%i \Br &= -r \Be_2 e^{i\theta}.
%\end{aligned}
%\end{equation}
%
%Both \( i \Br \) and \( \Br i \) are perpendicular to \( \Br \) with the same magnitude.
%The direction they point depends on the orientation of the pseudoscalar chosen.  For example,
%had we used an oppositely oriented pseudoscalar \( i = \Be_2 \Be_1 \), these orthogonal vectors would each have the opposite direction.
%

In three dimensions, a bivector can be factored into two orthogonal vector factors, say \( B = \Ba \Bb \), and
pseudoscalar multiplication of \( B I = \Bc \) produces a vector \( \Bc \) that is orthogonal to the factors \( \Ba, \Bb \).
For example, the unit vectors and bivectors are related in the following fashion
\begin{equation}\label{eqn:dual:1580}
\begin{aligned}
\Be_2 \Be_3 &= \Be_1 I \qquad \Be_2 \Be_3 I = -\Be_1 \\
\Be_3 \Be_1 &= \Be_2 I \qquad \Be_3 \Be_1 I = -\Be_2 \\
\Be_1 \Be_2 &= \Be_3 I \qquad \Be_1 \Be_2 I = -\Be_3.
\end{aligned}
\end{equation}

For example, with \( \Br = a \Be_1 + b \Be_2 \), the dual is
\begin{equation}\label{eqn:dual:1640}
\begin{aligned}
\Br I
&= \lr{ a \Be_1 + b \Be_2 } \Be_{123} \\
&= a \Be_{23} + b \Be_{31} \\
&= \Be_3 \lr{ -a \Be_{2} + b \Be_{1} }.
\end{aligned}
\end{equation}

Here \( \Be_3 \) was factored out of the resulting bivector, leaving two factors both perpendicular to the original vector.  Every vector that lies in the span of the plane represented by this bivector is perpendicular to the original vector.
%%
This is illustrated in \cref{fig:dualityInR3:dualityInR3Fig1}.
\pmathImageFigure{../figures/GAelectrodynamics/\subfigdir/}{dualityInR3Fig1}{\R{3} duality illustration.}{fig:dualityInR3:dualityInR3Fig1}{0.2}{dualityInR3Fig1.nb}

%Using pseudoscalar multiplication to produce orthogonal subspaces
%is referred to as a duality transformation.
Some notes on duality

\begin{itemize}
\item The dual of any scalar is a pseudoscalar, whereas the dual of a pseudoscalar is a scalar.
\item The dual of a k-vector in a N-dimensional space is an \((N-k)\)-vector.
For example, \cref{eqn:dual:1580} showed that the dual of a 1-vector in \R{3} was a \((3-1)\)-vector, and the dual of a 2-vector is a \((3-2)\)-vector.  In \R{7}, say, the dual of a 2-vector is a 5-vector, the dual of a 3-vector is a 4-vector, and so forth.
\item All factors of the dual \((N-k)\)-vector are orthogonal to all the factors of the k-vector.  Looking to \cref{eqn:dual:1580} for examples, we see that the dual of the bivector \( \Be_2 \Be_3 \) is \( \Be_1 \), and both factors of the bivector \( \Be_2, \Be_3 \) are orthogonal to the dual of that bivector \( \Be_1 \).
%\item If \( A \) is a non-scalar k-vector, none of the vector factors of \( A \) and \( A^\conj \) are common.
%\item In a sense that can be defined more precisely once the general dot product operator is defined, the dual to a given k-vector \( A \) represents a \((N-k)\)-vector object that is orthogonal to \( A \).
\item Some authors use different sign conventions for duality, in particular, designating the dual as \( M I^{-1} \), which can have a different sign.  As one may choose pseudoscalars that differ in sign anyways, the duality convention doesn't matter too much, provided one is consistent.
%\item For non-Euclidean spaces, in particular ``null spaces'' where the square of vectors can be null, a different definition of duality may be required.
\end{itemize}

%%%As a multivector may not have an obvious geometric interpretation, a geometric
%%%interpretation of a general duality transformation may not be any more obvious.
%%%For example in \R{3}, given \( M = 1 + \Be_{23} - \Be_{123} \), its dual is
%%%
%%%\begin{equation}\label{eqn:dual:460}
%%%M I
%%%=
%%%\lr{ 1 + \Be_{23} - \Be_{123} } I
%%%=
%%%I - \Be_1 + 1.
%%%\end{equation}

%R^2 : M = 1 + i. M^* = i - 1.
%Because this particular multivector had a complex structure the duality operation can be interpreted as a rotation of a vector.

%When working with multivector integrals it will be useful to consider the differential volume element a volume weighted pseudoscalar.
%The dual of a k-vector \( X \) is an (n-k)vector, and can be thought of as an object of the same magnitude containing all the
%vectors that are not factors of \( X \).

%%%The dual vectors to the \R{2} basis vectors are those same vectors rotated by \( \pi/2 \)
%%%
%%%\begin{equation}\label{eqn:dual:360}
%%%\begin{aligned}
%%%\Be_1 \Be_{12} &= \Be_2 \\
%%%\Be_2 \Be_{12} &= -\Be_1,
%%%\end{aligned}
%%%\end{equation}
%%%
%%%with an inverse duality transformation given by the multiplication with \( \Be_{12}^{-1} = \Be_{21} \)
%%%
%%%\begin{equation}\label{eqn:dual:440}
%%%\begin{aligned}
%%%\Be_2 \Be_{21} &= \Be_1 \\
%%%-\Be_1 \Be_{21} &= \Be_2.
%%%\end{aligned}
%%%\end{equation}
%%%
%%%The \R{3} duals to the basis vectors are bivectors
%%%
%%%\begin{equation}\label{eqn:dual:380}
%%%\begin{aligned}
%%%\Be_1 \Be_{123} &= \Be_{23} \\
%%%\Be_2 \Be_{123} &= \Be_{31} \\
%%%\Be_3 \Be_{123} &= \Be_{12},
%%%\end{aligned}
%%%\end{equation}
%%%
%%%whereas the duals to those bivectors with respect to the pseudoscalar \( I^{-1} = \Be_{321} \) are the original basis vectors
%%%
%%%\begin{equation}\label{eqn:dual:400}
%%%\begin{aligned}
%%%\Be_{23} \Be_{321} &= \Be_1 \\
%%%\Be_{31} \Be_{321} &= \Be_2 \\
%%%\Be_{12} \Be_{321} &= \Be_3.
%%%\end{aligned}
%%%\end{equation}
