%
% Copyright © 2023 Peeter Joot.  All Rights Reserved.
% Licenced as described in the file LICENSE under the root directory of this GIT repository.
%
%{
%\input{../latex/blogpost.tex}
%\renewcommand{\basename}{ellipticproblem}
%\renewcommand{\dirname}{notes/phy1520/}
%\renewcommand{\dirname}{notes/ece1228-electromagnetic-theory/}
%\newcommand{\dateintitle}{}
%\newcommand{\keywords}{}
%
%\input{../latex/peeter_prologue_print2.tex}
%
%\usepackage{peeters_layout_exercise}
%\usepackage{peeters_braket}
%\usepackage{peeters_figures}
%\usepackage{amsthm}
%\usepackage{siunitx}
%\usepackage{verbatim}
%\usepackage{mhchem} % \ce{}
%\usepackage{macros_bm} % \bcM
%\usepackage{macros_qed} % \qedmarker
%\usepackage{txfonts} % \ointclockwise
%
%\beginArtNoToc
%
%\generatetitle{XXX}
%\chapter{XXX}
%\label{chap:ellipticproblem}
%
\makeproblem{Hyperbolic identities.}{lemma:ellipticproblem:280}{
Show that
\begin{equation}\label{eqn:ellipticproblem:300}
2 \cosh\lr{ \mu - i \theta } \sinh\lr{ \mu + i \theta } = \sinh(2 \mu) + i \sin(2 \theta).
\end{equation}
\begin{equation}\label{eqn:ellipticproblem:420}
2 \cosh\lr{ \mu } \sinh\lr{ \mu } = \sinh(2 \mu).
\end{equation}
\begin{equation}\label{eqn:ellipticproblem:440}
\cosh\lr{ \mu + i \theta } = \cosh\mu \cos\theta + i \sinh\mu \sin\theta.
\end{equation}
} % problem
\makeanswer{lemma:ellipticproblem:280}{
\begin{equation}\label{eqn:ellipticproblem:320}
\begin{aligned}
\cosh\lr{ \mu - i \theta } \sinh\lr{ \mu + i \theta }
&=
\frac{1}{4}
   \lr{ e^{\mu - i\theta} - e^{-\mu + i \theta } } \lr{ e^{\mu + i\theta} - e^{-\mu - i \theta } }
\\
&=
\frac{1}{4} \lr{
   e^{2 \mu} - e^{-2\mu} + e^{2 i \theta} - e^{-2 i \theta}
} \\
&=
\inv{2} \lr{  \sinh(2 \mu) + i \sin(2 \theta) }.
\end{aligned}
\end{equation}
The second identity follows from the first, setting \( \theta = 0 \).
Finally, for the third expanding the \( \cosh \) in terms of exponentials, we find
\begin{equation}\label{eqn:ellipticproblem:50}
\begin{aligned}
\cosh\lr{ \mu + i \theta }
&=
\frac{1}{2} \lr{ e^{\mu + i \theta} + e^{-\mu - i\theta} } \\
&=
\frac{e^\mu}{2} \lr{ \cos\theta + i \sin\theta }
+
\frac{e^{-\mu}}{2} \lr{ \cos\theta - i \sin\theta } \\
&=
\frac{ e^\mu + e^{-\mu} }{2} \cos\theta
+ i \frac{ e^\mu - e^{-\mu} }{2} \sin\theta \\
&=
\cosh\mu \cos\theta + i \sinh\mu \sin\theta.
\end{aligned}
\end{equation}
}

\makeproblem{Elliptic curvilinear and reciprocal basis.}{problem:ellipticproblem:10}{
\makesubproblem{}{problem:ellipticproblem:10:a}
Show that an ellipse can be parameterized by
\begin{equation}\label{eqn:ellipticproblem:20}
   \Bx = u_1 \Be_1 \cosh\lr{ \mu + i u_2 },
\end{equation}
where \( i = \Be_{12} \), and find the values of the semi-major and semi-minor axes.
\makesubproblem{}{problem:ellipticproblem:10:b}
Determine how \( \mu \) and the eccentricity \( \epsilon = \sqrt{1 - b^2/a^2} \) are related.
\makesubproblem{}{problem:ellipticproblem:10:c}
Compute the curvilinear and reciprocal frame vectors for the parameterization \( \Bx(u_1, u_2) \) above.
%, and use this to verify
%\cref{eqn:curvilinearDefined:520} and \cref{eqn:curvilinearDefined:540} respectively.
\makesubproblem{}{problem:ellipticproblem:10:d}
Check that \( \Bx^i \cdot \Bx_j = {\delta^i}_j \).
} % problem
\makeanswer{problem:ellipticproblem:10}{
\makesubanswer{}{problem:ellipticproblem:10:a}
Using the multiple angle \( \cosh \) expansion, we find
\begin{equation}\label{eqn:ellipticproblem:40}
\begin{aligned}
\Be_1 \cosh\lr{ \mu + i u_2 }
&=
\Be_1 \lr{
\cosh\mu \cos u_2 + i \sinh\mu \sin u_2
} \\
&=
\Be_1 \cosh\mu \cos u_2 + \Be_2 \sinh\mu \sin u_2,
\end{aligned}
\end{equation}
so
\begin{equation}\label{eqn:ellipticproblem:60}
\Bx = u_1 \Be_1 \cosh\lr{ \mu + i u_2 } = \Be_1 a \cos u_2 + \Be_2 b \sin u_2,
\end{equation}
where
\begin{equation}\label{eqn:ellipticproblem:80}
\begin{aligned}
   a &= u_1 \cosh\mu \\
   b &= u_1 \sinh\mu,
\end{aligned}
\end{equation}
are the semi-major and semi-minor axis values.
\makesubanswer{}{problem:ellipticproblem:10:b}
The eccentricity (squared) is
\begin{equation}\label{eqn:ellipticproblem:100}
\begin{aligned}
   \epsilon^2
   &= 1 - \tanh^2\mu \\
   &= \frac{\cosh^2\mu - \sinh^2\mu}{\cosh^2\mu} \\
   &= \inv{\cosh^2\mu},
\end{aligned}
\end{equation}
so the eccentricity is
\begin{equation}\label{eqn:ellipticproblem:120}
   \epsilon = \inv{\cosh\mu}.
\end{equation}
\makesubanswer{}{problem:ellipticproblem:10:c}
Our curvilinear basis vectors are
\begin{equation}\label{eqn:ellipticproblem:140}
\begin{aligned}
\Bx_1 &= \Be_1 \cosh\lr{ \mu + i u_2 } \\
\Bx_2 &= \Be_2 u_1 \sinh\lr{ \mu + i u_2 } \\
\end{aligned}
\end{equation}

To compute the reciprocals we need the area element
\begin{equation}\label{eqn:ellipticproblem:160}
\begin{aligned}
\Bx_1 \wedge \Bx_2
&=
\gpgradetwo{
   \Be_1 \cosh\lr{ \mu + i u_2 } \Be_2 u_1 \sinh\lr{ \mu + i u_2 }
} \\
&=
u_1 \gpgradetwo{
   i \cosh\lr{ \mu - i u_2 } \sinh\lr{ \mu + i u_2 }
} \\
&=
\frac{u_1}{2} \gpgradetwo{
   i \lr{ \sinh(2 \mu) + i \sin(2 u_2) }
} \\
&=
u_1 i \cosh \mu \sinh \mu.
\end{aligned}
\end{equation}

Our recipocal basis vectors are
\begin{equation}\label{eqn:ellipticproblem:240}
\begin{aligned}
   \Bx^1
   &= \Bx_2 \inv{ \Bx_1 \wedge \Bx_2 } \\
   &= \Be_2 u_1 \sinh\lr{ \mu + i u_2 } \inv{u_1 i \cosh \mu \sinh \mu} \\
   &= \Be_1 \frac{\sinh\lr{ \mu + i u_2 }}{\cosh \mu \sinh \mu},
\end{aligned}
\end{equation}
and
\begin{equation}\label{eqn:ellipticproblem:180}
\begin{aligned}
\Bx^2
   &= -\Bx_1 \inv{ \Bx_1 \wedge \Bx_2 } \\
   &= -\lr{ \Be_1 \cosh\lr{ \mu + i u_2 } } \inv{ u_1 i \cosh \mu \sinh \mu} \\
   &= \frac{ \Be_2 \cosh\lr{ \mu + i u_2 } }{ u_1 \cosh \mu \sinh \mu}.
\end{aligned}
\end{equation}
\makesubanswer{}{problem:ellipticproblem:10:d}
\begin{equation}\label{eqn:ellipticproblem:260}
\begin{aligned}
\Bx_1 \cdot \Bx^1
   &= \gpgradezero{
\Be_1 \cosh\lr{ \mu + i u_2 } \Be_1 \frac{\sinh\lr{ \mu + i u_2 }}{\cosh \mu \sinh \mu}
} \\
   &=
\inv{\cosh \mu \sinh \mu}
\gpgradezero{
\Be_1^2 \cosh\lr{ \mu - i u_2 } \lr{\sinh\lr{ \mu + i u_2 }}
} \\
   &=
\inv{\cosh \mu \sinh \mu}
\gpgradezero{
   \inv{2} \lr{ \sinh(2 \mu) + i \sin(2 u_2) }
} \\
&=
\frac{\sinh(2 \mu) }{2 \cosh \mu \sinh \mu} \\
&= 1.
\end{aligned}
\end{equation}
\begin{equation}\label{eqn:ellipticproblem:360}
\begin{aligned}
\Bx_2 \cdot \Bx^2
&=
\gpgradezero{
\Be_2 u_1 \sinh\lr{ \mu + i u_2 }
\frac{ \Be_2 \cosh\lr{ \mu + i u_2 } }{ u_1 \cosh \mu \sinh \mu}
} \\
&=
\inv{ \cosh \mu \sinh \mu}
\gpgradezero{
\sinh\lr{ \mu + i u_2 }
\Be_2^2 \cosh\lr{ \mu - i u_2 }
} \\
&= 1.
\end{aligned}
\end{equation}
\begin{equation}\label{eqn:ellipticproblem:380}
\begin{aligned}
\Bx_1 \cdot \Bx^2
&=
\gpgradezero{
\Be_1 \cosh\lr{ \mu + i u_2 }
\frac{ \Be_2 \cosh\lr{ \mu + i u_2 } }{ u_1 \cosh \mu \sinh \mu}
} \\
&=
\inv{ u_1 \cosh \mu \sinh \mu}
\gpgradezero{
   \Be_{12} \cosh\lr{ \mu - i u_2 } \cosh\lr{ \mu + i u_2 }
} \\
&=
\frac{ \Abs{ \cosh\lr{ \mu + i u_2 } }^2 }
{ u_1 \cosh \mu \sinh \mu}
\gpgradezero{
   \Be_{12}
} \\
&= 0.
\end{aligned}
\end{equation}
\begin{equation}\label{eqn:ellipticproblem:400}
\begin{aligned}
\Bx^1 \cdot \Bx_2
&=
\gpgradezero{
\Be_1 \frac{\sinh\lr{ \mu + i u_2 }}{\cosh \mu \sinh \mu}
\Be_2 u_1 \sinh\lr{ \mu + i u_2 }
} \\
&=
\frac{ u_1 \Abs{\sinh\lr{ \mu + i u_2 }}^2 }{\cosh \mu \sinh \mu} \gpgradezero{ \Be_{12} } \\
&= 0.
\end{aligned}
\end{equation}
} % answer
%}
%\EndNoBibArticle
