%
% Copyright � 2017 Peeter Joot.  All Rights Reserved.
% Licenced as described in the file LICENSE under the root directory of this GIT repository.
%
%{
\index{curvilinear coordinates}
\index{spherical coordinates}
The spherical vector parameterization admits a compact GA representation.
From the coordinate representation, some factoring gives
\begin{equation}\label{eqn:curvilinearspherical:20}
\begin{aligned}
\Bx
&= r \lr{ \Be_1 \sin\theta \cos\phi + \Be_2 \sin\theta \sin\phi + \Be_3 \cos\theta } \\
&= r \lr{ \sin\theta \Be_1 (\cos\phi + \Be_{12} \sin\phi ) + \Be_3 \cos\theta } \\
&= r \lr{ \sin\theta \Be_1 e^{\Be_{12} \phi } + \Be_3 \cos\theta } \\
&= r \Be_3 \lr{ \cos\theta + \sin\theta \Be_3 \Be_1 e^{\Be_{12} \phi } }.
\end{aligned}
\end{equation}

With
\begin{equation}\label{eqn:curvilinearspherical:40}
\begin{aligned}
i &= \Be_{12} \\
j &= \Be_{31} e^{i \phi},
\end{aligned}
\end{equation}
this is
\begin{equation}\label{eqn:curvilinearspherical:60}
\Bx = r \Be_3 e^{j \theta}.
\end{equation}

The curvilinear basis can be easily be computed in software
\begin{mmaCell}[moredefined={i, j, ej, x, xr, xt, xp, e, GeometricProduct},morepattern={phi_, t_, p_, r_, theta_}]{Input}
  ClearAll[i, j, ej, x, xr, xt, xp]
  i = e[1, 2];
  j[phi_] = GeometricProduct[e[3, 1], Cos[phi] + i Sin[phi]];
  ej[t_, p_] = Cos[t] + j[p] Sin[t];
  x[r_, t_, p_] = r GeometricProduct[e[3], ej[t, p]];

  xr[r_, theta_, phi_] = D[x[a, theta, phi], a] /. a\(\pmb{\to}\)r;
  xt[r_, theta_, phi_] = D[x[r, t, phi], t] /. t\(\pmb{\to}\)theta;
  xp[r_, theta_, phi_] = D[x[r, theta, p], p] /. p\(\pmb{\to}\)phi;

  \{x[r, \mmaUnd{\(\pmb{\theta}\)}, \mmaUnd{\(\pmb{\phi}\)}],
  xr[r, \mmaUnd{\(\pmb{\theta}\)}, \mmaUnd{\(\pmb{\phi}\)}],
  xt[r, \mmaUnd{\(\pmb{\theta}\)}, \mmaUnd{\(\pmb{\phi}\)}],
  xp[r, \mmaUnd{\(\pmb{\theta}\)}, \mmaUnd{\(\pmb{\phi}\)}]\} // Column
\end{mmaCell}
\begin{mmaCell}{Output}
  r (Cos[\(\theta\)] e[3] + Cos[\(\phi\)] e[1] Sin[\(\theta\)] + e[2] Sin[\(\theta\)] Sin[\(\phi\)])
  Cos[\(\theta\)] e[3] + Cos[\(\phi\)] e[1] Sin[\(\theta\)] + e[2] Sin[ \(\theta\)] Sin[\(\phi\)]
  r (Cos[\(\theta\)] Cos[\(\phi\)] e[1] - e[3] Sin[\(\theta\)] + Cos[\(\theta\)] e[2] Sin[\(\phi\)])
  r (Cos[\(\phi\)] e[2] Sin[\(\theta\)] - e[1] Sin[\(\theta\)] Sin[\(\phi\)])
\end{mmaCell}

Unfortunately the compact representation is lost doing so.  Computing the basis elements manually, we find

\begin{subequations}
\label{eqn:curvilinearspherical:80}
\begin{equation}\label{eqn:curvilinearspherical:100}
\Bx_r = \Be_3 e^{j \theta}
\end{equation}
\begin{equation}\label{eqn:curvilinearspherical:120}
\begin{aligned}
\Bx_\theta
&= r \Be_3 j e^{j \theta} \\
&= r \Be_3 \Be_{31} e^{i\phi} e^{j \theta} \\
&= r \Be_1 e^{i\phi} e^{j \theta}
\end{aligned}
\end{equation}
\begin{equation}\label{eqn:curvilinearspherical:140}
\begin{aligned}
\Bx_\phi
&= \PD{\phi}{} \lr{ r \Be_3 \sin\theta \Be_{31} e^{i \phi} } \\
&= r \sin\theta \Be_1 \Be_{12} e^{i \phi} \\
&= r \sin\theta \Be_2 e^{i \phi}.
\end{aligned}
\end{equation}
\end{subequations}
These are all mutually orthogonal, which can be verified by computing dot products.

\begin{mmaCell}[moredefined={x1, x2, x3, xr, xt, xp, InnerProduct}]{Input}
  ClearAll[x1, x2, x3]
  x1 = xr[r, \mmaUnd{\(\pmb{\theta}\)}, \mmaUnd{\(\pmb{\phi}\)}];
  x2 = xt[r, \mmaUnd{\(\pmb{\theta}\)}, \mmaUnd{\(\pmb{\phi}\)}];
  x3 = xp[r, \mmaUnd{\(\pmb{\theta}\)}, \mmaUnd{\(\pmb{\phi}\)}];
  MapThread[InnerProduct, \{\{x1, x2, x3\}, \{x2, x3, x1\}\}]// Simplify
\end{mmaCell}
\begin{mmaCell}{Output}
  \{0,0,0\}
\end{mmaCell}

An algebraic computation of these dot products is left as a problem for the student (\cref{problem:sphericaldot:1}).
Orthonormalization of the curvilinear basis is now possible by inspection
\index{\(\rcap\)!spherical}
\index{\(\thetacap\)!spherical}
\index{\(\phicap\)!spherical}
\begin{equation}\label{eqn:curvilinearspherical:240}
\begin{aligned}
\rcap &= \Bx_r = \Be_3 e^{j \theta} \\
\thetacap &= \inv{r} \Bx_\theta = \Be_1 e^{i\phi} e^{j \theta} \\
\phicap &= \inv{r \sin\theta} \Bx_\phi = \Be_2 e^{i \phi},
\end{aligned}
\end{equation}
so
\index{\(\Bx^r\)!spherical}
\index{\(\Bx^\theta\)!spherical}
\index{\(\Bx^\phi\)!spherical}
\begin{equation}\label{eqn:curvilinearspherical:260}
\begin{aligned}
\Bx^r &= \rcap = \Be_3 e^{j \theta} \\
\Bx^\theta &= \inv{r} \thetacap = \inv{r} \Be_1 e^{i\phi} e^{j \theta} \\
\Bx^\phi &= \inv{r \sin\theta} \phicap = \inv{r \sin\theta} \Be_2 e^{i \phi}.
\end{aligned}
\end{equation}

\index{gradient!spherical}
In particular, this shows that the spherical representation of the gradient is
\begin{equation}\label{eqn:curvilinearspherical:280}
\begin{aligned}
\spacegrad
&=
\Bx^r \PD{r}{}
+ \Bx^\theta \PD{\theta}{}
+ \Bx^\phi \PD{\phi}{} \\
&=
\rcap \PD{r}{}
+\inv{r} \thetacap \PD{\theta}{}
+\inv{r \sin\theta} \phicap \PD{\phi}{}.
\end{aligned}
\end{equation}

The spherical (oriented) volume element can also be computed in a compact fashion
\begin{dmath}\label{eqn:curvilinearspherical:300}
\frac{d^3 \Bx}{dr d\theta d\phi}
=
\Bx_r \wedge \Bx_\theta \wedge \Bx_\phi
=
\gpgradethree{
\Bx_r \Bx_\theta \Bx_\phi
}
=
\gpgradethree{
\Be_3 e^{j \theta}
r \Be_1 e^{i\phi} e^{j \theta}
r \sin\theta \Be_2 e^{i \phi}
}
=
r^2 \sin\theta
\gpgradethree{
\Be_3 e^{j \theta}
\Be_1 e^{i\phi} e^{j \theta}
\Be_2 e^{i \phi}
}
=
r^2 \sin\theta\, \Be_{123}
.
\end{dmath}

The scalar factor is in fact the Jacobian with respect to the spherical parameterization
\begin{dmath}\label{eqn:curvilinearspherical:320}
\frac{dV}{
dr d\theta d\phi}
=
\frac{\partial( x_1, x_2, x_3)}{\partial(r, \theta, \phi)}
=
\begin{vmatrix}
\sin\theta \cos\phi & \sin\theta \sin\phi & \cos\theta \\
r \cos\theta \cos\phi & r \cos\theta \sin\phi & -r \sin\theta \\
-r \sin\theta \sin\phi & r \sin\theta \cos\phi & 0 \\
\end{vmatrix}
=
r^2 \sin\theta.
\end{dmath}

The final reduction of \cref{eqn:curvilinearspherical:300}, and the expansion of the Jacobian
\cref{eqn:curvilinearspherical:320}, are both easily verified with software

\begin{mmaCell}[moredefined={OuterProduct, xr, xt, xp, e1, e2, e3, jacobian, e}]{Input}
  OuterProduct[ xr[r, \mmaUnd{\(\pmb{\theta}\)}, \mmaUnd{\(\pmb{\phi}\)}],
  xt[r, \mmaUnd{\(\pmb{\theta}\)}, \mmaUnd{\(\pmb{\phi}\)}],
  xp[r, \mmaUnd{\(\pmb{\theta}\)}, \mmaUnd{\(\pmb{\phi}\)}]]

  \{e1,e2,e3\} = IdentityMatrix[3];
  jacobian = \{xr[r, \mmaUnd{\(\pmb{\theta}\)}, \mmaUnd{\(\pmb{\phi}\)}],
  xt[r, \mmaUnd{\(\pmb{\theta}\)}, \mmaUnd{\(\pmb{\phi}\)}],
  xp[r, \mmaUnd{\(\pmb{\theta}\)}, \mmaUnd{\(\pmb{\phi}\)}]\} /. \{e[1] \(\pmb{\to}\) e1, e[2] \(\pmb{\to}\) e2, e[3]\(\pmb{\to}\)e3\};
  Det[ jacobian ] // Simplify
\end{mmaCell}
\begin{mmaCell}{Output}
  \mmaSup{r}{2} e[1,2,3] Sin[\(\theta\)]
\end{mmaCell}
\begin{mmaCell}{Output}
  \mmaSup{r}{2} Sin[\(\theta\)]
\end{mmaCell}

Performing these calculations manually are left as problems
for the student (\cref{problem:curvilinearspherical:1}, \cref{problem:volumeselection:1}).
%}
