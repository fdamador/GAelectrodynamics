%
% Copyright © 2018 Peeter Joot.  All Rights Reserved.
% Licenced as described in the file LICENSE under the root directory of this GIT repository.
%
%{
How are the reciprocal frame vectors computed?  While these vectors have a natural GA representation, this is not intrinsically a GA problem, and can be solved with standard linear algebra, using a matrix inversion.
For example, given a 2D basis \( \setlr{ \Bx_1, \Bx_2 } \), the reciprocal basis can be assumed to have a coordinate representation in the original basis
\begin{equation}\label{eqn:reciprocal_R2:100}
\begin{aligned}
\Bx^1 &= a \Bx_1 + b \Bx_2 \\
\Bx^2 &= c \Bx_1 + d \Bx_2.
\end{aligned}
\end{equation}

Imposing the constraints of \cref{dfn:reciprocal:frame} leads to a pair of 2x2 linear systems that are easily solved to find
\begin{equation}\label{eqn:reciprocal_R2:120}
\begin{aligned}
\Bx^1 &= \inv{ (\Bx_1)^2 (\Bx_2)^2 - \lr{ \Bx_1 \cdot \Bx_2}^2 } \lr{ (\Bx_2)^2 \Bx_1 - \lr{ \Bx_1 \cdot \Bx_2 } \Bx_2 } \\
\Bx^2 &= \inv{ (\Bx_1)^2 (\Bx_2)^2 - \lr{ \Bx_1 \cdot \Bx_2}^2 } \lr{ (\Bx_1)^2 \Bx_2 - \lr{ \Bx_1 \cdot \Bx_2 } \Bx_1 }.
\end{aligned}
\end{equation}

The reader may notice that for \R{3} the denominator is related to the norm of the cross product \( \Bx_1 \cross \Bx_2 \).
More generally, this can be expressed as the square of the bivector \( \Bx_1 \wedge \Bx_2 \)
\begin{equation}\label{eqn:reciprocal_R2:140}
\begin{aligned}
-\lr{\Bx_1 \wedge \Bx_2 }^2
&= -\lr{\Bx_1 \wedge \Bx_2 } \cdot \lr{\Bx_1 \wedge \Bx_2 } \\
&= -\lr{ \lr{\Bx_1 \wedge \Bx_2 } \cdot \Bx_1 } \cdot \Bx_2 \\
&= (\Bx_1)^2 (\Bx_2)^2 - \lr{\Bx_1 \cdot \Bx_2}^2.
\end{aligned}
\end{equation}

Additionally, the numerators are each dot products of \( \Bx_1, \Bx_2 \) with that same bivector
\begin{equation}\label{eqn:reciprocal_R2:160}
\begin{aligned}
\Bx^1 &= \frac{\Bx_2 \cdot \lr{ \Bx_1 \wedge \Bx_2 } }{ \lr{ \Bx_1 \wedge \Bx_2}^2 } \\
\Bx^2 &= \frac{\Bx_1 \cdot \lr{ \Bx_2 \wedge \Bx_1 } }{ \lr{ \Bx_1 \wedge \Bx_2}^2 },
\end{aligned}
\end{equation}
or
\boxedEquation{eqn:reciprocal_R2:180}{
\begin{aligned}
\Bx^1 &= \Bx_2 \cdot \inv{ \Bx_1 \wedge \Bx_2 } \\
\Bx^2 &= \Bx_1 \cdot \inv{ \Bx_2 \wedge \Bx_1 }.
\end{aligned}
}

Recall that dotting with the unit bivector of a plane (or its inverse) rotates a vector in that plane by \( \pi/2 \).
In a plane subspace, such a rotation is exactly the transformation to ensure that \( \Bx_1 \cdot \Bx^2 = \Bx_2 \cdot \Bx^1 = 0 \).
This shows that the reciprocal frame for the basis of a two dimensional subspace is found by a duality transformation of each of the curvilinear coordinates, plus a subsequent scaling operation.
As \( \Bx_1 \wedge \Bx_2 \), the pseudoscalar for the subspace spanned by \( \setlr{ \Bx_1, \Bx_2 } \), is not generally a unit bivector, the dot product with its inverse also has a scaling effect.

%\makeexample{Numerical example}{example:reciprocal_R2:320}{
\paragraph{Numerical example:}
%\iftoggle{kindle-version}{It is left to the reader to calculate the reciprocal frame depicted in \cref{fig:obliqueReciprocal:obliqueReciprocalFig2}.
%One should find that the defining property \( \Bx_i \cdot \Bx^j = {\delta_i}^j \) of the reciprocal frame vectors is satisfied.
%}
%{
Here is a Mathematica calculation of the reciprocal frame depicted in \cref{fig:obliqueReciprocal:obliqueReciprocalFig2}

\begin{mmaCell}[moredefined={x1, x2, inverse, e, x12, OuterProduct, GeometricProduct, x12inverse, s1, InnerProduct, s2, dots},morepattern={a_, a, b_, b}]{Input}
  ClearAll[x1, x2, inverse]
  x1 = e[1] + e[2]; x2 = e[1] + 2 e[2];
  x12 = OuterProduct[x1, x2];
  inverse[a_] := a / GeometricProduct[a, a] ;
  x12inverse = inverse[x12];
  s1 = InnerProduct[x2, x12inverse];
  s2 = InnerProduct[x1, -x12inverse];
  s1
  s2
  dots[a_,b_] := \{a , "\(\pmb{\cdot}\)", b, " = ",
                   InnerProduct[a // ReleaseHold, b // ReleaseHold]\};
  MapThread[dots, \{\{x1 // HoldForm, x2 // HoldForm,
                      x1 // HoldForm, x2 // HoldForm\},
                  \{s1 // HoldForm, s1 // HoldForm,
                    s2 // HoldForm, s2 // HoldForm\}\}] // Grid
\end{mmaCell}
\begin{mmaCell}{Output}
  2 e[1] - e[2]
\end{mmaCell}
\begin{mmaCell}{Output}
  -e[1] + e[2]
\end{mmaCell}
\begin{mmaCell}{Output}
  x1	\(\cdot\)	s1	 = 	1
  x2	\(\cdot\)	s1	 = 	0
  x1	\(\cdot\)	s2	 = 	0
  x2	\(\cdot\)	s2	 = 	1
\end{mmaCell}

This shows the reciprocal vector calculations using \cref{eqn:reciprocal_R2:180} and that the
%This gives \( \Bx^1 = 2 \Be_1 - \Be_2, \Bx^2 = \Be_2 - \Be_1 \), which
%satisfies the
defining property
\( \Bx_i \cdot \Bx^j = {\delta_i}^j \)
of the reciprocal frame vectors
is satisfied.
%}
%} % example

\paragraph{Example: \R{2}:}
Given a pair of arbitrary oriented vectors in \R{2}, \( \Bx_1 = a_1 \Be_1 + a_2 \Be_2, \Bx_2 = b_1 \Be_1 + b_2 \Be_2 \), the pseudoscalar associated with the basis \( \setlr{ \Bx_1, \Bx_2} \) is
\begin{equation}\label{eqn:reciprocal_R2:260}
\begin{aligned}
\Bx_1 \wedge \Bx_2
&= \lr{ a_1 \Be_1 + a_2 \Be_2 } \wedge \lr{ b_1 \Be_1 + b_2 \Be_2 } \\
&= \lr{ a_1 b_2 - a_2 b_1 } \Be_{12}.
\end{aligned}
\end{equation}

The inverse of this pseudoscalar is
\begin{equation}\label{eqn:reciprocal_R2:280}
\inv{\Bx_1 \wedge \Bx_2 }
=
\inv{ a_1 b_2 - a_2 b_1 } \Be_{21}.
\end{equation}

So for this fixed oblique \R{2} basis, the reciprocal frame is just
\begin{equation}\label{eqn:reciprocal_R2:300}
\begin{aligned}
\Bx^1 &= \Bx_2 \frac{\Be_{21}}{ a_1 b_2 - a_2 b_1 } \\
\Bx^2 &= \Bx_1 \frac{\Be_{12}}{ a_1 b_2 - a_2 b_1 }.
\end{aligned}
\end{equation}

The vector \( \Bx^1 \) is obtained by rotating \( \Bx_2 \) by \( -\pi/2 \), and rescaling it by the area of the parallelogram spanned by \( \Bx_1, \Bx_2 \).
The vector \( \Bx^2 \) is obtained with the same scaling plus a rotation of \( \Bx_1 \) by \( \pi/2 \).

%}
