%
% Copyright � 2018 Peeter Joot.  All Rights Reserved.
% Licenced as described in the file LICENSE under the root directory of this GIT repository.
%
%{
%\input{../latex/blogpost.tex}
%\renewcommand{\basename}{amperes}
%%\renewcommand{\dirname}{notes/phy1520/}
%\renewcommand{\dirname}{notes/ece1228-electromagnetic-theory/}
%%\newcommand{\dateintitle}{}
%%\newcommand{\keywords}{}
%
%\input{../latex/peeter_prologue_print2.tex}
%
%\usepackage{peeters_layout_exercise}
%\usepackage{peeters_braket}
%\usepackage{peeters_figures}
%\usepackage{siunitx}
%%\usepackage{mhchem} % \ce{}
%%\usepackage{macros_bm} % \bcM
%%\usepackage{macros_qed} % \qedmarker
%\usepackage{txfonts} % \ointclockwise
%
%\beginArtNoToc
%
%\generatetitle{Generalizing Ampere's law using geometric algebra.}
%%\chapter{Field flux.}
\label{chap:amperes}

%Ampere's law is a relationship for static configurations between the line integral of the magnetic field to the enclosed current, the flux of the current density through a surface
In this section we will present the generalization of Ampere's law to line integrals of the total electromagnetic field strength.

\input{Theorem_line_integral_of_field.tex}
The last of the scalar equations in
\cref{thm:amperes:280}
is Ampere's law
\begin{equation}\label{eqn:amperes:20}
\ointctrclockwise_{\partial A} d\Bx \cdot \BH = \int_A \ncap \cdot \BJ = I_{\textrm{enc}},
\end{equation}
and the first is the dual of Ampere's law for (fictitious) magnetic current density\footnote{Even without the fictitious magnetic sources, neither the name nor applications of the two cross product line integrals with the normal derivatives are familiar to the author.}.
In \cref{eqn:amperes:20} the flux of the electric current density equals the enclosed current flowing through an open surface.  This enclosed current equals the line integral of the magnetic field around the boundary of that surface.

\begin{proof}
To prove
\cref{thm:amperes:280}
we compute the surface integral of the current \( J = \spacegrad F \)
\begin{dmath}\label{eqn:amperes:240}
\int_A d^2 \Bx\, J
=
\int_A d^2 \Bx \spacegrad F.
\end{dmath}
As we are working in \R{3} not \R{2}, the gradient may not be replaced by the vector derivative in \cref{eqn:amperes:240}.  Instead we
must split the gradient into its vector derivative component, the projection of the gradient onto the tangent plane of the integration surface, and its normal component
\begin{equation}\label{eqn:amperes:160}
\spacegrad = \boldpartial + \ncap (\ncap \cdot \spacegrad).
\end{equation}
The surface integral form \cref{eqn:surfaceintegral:300}
of the fundamental theorem of geometric calculus may be
applied to the vector derivative portion of the field integral
\begin{equation}\label{eqn:amperes:100}
\int_A d^2 \Bx \spacegrad F
=
\int_A d^2 \Bx\, \boldpartial F
+
\int_A d^2 \Bx\, \ncap \lr{ \ncap \cdot \spacegrad } F,
\end{equation}
so
\begin{equation}\label{eqn:amperes:120}
\begin{aligned}
\ointclockwise_{\partial A} d\Bx\, F
&= \int_A d^2 \Bx \lr{ J - \ncap \lr{ \ncap \cdot \spacegrad } F } \\
&= \int_A dA \lr{ I \ncap J - \lr{ \ncap \cdot \spacegrad } I F } \\
&= \int_A dA \lr{ I \ncap J - I \PD{n}{F} },
\end{aligned}
\end{equation}
where the surface area bivector has been written in its dual form \( d^2 \Bx = I \ncap dA \) in terms of a scalar area element, and the directional derivative has been written in scalar form with respect to a parameter \( n \) that represents the length along the normal direction.  This proves the first part of
\cref{thm:amperes:280}.

Observe that the \( d\Bx\, F \) product has all possible grades
\begin{equation}\label{eqn:amperes:180}
\begin{aligned}
d\Bx\, F
&= d\Bx \lr{ \BE + I \eta \BH } \\
&=
d\Bx \cdot \BE + I \eta d\Bx \cdot \BH
+
d\Bx \wedge \BE + I \eta d\Bx \wedge \BH \\
&=
d\Bx \cdot \BE
- \eta (d\Bx \cross \BH)
+ I (d\Bx \cross \BE )
+ I \eta (d\Bx \cdot \BH),
\end{aligned}
\end{equation}
as does the \( I \ncap J \) product (in general)
\begin{equation}\label{eqn:amperes:200}
\begin{aligned}
I \ncap J
&=
I \ncap
\lr{
   \frac{\rho}{\epsilon} - \eta \BJ + I \lr{ c \rho_\txtm - \BM }
} \\
&=
\ncap I \frac{\rho}{\epsilon} - \eta \ncap I \BJ - \ncap c \rho_\txtm + \ncap \BM \\
&=
% 0
\ncap \cdot \BM
% 1
+ \eta (\ncap \cross \BJ)
- \ncap c \rho_\txtm
% 2
+ I (\ncap \cross \BM)
+ \ncap I \frac{\rho}{\epsilon}
% 3
- \eta I (\ncap \cdot \BJ).
\end{aligned}
\end{equation}
On the other hand \( I F = I \BE - \eta \BH \) has only grades 1,2, like \( F \) itself.  This allows the line integrals to be split by grade selection into components with and without a normal derivative
\begin{equation}\label{eqn:amperes:140}
\begin{aligned}
\ointclockwise_{\partial A} \gpgrade{d\Bx\, F}{0,3}
&=
\int_A dA\, \gpgrade{ I \ncap J }{0,3} \\
\ointclockwise_{\partial A} \gpgrade{d\Bx\, F}{1,2}
&=
\int_A dA \lr{ \gpgrade{ I \ncap J }{1,2} - \lr{ \ncap \cdot \spacegrad } I F }.
\end{aligned}
\end{equation}
The first of \cref{eqn:amperes:140} contains Ampere's law and its dual as one multivector equation, which can be seen more readily by explicit expansion in the constituent fields and sources using \cref{eqn:amperes:180}, \cref{eqn:amperes:200}
\begin{equation}\label{eqn:amperes:220}
\begin{aligned}
\ointclockwise_{\partial A}
\lr{
   d\Bx \cdot \BE + I \eta (d\Bx \cdot \BH)
}
&=
\int_A dA
\lr{
   \ncap \cdot \BM - \eta I (\ncap \cdot \BJ)
} \\
\ointclockwise_{\partial A}
\lr{
   - \eta (d\Bx \cross \BH)
   + I (d\Bx \cross \BE )
}
&=
\int_A dA
\Biglr{
   % 1
     \eta (\ncap \cross \BJ)
   - \ncap c \rho_\txtm \\
&\qquad
   % 2
   + I (\ncap \cross \BM)
   + \ncap I \frac{\rho}{\epsilon}
   -\PD{n}{} \lr{ I \BE - \eta \BH }
}.
\end{aligned}
\end{equation}

Further grade selection operations, and minor adjustments of the leading constants completes the proof.

It is also worth pointing out that for pure magnetostatics problems where \( J = \eta \BJ, F = I \eta \BH \), that Ampere's law can be written in a trivector form
\begin{equation}\label{eqn:amperes:260}
\ointclockwise_{\partial A} d\Bx \wedge F = I \int_A dA\, \ncap \cdot J = I \eta \int_A dA\, \ncap \cdot \BJ.
\end{equation}
This encodes the fact that the magnetic field component of the total electromagnetic field strength is most naturally expressed in
geometric algebra as a bivector.
\end{proof}
% \( I \eta \BH \), and not just the vector \( \BH \).
%}
%\EndArticle
