%
% Copyright � 2018 Peeter Joot.  All Rights Reserved.
% Licenced as described in the file LICENSE under the root directory of this GIT repository.
%
%{
\index{area element}
\index{differential form}
\makedefinition{Multivector surface integral.}{dfn:lineintegraldef:multivectorsurfaceintegral}{
Given a continuous and differentiable surface described by a vector function \( \Bx(a, b) \), parameterized by two scalars \( a, b \) with differential
\begin{equation*}
d^2 \Bx \equiv d\Bx_a \wedge d\Bx_b =
\PD{a}{\Bx} \wedge \PD{b}{\Bx}
da db = {\Bx_a \wedge \Bx_b } da db,
\end{equation*}
and multivector functions \( F, G \), the integral
\begin{equation*}
\int F d^2 \Bx G
\end{equation*}
is called a multivector surface integral.
} % definition

%Unless \( F, G \) are both scalars, such a surface integral is not generally bivector valued like the area element.
An example of a two parameter surface, and the corresponding differentials with respect to those parameters, is illustrated in
\cref{fig:twoParameterDifferential:twoParameterDifferentialFig1}.

\mathImageFigure{../figures/GAelectrodynamics/\subfigdir/twoParameterDifferentialFig1}{Two parameter manifold differentials.}{fig:twoParameterDifferential:twoParameterDifferentialFig1}{0.4}{twoParameterDifferentialFig.nb}

In \R{3} it will often be convenient to utilize a dual representation of the area element \( d^2 \Bx = I \ncap dA \), where \( dA \) is a scalar area element, and \( \ncap \) is a normal vector to the surface.  With such an area element representation we will call \( I \int dA\, F \ncap G \) a surface integral.

\paragraph{Example: Spherical surface integral.}

From \cref{eqn:curvilinearspherical:300}, we know that
\begin{equation}\label{eqn:surfaceintegraldef:140}
\Bx_r \Bx_\theta \Bx_\phi = I r^2 \sin\theta,
\end{equation}
so
\begin{equation}\label{eqn:surfaceintegraldef:160}
\begin{aligned}
\Bx_\theta \wedge \Bx_\phi
&= \Bx_\theta \Bx_\phi \\
&= \Bx_r I r^2 \sin\theta,
\end{aligned}
\end{equation}
so the (bivector-valued) area element for a spherical surface is
\begin{equation}\label{eqn:surfaceintegraldef:180}
d^2 \Bx = I \Bx_r r^2 \sin\theta d\theta d\phi.
\end{equation}

Suppose we integrate a vector valued function \( F(\theta, \phi) = \alpha \Bx^r + \beta \Bx^\theta + \gamma \Bx^\phi \), where \( \alpha, \beta, \gamma\) are constants, over the surface of a sphere of radius \( r \), then the surface integral (with the area element on the right) is
\begin{dmath}\label{eqn:surfaceintegraldef:200}
\int F d^2\Bx
=
\alpha I r^2 \int \Bx^r \Bx_r \sin\theta d\theta d\phi
+
\beta I r^2 \int \Bx^\theta \Bx_r \sin\theta d\theta d\phi
+
\gamma I r^2 \int \Bx^\phi \Bx_r \sin\theta d\theta d\phi.
\end{dmath}
This can be simplified using \( \rcap \thetacap \phicap = I \), and \cref{eqn:curvilinearspherical:260}, to find
\begin{equation}\label{eqn:surfaceintegraldef:220}
\begin{aligned}
\Bx^r \Bx_r &= 1 \\
I \Bx^\theta \Bx_r &= \inv{r} I \thetacap \rcap = \inv{r} \phicap \\
I \Bx^\phi \Bx_r &= \inv{r \sin\theta} I \phicap \rcap = -\inv{r \sin\theta} \thetacap,
\end{aligned}
\end{equation}
so
\begin{dmath}\label{eqn:surfaceintegraldef:240}
\int F d^2\Bx =
\alpha I 4 \pi r^2
+
\beta r \int \phicap \sin\theta d\theta d\phi
-
\gamma r \int \thetacap d\theta d\phi
=
\alpha I 4 \pi r^2,
\end{dmath}
where the integrands containing \( \thetacap, \phicap \) are killed by the integral over \( \phi \in [0, 2\pi] \).  If integrated over a subset of the spherical surface, where such perfect cancellation does not occur, this surface integral may have both vector and trivector components.

\paragraph{Example: Bivector function.}
Given a bivector valued function \( F(a,b) = (a + b) \Be_2 \Be_1 + 2 (a \Be_1 - b \Be_2) \Be_3 \) defined over the unit square \( a,b \in [0, 1] \), and a surface \( \Bx(a,b) = a \Be_1 + b \Be_2 \), the multivector surface integral (with the area element on the right) is
\begin{equation}\label{eqn:surfaceintegraldef:260}
\begin{aligned}
\int F d^2 \Bx
&= \int_0^1\int_0^1 (a + b) \,da db + 2 \int_0^1\int_0^1 (a \Be_1 - b \Be_2) \Be_3 \Be_1 \Be_2\, da db \\
&= 1+ I \int_0^1 \evalrange{a^2}{0}{1} \Be_1 db - I \int_0^1 \evalrange{b^2}{0}{1} \Be_2 da \\
&= 1+ I \lr{ \Be_1 - \Be_2 } \\
&= 1+ \lr{ \Be_{1} + \Be_{2} } \Be_3.
\end{aligned}
\end{equation}
In this example, the integral of a bivector valued function over a (bivector-valued) surface area element results in a multivector with a scalar and bivector grade.  In higher dimensional spaces, such an integral may also have grade-4 components.
%}
