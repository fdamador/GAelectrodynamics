%
% Copyright © 2017 Peeter Joot.  All Rights Reserved.
% Licenced as described in the file LICENSE under the root directory of this GIT repository.
%
\index{curvilinear coordinates}
Curvilinear coordinates can be defined for any subspace spanned by a parameterized vector into that space.
%Consider a continuous subspace parameterized by a two parameter vector function \( \Bx = \Bx(u_1, u_2) \) that is differentiable with respect to either parameter
As an example, consider a two parameter planar subspace parameterized by the following continuous vector function
%\begin{equation}\label{eqn:curvilinearDefined:480}
%\Bx(u_1, u_2) = u_1 \Be_1 \frac{\sqrt{3}}{2} \cosh\lr{ \Atanh(1/2) + \Be_{12} u_2 },
%\end{equation}
\begin{equation}\label{eqn:curvilinearDefined:480}
\Bx(u_1, u_2) = u_1 \lr{ \Be_1 \cos u_2 + \inv{2} \Be_2 \sin u_2 },
\end{equation}
where \( u_1 \in [0,1] \) and \( u_2 \in [0, \pi/2] \).
This parameterization spans the first quadrant of the ellipse with semi-major axis length 1, and semi-minor axis length \( 1/2 \).
%\footnote{
A parameterization of an elliptic area may or may not be of much use in electrodynamics, but it happens to provide a non-trivial, yet simple, example of a non-orthonormal parameterization.
%}
Contours for this parameterization are plotted in \cref{fig:ellipticalContours:ellipticalContoursFig1}.
The radial contours are for fixed values of \( u_2 \) and the elliptical contours fix the value of \( u_1 \), and depict a set of elliptic curves
with a semi-major/major axis ratio of \( 1/2 \).
\pmathImageFigure{../figures/GAelectrodynamics/}{ellipticalContoursFig1}{Contours for an elliptical region.}{fig:ellipticalContours:ellipticalContoursFig1}{0.3}{ellipticalContoursFigures.nb}

We define a curvilinear basis associated with each point in the region by the partials
\begin{equation}\label{eqn:curvilinearDefined:80}
\begin{aligned}
\Bx_{1} &= \PD{u_1}{\Bx} \\
\Bx_{2} &= \PD{u_2}{\Bx}.
\end{aligned}
\end{equation}

For \cref{eqn:curvilinearDefined:480} our curvilinear basis elements are
\begin{equation}\label{eqn:curvilinearDefined:520}
%\begin{aligned}
%\Bx_{1} &= \Be_1 \frac{\sqrt{3}}{2} \cosh\lr{ \Atanh(1/2) + \Be_{12} u_2 } \\
%\Bx_{2} &= u_1 \Be_2 \frac{\sqrt{3}}{2} \sinh\lr{ \Atanh(1/2) + \Be_{12} u_2 }.
%\end{aligned}
\begin{aligned}
\Bx_1 &= \Be_1 \cos u_2 + \inv{2} \Be_2 \sin u_2 \\
\Bx_2 &= u_1 \lr{ - \Be_1 \sin u_2 + \inv{2} \Be_2 \cos u_2 },
\end{aligned}
\end{equation}

We form vector valued differentials for each parameter
\begin{equation}\label{eqn:curvilinearDefined:500}
\begin{aligned}
d\Bx_{1} &= \Bx_1 du_1 \\
d\Bx_{2} &= \Bx_2 du_2.
\end{aligned}
\end{equation}

For \cref{eqn:curvilinearDefined:480},
the values of these differentials \( d\Bx_1, d\Bx_2 \) with \( du_1 = du_2 = 0.1 \) are plotted
in
\cref{fig:ellipticalContours:ellipticalContoursFig2}
for the points
\begin{equation}\label{eqn:curvilinearDefined:600}
(u_1, u_2) = (0.7, 5 \pi/20), (0.9, 3 \pi/20), (1.0, 5 \pi/20)
\end{equation}
in
(dark-thick) red, blue and purple respectively.
\pmathImageFigure{../figures/GAelectrodynamics/}{ellipticalContoursFig2}{Differentials for an elliptical parameterization.}{fig:ellipticalContours:ellipticalContoursFig2}{0.3}{ellipticalContoursFigures.nb}

In this case and in general there is no reason to presume that there is any orthonormality constraint on the basis \( \setlr{ \Bx_{1}, \Bx_{2} } \) for a given two parameter subspace.

Should we wish to calculate the reciprocal frame
for \cref{eqn:curvilinearDefined:480}, we would find
%(\cref{problem:ellipticproblem:10}) that
(\cref{problem:ellipseCurvilinear:200}) that
\begin{equation}\label{eqn:curvilinearDefined:540}
\begin{aligned}
\Bx^{1} &= \Be_1 \cos u_2 + 2 \Be_2 \sin u_2 \\
\Bx^{2} &= \inv{u_1} \lr{ - \Be_1 \sin u_2 + 2 \Be_2 \cos u_2 }
%\Bx^{1} &= \Be_1 \sqrt{3} \sinh\lr{ \Atanh(1/2) + \Be_{12} u_2 } \\
%\Bx^{2} &= \frac{\Be_2}{u_1} \sqrt{3} \cosh\lr{ \Atanh(1/2) + \Be_{12} u_2 }.
\end{aligned}
\end{equation}

These are plotted (scaled by \( da = 0.1 \) so they fit in the image nicely) in \cref{fig:ellipticalContours:ellipticalContoursFig2} using thin light arrows.

When evaluating surface integrals, we will form
oriented (bivector) area elements from the wedge product of the differentials
\begin{equation}\label{eqn:curvilinearDefined:60}
d^2 \Bx \equiv d\Bx_{1} \wedge d\Bx_{2}.
\end{equation}

This absolute value of this area element \( \sqrt{-(d^2 \Bx)^2} \) is the area of the parallelogram spanned by \( d\Bx_1, d\Bx_2 \).
In this example, all such area elements lie in the \( x-y \) plane, but that need not be the case.

Also note that we will only perform integrals for those parametrizations for which the area element \( d^2 \Bx \) is non-zero.

%If the spacing between the contours is made small enough, the boundaries of each partition will define a planar region at the point of evaluation.
%All points in the interior will be accessible by a combination of the vectors formed from the partials of \( \Bx \) at that point.
%
% Copyright � 2023 Peeter Joot.  All Rights Reserved.
% Licenced as described in the file LICENSE under the root directory of this GIT repository.
%
%{
%\input{../latex/blogpost.tex}
%\renewcommand{\basename}{ellipseCurvilinear}
%%\renewcommand{\dirname}{notes/phy1520/}
%\renewcommand{\dirname}{notes/ece1228-electromagnetic-theory/}
%%\newcommand{\dateintitle}{}
%%\newcommand{\keywords}{}
%
%\input{../latex/peeter_prologue_print2.tex}
%
%\usepackage{peeters_layout_exercise}
%\usepackage{peeters_braket}
%\usepackage{peeters_figures}
%\usepackage{siunitx}
%\usepackage{verbatim}
%%\usepackage{mhchem} % \ce{}
%%\usepackage{macros_bm} % \bcM
%%\usepackage{macros_qed} % \qedmarker
%%\usepackage{txfonts} % \ointclockwise
%
%\beginArtNoToc
%
%\generatetitle{XXX}
%%\chapter{XXX}
%%\label{chap:ellipseCurvilinear}
%
\makeproblem{Elliptic parameterization.}{problem:ellipseCurvilinear:200}{
An elliptical area can be parameterized as
\begin{equation}\label{eqn:ellipseCurvilinear:20}
   \Bx(u_1, u_2) = u_1 \lr{ \Be_1 \cos u_2 + \beta \Be_2 \sin u_2 },
\end{equation}
where \( \beta = \sqrt{1 - \epsilon^2} \), and \( \epsilon \) is the eccentricity of the ellipse.
\makesubproblem{}{problem:ellipseCurvilinear:200:a}
Compute the curvilinear vectors
\begin{equation}\label{eqn:ellipseCurvilinear:200}
\begin{aligned}
   \Bx_1 &= \PDi{u_1}{\Bx} \\
   \Bx_2 &= \PDi{u_2}{\Bx}.
\end{aligned}
\end{equation}
\makesubproblem{}{problem:ellipseCurvilinear:200:b}
Compute the reciprocal frame vectors
\begin{equation}\label{eqn:ellipseCurvilinear:220}
\begin{aligned}
\Bx^1 &= \Bx_2 \cdot \inv{ \Bx_1 \wedge \Bx_2 } \\
\Bx^2 &= -\Bx_1 \cdot \inv{ \Bx_1 \wedge \Bx_2 }.
\end{aligned}
\end{equation}
\makesubproblem{}{problem:ellipseCurvilinear:200:c}
Verify that \( \Bx_i \cdot \Bx^j = {\delta_i}^j \).
} % problem

\makeanswer{problem:ellipseCurvilinear:200}{
\makesubanswer{}{problem:ellipseCurvilinear:200:a}
The curvilinear basis associated with this parameterization can be computed by inspection
\begin{equation}\label{eqn:ellipseCurvilinear:40}
\begin{aligned}
   \Bx_1 &= \Be_1 \cos u_2 + \beta \Be_2 \sin u_2 \\
   \Bx_2 &= u_1 \lr{ -\Be_1 \sin u_2 + \beta \Be_2 \cos u_2 }.
\end{aligned}
\end{equation}
\makesubanswer{}{problem:ellipseCurvilinear:200:b}

We need to compute the area element first
\begin{equation}\label{eqn:ellipseCurvilinear:60}
\begin{aligned}
\Bx_1 \wedge \Bx_2
&= \lr{ \Be_1 \cos u_2 + \beta \Be_2 \sin u_2 } \wedge u_1 \lr{ -\Be_1 \sin u_2 + \beta \Be_2 \cos u_2 } \\
&= u_1 \gpgradetwo{ \lr{ \Be_1 \cos u_2 + \beta \Be_2 \sin u_2 } \lr{ -\Be_1 \sin u_2 + \beta \Be_2 \cos u_2 } } \\
&= u_1 \lr{
\beta \Be_{12} \cos^2 u_2 - \beta \Be_{21} \sin^2 u_2
} \\
&= u_1 \beta i,
\end{aligned}
\end{equation}
where \( i = \Be_{12} \).

The reciprocal frame vectors are given by
\begin{equation}\label{eqn:ellipseCurvilinear:80}
\begin{aligned}
\Bx^1
   &= \Bx_2 \cdot \inv{ \Bx_1 \wedge \Bx_2 } \\
   &= u_1 \lr{ -\Be_1 \sin u_2 + \beta \Be_2 \cos u_2 } \inv{ u_1 \beta i } \\
   %&= \lr{ - \inv{\beta} \Be_1 \sin u_2 + \Be_2 \cos u_2 } (-i)
   &= \inv{\beta} \Be_2 \sin u_2 + \Be_1 \cos u_2,
\end{aligned}
\end{equation}
\begin{equation}\label{eqn:ellipseCurvilinear:100}
\begin{aligned}
   \Bx^2
   &= -\Bx_1 \cdot \inv{ \Bx_1 \wedge \Bx_2 } \\
   &= - \lr{ \Be_1 \cos u_2 + \beta \Be_2 \sin u_2 } \inv{ u_1 \beta i } \\
   %&= \lr{ \Be_1 \cos u_2 + \beta \Be_2 \sin u_2 } \frac{i}{ u_1 \beta }
   &= \inv{u_1} \lr{ \inv{\beta} \Be_2 \cos u_2 - \Be_1 \sin u_2 }.
\end{aligned}
\end{equation}

\makesubanswer{}{problem:ellipseCurvilinear:200:c}
To verify that \( \Bx_i \cdot \Bx^j = {\delta_i}^j \) we can compute each of the dot products
\begin{equation}\label{eqn:ellipseCurvilinear:120}
\begin{aligned}
\Bx^1 \cdot \Bx_1
&= \gpgradezero{
\lr{ \Be_1 \cos u_2 + \beta \Be_2 \sin u_2 } \lr{ \inv{\beta} \Be_2 \sin u_2 + \Be_1 \cos u_2 }
} \\
&=
\cos^2 u_2 + \sin^2 u_2 \\
&= 1,
\end{aligned}
\end{equation}
\begin{equation}\label{eqn:ellipseCurvilinear:140}
\begin{aligned}
\Bx^2 \cdot \Bx_2
&= \gpgradezero{
u_1 \lr{ -\Be_1 \sin u_2 + \beta \Be_2 \cos u_2 } \inv{u_1} \lr{ \inv{\beta} \Be_2 \cos u_2 - \Be_1 \sin u_2 }
} \\
&=
% \lr{ -\Be_1 \sin u_2 + \beta \Be_2 \cos u_2 } \lr{ \inv{\beta} \Be_2 \cos u_2 - \Be_1 \sin u_2 }
\sin^2 u_2 + \cos^2 u_2 \\
&= 1,
\end{aligned}
\end{equation}
\begin{equation}\label{eqn:ellipseCurvilinear:160}
\begin{aligned}
\Bx^1 \cdot \Bx_2
&= \gpgradezero{
\lr{ \inv{\beta} \Be_2 \sin u_2 + \Be_1 \cos u_2 } u_1 \lr{ -\Be_1 \sin u_2 + \beta \Be_2 \cos u_2 }
} \\
&=
u_1 \sin u_2 \cos u_2 - u_1 \cos u_2 \sin u_2 \\
&= 0.
\end{aligned}
\end{equation}
\begin{equation}\label{eqn:ellipseCurvilinear:180}
\begin{aligned}
\Bx^2 \cdot \Bx_1
&= \gpgradezero{
\inv{u_1} \lr{ \inv{\beta} \Be_2 \cos u_2 - \Be_1 \sin u_2 } \lr{ \Be_1 \cos u_2 + \beta \Be_2 \sin u_2 }
} \\
&=
\inv{u_1} \lr{
\cos u_2 \sin u_2 - \sin u_2 \cos u_2
}
\\
&= 0.
\end{aligned}
\end{equation}
} % answer
%}
%\EndNoBibArticle

%
% Copyright © 2023 Peeter Joot.  All Rights Reserved.
% Licenced as described in the file LICENSE under the root directory of this GIT repository.
%
%{
%\input{../latex/blogpost.tex}
%\renewcommand{\basename}{ellipticproblem}
%\renewcommand{\dirname}{notes/phy1520/}
%\renewcommand{\dirname}{notes/ece1228-electromagnetic-theory/}
%\newcommand{\dateintitle}{}
%\newcommand{\keywords}{}
%
%\input{../latex/peeter_prologue_print2.tex}
%
%\usepackage{peeters_layout_exercise}
%\usepackage{peeters_braket}
%\usepackage{peeters_figures}
%\usepackage{amsthm}
%\usepackage{siunitx}
%\usepackage{verbatim}
%\usepackage{mhchem} % \ce{}
%\usepackage{macros_bm} % \bcM
%\usepackage{macros_qed} % \qedmarker
%\usepackage{txfonts} % \ointclockwise
%
%\beginArtNoToc
%
%\generatetitle{XXX}
%\chapter{XXX}
%\label{chap:ellipticproblem}
%
\makeproblem{Hyperbolic identities.}{lemma:ellipticproblem:280}{
Show that
\begin{equation}\label{eqn:ellipticproblem:300}
2 \cosh\lr{ \mu - i \theta } \sinh\lr{ \mu + i \theta } = \sinh(2 \mu) + i \sin(2 \theta).
\end{equation}
\begin{equation}\label{eqn:ellipticproblem:420}
2 \cosh\lr{ \mu } \sinh\lr{ \mu } = \sinh(2 \mu).
\end{equation}
\begin{equation}\label{eqn:ellipticproblem:440}
\cosh\lr{ \mu + i \theta } = \cosh\mu \cos\theta + i \sinh\mu \sin\theta.
\end{equation}
} % problem
\makeanswer{lemma:ellipticproblem:280}{
\begin{equation}\label{eqn:ellipticproblem:320}
\begin{aligned}
\cosh\lr{ \mu - i \theta } \sinh\lr{ \mu + i \theta }
&=
\frac{1}{4}
   \lr{ e^{\mu - i\theta} - e^{-\mu + i \theta } } \lr{ e^{\mu + i\theta} - e^{-\mu - i \theta } }
\\
&=
\frac{1}{4} \lr{
   e^{2 \mu} - e^{-2\mu} + e^{2 i \theta} - e^{-2 i \theta}
} \\
&=
\inv{2} \lr{  \sinh(2 \mu) + i \sin(2 \theta) }.
\end{aligned}
\end{equation}
The second identity follows from the first, setting \( \theta = 0 \).
Finally, for the third expanding the \( \cosh \) in terms of exponentials, we find
\begin{equation}\label{eqn:ellipticproblem:50}
\begin{aligned}
\cosh\lr{ \mu + i \theta }
&=
\frac{1}{2} \lr{ e^{\mu + i \theta} + e^{-\mu - i\theta} } \\
&=
\frac{e^\mu}{2} \lr{ \cos\theta + i \sin\theta }
+
\frac{e^{-\mu}}{2} \lr{ \cos\theta - i \sin\theta } \\
&=
\frac{ e^\mu + e^{-\mu} }{2} \cos\theta
+ i \frac{ e^\mu - e^{-\mu} }{2} \sin\theta \\
&=
\cosh\mu \cos\theta + i \sinh\mu \sin\theta.
\end{aligned}
\end{equation}
}

\makeproblem{Elliptic curvilinear and reciprocal basis.}{problem:ellipticproblem:10}{
\makesubproblem{}{problem:ellipticproblem:10:a}
Show that an ellipse can be parameterized by
\begin{equation}\label{eqn:ellipticproblem:20}
   \Bx = u_1 \Be_1 \cosh\lr{ \mu + i u_2 },
\end{equation}
where \( i = \Be_{12} \), and find the values of the semi-major and semi-minor axes.
\makesubproblem{}{problem:ellipticproblem:10:b}
Determine how \( \mu \) and the eccentricity \( \epsilon = \sqrt{1 - b^2/a^2} \) are related.
\makesubproblem{}{problem:ellipticproblem:10:c}
Compute the curvilinear and reciprocal frame vectors for the parameterization \( \Bx(u_1, u_2) \) above.
%, and use this to verify
%\cref{eqn:curvilinearDefined:520} and \cref{eqn:curvilinearDefined:540} respectively.
\makesubproblem{}{problem:ellipticproblem:10:d}
Check that \( \Bx^i \cdot \Bx_j = {\delta^i}_j \).
} % problem
\makeanswer{problem:ellipticproblem:10}{
\makesubanswer{}{problem:ellipticproblem:10:a}
Using the multiple angle \( \cosh \) expansion, we find
\begin{equation}\label{eqn:ellipticproblem:40}
\begin{aligned}
\Be_1 \cosh\lr{ \mu + i u_2 }
&=
\Be_1 \lr{
\cosh\mu \cos u_2 + i \sinh\mu \sin u_2
} \\
&=
\Be_1 \cosh\mu \cos u_2 + \Be_2 \sinh\mu \sin u_2,
\end{aligned}
\end{equation}
so
\begin{equation}\label{eqn:ellipticproblem:60}
\Bx = u_1 \Be_1 \cosh\lr{ \mu + i u_2 } = \Be_1 a \cos u_2 + \Be_2 b \sin u_2,
\end{equation}
where
\begin{equation}\label{eqn:ellipticproblem:80}
\begin{aligned}
   a &= u_1 \cosh\mu \\
   b &= u_1 \sinh\mu,
\end{aligned}
\end{equation}
are the semi-major and semi-minor axis values.
\makesubanswer{}{problem:ellipticproblem:10:b}
The eccentricity (squared) is
\begin{equation}\label{eqn:ellipticproblem:100}
\begin{aligned}
   \epsilon^2
   &= 1 - \tanh^2\mu \\
   &= \frac{\cosh^2\mu - \sinh^2\mu}{\cosh^2\mu} \\
   &= \inv{\cosh^2\mu},
\end{aligned}
\end{equation}
so the eccentricity is
\begin{equation}\label{eqn:ellipticproblem:120}
   \epsilon = \inv{\cosh\mu}.
\end{equation}
\makesubanswer{}{problem:ellipticproblem:10:c}
Our curvilinear basis vectors are
\begin{equation}\label{eqn:ellipticproblem:140}
\begin{aligned}
\Bx_1 &= \Be_1 \cosh\lr{ \mu + i u_2 } \\
\Bx_2 &= \Be_2 u_1 \sinh\lr{ \mu + i u_2 } \\
\end{aligned}
\end{equation}

To compute the reciprocals we need the area element
\begin{equation}\label{eqn:ellipticproblem:160}
\begin{aligned}
\Bx_1 \wedge \Bx_2
&=
\gpgradetwo{
   \Be_1 \cosh\lr{ \mu + i u_2 } \Be_2 u_1 \sinh\lr{ \mu + i u_2 }
} \\
&=
u_1 \gpgradetwo{
   i \cosh\lr{ \mu - i u_2 } \sinh\lr{ \mu + i u_2 }
} \\
&=
\frac{u_1}{2} \gpgradetwo{
   i \lr{ \sinh(2 \mu) + i \sin(2 u_2) }
} \\
&=
u_1 i \cosh \mu \sinh \mu.
\end{aligned}
\end{equation}

Our recipocal basis vectors are
\begin{equation}\label{eqn:ellipticproblem:240}
\begin{aligned}
   \Bx^1
   &= \Bx_2 \inv{ \Bx_1 \wedge \Bx_2 } \\
   &= \Be_2 u_1 \sinh\lr{ \mu + i u_2 } \inv{u_1 i \cosh \mu \sinh \mu} \\
   &= \Be_1 \frac{\sinh\lr{ \mu + i u_2 }}{\cosh \mu \sinh \mu},
\end{aligned}
\end{equation}
and
\begin{equation}\label{eqn:ellipticproblem:180}
\begin{aligned}
\Bx^2
   &= -\Bx_1 \inv{ \Bx_1 \wedge \Bx_2 } \\
   &= -\lr{ \Be_1 \cosh\lr{ \mu + i u_2 } } \inv{ u_1 i \cosh \mu \sinh \mu} \\
   &= \frac{ \Be_2 \cosh\lr{ \mu + i u_2 } }{ u_1 \cosh \mu \sinh \mu}.
\end{aligned}
\end{equation}
\makesubanswer{}{problem:ellipticproblem:10:d}
\begin{equation}\label{eqn:ellipticproblem:260}
\begin{aligned}
\Bx_1 \cdot \Bx^1
   &= \gpgradezero{
\Be_1 \cosh\lr{ \mu + i u_2 } \Be_1 \frac{\sinh\lr{ \mu + i u_2 }}{\cosh \mu \sinh \mu}
} \\
   &=
\inv{\cosh \mu \sinh \mu}
\gpgradezero{
\Be_1^2 \cosh\lr{ \mu - i u_2 } \lr{\sinh\lr{ \mu + i u_2 }}
} \\
   &=
\inv{\cosh \mu \sinh \mu}
\gpgradezero{
   \inv{2} \lr{ \sinh(2 \mu) + i \sin(2 u_2) }
} \\
&=
\frac{\sinh(2 \mu) }{2 \cosh \mu \sinh \mu} \\
&= 1.
\end{aligned}
\end{equation}
\begin{equation}\label{eqn:ellipticproblem:360}
\begin{aligned}
\Bx_2 \cdot \Bx^2
&=
\gpgradezero{
\Be_2 u_1 \sinh\lr{ \mu + i u_2 }
\frac{ \Be_2 \cosh\lr{ \mu + i u_2 } }{ u_1 \cosh \mu \sinh \mu}
} \\
&=
\inv{ \cosh \mu \sinh \mu}
\gpgradezero{
\sinh\lr{ \mu + i u_2 }
\Be_2^2 \cosh\lr{ \mu - i u_2 }
} \\
&= 1.
\end{aligned}
\end{equation}
\begin{equation}\label{eqn:ellipticproblem:380}
\begin{aligned}
\Bx_1 \cdot \Bx^2
&=
\gpgradezero{
\Be_1 \cosh\lr{ \mu + i u_2 }
\frac{ \Be_2 \cosh\lr{ \mu + i u_2 } }{ u_1 \cosh \mu \sinh \mu}
} \\
&=
\inv{ u_1 \cosh \mu \sinh \mu}
\gpgradezero{
   \Be_{12} \cosh\lr{ \mu - i u_2 } \cosh\lr{ \mu + i u_2 }
} \\
&=
\frac{ \Abs{ \cosh\lr{ \mu + i u_2 } }^2 }
{ u_1 \cosh \mu \sinh \mu}
\gpgradezero{
   \Be_{12}
} \\
&= 0.
\end{aligned}
\end{equation}
\begin{equation}\label{eqn:ellipticproblem:400}
\begin{aligned}
\Bx^1 \cdot \Bx_2
&=
\gpgradezero{
\Be_1 \frac{\sinh\lr{ \mu + i u_2 }}{\cosh \mu \sinh \mu}
\Be_2 u_1 \sinh\lr{ \mu + i u_2 }
} \\
&=
\frac{ u_1 \Abs{\sinh\lr{ \mu + i u_2 }}^2 }{\cosh \mu \sinh \mu} \gpgradezero{ \Be_{12} } \\
&= 0.
\end{aligned}
\end{equation}
} % answer
%}
%\EndNoBibArticle

\todo{Don't introduce the idea of tangent space until a 3D example.
Remove the \R{3} reference above, and keep this first example planar.}

At the point of evaluation, the span of these differentials is called the tangent space.
In this particular case the tangent space at all points in the region is the entire x-y plane.
These partials locally span the tangent space at a given point on the surface.
\subsubsection{Curved two parameter surfaces.}
Continuing to illustrate by example, let's now consider a non-planar two parameter surface
\begin{equation}\label{eqn:curvilinearDefined:560}
\Bx(u_1, u_2) =
(u_1-u_2)^2
\Be_1
+ (1-(u_2)^2 ) \Be_2
+ u_1 u_2 \Be_3.
\end{equation}

The curvilinear basis elements, and the area element, are
\begin{equation}\label{eqn:curvilinearDefined:580}
\begin{aligned}
\Bx_1 &= 2 (u_1 - u_2) \Be_1 + u_2 \Be_3 \\
\Bx_2 &= 2 (u_2 - u_1) \Be_1 - 2 u_2 \Be_2 + u_1 \Be_3 \\
\Bx_1 \wedge \Bx_2 &=
-4 u_2
\left(u_1-u_2\right)
\Be_{12}
+2 u_2^2 \Be_{23}
+ 2
\lr{u_1^2 - u_2^2}
\Be_{13}.
\end{aligned}
\end{equation}
Two examples of these vectors and the associated area element (rescaled to fit) is plotted in
\cref{fig:2dmanifold:2dmanifoldFig1}.
This plane is called the tangent space at the point in question, and has been evaluated at \( (u_1, u_2) = (0.5,0.5), (0.35, 0.75) \).
\pmathImageFigure{../figures/GAelectrodynamics/}{2dmanifoldFig1}{Two parameter manifold.}{fig:2dmanifold:2dmanifoldFig1}{0.3}{2dmanifoldPlot.nb}
%\mathImageFigure{../figures/GAelectrodynamics/twoParameterDifferentialFieldFig1}{Curvilinear coordinates along a two parameter surface.}{fig:twoParameterDifferentialField:twoParameterDifferentialFieldFig1}{0.3}{twoParameterDifferentialFieldFig1.nb}
The results of \cref{eqn:curvilinearDefined:580} can be calculated easily by hand for this particular parameterization, but also submit to symbolic calculation software.
%\iftoggle{kindle-version}{}{
Here's a complete example using CliffordBasic

\begin{mmaCell}[moredefined={$SetSignature}]{Input}
  << CliffordBasic`;
  $SetSignature=\{3,0\};
\end{mmaCell}

\begin{mmaCell}[moredefined={e, OuterProduct, GFactor},morepattern={u_, v_}]{Input}
  ClearAll[xp, x, x1, x2]
  (* Use dummy parameter values for the derivatives,
     and then switch them to function parameter values. *)
  xp :=  (a - b)^2  e[1] + (1 - b^2) e[2] + b a e[3];
  x[u_, v_] := xp /. \{a \(\pmb{\to}\) \mmaPat{u}, b\(\pmb{\to}\)\mmaPat{v}\};
  x1[u_, v_] := D[xp, a] /. \{a \(\pmb{\to}\) \mmaPat{u}, b\(\pmb{\to}\)\mmaPat{v}\};
  x2[u_, v_] := D[xp, b] /. \{a \(\pmb{\to}\) \mmaPat{u}, b\(\pmb{\to}\)\mmaPat{v}\};

  x1[u,v]
  x2[u,v]
  OuterProduct[x1[u, v], x2[u, v]] // GFactor
\end{mmaCell}
%  (* CliffordBasic display of wedge doesn't currently group by the wedge basis bivectors. *)

\begin{mmaCell}{Output}
  2 (u-v) e[1] + v e[3]
\end{mmaCell}

\begin{mmaCell}{Output}
  -2 (u - v) e[1] - 2 v e[2] + u e[3]
\end{mmaCell}

\begin{mmaCell}{Output}
  (-4 u v + 4 \mmaSup{v}{2}) e[1,2] + (2 \mmaSup{u}{2} - 2 \mmaSup{v}{2}) e[1,3] + 2 \mmaSup{v}{2} e[2,3]
\end{mmaCell}
%}
