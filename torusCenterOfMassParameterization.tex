%
% Copyright � 2012 Peeter Joot.  All Rights Reserved.
% Licenced as described in the file LICENSE under the root directory of this GIT repository.
%
\index{toroid}
\index{differential form}
%\imageFigure{../figures/gabook/toriodalSegment}{Toroidal parameterization.}{fig:toriodalSegment}{0.5}
\mathImageFigure{../figures/GAelectrodynamics/toroidFig1}{Toroidal parameterization.}{fig:toriodalSegment}{0.3}{gaToroid.nb}
Here is a 3D example of a parameterization with a non-orthogonal curvilinear basis, that of a
toroidal subspace specified by two angles and a radial distance to the center of the toroid, as illustrated in \cref{fig:toriodalSegment}.

The position vector on the surface of a toroid of radius \( \rho \) within the torus can be stated directly
\begin{subequations}
\begin{align}\label{eqn:torusCenterOfMassParameterization:1}
\Bx(\rho, \theta, \phi) &= e^{-j\theta/2} \left( \rho \Be_1 e^{ i \phi } + R \Be_3 \right) e^{j \theta/2} \\
i &= \Be_1 \Be_3 \\
j &= \Be_3 \Be_2
\end{align}
\end{subequations}

It happens that the unit bivectors \(i\) and \(j\) used in this construction happen
to have the
quaternion-ic properties \(i j = -j i\), and \(i^2 = j^2 = -1\) which can be verified easily.

The curvilinear basis is found (\cref{problem:toriodalProblem:1}) to be
\begin{subequations}\label{eqn:torusCenterOfMassParameterization:3}
\begin{align}
\Bx_\rho &= \PD{\rho}{\Bx} = e^{-j\theta/2} \Be_1 e^{ i \phi } e^{j \theta/2} \\
\Bx_\theta &= \PD{\theta}{\Bx}
%&= e^{-j\theta/2} \left( \rho \inv{2} \left( -\Be_3 \Be_2 \Be_1 e^{ i \phi } + \Be_1 e^{ i \phi } \Be_3 \Be_2 \right) + R \Be_2 \right) e^{j \theta/2} \\
= e^{-j\theta/2} \left( R + \rho \sin\phi \right) \Be_2 e^{j \theta/2} \\
\Bx_\phi &= \PD{\phi}{\Bx} = e^{-j\theta/2} \rho \Be_3 e^{ i \phi } e^{j \theta/2}.
\end{align}
\end{subequations}

The oriented
volume element can be computed using a trivector selection operation, which conveniently wipes out a number of the interior exponentials
%\begin{align}\label{eqn:torusCenterOfMassParameterization:4}
\begin{equation}\label{eqn:torusCenterOfMassParameterization:4}
\PD{\rho}{\Bx} \wedge \PD{\theta}{\Bx} \wedge \PD{\phi}{\Bx}
=
\rho \left( R + \rho \sin\phi \right) \gpgradethree{ e^{-j\theta/2} \Be_1 e^{ i \phi } \Be_2 \Be_3 e^{ i \phi } e^{j \theta/2} }.
%\end{align}
\end{equation}

Note that \(\Be_1\) commutes with \(j = \Be_3 \Be_2\), so also with \(e^{-j\theta/2}\).
Also \(\Be_2 \Be_3 = -j\) anticommutes with \(i\), so
there is a conjugate commutation effect \(e^{i\phi} j = j e^{-i\phi}\).  This gives
\begin{equation}\label{eqn:torusCenterOfMassParameterization:28}
\begin{aligned}
\gpgradethree{ e^{-j\theta/2} \Be_1 e^{ i \phi } \Be_2 \Be_3 e^{ i \phi } e^{j \theta/2} }
&=
-\gpgradethree{ \Be_1 e^{-j\theta/2} j e^{ -i \phi } e^{ i \phi } e^{j \theta/2} } \\
&=
-\gpgradethree{ \Be_1 e^{-j\theta/2} j e^{j \theta/2} } \\
&=
-\gpgradethree{ \Be_1 j } \\
&=
I.
\end{aligned}
\end{equation}

Together the trivector grade selection reduces almost magically to just
\begin{equation}\label{eqn:torusCenterOfMassParameterization:5}
\PD{\rho}{\Bx} \wedge \PD{\theta}{\Bx} \wedge \PD{\phi}{\Bx}
=
\rho \left( R + \rho \sin\phi \right) I.
\end{equation}

\todo{Show this with Mathematica too.}

Thus the (scalar) volume element is
\begin{align}\label{eqn:torusCenterOfMassParameterization:6}
dV = \rho \left( R + \rho \sin\phi \right) d\rho d\theta d\phi.
\end{align}

As a check, it should be the case that the
volume of the complete torus using this volume element has the
expected \(V = (2 \pi R) (\pi r^2)\) value.

That volume is
\begin{align}\label{eqn:torusCenterOfMassParameterization:7}
V = \int_{\rho=0}^r \int_{\theta=0}^{2\pi} \int_{\phi=0}^{2\pi} \rho \left( R + \rho \sin\phi \right) d\rho d\theta d\phi.
\end{align}

The sine term conveniently vanishes over the \(2\pi\) interval, leaving just
\begin{align}\label{eqn:torusCenterOfMassParameterization:8}
V = \inv{2} r^2 R (2 \pi)(2 \pi),
\end{align}

as expected.
