%
% Copyright � 2018 Peeter Joot.  All Rights Reserved.
% Licenced as described in the file LICENSE under the root directory of this GIT repository.
%
%{
%
% Copyright © 2018 Peeter Joot.  All Rights Reserved.
% Licenced as described in the file LICENSE under the root directory of this GIT repository.
%
%\input{../latex/blogpost.tex}
\newcommand{\authorname}{Peeter Joot}
\newcommand{\email}{peeter.joot@mail.utoronto.ca, 920798560}
\newcommand{\basename}{FIXMEbasenameUndefined}
\newcommand{\dirname}{notes/FIXMEdirnameUndefined/}
\renewcommand{\basename}{ece2500report}
%\renewcommand{\dirname}{notes/phy1520/}
\renewcommand{\dirname}{notes/ece1228-electromagnetic-theory/}
%\newcommand{\dateintitle}{}
%\newcommand{\keywords}{}

% too many alphabets fix:
% https://tex.stackexchange.com/a/243541/15
\newcommand\hmmax{0}
\newcommand\bmmax{0}

%\input{../latex/peeter_prologue_print2.tex}
\newcommand{\onlineurl}{https://peeterjoot.com/archives/math2018/\basename.pdf}
\newcommand{\sourcepath}{\dirname\basename.tex}
\newcommand{\generatetitle}[1]{\chapter{#1}}

% this has a different implementation in various thisbook.sty's:
\newcommand{\underlineAndIndex}[1]{\underline{#1}}
\newcommand{\paragraphAndIndex}[1]{\paragraph{#1}}

\newcommand{\vcsinfo}{%
\section*{}
\noindent{\color{DarkOliveGreen}{\rule{\linewidth}{0.1mm}}}
\paragraph{Document version}
%\paragraph{\color{Maroon}{Document version}}
{
\small
\begin{itemize}
\item Available online at:\\
\href{\onlineurl}{\onlineurl}
\item Git Repository: \input{./.revinfo/gitRepo.tex}
\item Source: \sourcepath
\item last commit: \input{./.revinfo/gitCommitString.tex}
\item commit date: \input{./.revinfo/gitCommitDate.tex}
\end{itemize}
}
}

%\PassOptionsToPackage{dvipsnames,svgnames}{xcolor}
\PassOptionsToPackage{square,numbers}{natbib}

\documentclass{scrreprt}
%\documentclass[12pt]{scrreprt}
%\usepackage[left=1.2cm,right=1.2cm,top=1.5cm,bottom=1.5cm]{geometry}
%\usepackage[left=1.5cm,right=1.5cm,top=1.5cm,bottom=1.5cm]{geometry}
%\usepackage[top=1.5cm,bottom=1.5cm]{geometry}
%\usepackage[left=3cm,right=3cm,top=4cm,bottom=4cm]{geometry}
\usepackage[left=2.4cm,right=2.4cm,top=2.9cm,bottom=2.9cm]{geometry}
%\usepackage[left=1.9cm,right=1.9cm,top=2.2cm,bottom=2.2cm]{geometry}
%\usepackage{geometry}

%\documentclass{scrreprt}
%\usepackage[left=2cm,right=2cm]{geometry}

%\documentclass[10pt]{scrreprt}
%\usepackage[left=2cm,right=2cm]{geometry}

\usepackage[svgnames]{xcolor}
%\usepackage[english]{cleveref}
\usepackage{peeters_layout}
%\usepackage{peeters_layout_exercise}

\usepackage{natbib}

\usepackage[
colorlinks=true,
bookmarks=false,
pdfauthor={\authorname, \email},
backref
]{hyperref}

% http://tex.stackexchange.com/questions/75773/how-to-reference-problems-by-the-text-label-in-an-exercise-envioronment
\usepackage[english]{cleveref}
\crefname{Exercise}{exercise}{exercises}
\Crefname{Exercise}{Exercise}{Exercises}

\crefname{theorem}{theorem}{theorems}
\Crefname{theorem}{Theorem}{Theorems}

\crefname{lemma}{lemma}{lemmas}
\Crefname{lemma}{Lemma}{Lemmas}

\usepackage{tikz}

\RequirePackage{titlesec}
\RequirePackage{ifthen}

% http://stackoverflow.com/questions/4932910/date-in-the-tabular-environment
\makeatletter
\let\insertdate\@date
\makeatother

\titleformat{\chapter}[display]
{\bfseries\Large}
{\color{DarkSlateGrey}\filleft \authorname
\ifthenelse{\isundefined{\studentnumber}}{}{\\ \studentnumber}
\ifthenelse{\isundefined{\email}}{}{\\ \email}
\ifthenelse{\isundefined{\dateintitle}}{}{\\ \insertdate}
%\ifthenelse{\isundefined{\coursename}}{}{\\ \coursename} % put in title instead.
}
{4ex}
{\color{DarkOliveGreen}{\titlerule}\color{Maroon}
\vspace{2ex}%
\filright}
[\vspace{2ex}%
\color{DarkOliveGreen}\titlerule
]

\newcommand{\beginArtWithToc}[0]{\begin{document}\tableofcontents}
\newcommand{\beginArtNoToc}[0]{\begin{document}}
\newcommand{\EndNoBibArticle}[0]{\end{document}}
\newcommand{\EndArticle}[0]{\bibliography{Bibliography}\bibliographystyle{plainnat}\end{document}}

%
%\newcommand{\citep}[1]{\cite{#1}}

\colorSectionsForArticle

% FIXME: this doesn't work anymore, now that peeters_layout_exercise.sty is pulled out.
%% (a), (b), (c), ... numbering in ex (exercise pulled in by peeters_layout)
%\renewcommand{\QuestionNB}{\alph{Question}.\ }
%\renewcommand{\theQuestion}{\alph{Question}}

% have \cref fixup in book_layout.sty.  Use this macro instead:
\newcommand{\eqnref}[1]{eq. (\ref{#1})}
\newcommand{\Eqnref}[1]{Eq. (\ref{#1})}

%
% example:
%\statmechchapcite{nonIntegralBinomialSeries}
\newcommand{\quantumsolidschapcite}[1]{\citep{phy487:#1}}
%\newcommand{\statmechchapcite}[1]{\citep{phy452:#1}}
%\newcommand{\statmechchapcite}[1]{\cref{chap:#1}}

\usepackage{peeters_layout_exercise}
\usepackage{peeters_braket}
\usepackage{peeters_figures}

%\newcommand{\dAlembertian}[0]{\square}
\newcommand{\dAlembertian}[0]{\Box}

\newcommand{\dispdot}[2][.2ex]{\dot{\raisebox{0pt}[\dimexpr\height+#1][\depth]{$#2$}}}% \dispdot[<displace>]{<stuff>}
\newcommand{\dotBJ}[0]{\dispdot{\mathbf{J}}}

\newcommand{\stgrad}[0]{\lr{ \spacegrad + \inv{c} \PD{t}{}}}
\newcommand{\conjstgrad}[0]{\lr{ \spacegrad - \inv{c} \PD{t}{}}}
\newcommand{\stgradi}[0]{\lr{ \spacegrad + (1/c) \PDi{t}{}}}
\newcommand{\conjstgradi}[0]{\lr{ \spacegrad - (1/c) \PDi{t}{}}}

\usepackage{siunitx}
%\usepackage{mhchem} % \ce{}
\usepackage{macros_bm} % \bcM
\usepackage{macros_cal}
%\usepackage{macros_qed} % \qedmarker
\usepackage{txfonts} % \ointclockwise
\usepackage{enumerate}
\usepackage{mmacells}



\beginArtNoToc

\generatetitle{Project report ECE2500.  Geometric Algebra for Electrical Engineers}

\section{Motivation.}
This is the report for an ECE2500 M.Eng project course.

\subsection{Goals.}
This project had a few goals
\begin{enumerate}
\item Perform a literature review of applications of geometric algebra
to the study of electromagnetism.
\item Identify the subset of the literature that had direct relevance to electrical engineering.
\item Create a complete, and as compact as possible, introduction of the prerequisites required
for a graduate or advanced undergraduate electrical engineering student to be able to apply
geometric algebra to problems in electromagnetism.
%\item Develop some computer algebra methods for geometric algebra manipulation.
\end{enumerate}

\subsection{What is geometric algebra?}

In a geometric sense, vector algebra provides a representation of directed line segments.
Geometric algebra generalizes vector algebra also providing an algebraic representation of points, plane segments, volumes, and higher degree geometric objects (hypervolumes.)
The geometric algebra representation of planes, volumes and hypervolumes requires a vector dot product, a vector multiplication operation, and a generalized addition operation.
The dot product provides the length of a vector and a test for vector perpendicularity.
The vector multiplication operation is used to construct
directed plane segments (bivectors),
and directed volumes (trivectors), which are built from the respective products of two or three mutually perpendicular vectors.
The geometric algebra
sum of scalars, vectors, or products of vectors is called a multivector.

The power to multiply and add scalars, vectors, and products of vectors can be exploited to simplify much of electromagnetism.
In particular, Maxwell's equations for isotropic media can be merged into a single multivector equation
\begin{dmath}\label{eqn:quaternion2maxwellWithGA:20}
\stgrad \lr{ \BE + I \eta \BH } = \eta\lr{ c \rho - \BJ },
\end{dmath}
where \( \spacegrad \) is the gradient, \( I = \Be_1 \Be_2 \Be_3 \) is the ordered product of the three \R{3} basis vectors, \( c = 1/\sqrt{\mu\epsilon}\) is the group velocity of the medium, \( \eta = \sqrt{\mu/\epsilon} \) is the impedance of the medium, \( \BE, \BH \) are the electric and magnetic fields, and
\( \rho \) and \( \BJ \) are the charge and current densities.
We will write this as
\begin{dmath}\label{eqn:ece2500report:40}
\stgrad F = J,
\end{dmath}
where \( F = \BE + I \eta \BH \) is the combined (multivector) electromagnetic field, and \( J = \eta\lr{ c \rho - \BJ } \) is the multivector current.
The complex interdependencies of Maxwell's equations require the student to learn a variety of special techniques and tricks to find solutions.
With Maxwell's equations reduced a single equation, we may throw the
complete Green's function toolbox into the fight, and directly
invert Maxwell's equation for \( F \) for any charge and current density distribution by convolution with the Green's function for the operator \( \stgradi \) as follows
\begin{dmath}\label{eqn:ece2500report:60}
F(\Bx, t)
= \int dt' dV' \, G(\Bx, \Bx' ; t, t') J(\Bx', t').
\end{dmath}
Green's functions may also be applied to static and frequency domain field configurations.

Working with the combined field \( F \) shows the
hidden structure behind a number of seemingly
disparate ideas in electromagnetism.
This project explored a number of ideas along these lines.
For example, a
Green's function solutions for the static field configurations simultaneously yields Coulomb's and the Biot-Savart law.
Plane, circular and elliptical waves may be expressed compactly in a multivector form, naturally expressing the mutual perpendicularity of the electric field, magnetic field and the propagation directions, as well
as the relationships between the electric and magnetic field components.
The field energy density and the Poynting vector have a simple multivector form expressed in terms of \( F \) alone.
Calculations of radiation pressure can be performed using only the normal component of what is known as the energy momentum tensor in the conventional representation, which has a particularly compact multivector
representation.

\subsection{Results: ``\textit{The book}''.}

Much of the geometric algebra literature for electrodynamics is presented with a relativistic bias, or assumes high levels of
physics and mathematical sophistication.
The aim of this work was an attempt to make the study of electromagnetism using geometric algebra more accessible, especially to an electrical engineering audience.
In particular, this project explored non-relativistic applications of geometric algebra to electromagnetism.
The end product of this project was a fairly small self contained book, titled ``Geometric Algebra for Electrical Engineers''.
This book includes an introduction to Euclidean geometric algebra focused on \R{2} and \R{3} (64 pages), an introduction to geometric calculus and multivector Green's functions (64 pages), applications to electromagnetism (82 pages), plus some appendices.
From this point on this work will be referred to as \textit{the book}.

In the book, many of the fundamental results of electromagnetism are derived directly from
\cref{eqn:ece2500report:40}, the multivector Maxwell's equation, in a streamlined and compact fashion.
This includes some new results, and many of the existing non-relativistic results from geometric algebra literature.
As a conceptual bridge, the book includes many examples of how to extract
familiar conventional results from simpler multivector representations.
Also included in the book are some sample calculations exploiting unique capabilities that geometric algebra provides.
In particular, vectors in a plane may be manipulated much like complex numbers, which has a number of advantages over working with coordinates explicitly.

The book produced in this project provides the prerequisite material for such exploration, and some first steps along the path of such an expedition.

\section{This report.}

The research focus of this project was multivector calculus, multivector Green's functions, and multivector electromagnetism.
A great deal of work was done to try to describe all of these and all the prerequisite geometric algebra concepts in a comprehensive and logically sequenced fashion.
This report aims to provide an overview of the work done on this project, but due to space constraints, it is not possible to
simultaneously summarize the work and explain it all in a meaningful and self contained fashion.
In the book, a great deal of work was spent assembling
figures, Mathematica listings\footnote{Using Mathematica modules developed in this project, and other existing Mathematica modules.}, worked examples, citations, explanations and derivations.  These supplemented the ideas presented in this summary report, but rather painfully, almost all such detail had to be omitted from this report.

\section{Chapter I: Geometric Algebra.}
Chapter I of the book provides a self contained introduction to geometric algebra, with all the prerequisites required for geometric calculus and the subsequent applications to electromagnetism.  Concepts included in chapter 1 include bivector, trivector, multivector, pseudoscalar, grade selection, commutation and anticommutation conditions, rotation, complex representation, duality, vector product, multivector dot product, wedge product, wedge and cross product relationships, reversion, blade, distribution rules, projection, rejection, vector inverse, reflection, and linear system solution.  As new ideas are introduced there is a continual attempt to show how geometric algebra identites map to thier equivalents in conventional vector algebra.  A number of such equivalencies are summarized in a table at the end of the chapter.

\section{Chapter II: Geometric Calculus.}

Chapter II had two goals.
The first goal was to present the theory of multivector integration along curves and surfaces (i.e. manifold calculus), and the second goal was to derive all the multivector Green's functions required for application to problems of electromagnetism.
The multivector integration theory incorporates all of vector calculus as well as the more abstract theory of differential forms or integration on manifolds, but also generalizes both.
The key ideas of this chapter will be summarized below.
Examples, figures, Mathematica listings, derivations, and detailed explanations of the ideas summarized here are available in the book.

\subsection{Reciprocal frames.}
Most elementary texts on electromagnetism will state or derive Stokes theorem and the divergence theorem in their rectangular coordinate form.
The fundamental theorem of geometric calculus is simplest to prove in the context of curvilinear coordinates.
This requires some additional mathematical baggage, one part of which is the reciprocal frame, a key concept to working with
curvilinear and non-orthonormal bases.

%
% Copyright � 2018 Peeter Joot.  All Rights Reserved.
% Licenced as described in the file LICENSE under the root directory of this GIT repository.
%
\makedefinition{Reciprocal frame}{dfn:reciprocal:frame}{
Given a subspace basis \( \beta = \setlr{ \Bx_1, \Bx_2, \cdots \Bx_m } \), not
necessarily orthonormal, the \textit{reciprocal frame} is defined as the set of vectors \( \setlr{ \Bx^1, \Bx^2, \cdots \Bx^m } \in \Span \beta \) satisfying
\begin{dmath*}
\Bx_i \cdot \Bx^j = {\delta_i}^j,
\end{dmath*}
where the vector \( \Bx^j \) is not the j-th power of \( \Bx \), but is a superscript index, the conventional way of denoting a reciprocal frame vector, and \( {\delta_i}^j \) is the Kronecker delta.
} % definition


Any orthonormal basis is also its own reciprocal frame (or basis).
Mixed index (upper and lower) variables are used when working with curvilinear coordinates.
Summation (Einstein) convention was not used in the book, so sums over matched pairs of upper and lower indexes, are
marked with an explicit summation symbol.  Since geometric algebra identities are often coordinate free, this is not a terrible imposition.

A vector may be expressed in terms of either the curvilinear or reciprocal basis
\begin{equation}\label{eqn:ece2500report:2600}
\Ba
= \sum_i \lr{ \Ba \cdot \Bx^i } \Bx_i
= \sum_i \lr{ \Ba \cdot \Bx_i } \Bx^i.
\end{equation}
Either basis may be used to compute coordinates \( a_i = \Ba \cdot \Bx_i, a^i = \Ba \cdot \Bx^i \).  Coordinates are never used with an implied basis,
as is typical in both
engineering (column representation of vectors), and
relativistic physics (four-vectors), so we have
no need of the covariant nor contravariant tensor nomenclature despite using mixed index representations.
To help the reader become familiar with curvilinear coordinates and reciprocal frames,
figures, hand calculations, Mathematica listings, and examples are provided in the book.

\subsection{Curvilinear coordinates and hypervolume elements.}
The book provides a layman's introduction to manifold calculus in its geometric algebra form, where a manifold is, loosely speaking,
a connected and orientable subspace defined by a vector parameterization \( \Bx = \Bx(u_1, u_2, \cdots ) \).
Here connectivity is mentioned in passing, but is generally ignored in the book.
Infinitesimal partitioning (i.e. triangularization) is also generally ignored in the book, although some references are provided for further study by the interested student.

Given a manifold defined by a parameterization,
curvilinear basis elements are defined by partials with respect to the parameters \( \Bx_i = \PDi{u_i}{\Bx} \), as are the associated vector valued differentials \( d\Bx_i = \Bx_i du_i \).
Bivector valued area elements \( d^2 \Bx \equiv d\Bx_1 \wedge d\Bx_2 \) are formed by wedging the two vector differentials for a two parameter manifold, whereas trivector valued volume elements  \( d^3 \Bx \equiv d\Bx_1 \wedge d\Bx_2 \wedge d\Bx_3 \) are formed by wedging all three of the differentials for the parameterized space.
In general, it is assumed in the book that no (hyper)volume element \( d^k \Bx = d\Bx_1 \wedge \cdots \wedge d\Bx_k \) is ever zero (and thus never changes sign).
Such hypervolume elements have an orientation, as changing the order of any two differentials toggles the sign.
Orientability is also ignored in the book, and it should be assumed that the theorems stated do not hold for Mobius-strip like manifolds.
For Euclidean spaces, the focus of the book, the absolute values of such area and volume differentials are \( dA = \sqrt{} -( d^2 \Bx)^2, dV = \sqrt{} -( d^3 \Bx)^2 \), which are the respective areas and volumes of the parallelogram and parallelepiped spanned by the vector differentials in each of the parameterization directions.

The required concepts are explained in detail in the book.
Many examples, figures, and Mathematica listings are used to facilitate that explanation.

\subsection{Gradient and vector derivative.}

The final tool required to formulate multivector integration theory is the vector derivative, the projection of the gradient onto the integration manifold.

%
% Copyright � 2018 Peeter Joot.  All Rights Reserved.
% Licenced as described in the file LICENSE under the root directory of this GIT repository.
%
\maketheorem{Reciprocal frame vectors}{thm:curvilinearGradient:1}{
Given a curvilinear basis with elements \( \Bx_i = \PDi{u_i}{\Bx} \), the \textit{reciprocal frame} vectors are given by
\begin{equation*}
\Bx^i = \spacegrad u_i.
\end{equation*}
} % theorem


%
% Copyright � 2018 Peeter Joot.  All Rights Reserved.
% Licenced as described in the file LICENSE under the root directory of this GIT repository.
%
\maketheorem{Curvilinear representation of the gradient}{thm:curvilinearGradient:2}{
Given an N-parameter vector parameterization
\( \Bx = \Bx(u_1, u_2, \cdots, u_N) \)
of \R{N},
with curvilinear basis elements \( \Bx_i = \PDi{u_i}{\Bx} \), the \textit{gradient} is
\begin{equation*}
\spacegrad = \sum_i \Bx^i \PD{u_i}{}.
\end{equation*}
It is convenient to define \( \partial_i \equiv \PDi{u_i}{} \), so that the gradient can be expressed in mixed index representation
\begin{equation*}
\spacegrad = \sum_i \Bx^i \partial_i.
\end{equation*}
%or the same with sums over mixed indexes implied.
} % theorem


%
% Copyright � 2018 Peeter Joot.  All Rights Reserved.
% Licenced as described in the file LICENSE under the root directory of this GIT repository.
%
\makedefinition{Vector derivative}{dfn:gradient:100}{
Given a k-parameter vector parameterization
\( \Bx = \Bx(u_1, u_2, \cdots, u_k) \) of \R{N} with \( k \le N \),
and curvilinear basis elements \( \Bx_i = \PDi{u_i}{\Bx} \), the \textit{vector derivative} \( \boldpartial \) is defined as
\begin{equation*}
\boldpartial = \sum_{i=1}^k \Bx^i \partial_i.
\end{equation*}
} % theorem


When the dimension of the subspace (number of parameters) equals the dimension of the underlying vector space, the vector derivative and gradient are identical.
In other situations what is the vector derivative?  This can be answered by introducing the concept of the tangent space.
The tangent space \( T_\Bp \) is a curvilinear concept that represents the span of the differentials \( \Bx_i \) at the point of evaluation \( \Bp \).  For a curve this is the tangent line, and for a surface, this is the tangent plane.  Both are illustrated in the book.
The projection of any vector onto a k-dimensional tangent space with basis \( \setlr{\Bx_i} \) all evaluated at the point \( \Bp \) is just \(
\Proj_{T_\Bp} \Bf = \sum_{i = 1}^k \Bx^i (\Bx_i \cdot \Bf) \).
For the vector derivative, this is just
\begin{equation}\label{eqn:ece2500report:2640}
\Proj_{T_\Bp} \spacegrad
= \sum_{i = 1}^k \sum_{j = 1}^N \Bx^i (\Bx_i \cdot \Bx^j) \partial_j
= \sum_{i = 1}^k \sum_{j = 1}^N \Bx^i {\delta_i}^j \partial_j
= \sum_{i = 1}^k \Bx^i \partial_i,
\end{equation}
which is the vector derivative.
The vector derivative is the projection of the gradient onto the tangent space at the point of evaluation.

Examples, figures, and Mathematica calculations are used in the book to illustrate and explain the vector derivative and associated curvilinear parametrizations.
This includes calculations of curvilinear coordinates, area and volume elements, and the vector derivative
for polar, spherical and a toroidal coordinate systems.

\subsection{Integration theory.}

The vector derivative
may not commute the functions it acts on nor a k-volume element \( d^k \Bx \), so we are forced to use some notation to indicate what the vector derivative (or gradient) acts on.
%
% Copyright � 2018 Peeter Joot.  All Rights Reserved.
% Licenced as described in the file LICENSE under the root directory of this GIT repository.
%
\makedefinition{k-volume integrand.}{dfn:fundamentalTheoremOfCalculus:240}{
Given a k-volume volume element \( d^k \Bx \), and
multivector functions \( F, G \), a k-volume integrand is a double sided application of the vector derivative on \( F, G \)
\begin{equation*}
F d^k \Bx \lrboldpartial G
=
(F d^k \Bx \lboldpartial) G
+
F d^k \Bx (\rboldpartial G),
\end{equation*}
The explicit meaning of these directionally acting derivative operations is given by the following chain rule coordinate expansion
\begin{equation*}
\begin{aligned}
F d^k \Bx \lrboldpartial G
&= F d^k \Bx \lr{ \sum_i \Bx^i {\stackrel{ \leftrightarrow }{\partial_i}} } G \\
&= (\partial_i F) d^k \Bx \sum_i \Bx^i G + F d^k \Bx \sum_i \Bx^i (\partial_i G).
\end{aligned}
\end{equation*}
} % definition

In conventional right acting cases, where there is no ambiguity, arrows will usually be omitted, but braces may also be used to indicate the scope of derivative operators.
This bidirectional notation will also be used for the gradient, especially for volume integrals in \R{3} where the vector derivative is identical to the gradient.
Some authors use the Hestenes dot notation, with overdots or primes to indicating the exact scope of multivector derivative operators, as in
\begin{dmath}\label{eqn:fundamentalTheoremOfCalculus:260}
\dot{F} d^k \Bx \dot{\boldpartial} \dot{G} =
\dot{F} d^k \Bx \dot{\boldpartial} G
+
F d^k \Bx \dot{\boldpartial} \dot{G}.
\end{dmath}
The dot notation has the advantage of emphasizing that the action of the vector derivative (or gradient) is on the functions \( F, G \), and not on the hypervolume element \( d^k \Bx \).
However, in the book, where primed operators such as \( \spacegrad' \) are used to indicate that derivatives are taken with respect to primed \( \Bx' \) variables, a mix of dots and ticks would have been confusing.

The generalization of line, surface and volume integrals to hypervolumes and multivector functions can now be stated.
\input{Theorem_fundamental_theorem_of_gc.tex}

The fundamental theorem of geometric calculus is a generalization of many conventional scalar and vector integral theorems, and relates a hypervolume integral to its boundary.
This is a a powerful theorem, used in the book with Green's functions to solve Maxwell's equation.

The book includes a general k-volume proof of
\cref{thm:fundamentalTheoremOfCalculus:1} in the appendix.
Since most of the complexity of the proof is some hellish index manipulation required for the general k-volume case,
to aid accessibility, this theorem is stated and proven separately in the book for each of the line, surface, and volume integral cases.

\subsubsection{Stokes' theorem.}
An important consequence of the fundamental theorem of geometric calculus is the geometric algebra form of Stokes' theorem.
The Stokes' theorem that we know from conventional vector calculus relates \R{3} surface integrals to the line integral around a bounding surface.
The geometric algebra form of Stokes' theorem is equivalent to Stokes' theorem from the theory of differential forms, a more general theorem.
%, which relates
%hypervolume integrals of blades\footnote{Blades are isomorphic to the k-forms found in the theory of differential forms.} to the integrals over their hypersurface boundaries, a much more general result.

\input{Theorem_stokes_theorem_general.tex}

In the book we will see that most of the well known scalar and vector integral theorems can easily be derived as direct consequences of \cref{thm:stokesTheoremGeometricAlgebra:1740}, itself a special case of \cref{thm:fundamentalTheoremOfCalculus:1}.
This report states those results for the special cases of line, area and volume integrals.

\input{Theorem_multivector_line_integral.tex}

\input{Theorem_line_integral_scalar_function.tex}

\input{Theorem_multivector_surface_integral.tex}

\input{Theorem_surface_integral_of_scalar_stokes.tex}

\input{Theorem_surface_integral_of_vector_stokes.tex}

\input{Theorem_greens.tex}

\input{Theorem_multivector_volume_integral.tex}

\input{Theorem_volume_integral_of_vector_stokes.tex}

\input{Theorem_volume_integral_of_bivector_stokes_divergence.tex}

\input{Theorem_divergence.tex}

All of \cref{thm:stokesTheoremGeometricAlgebra:1740}-\cref{thm:volumeintegral:2661} are direct consequences (or specializations) of
\cref{thm:fundamentalTheoremOfCalculus:1}.

\subsection{Multivector Fourier transform and phasors.}
The book makes use of time harmonic (frequency domain) representations when convenient.
Provided we utilize a scalar (non-geometric) imaginary, a standard Fourier transform generalizes to multivectors.

\input{Definition_multivector_fourier.tex}
By non-geometric, we mean that the imaginary need not have a geometric interpretation such as \( j = \Be_{12}, \Be_{123}, \cdots \).  Using a geometric representation of the imaginary requires special care as it may not commute with the multivector functions.  This is an active research topic in the literature, and was avoided in the book.

The book uses the engineering convention for
phasors, with a positive sign on the angular frequency complex exponentials (i.e. \( e^{j\omega t} \), not \( e^{-i \omega t} \)).

\input{Definition_multivector_phasor.tex}

The complex valued multivector \( F(\Bx) \) is still generated from the real Euclidean basis for \R{3}, so there will be
no reason to introduce complex inner products spaces into the mix.

\subsection{Multivector Green's functions.}
An attempt to motivate the concept of a Green's function is provided in the book, omitted here.
The book specifies the chosen sign convention for the Green's functions used, as those vary in the literature.
Bounded vs. unbounded integration volumes and some theorems related to bounded applications of Green's functions (Green's theorem, and a related identity for the (first order) gradient) are also discussed.

\input{Theorem_greens_function_helmholtz_report.tex}

We will use the advancing (causal) Green's function, and refer to this function as \( G(\Bx, \Bx') \) without any subscript.
Observe that as a special case, the Helmholtz Green's function reduces to the Green's function for the Laplacian when \( k = 0 \)
\begin{dmath}\label{eqn:greensFunctionHelmholtz:80}
G(\Bx, \Bx') = -\inv{ 4 \pi \Norm{\Bx - \Bx'}}.
\end{dmath}

The Helmholtz operator can be factored in geometric algebra as \( \spacegrad^2 + k^2 = \lr{ \spacegrad \pm j k }\lr{ \spacegrad \mp j k } \).
We will call these factors first order Helmholtz operators, and derive their (multivector valued) Green's functions, which are as follows.
\input{Theorem_greens_function_helmholtz_first_order.tex}

This theorem is proven in the book from the Green's function for the (second order) Helmholtz operator.
The solution of the
the first order Helmholtz system \( \lr{ \spacegrad + j k } F = J \) follows immediately
\begin{dmath}\label{eqn:greensFunctionFirstOrderHelmholtz:880}
F(\Bx)
=
\int_V G(\Bx, \Bx') J(\Bx') dV'
-
\int_{\partial V} G(\Bx, \Bx') \ncap' F(\Bx') dA'
+ F_0,
\end{dmath}
where \( F_0 \) is a solution to the homogeneous first order Helmholtz equation \( \lr{ \spacegrad + j k } F_0 = 0 \).
Given a ``well-behaved'' \( F(\Bx') \) the boundary term is required to vanish when the integral is taken to infinity, which leaves a solution that depends only on the sources though the convolution with the Green's function.

A special but important case of \cref{thm:gradientGreensFunctionEuclidean:720}
is the \( k = 0 \) condition, which provides the
Green's function for the gradient, which is vector valued
\begin{equation}\label{eqn:greensFunctionFirstOrderHelmholtz:900}
G(\Bx, \Bx' ; k = 0) = \inv{4 \pi} \frac{\rcap}{r^2}.
\end{equation}

For the wave equation operator, it is helpful to introduce a d'Alembertian operator.  Because the sign of this operator varies in the literature, we must state our choice explicitly.
%
% Copyright � 2018 Peeter Joot.  All Rights Reserved.
% Licenced as described in the file LICENSE under the root directory of this GIT repository.
%
\makedefinition{d'Alembertian (wave equation) operator.}{dfn:continuity:120}{
The symbol \( \dAlembertian \) is used to represent the \textit{d'Alembertian (wave equation) operator}, with a positive sign on the Laplacian term
\begin{equation*}
\dAlembertian =
\conjstgrad
\stgrad
=
\spacegrad^2 - \inv{c^2} \PDSq{t}{}.
\end{equation*}
} % definition


The d'Alembertian operator may be factored in geometric algebra as \( \dAlembertian = \conjstgradi \stgradi \).
The operator \( \stgradi \) will be called the \textit{spacetime gradient}\footnote{This form of spacetime gradient is given a special symbol (\(\gamma_0 \grad, \overbar{D}, \calD, \cdots\)) by a number of authors, but there is no general agreement on what to use.
Instead of entering the fight, it was written out in full in the book.}.

In the book, the Green's function for the spacetime gradient is derived from the Green's function for the d'Alembertian, also well known.
%
% Copyright � 2018 Peeter Joot.  All Rights Reserved.
% Licenced as described in the file LICENSE under the root directory of this GIT repository.
%
\maketheorem{Green's function for the spacetime gradient.}{thm:greensFunctionSpacetimeGradient:120}{
The \textit{Green's function for the spacetime gradient} \( \spacegrad + (1/c) \partial_t \) satisfies
\begin{equation*}
\stgrad G(\Bx - \Bx', t - t') = \delta(\Bx - \Bx') \delta(t - t'),
\end{equation*}
and has the value
\begin{equation*}
G(\Bx - \Bx', t - t')
=
\inv{4\pi} \lr{
- \frac{\rcap}{r^2} \PD{r}{}
+ \frac{\rcap}{r^2}
+ \inv{c r} \PD{t}{}
}
\delta( -r/c + t - t' ),
\end{equation*}
where \( \Br = \Bx - \Bx', r = \Norm{\Br} \) and \( \rcap = \Br/r \).
} % theorem

Like the Green's function for the first order Helmholtz operator, the Green's function for the spacetime gradient is also multivector.
These are all the Green's functions required for the electromagnetic solutions considered in chapter 3 of the book.

\subsection{Helmholtz theorem.}
In the book first and second order proofs of Helmholtz theorem are provided, both using geometric algebra techniques.
Discussion of those derivations are omitted here for brevity.
\section{Chapter III: Electromagnetism.}
\subsection{Conventional Maxwell's equations.}
In the book, it is presumed that the reader is familiar with Maxwell's equations.
No attempt to motivate them was made.
The starting point was Maxwell's equations with antenna theory extensions (fictitious magnetic sources)
\begin{dmath}\label{eqn:ece2500report:2540}
\begin{aligned}
\spacegrad \cross \BE &= - \BM - \PD{t}{\BB} \\
\spacegrad \cross \BH &= \BJ + \PD{t}{\BD} \\
\spacegrad \cdot \BD &= \rho \\
\spacegrad \cdot \BB &= \rho_\txtm.
\end{aligned}
\end{dmath}
where \( \BE, \BH, \BD, \BB \) are the conventional electric and magnetic field intensities and flux densities,
\( \rho, \rho_\txtm \) are the electric and (fictitious-)magnetic charge densities,
and \( \BJ, \BM \) are the electric and (fictitious-)magnetic current densities.

There is some limited discussion of the geometric algebra form of Maxwell's equations for more general media at the end of the book, however
much of the book presumes isotropic constitutive relationships between the electric and magnetic fields
\begin{dmath}
\label{eqn:freespace:300}
\begin{aligned}
\BB &= \mu \BH \\
\BD &= \epsilon \BE,
\end{aligned}
\end{dmath}
where \( \epsilon = \epsilon_r \epsilon_0 \) is the permittivity of the medium, and \( \mu = \mu_r \mu_0 \) is the permeability of the medium.
\subsection{Maxwell's equation.}
For isotropic media and constitutive relationships \cref{eqn:freespace:300} a multivector that includes both electric and magnetic fields is defined.
\input{Definition_electromagnetic_field_strength.tex}

The factors of \( \eta \) (or \( c \)) that multiply the magnetic field are for dimensional consistency, since \( [\sqrt{\epsilon} \BE] = [\sqrt{\mu} \BH] = [\BB/\sqrt{\mu}]\).
The justification for imposing a dual (or complex) structure on the electromagnetic field strength can be found in the historical development of
Maxwell's equations.
This structure also arises naturally when assembling the multivector Maxwell's equation.

No information is lost by imposing the complex structure of
\cref{dfn:isotropicMaxwells:640}, since we can always obtain the
electric field vector \( \BE \) and the magnetic field bivector \( I \BH \) by grade selection
from the electromagnetic field strength when desired using \(
\BE = \gpgradeone{ F },
I \BH = (1/\eta) \gpgradetwo{ F } \).

A multivector current containing all charge densities and current densities is defined as follows.
\input{Definition_multivector_current.tex}
When fictitious magnetic source terms \((\rho_\txtm, \BM)\) are included, the current has one grade for each possible source (scalar, vector, bivector, trivector).
With only conventional electric sources, the current is still a multivector, but contains only scalar and vector grades.

Given the multivector field and current, it is now possible to state the multivector form of Maxwell's equation (singular).
\input{Definition_maxwells.tex}

%See the book for a proof of \cref{dfn:isotropicMaxwells:680}.
The workhorse of the proof is the identity \( \spacegrad \Bb = \spacegrad \cdot \Bb + I \spacegrad \cross \Bb \) which allows Maxwell's equations to be grouped into two gradient equations, one for each of \( \spacegrad \BE \), and \( \spacegrad \BH \)
\begin{dmath}\label{eqn:ece2500report:2660}
\begin{aligned}
\spacegrad \BE &= \inv{\epsilon} \rho + I \lr{ - \BM - \mu \PD{t}{\BH} } \\
\spacegrad \BH &= \inv{\mu} \rho_\txtm + I \lr{ \BJ + \epsilon \PD{t}{\BE} },
\end{aligned}
\end{dmath}
which can then be further grouped after dimensional rescaling to find
\begin{dmath}\label{eqn:isotropicMaxwells:580}
\stgrad \lr{ \BE + I \eta \BH } = \eta\lr{ c \rho - \BJ } + I \lr{ c \rho_\txtm - \BM },
\end{dmath}
which is \cref{dfn:isotropicMaxwells:680} expanded explicitly.  There is a lot of information packed into this single equation.
All the subsequent analysis in the book utilizes the multivector form of Maxwell's equation.

\subsection{Wave equation and continuity.}
It can be
argued that the conventional form
\cref{eqn:ece2500report:2540}
of Maxwell's equations has a built in redundancy since continuity equations on the charge and current densities couple some of the equations.
An opposing argument is also possible, where the continuity equations are viewed as necessary consequences of Maxwell's equation.
This amounts to a statement that the multivector current \( J \) is not completely unconstrained.

\input{Theorem_continuity_wave.tex}

The proof is in the book, but basically just requires operating on Maxwell's equation with \( \conjstgradi \), which yields two equations, one for grades 1,2 and one for grades 0,3
\begin{dmath}\label{eqn:continuity:130}
\begin{aligned}
\dAlembertian
F &= \gpgrade{ \conjstgrad J }{1,2} \\
                                           0 &= \gpgrade{ \conjstgrad J }{0,3}.
\end{aligned}
\end{dmath}
Expansion of the grade 0,3 selection of \cref{eqn:continuity:130} provides the continuity equations.
\subsection{Plane waves.}
With all sources zero,
the free space Maxwell's equation as given by \cref{dfn:isotropicMaxwells:680} for the
electromagnetic field strength reduces to just
\begin{dmath}\label{eqn:planewavesMultivector:300}
\stgrad F(\Bx, t) = 0.
\end{dmath}

Utilizing a phasor representation of the form \cref{dfn:greensFunctionOverview:300},
we will define the
phasor representation of the field as follows.
\input{Definition_plane_wave.tex}

In the book, we show that solutions of the electromagnetic field wave equation have the following form.
\input{Theorem_plane_wave_solutions.tex}

A full proof and discussion is in the book.
The key step is that after insertion of the presumed phasor relationship, we find that
\begin{dmath}\label{eqn:planewavesMultivector:60}
0
=
-j \lr{ \Bk - \frac{\omega}{c} } F(\Bk),
\end{dmath}
which can be satisfied by insisting that \( F \) has a \( \Bk + \omega/c \) factor and that \( \Norm{\Bk} = \omega/c \).
The observation that
\( \kcap, \BE, \BH \) form a right handed triple, is expressed geometrically by \( I = \kcap \Ecap \Hcap \), from which we can also find \( \kcap = \Ecap \cross \Hcap \).

\subsection{Statics solution.}
If we restrict attention to time invariant fields (\( \partial_t F = 0\)) and time invariant sources (\(\partial_t J = 0\)),
Maxwell's equation is reduced to an invertible first order gradient equation
\begin{dmath}\label{eqn:statics:20}
\spacegrad F(\Bx) = J(\Bx),
\end{dmath}

\input{Theorem_maxwell_statics_solution.tex}

The solution incorporates a {\color{DarkOliveGreen}Coulomb's law} contribution and a {\color{Maroon}Biot-Savart law} contribution, as well as their magnetic source analogues.

The proof is essentially a convolution with the (vector valued) Green's function for the (first order) gradient \cref{eqn:greensFunctionFirstOrderHelmholtz:900}.
\subsection{Statics: Enclosed charge.}
In conventional electrostatics we obtain a relation between the normal electric field component and the enclosed charge by integrating the electric field divergence.
The geometric algebra generalization relates the product of the normal and the electromagnetic field strength related to the enclosed multivector current as follows.
\input{Theorem_enclosed_multivector_current.tex}

The proof requires evaluation of the volume integral of the gradient of the field using \cref{thm:volumeintegral:100}, then a grade selection.  Full details are in the book.
The results of the grade selection
could have been obtained directly from Maxwell's equations in their conventional form.
However, integration of the conventional Maxwell's equations would not have shown that this crazy mix of
fields, sources, dot and cross products has a hidden structure as simple as
\( \int_{\partial V} dA \ncap F = \int_V dV J \).

\subsection{Statics: Enclosed current.}
Ampere's law may be generalized to line integrals of the total electromagnetic field strength.
\input{Theorem_line_integral_of_field.tex}
Flipping the direction of integration in the the last of the scalar equations in
\cref{thm:amperes:280}
provides the conventional form of Ampere's law
\begin{equation}\label{eqn:amperes:20}
\ointctrclockwise_{\partial A} d\Bx \cdot \BH = \int_A \ncap \cdot \BJ = I_{\textrm{enc}}.
\end{equation}
The proof and additional details can be found in the book.

It is worth pointing out that for pure magnetostatics problems where \( J = \eta \BJ, F = I \eta \BH \), that Ampere's law can be written in a trivector form
\begin{equation}\label{eqn:amperes:260}
\ointclockwise_{\partial A} d\Bx \wedge F = I \int_A dA\, \ncap \cdot J = I \eta \int_A dA\, \ncap \cdot \BJ.
\end{equation}
This encodes the fact that the magnetic field component of the total electromagnetic field strength is most naturally expressed in
geometric algebra as a bivector.

\subsection{Statics: Example field calculations.}
A number of worked examples were calculated to illustrate geometric algebra techniques.
\begin{itemize}
\item A finite line charge with line charge density \( \lambda \).  This problem is worked with conventional and geometric algebra.  With the conventional approach a compact factorization of the end result is possible by introducing a 3x3 rotation matrix.
In the geometric algebra approach the field observation point is directly encoded in ``complex exponential'' form using
\( i = \Be_1 \Be_3 \) to represent the pseudoscalar for the x-z plane.
The (electric) field is found to have the form \( F = (\lambda/r) \int du \lr{ e^{i\theta} - u } f(u) \), where
\( f(u) \) is a scalar function (specified in the book.)
The scalar portion of the integral is strictly a scale factor for the component of the field that lies along the x-axis, whereas the ``complex exponential'' factor of the integrand represents a rotational term along the direction \( \Be_1 e^{i\theta} = \Be_1 \cos\theta + \Be_3 \sin\theta \).
In problems like this, with only two degrees of freedom, there will often be a complex like representation possible using geometric algebra.
\item The field for infinite static charge and current densities lying along the z-axis \( \rho(\Bx) = \lambda \delta(x) \delta(y), \BJ(\Bx) = \Bv \rho(\Bx) \) is found to be \( F = \lambda \rhocap \lr{ 1 - \Bv/c}/(2 \pi \epsilon R)\).
The field splits naturally into electric (vector) and magnetic (bivector) grades as \( F = \BE \lr{ 1 - \Bv/c } = \BE + I \lr{ \ifrac{\Bv}{c} \cross \BE } \).
\item A similar problem is left for the reader, who is asked to compute the field for
the magnetic charge density \( \rho_m = \lambda_m \delta(x) \delta(y) \), and current density \( \BM = v \Be_3 \rho_m = \Bv \rho_m \).
That field is
\( F = \lambda_m c I \rhocap \lr{ 1 - \ifrac{\Bv}{c} }/(4 \pi R) \), which may be split into electric and magnetic components as
\( F = \BB \cross \Bv + c I \BB \), where \( \BB = \lambda_m \rhocap/(4 \pi R) \).
\item The field for a uniform infinite planar charge density \( \rho(\Bx) = \sigma \delta(z) \) and associated current density \( \BJ(\Bx) = \Bv \rho(\Bx) \), where \( \Bv = v \Be_1 e^{i\theta}, \quad i = \Be_{12} \) is found to be
\( F = \sigma \sgn(z) \Be_3 \lr{ 1 - \ifrac{\Bv}{c}}/(4 \pi \epsilon) \).
As should be expected by superposition, the field splits neatly into electric field (vector) and magnetic field (bivector) components associated with the respective pure electrostatic and magnetostatics problems.
\item As a problem the reader is asked to show that the field for an infinite planar
magnetic charge density \( \rho_m = \sigma_m \delta(z) \), and current density \( \BM = \Bv \rho_m, \Bv = v \Be_1 e^{i\theta}, i = \Be_{12}\) is \( F = \ifrac{\sigma_m c \sgn(z)}{(4 \pi)} i \lr{ 1 - \ifrac{\Bv}{c} } \).
\item The field for a line charge density \( \lambda \) along a circular arc segment \( \phi' \in [a,b] \), of radius \( r \) in the x-y plane is found to be \( F = \ifrac{\lambda r}{(4 \pi \epsilon_0 R^2)} \int_{a-\phi}^{b-\phi} du \lr{ \rcap + \phicap u i e^{i \alpha } } \lr{ 1 + u^2 - 2 u \sin\theta \cos \alpha }^{-3/2} \), where \( i = \Be_{12} \).
This problem is often given as an example or problem in electrostatics, but usually for circular charge distribution, and an observation point on the z-axis where symmetries kill off all but the z-axis component of the field.
The freedom to represent rotational terms as complex exponentials in geometric algebra allows the more general problem to be calculated without much additional difficulty.
The resulting integrals can be evaluated easily with any existing numerical integration software as the vector factors \( \rcap, \phicap \) may be pulled out of the integrals, leaving strict scalar or complex valued integrands.
\item To illustrate the algebraic flexibility available, the circular ring charge problem is tackled in cylindrical coordinates instead of spherical (as previous).
For a static charge line density \( \lambda \) on a ring at \( z = 0 \), and an azimuthal current density \( \BJ = \Bv \rho \), we find a closed form solution for the field is found.
The symmetry of the ring configuration allows for a closed form solution (numerical integration not required) of the field, but comes with the cost of requiring
elliptic integrals, which are detailed in the book along with the derivation and plots of the resulting fields.
\item The final worked statics problem in the book is a use of Ampere's law, to compute the magnetic field for a
pair of z-axis oriented electric currents of magnitude \( I_1, I_2 \) flowing through the \( \Bp_1, \Bp_2 \) on the x-y plane.
The geometry and derivation is detailed in the book, but we use the multivector line integral form of Ampere's law \( \ointctrclockwise_{\partial A} d\Bx\, F = -I \int_A dA \Be_3 (-\eta \BJ) = I \eta I_\txte \), and superposition to compute the field
\( F = \sum_{k = 1,2} \ifrac{\eta I_k}{(2 \pi)} \ifrac{1}{(\Be_3 \wedge \lr{ \Br - \Bp_k})} \).
The bivector (magnetic) nature of a field with only electric current density sources is naturally represented by the wedge product \( \Be_3 \wedge \lr{ \Br - \Bp_k} \) which is a vector product of \( \Be_3 \) and the projection of \( \Br - \Bp_k \) onto the x-y plane.
\end{itemize}
\subsection{Dynamics.}
Maxwell's equation (\cref{dfn:isotropicMaxwells:680}) is invertible, with solution.
%
% Copyright � 2018 Peeter Joot.  All Rights Reserved.
% Licenced as described in the file LICENSE under the root directory of this GIT repository.
%
\maketheorem{Jefimenkos solution.}{thm:jefimenkosEquations:120}{
The general solution of Maxwell's equation is given by
\begin{equation*}
F(\Bx, t)
=
F_0(\Bx, t)
+
\inv{4 \pi}
\int dV'
\lr{
   \frac{\rcap}{r^2} J(\Bx', t_r)
   +
   \inv{c r} \lr{ 1 + \rcap } \dispdot{J}(\Bx', t_r)
},
\end{equation*}
where \( F_0(\Bx, t) \) is any specific solution of the homogeneous equation \( \lr{ \spacegrad + (1/c) \partial_t } F_0 = 0 \),
time derivatives are denoted by overdots, and all times are evaluated at the retarded time \( t_r = t - r/c \).
When expanded in terms of the electric and magnetic fields (ignoring magnetic sources), the non-homogeneous portion of this solution is known as
Jefimenkos' equations
\begin{equation}\label{eqn:jefimenkosEquations:100}
\begin{aligned}
\BE &=
\inv{4 \pi}
\int dV'
\lr{
\frac{\rcap}{\epsilon r} \lr{
\frac{\rho(\Bx', t_r)}{r} + \frac{\dispdot{\rho}(\Bx', t_r) }{c} }
   - \frac{\eta }{ c r } \dotBJ(\Bx', t_r)
} \\
\BH &=
\inv{4 \pi}
\int dV'
\lr{
   \frac{1}{c r} \dotBJ(\Bx', t_r)
+
   \frac{1}{r^2} \BJ(\Bx', t_r)
} \cross \rcap,
\end{aligned}
\end{equation}
%which checks against Griffiths.
} % theorem

This is found fairly easily using the Green's function for the spacetime gradient \cref{thm:greensFunctionSpacetimeGradient:120}, and the details can be found in the book.
Unlike the conventional approach, we are able to find the field directly without first having to determine the retarded time potentials, nor having to take their derivatives.

\subsection{Energy and momentum.}
The energy and momentum section of the book discusses field energy density, the Poynting vector, Maxwell stress tensor, and more generally, the energy momentum tensor.
The results in the conventional and geometric algebra formalism are detailed, showing how the two relate.
\input{Definition_energy_momentum_poynting.tex}
In geometric algebra the energy momentum tensor, and the Maxwell stress tensor may be represented as linear grade \((0,1)\)-multivector valued functions of a grade \((0,1)\)-multivector, as follows.
\input{Definition_energy_momentum_and_maxwell_stress.tex}

\input{Theorem_energy_momentum_tensor_expansion.tex}

\Cref{thm:poyntingF:1240} relates the geometric algebra definition of the energy momentum tensor to the quantities found in the conventional
electromagnetism literature.
In the book, the conventional indexed representation is detailed more completely for comparison purposes.

Associated with the energy momentum tensor are a number of conservation relationships, which are most compactly stated utilizing the adjoint of the energy momentum tensor.
\input{Definition_adjoint.tex}
\input{Theorem_poynting_differential.tex}
or in an integral form
%
% Copyright � 2018 Peeter Joot.  All Rights Reserved.
% Licenced as described in the file LICENSE under the root directory of this GIT repository.
%
\maketheorem{Poynting's theorem (integral form.)}{thm:poyntingTheoremRewrite:1420}{
%\begin{equation}\label{eqn:poyntingTheoremRewrite:1400}
\begin{equation*}
\begin{aligned}
&\PD{t}{}
\int_V dV\, \calE
=
-\int_{\partial V} dA\, \ncap \cdot \BS
-
\int_V dV \lr{
   \BJ \cdot \BE
   +
   \BM \cdot \BH
} \\
&
\int_V dV \lr{ \rho \BE + \BJ \cross \BB } \\
&\qquad + \int_V dV \lr{ \rho_\txtm \BH - \epsilon \BM \cross \BE }
=
-
\PD{t}{ }
\int_V dV\, \bcP
+
\int_{\partial V} dA\, \BT(\ncap).
\end{aligned}
\end{equation*}
%\end{equation}
} % theorem


As the field in the volume is carrying the (electromagnetic) momentum \( \Bp_{\textrm{em}} = \int_V dV\, \bcP \), we can identify the sum of the Maxwell stress tensor's normal component over the bounding integral as time rate of change of the mechanical and electromagnetic momentum
\begin{equation}\label{eqn:ece2500report:2560}
\frac{d}{dt} \Bp_{\textrm{mech}} + \frac{d}{dt} \Bp_{\textrm{em}} = \int_{\partial V} dA \BT(\ncap).
\end{equation}
The rate of change of mechanical momentum density \( \ifrac{d\Bp_{\textrm{mech}}}{dt} \) is the continuous equivalent of the Lorentz force.
%, which is found to be a direct consequence of conservation relationships associated with Maxwell's equation.

Please refer to the book for additional details.

\subsubsection{Example energy momentum calculations.}
To illustrate the ideas above, the energy momentum tensor components for all of the static fields computed previously are determined.
For brevity, these are omitted from this report, but it should be noted that we see that geometric algebra allows for a particularly compact coordinate free representation of the energy momentum tensor components.
\subsubsection{Complex power.}
The geometric algebra forms of the \( T(1) \) (field energy density and Poynting vector) are found to be
\input{Theorem_complex_power.tex}

\subsection{Lorentz force.}
The Lorentz force equation can be stated in terms of the total electromagnetic field strength and current density
\input{Theorem_lorentz_force_power_report.tex}

\subsubsection{Constant magnetic field.}
As another example of geometric algebra in action, the Lorentz force equation for a constant external magnetic field bivector \( F = I c \BB \)
\begin{dmath}\label{eqn:lorentzForce_constantMagnetic:60}
m \frac{d\Bv}{dt} = q F \cdot \frac{\Bv}{c},
\end{dmath}
is solved in a fashion unique to this algebra.
With
\( \Omega = -\ifrac{q F}{m c} \), the Lorentz force equation is reduced to \( \ifrac{d\Bv}{dt} = \Bv \cdot \Omega \), which may be solved
using a multivector integration factor.
The solution is shown to be
\begin{dmath}\label{eqn:lorentzForce_constantMagnetic:200}
\Bv(t) = e^{-\Omega t/2} \Bv(0) e^{\Omega t/2}.
\end{dmath}
Any component of the initial velocity \( \Bv(0)_\perp \) perpendicular to the \( \Omega \) plane is untouched by this rotation operation, whereas components of the initial velocity \( \Bv(0)_\parallel \) that lie in the \( \Omega \) plane will trace out a circular path, so the velocity of the charged particle traces out a helical path.

More general examples are considered in the literature cited in the book.

\subsection{Polarization}
Following the usual convention, the geometric algebra treatment of polarization in the book
aligns the propagation direction along the z-axis.
The field is
\begin{dmath}\label{eqn:polarization:20}
\begin{aligned}
F(\Bx, \omega) &= (1 + \Be_3) \BE e^{-j \beta z} \\
F(\Bx, t) &= \Real\lr{ F(\Bx, \omega) e^{j \omega t} },
\end{aligned}
\end{dmath}
where \( \BE \cdot \Be_3 = 0 \).
Here the imaginary \( j \) has no intrinsic geometrical interpretation, but we are able to dispense with it and use geometric imaginaries instead.
This is done by first assuming the electric field is given by \( \BE = \lr{ \alpha_1 + j \beta_1 } \Be_1 + \lr{ \alpha_2 + j \beta_2 } \Be_2 \), so that the
time domain representation of the field is given by
\begin{dmath}\label{eqn:polarization_circular:160}
F(\Bx, t) = (1 + \Be_3) \lr{
\lr{ \alpha_1 \Be_1 + \alpha_2 \Be_2 } \cos\lr{ \omega t - \beta z }
-\lr{ \beta_1 \Be_1 + \beta_2 \Be_2 } \sin\lr{ \omega t - \beta z }
}.
\end{dmath}

Two geometric representations are possible.
The first uses the pseudoscalar for the transverse plane \( \Be_{12} \), denoted \( i \) here, and the other uses the \R{3} pseudoscalar as the imaginary.
\subsubsection{Transverse plane imaginary.}
\input{Theorem_circular_polarization_coeff_report.tex}

\begin{itemize}
\item
Linear polarization at an angle \( \psi\) from the x-axis in the transverse plane is given by
\( \alpha_\txtR = \inv{2}\Norm{\BE} e^{i(\psi + \theta)},
\alpha_\txtL = \inv{2}\Norm{\BE} e^{i(\psi - \theta)} \), for which the field is \( F = \lr{ 1 + \Be_3 } \Norm{\BE} \Be_1 e^{i \psi} \cos(\phi + \theta) \),
where \( \theta \) is an initial phase angle.
\item
Following the IEEE antenna convention, we define right(left) circular polarization as the
a change in phase that
results in the electric field tracing out a (clockwise,counterclockwise) circle
\begin{dmath}\label{eqn:polarization_circular:180}
\begin{aligned}
\BE_\txtR &= \Norm{\BE} \lr{ \Be_1 \cos\phi + \Be_2 \sin\phi } = \Norm{\BE} \Be_1 \exp\lr{  \Be_{12} \phi } \\
\BE_\txtL &= \Norm{\BE} \lr{ \Be_1 \cos\phi - \Be_2 \sin\phi } = \Norm{\BE} \Be_1 \exp\lr{ -\Be_{12} \phi }.
\end{aligned}
\end{dmath}
Right and left circular polarization in this representation are given by
\(\alpha_\txtR = \Norm{\BE}, \alpha_\txtL = 0 \) and \(\alpha_\txtL = \Norm{\BE}, \alpha_\txtR = 0 \) respectively.
The right(left) polarized fields are just
\( F = (1 + \Be_3) \Norm{\BE} \Be_1 e^{\pm i(\omega t - k z)} \).
\item An elliptically polarized field is given by
\( \alpha_\txtR = \inv{2}\lr{ E_a - E_b },
\alpha_\txtL = \inv{2}\lr{ E_a + E_b } \), or
\begin{dmath}\label{eqn:ece2500report:2580}
F = \inv{2} (1 + \Be_3) \Be_1 \lr{ (E_a + E_b) e^{i\phi} + (E_a - E_b) e^{-i\phi} }
\end{dmath}
A hyperbolic parameterization of the elliptically polarized wave is also discussed in the book
\begin{dmath}\label{eqn:polarization_elliptical:380}
\begin{aligned}
F &= e E_a \lr{ 1 + \Be_3 } \Be_1 e^{ i \psi } \cosh\lr{ m + i \phi} \\
m &= \tanh^{-1}\lr{ E_b/E_a } \\
e &= \sqrt{1 - {(E_b/E_a)}^2 },
\end{aligned}
\end{dmath}
where \( E_a(E_b) \) are the magnitudes of the electric field components lying along the semi-major axis directed along \(
\begin{bmatrix}
\Be_1 \\
\Be_2
\end{bmatrix}
e^{i\psi} \) respectively.
Additional discussion and diagrams can be found in the book.
\end{itemize}

Each polarization considered above (linear, circular, elliptical) have the same general form
\begin{dmath}\label{eqn:polarizationRewrite:760}
F = \lr{ 1 + \Be_3 } \Be_1 e^{i\psi} f(\phi),
\end{dmath}
where \( f(\phi) \) is a complex valued function (i.e. grade 0,2).
The structure of \cref{eqn:polarizationRewrite:760} could be more general than considered so far.
For example, a Gaussian modulation could be added into the mix with \( f(\phi) = e^{i \phi - (\phi/\sigma)^2/2 } \).
The simple complex structure that encodes all the phase dependence allows the
energy, momentum and Maxwell stress tensor to be computed easily.
\input{Theorem_energy_momentum_tensor_plane_wave.tex}

Only the propagation direction of a plane wave, regardless of its polarization (or even whether or not there are Gaussian or other damping factors), carries any energy or momentum, and only the propagation direction component of the Maxwell stress tensor \( \BT(\Ba) \) is non-zero.

Using \cref{thm:polarizationRewrite:780} the energy momentum vector may be computed for each of the polarizations considered above.
\begin{itemize}
\item
For the linearly polarization \( T(1) = \ifrac{\epsilon}{2} \lr{ 1 + \Be_3 } \Norm{\BE}^2 \cos^2( \phi + \theta ) \).
\item For the circularly polarization \( T(1) = \ifrac{\epsilon}{2} (1 + \Be_3) \Norm{\BE}^2 \).
A circularly polarized wave carries maximum energy and momentum, whereas the energy and momentum of a linearly polarized wave
oscillates with the phase angle.
\item For the elliptical polarization \(
T(1)
= \ifrac{\epsilon}{2} \lr{ 1 + \Be_3 } e^2 \lr{ E_b^2 + 2 \lr{ E_a^2 - E_b^2} \cos^2 \phi } \).
As expected, the phase dependent portion of the energy momentum tensor vanishes as the wave function approaches circular polarization.
\end{itemize}

\subsubsection{Pseudoscalar imaginary.}
Alternatively, it is possible to encode the sines and cosines in the time domain representation of the field in terms of the \R{3} pseudoscalar.
\input{Theorem_circular_polarization_coeff_pseudoscalar_report.tex}

There appear to be some advantages to pseudoscalar description of polarization, especially for computing energy momentum tensor components since \( I \) commutes with all grades.
For example, we can see practically by inspection that
\begin{equation}\label{eqn:polarization_pseudoscalarImaginary:620}
T(1) = \calE + \frac{\BS}{v} =
\epsilon \lr{ 1 + \Be_3 } \lr{ \Abs{\alpha_\txtR}^2 + \Abs{\alpha_\txtL}^2 },
\end{equation}
where the absolute value is computed using the reverse as the conjugation operation \( \Abs{z}^2 = z z^\dagger \).

\subsection{Transverse fields in a waveguide.}
One topic from waveguide theory is considered in the book.
\input{Theorem_transverse_and_propagation_solutions_report.tex}

Details and proof are in the book.
This and many other examples in the book show that we pay a significant additional cost to work with separate electric and magnetic field components compared to working with the complete electromagnetic field multivector \( F \) in its entirety.

\subsection{Multivector potential.}
Conventional electromagnetism utilizes scalar and vector potentials, which may be generalized to a multivector potential containing their sums.
\input{multivector_potential}

The grades of the multivector potentials may be chosen to match SI conventions (as specified in Balanis' ``Antenna Theory'',
which includes fictitious magnetic sources),
\input{Theorem_fields_and_potential_wave_equations_report.tex}

Also detailed in the book is the multivector formulation of gauge transformation that allows the grade selection operation in
\cref{thm:generalPotential:80} to be removed.

\input{Theorem_gauge_invariance.tex}

We say that we are working in the Lorenz gauge, if the 0,3 grades of \( \conjstgradi A \) are zero, or a transformation that kills those grades is made.
\input{Theorem_lorentz_gauge_transformation.tex}

Please see the book for proofs and additional details, including the explicit integral solution of \( \Psi \) required for the transformation to the Lorentz gauge.

\subsection{Far field.}
The geometric algebra form for the far field associated with a vector (or dual-vector) potential has a fairly simple coordinate free form.
\input{Theorem_far_field_magnetic_vector_potential_report.tex}
Noting that \( \rcap (\rcap \wedge \BA) \) is the rejection of the radial component of \( \BA \) (i.e. is a vector not a multivector) allows the far-field solution to easily be split into electric and magnetic field components if desired.
These are detailed in the book, along with the proof of \cref{thm:potentialSection_farfield:1}.
Also included in the book is an example calculation of \( F = -j \omega \lr{ 1 + \rcap } \lr{ \rcap \wedge \BA} \) for the dipole vector potential.

\subsection{Dielectric and magnetic media.}
The majority of the electromagnetic theory covered in the book focused on fields with the
isotropic constitutive relationships \cref{eqn:freespace:300}.
For more general constitutive relationships the geometric algebra form of Maxwell's equations requires a pair of
multivector equations, fields and sources as follows.
\input{Theorem_maxwells_equations_in_media.tex}

Along with some discussion of solution of these more complicated equations, gauge-like transformations of the fields are discussed.
More work is required to fully flesh out this topic, especially given that \( F, G \) may be directly coupled, allowing neither field to be solved for independently.

\subsection{Boundary value conditions.}
Independent of the techniques used to find the multivector fields \( F, G \) for electromagnetism in matter, we may use \cref{thm:dielectric:20} to easily derive the boundary
value conditions for the fields spanning a surface with surface currents or charges, or even just a discontinuity in the media.
%\newpage
\input{Theorem_boundary_value_relations.tex}

A simple proof is possible by integrating the pair of Maxwell's equations over the pillbox configuration, allowing the height \( n \) of that pillbox above or below the surface to tend to zero,
and the area of the pillbox top to also tend to zero, using \cref{thm:volumeintegral:100} to transform the multivector integrals to boundary integrals.
Figures and the full proof are available in the book.

Note that in the special case where there are surface charge and current densities along the interface surface, but the media is uniform (\(\epsilon_1 = \epsilon_2, \mu_1 = \mu_2\)), then the field and current relationship has a particularly simple form% \citep{chappell2014geometric}
\begin{dmath}\label{eqn:boundarySurfaceSources:421}
\ncap (F_2 - F_1) = J_s.
\end{dmath}

\section{Conclusions.}

The multivector representation of Maxwell's equation has an enticing and striking compactness and simplicity.
However, a student, engineer, or physicist interested in application of geometric algebra to electromagnetism faces a number of obstacles.
The first is learning geometric algebra itself.  This is not insurmountable, as there are now many resources available.
A second hurdle is multivector calculus, for which much of the literature is opaque and inaccessible.
Another obstacle, mentioned earlier, is the relativistic bias in much of the literature, and high levels of presumed physics and mathematical softistication.
Additionally, while the multivector form of Maxwell's equation occurs frequently in the geometric algebra literature,
many such references stop after showing that the conventional Maxwell's equations can be obtained from the multivector representation, and do dive into no or few additional details.
Encountering the multivector representation of Maxwell's equation can leave the student with more questions than answers.

The book produced on this project attempts to systematically collect all the prerequisite material required for a study of electromagnetism using geometric algebra.  This includes
\begin{itemize}
\item
Questions of how Stokes', Green's and the divergence theorem generalize to multivector integrands are answered by showing how these theorems are special cases of a more general integration theory.
\item
The multivector (time and frequency domain) Green's functions that can be used to invert Maxwell's equation are both derived and applied to this inversion task.
\item
A number of demonstrations are provided that illustrate how geometric algebra techniques can be applied to electromagnetic problems.
\item
The (multivector) wave equation form of Maxwell's equation is introduced and multivector plane wave solutions for the field are found.
\item
A multivector form of the energy momentum tensor is defined and related to the conventional form.  The Poynting conservation laws are derived, including both the conventional Poynting theorem, and the continuum equivalent of the Lorentz force law.
\item
Polarization is described with two (multivector) complex exponential representations, which are both related to the conventional polarization formalism by multivector representations of the Jones vector.
\item
Waveguide relationships between transverse and propagation-direction components of the field are obtained.
\item
Multivector potentials and gauge transformations are formulated and derived.
\item
The multivector form of Maxwell's equations in matter are touched on briefly, and used to derive the boundary value conditions given surface current and charge densities.
\end{itemize}

All this is really only a starting point.
Many more worked examples, problems, figures and computer algebra listings should be added.
Problems customarily tackled with separate electric and magnetic field equations should also be incorporated, showing how geometric algebra techniques could be used to solve those problems.
There are a number of topics covered in introductory electromagnetism texts that are missing.
Examples include the Fresnel relationships for transmission and reflection at an interface,
in depth treatment of waveguides,
dipole radiation,
motion of charged particles,
bound charges,
meta materials,
Lorentz transformations and relativity,
and more.

Despite limitations like those mentioned above, there is a great deal of information assembled in the book.
The book shows how the familiar identities and relationships of electromagnetism can be obtained from the geometric algebra formalism, and how to work with the geometric algebra formalism directly.
The student is provided with a powerful theoretical framework, without sacrificing the ability to use any familiar conventional tools when desired.

It is the belief of the author that systematically working through a larger set of topics in electromagnetism using geometric algebra would provide electrical engineers and physics practitioners many powerful new tools and procedures.  Hopefully, this book is a good first step in that direction.

\EndNoBibArticle
