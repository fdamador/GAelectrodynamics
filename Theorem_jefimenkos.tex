%
% Copyright � 2018 Peeter Joot.  All Rights Reserved.
% Licenced as described in the file LICENSE under the root directory of this GIT repository.
%
\maketheorem{Jefimenkos solution.}{thm:jefimenkosEquations:120}{
The solution of Maxwell's equation is given by
\begin{equation*}
F(\Bx, t)
=
F_0(\Bx, t)
+
\inv{4 \pi}
\int dV'
\lr{
   \frac{\rcap}{r^2} J(\Bx', t_r)
   +
   \inv{c r} \lr{ 1 + \rcap } \dispdot{J}(\Bx', t_r)
},
\end{equation*}
where \( F_0(\Bx, t) \) is any specific solution of the homogeneous equation \( \lr{ \spacegrad + (1/c) \partial_t } F_0 = 0 \),
time derivatives are denoted by overdots, and all times are evaluated at the retarded time \( t_r = t - r/c \).
When expanded in terms of the electric and magnetic fields (ignoring magnetic sources), the non-homogeneous portion of this solution is known as
Jefimenkos' equations \citep{griffiths1999introduction}.
\begin{equation}\label{eqn:jefimenkosEquations:100}
\begin{aligned}
\BE &=
\inv{4 \pi}
\int dV'
\lr{
\frac{\rcap}{\epsilon r} \lr{
\frac{\rho(\Bx', t_r)}{r} + \frac{\dispdot{\rho}(\Bx', t_r) }{c} }
   - \frac{\eta }{ c r } \dotBJ(\Bx', t_r)
} \\
\BH &=
\inv{4 \pi}
\int dV'
\lr{
   \frac{1}{c r} \dotBJ(\Bx', t_r)
+
   \frac{1}{r^2} \BJ(\Bx', t_r)
} \cross \rcap,
\end{aligned}
\end{equation}
%which checks against Griffiths.
} % theorem
