%
% Copyright � 2018 Peeter Joot.  All Rights Reserved.
% Licenced as described in the file LICENSE under the root directory of this GIT repository.
%
%{
%\input{../latex/blogpost.tex}
%\renewcommand{\basename}{helmholtzGreens}
%%\renewcommand{\dirname}{notes/phy1520/}
%\renewcommand{\dirname}{notes/ece1228-electromagnetic-theory/}
%%\newcommand{\dateintitle}{}
%%\newcommand{\keywords}{}
%
%\input{../latex/peeter_prologue_print2.tex}
%
%\usepackage{peeters_layout_exercise}
%\usepackage{peeters_braket}
%\usepackage{peeters_figures}
%\usepackage{siunitx}
%%\usepackage{mhchem} % \ce{}
%%\usepackage{macros_bm} % \bcM
%%\usepackage{macros_qed} % \qedmarker
%%\usepackage{txfonts} % \ointclockwise
%
%\beginArtNoToc
%
%\generatetitle{Helmholtz Green's function}
%%\chapter{Helmholtz Green's function}
%%\label{chap:helmholtzGreens}
%% \citep{sakurai2014modern} pr X.Y
%% \citep{pozar2009microwave}
%% \citep{qftLectureNotes}
%% \citep{doran2003gap}
%% \citep{jackson1975cew}
%% \citep{griffiths1999introduction}
%
\paragraph{The goal.}
\label{chap:helmholtzGreens}

The Helmholtz equation to solve is
\begin{equation}\label{eqn:helmholtzGreens:20}
\lr{ \spacegrad^2 + k^2 } f(\Bx) = u(\Bx).
\end{equation}

To solve using the
Green's function of \cref{thm:gradientGreensFunctionEuclidean:3}, we require
\begin{equation}\label{eqn:helmholtzGreens:80}
\lr{ \spacegrad^2 + k^2 } G(\Bx, \Bx') = \delta^3( \Bx - \Bx' ).
\end{equation}

Verifying this requires two steps, first considering points \( \Bx \ne \Bx' \), and then considering an infinitesimal neighborhood around \( \Bx' \).

\paragraph{Case I.  \( \Bx \ne \Bx' \).}

We will absorb the sign associated with the causal and acausal Green's function variations by writing \( i = \pm j \), so that
for points \( \Bx \ne \Bx' \), (i.e. \( r = \Norm{\Bx - \Bx'} \ne 0 \)), working in spherical coordinates, we find
\begin{equation}\label{eqn:helmholtzGreens:100}
\begin{aligned}
- 4 \pi \lr{ \spacegrad^2 + k^2 } G(\Bx, \Bx')
&= \inv{r^2} \lr{ r^2 G' }' - 4 \pi k^2 G \\
&= \inv{r^2} \frac{d}{dr} \lr{ r^2 \lr{ \frac{i k r}{r} - \inv{r^2} } e^{ i k r } } + \frac{k^2}{r} e^{i k r} \\
&= \inv{r^2} \frac{d}{dr} \lr{ \lr{ r i k - 1 } e^{ i k r }} + \frac{k^2}{r} e^{i k r} \\
&= \inv{r^2} \lr{ \cancel{ i k}  + \lr{ r i k - \cancel{ 1 } } i k } e^{ i k r} + \frac{k^2}{r} e^{i k r} \\
&= \inv{r^2} \lr{ - r k^2 } e^{ i k r} + k^2 \frac{e^{i k r}}{r} \\
&= 0.
\end{aligned}
\end{equation}

\paragraph{Case II.  In the neighborhood of \(\Norm{\Bx - \Bx'} < \epsilon\)}

Having shown that we end up with zero everywhere that \(\Bx \ne \Bx'\) we are left to consider an infinitesimal neighborhood of the volume surrounding the point \(\Bx\) in our integral.  Following the Coulomb treatment in \S 2.2 of \citep{schwartz1987pe} we use a spherical volume element centered around \(\Bx\) of radius \(\epsilon\), and then convert a divergence to a surface area to evaluate the integral away from the problematic point.
\begin{equation}\label{eqn:helmholtzGreens:50}
\begin{aligned}
\int &\left(\spacegrad^2 + k^2\right) G(\Bx, \Bx') f(\Bx') dV' \\
&=
-\inv{4\pi} \int_{\Norm{\Bx - \Bx'} < \epsilon} \left(\spacegrad^2 + k^2\right) \frac{e^{i k \Norm{\Bx - \Bx'}}}{\Norm{\Bx - \Bx'}} f(\Bx') dV' \\
&=
-\frac{1}{4\pi} \int_{\Norm{\Bx''} < \epsilon}
f(\Bx + \Bx'')
\left( \spacegrad^2 + k^2 \right) \frac{e^{i k \Norm{\Bx''}}}{\Norm{\Bx''}}
dV'',
\end{aligned}
\end{equation}
where a change of variables \(\Bx'' = \Bx' - \Bx \), as illustrated in
\cref{fig:neighbourhood:neighbourhoodFig1},
 has been performed.
\pmathImageFigure{../figures/GAelectrodynamics/\subfigdir/}{neighbourhoodFig1}{Neighborhood \( \Norm{\Bx - \Bx'} < \epsilon \).}{fig:neighbourhood:neighbourhoodFig1}{0.5}{neighbourhoodFig1.nb}

We
assume that \( f(\Bx) \) is sufficiently continuous and ``well behaved'' that it can be
pulled it out of the integral, replaced with a mean value \( f(\Bx^\conj) \) in the integration neighborhood around \( \Bx'' = 0 \).
\begin{equation}\label{eqn:helmholtzGreens:430}
\int \left(\spacegrad^2 + k^2\right) G(\Bx, \Bx') f(\Bx') dV'
=
\lim_{\epsilon \rightarrow 0}
-\frac{f(\Bx^\conj)}{4\pi} \int_{\Norm{\Bx''} < \epsilon} \left( \spacegrad^2 + k^2 \right) \frac{e^{i k \Norm{\Bx''}}}{\Norm{\Bx''}} dV''.
\end{equation}

The \( k^2 \) term of \cref{eqn:helmholtzGreens:430} can be evaluated with a
spherical coordinate change of variables
\begin{equation}\label{eqn:helmholtzGreens:470}
\begin{aligned}
\int_{\Norm{\Bx''} < \epsilon}
k^2 \frac{e^{i k \Norm{\Bx''}}}{\Norm{\Bx''}} dV''
&=
\int_{r = 0}^\epsilon \int_{\theta = 0}^\pi \int_{\phi=0}^{2\pi}
k^2 \frac{e^{i k r}}{r} r^2 dr \sin\theta d\theta d\phi \\
&=
4\pi k^2
\int_{r = 0}^\epsilon
r e^{i k r} dr  \\
&=
4\pi
\int_{u = 0}^{k\epsilon}
u e^{i u} du  \\
&=
4\pi
{\left.
(-i u + 1) e^{i u} \right\vert}_0^{k \epsilon} \\
&=
4 \pi \left( (-i k \epsilon + 1)e^{i k \epsilon} - 1 \right).
\end{aligned}
\end{equation}

To evaluate the Laplacian term of \cref{eqn:helmholtzGreens:430}, we can make a change of variables for the Laplacian
\begin{equation}\label{eqn:helmholtzGreens:310}
\spacegrad \frac{e^{i k \Norm{\Bx''}}}{\Norm{\Bx''}}
=
\spacegrad_{\Bx''}^2 \frac{e^{i k \Norm{\Bx''}}}{\Norm{\Bx''}}
=
\spacegrad_{\Bx''} \cdot \left(\spacegrad_{\Bx''} \frac{e^{i k \Norm{\Bx''}}}{\Norm{\Bx''}} \right),
\end{equation}
and then employ the divergence theorem
\begin{equation}\label{eqn:helmholtzGreens:450}
\begin{aligned}
\int_{\Norm{\Bx''} < \epsilon} \spacegrad^2 \frac{e^{i k \Norm{\Bx''}}}{\Norm{\Bx''}} dV''
&= \int_{\Norm{\Bx''} < \epsilon} \spacegrad_{\Bx''} \cdot \left( \spacegrad_{\Bx''} \frac{e^{i k \Norm{\Bx''}}}{\Norm{\Bx''}} \right) dV'' \\
&= \int_{\partial V} \left( \spacegrad_{\Bx''} \frac{e^{i k \Norm{\Bx''}}}{\Norm{\Bx''}} \right) \cdot \ncap dA'',
\end{aligned}
\end{equation}
where \( \partial V \) represents the surface of the \( \Norm{\Bx''} < \epsilon \) neighborhood, and \( \ncap \) is the unit vector directed along \( \Bx'' = \Bx' - \Bx \).
To evaluate this surface integral we will require only the radial portion of the gradient.  With \( r = \Norm{\Bx''} \), that is
\begin{equation}\label{eqn:helmholtzGreens:490}
\begin{aligned}
\left( \spacegrad_{\Bx''} \frac{e^{i k \Norm{\Bx''}}}{\Norm{\Bx''}} \right) \cdot \ncap
&=
\left( \ncap \PD{r}{} \frac{e^{i k r}}{r} \right) \cdot \ncap \\
&=
\PD{r}{} \frac{e^{i k r}}{r} \\
&=
\left( i k \inv{r} - \inv{r^2} \right)
e^{i k r} \\
&=
\left( i k r - 1 \right)
\frac{e^{i k r}}{r^2}.
\end{aligned}
\end{equation}

Using a spherical area element \(dA'' = r^2 \sin\theta d\theta d\phi\), we obtain
\begin{equation}\label{eqn:helmholtzGreens:530}
\begin{aligned}
\int_{\Norm{\Bx''} < \epsilon} &\spacegrad^2 \frac{e^{i k \Norm{\Bx''}}}{\Norm{\Bx''}} dV'' \\
%=
%\int_{\partial V} \left( \spacegrad_{\Bx''} \frac{e^{i k \Norm{\Bx''}}}{\Norm{\Bx''}} \right) \cdot \ncap dA'',
&=
\evalbar{
\int_{\theta = 0}^\pi \int_{\phi=0}^{2\pi}
\left( i k r - 1 \right)
\frac{e^{i k r}}{r^2}
r^2 \sin\theta d\theta d\phi
}{r = \epsilon} \\
&=
4 \pi
\left( i k \epsilon - 1 \right) e^{i k \epsilon}.
\end{aligned}
\end{equation}

Putting everything back together we have
\begin{equation}\label{eqn:helmholtzGreens:510}
\begin{aligned}
-\inv{4\pi} \int &\left(\spacegrad^2 + k^2\right) \frac{e^{i k \Norm{\Bx - \Bx'}}}{\Norm{\Bx - \Bx'}} f(\Bx') dV' \\
&=
\lim_{\epsilon \rightarrow 0}
-f(\Bx^\conj)
\left(
(-i k \epsilon + 1)e^{i k \epsilon} - 1
+\left( i k \epsilon - 1 \right) e^{i k \epsilon}
\right) \\
&=
\lim_{\epsilon \rightarrow 0}
-f(\Bx^\conj)
\left(
(-i k \epsilon + 1 + i k \epsilon - 1 )e^{i k \epsilon} - 1
\right) \\
&=
\lim_{\epsilon \rightarrow 0}
f(\Bx^\conj).
\end{aligned}
\end{equation}

Observe the perfect cancellation of all the explicitly \( \epsilon \) dependent terms.  The mean value point \( \Bx^\conj \) is also \( \epsilon \) dependent, but tends to \( \Bx \) in the limit, leaving
\boxedEquation{eqn:helmholtzGreens:330}{
f(\Bx)
=
-\inv{4\pi} \int \left(\spacegrad^2 + k^2\right) \frac{e^{i k \Norm{\Bx - \Bx'}}}{\Norm{\Bx - \Bx'}} f(\Bx') dV'.
}

This proves the delta function property that we claimed the Green's function had.
%}
%\EndArticle
