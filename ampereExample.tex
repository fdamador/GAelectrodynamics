%
% Copyright � 2018 Peeter Joot.  All Rights Reserved.
% Licenced as described in the file LICENSE under the root directory of this GIT repository.
%
%{
%\input{../latex/blogpost.tex}
%\renewcommand{\basename}{ampereExample}
%%\renewcommand{\dirname}{notes/phy1520/}
%\renewcommand{\dirname}{notes/ece1228-electromagnetic-theory/}
%%\newcommand{\dateintitle}{}
%%\newcommand{\keywords}{}
%
%\input{../latex/peeter_prologue_print2.tex}
%
%\usepackage{peeters_layout_exercise}
%\usepackage{peeters_braket}
%\usepackage{peeters_figures}
%\usepackage{siunitx}
%%\usepackage{mhchem} % \ce{}
%%\usepackage{macros_bm} % \bcM
%%\usepackage{macros_qed} % \qedmarker
%\usepackage{txfonts} % \ointclockwise
%
%\beginArtNoToc
%
%\generatetitle{Ampere's law example.  Two currents.}
%%\chapter{Ampere's law example}
\label{chap:ampereExample}

Let's try using Ampere's law as stated in \cref{thm:amperes:280} two compute the field at a point in the blue region
illustrated in
\cref{fig:amperesLawBetweenTwoCurrents:amperesLawBetweenTwoCurrentsFig1}.
This represents a pair of z-axis electric currents of magnitude \( I_1, I_2 \) flowing through the \( z = 0 \) points \( \Bp_1, \Bp_2 \) on the x-y plane.
\pmathImageFigure{../figures/GAelectrodynamics/}{amperesLawBetweenTwoCurrentsFig1}{Magnetic field between two current sources.}{fig:amperesLawBetweenTwoCurrents:amperesLawBetweenTwoCurrentsFig1}{0.3}{amperesLawMultiplePoints.nb}

Solving the system with superposition, let's consider first one source flowing through \( \Bp = (p_x, p_y, 0) \) with current \( \BJ = \Be_3 I_\txte \delta( x - p_x) \delta( y - p_y ) \), and evaluate the field due to this source at the point \( \Br \).
With
only magnetic sources in the multivector current, Ampere's law takes the form
\begin{equation}\label{eqn:ampereExample:20}
\ointctrclockwise_{\partial A} d\Bx\, F = -I \int_A dA\, \Be_3 (-\eta \BJ) = I \eta I_\txte.
\end{equation}
The field \( F \) must be a bivector satisfying \( d\Bx \cdot F = 0 \).  The circle is parameterized by
\begin{equation}\label{eqn:ampereExample:40}
\Br = \Bp + R \Be_1 e^{i\phi},
\end{equation}
so
\begin{equation}\label{eqn:ampereExample:60}
d\Bx = R \Be_2 e^{i\phi} d\phi = R \phicap d\phi.
\end{equation}
With the line element having only a \( \phicap \) component, \( F \) must be a bivector proportional to \( \Be_3 \rcap \).
Let \( F = F_0 \Be_{31} e^{i\phi} \), where \( F_0 \) is a scalar, so that \( d\Br F \) is a constant multiple of the unit pseudoscalar
\begin{equation}\label{eqn:ampereExample:80}
\begin{aligned}
\int_0^{2\pi} d\Br F
&= R F_0 \int_0^{2\pi} \Be_2 e^{i\phi} d\phi \Be_{31} e^{i\phi} \\
&= R F_0 \int_0^{2\pi} \Be_{231} e^{-i\phi} e^{i\phi} d\phi \\
&= 2 \pi I R F_0,
\end{aligned}
\end{equation}
so
\begin{equation}\label{eqn:ampereExample:100}
\begin{aligned}
F_0
&= \inv{I 2 \pi R} I I_\txte \\
&= \frac{I_\txte}{2 \pi R}.
\end{aligned}
\end{equation}
The field strength relative to the point \( \Bp \) is
\begin{equation}\label{eqn:ampereExample:120}
\begin{aligned}
F
&= \frac{\eta I_\txte}{2 \pi R} \Be_3 \rcap \\
&= \frac{\eta I_\txte}{2 \pi R} \Be_3 \rcap.
\end{aligned}
\end{equation}

Switching to an origin relative coordinate system, removing the \( z = 0 \) restriction for \( \Br \) and \( \Bp_k \), and summing over both currents, the total field at any point \( \Br \) strictly between the currents is
\begin{equation}\label{eqn:ampereExample:140}
\begin{aligned}
F
&= \sum_{k = 1,2} \frac{\eta I_k}{2 \pi} \Be_3 \inv{\Be_3 \lr{ \Be_3 \wedge \lr{ \Br - \Bp_k}}} \\
&= \sum_{k = 1,2} \frac{\eta I_k}{2 \pi} \inv{ \Be_3 \wedge \lr{ \Br - \Bp_k} }.
\end{aligned}
\end{equation}
The bivector nature of a field with only electric current density sources is naturally represented by the wedge product \( \Be_3 \wedge \lr{ \Br - \Bp_k} \) which is a vector product of \( \Be_3 \) and the projection of \( \Br - \Bp_k \) onto the x-y plane.
%}
%\EndNoBibArticle
