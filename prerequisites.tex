%
% Copyright © 2017 Peeter Joot.  All Rights Reserved.
% Licenced as described in the file LICENSE under the root directory of this GIT repository.
%
Geometric algebra (GA for short), generalizes and extends vector algebra.
The following section contains a lightning review of some
foundational concepts, including
scalar, vector, vector space, basis, orthonormality, and metric.
%If you are inclined to skip this, please at least examine the
%stated dot product definition, since the conventional positive definite property is not assumed.

\subsection{Vector.}
A vector is a directed line segment, with a length, direction, and an orientation.  A number of different representations of vectors are possible.

\paragraph{Graphical representation.}
A vector may be represented graphically as an arrow, with the head indicating the direction of the vector.
Multiplication of vectors by positive numbers changes the length of the vector, whereas multiplication by negative numbers changes the direction of the vector and the length, as illustrated in
\cref{fig:VectorsWithOppositeOrientation:VectorsWithOppositeOrientationFig1}.
Addition of vectors is performed by connecting the arrows heads to tails as illustrated in
\cref{fig:vectorAddition:vectorAdditionFig1}.
\index{\(\bbR\)}
In this book a scalar is a number, usually real, but occasionally complex valued.  The set of real numbers will be designated \(\bbR\).
\pmathImageFigure{../figures/GAelectrodynamics/\subfigdir/}{VectorsWithOppositeOrientationFig1}{Scalar multiples of vectors.}{fig:VectorsWithOppositeOrientation:VectorsWithOppositeOrientationFig1}{0.15}{vectorOrientationAndAdditionFigures.nb}
\pmathImageFigure{../figures/GAelectrodynamics/\subfigdir/}{vectorAdditionFig1}{Addition of vectors.}{fig:vectorAddition:vectorAdditionFig1}{0.3}{vectorOrientationAndAdditionFigures.nb}

\paragraph{Coordinate representation.}
The length and orientation of a vector, relative to a chosen fixed point (the origin) may be specified algebraically as the coordinates of the head of the vector, as
illustrated in \cref{fig:coordinateRepresentation:coordinateRepresentationFig1}.
\pmathImageFigure{../figures/GAelectrodynamics/}{coordinateRepresentationFig1}{Coordinate representation of vectors.}{fig:coordinateRepresentation:coordinateRepresentationFig1}{0.3}{vectorOrientationAndAdditionFigures.nb}

Two dimensional vectors may be represented as pairs of coordinates
\( \lr{x,y}\), three dimensional vectors as triples of coordinates \(\lr{x,y,z}\), and more generally, \( N \) dimensional vectors may be represented as coordinate tuples \(\lr{x_1,x_2,\cdots, x_N}\).
Given two vectors, say \( \Bx = \lr{x,y}, \By = \lr{a,b} \), the sum of the vectors is just the sum of the coordinates \( \Bx + \By = \lr{ x + a, y + b} \).
Numeric multiplication of a vector rescales each of the coordinates, for example with \( \Bx = \lr{x,y,z} \), \( \alpha \Bx = \lr{\alpha x, \alpha y, \alpha z} \).

It is often convienient to assemble such lists of coordinates in matrix form as rows or columns, providing a few equivalent vector representations as
shown in table \ref{tab:prereq:equivalent}.
\begin{tablelabelbox}[tabularx={X|X|X}]{Equivalent vector coordinate representations.}{label=tab:prereq:equivalent}
Tuple & Row & Column \\ \hline
\(\lr{x_1,x_2,\cdots, x_N}\) &
\(
\begin{bmatrix}
x_1 &
x_2 & \hdots &
x_N
\end{bmatrix}
\)
&
\(
\begin{bmatrix}
x_1 \\
x_2 \\
\vdots \\
x_N
\end{bmatrix}
\)
\\ \hline
\end{tablelabelbox}

In this book, the length one (unit) vector in the i'th direction will be given the symbol \( \Be_i \).
For example, in three dimensional space with a column vector representation, the respective unit vectors along each of the \(x\), \(y\), and \(z\) directions are designated
\begin{equation}\label{eqn:prerequisites:20}
\Be_1 =
\begin{bmatrix}
1 \\
0 \\
0 \\
\end{bmatrix},\quad
\Be_2 =
\begin{bmatrix}
0 \\
1 \\
0 \\
\end{bmatrix},\quad
\Be_3 =
\begin{bmatrix}
0 \\
0 \\
1 \\
\end{bmatrix}.
\end{equation}

Such symbolic designation allows any vector to be encoded in a
representation agnostic fashion.  For example a vector \( \Bx \) with coordinates \( x, y, z \) is
\begin{equation}\label{eqn:prerequisites:40}
\Bx = x \Be_1 + y \Be_2 + z \Be_3,
\end{equation}
independent of a tuple, row, column, or any other representation.
