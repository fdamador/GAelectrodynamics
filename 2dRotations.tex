%
% Copyright © 2017 Peeter Joot.  All Rights Reserved.
% Licenced as described in the file LICENSE under the root directory of this GIT repository.
%

\index{rotation}
Plotting \cref{eqn:2dMultiplication:180}, as in
\cref{fig:rotationOfe1:rotationOfe1Fig1},
 shows that multiplication by \( i \) rotates the \R{2} basis vectors by \( \pm \pi/2 \) radians,
with the
rotation direction dependent on the order of multiplication.

\mathImageTwoFigures
{../figures/GAelectrodynamics/rotationOfe1Fig1}
{../figures/GAelectrodynamics/rotationOfe2Fig1}
{Multiplication by \( \Be_1 \Be_2 \).}{fig:rotationOfe1:rotationOfe1Fig1}{scale=0.5}
{orientedAreas.nb}

Multiplying a polar vector representation
\begin{equation}\label{eqn:2dRotations:280}
   \Bx = \rho \lr{ \Be_1 \cos\theta + \Be_2 \sin\theta },
\end{equation}
by \( i \) shows that a \( \pi/2 \) rotation is induced.
\index{pseudoscalar}

Multiplying the vector from the right by \( i \) gives
\begin{equation}\label{eqn:2dRotations:300}
\begin{aligned}
\Bx i
&= \Bx \Be_1 \Be_2 \\
&= \rho \lr{ \Be_1 \cos\theta + \Be_2 \sin\theta } \Be_1 \Be_2 \\
&= \rho \lr{ \Be_2 \cos\theta - \Be_1 \sin\theta },
\end{aligned}
\end{equation}
a counterclockwise rotation of \( \pi/2 \) radians, and
multiplying the vector by \( i \) from the left gives
\begin{equation}\label{eqn:2dRotations:3}
\begin{aligned}
i \Bx
&= \Be_1 \Be_2 \Bx \\
&= \rho \Be_1 \Be_2 \lr{ \Be_1 \cos\theta + \Be_2 \sin\theta } \Be_1 \Be_2 \\
&= \rho \lr{ -\Be_2 \cos\theta + \Be_1 \sin\theta },
\end{aligned}
\end{equation}
a clockwise rotation by \( \pi/2 \) radians
(\cref{problem:2dRotations:1}).

The transformed vector \( \Bx' = \Bx \Be_1 \Be_2 = -\Be_1 \Be_2 \Bx \,(= \Bx i = -i \Bx) \) has been rotated in the direction that takes \( \Be_1 \) to \( \Be_2 \), as illustrated
in \cref{fig:rotationOfV:rotationOfVFig1}.

\mathImageFigure{../figures/GAelectrodynamics/rotationOfVFig1}{\( \pi/2\) rotation in the plane using pseudoscalar multiplication.}{fig:rotationOfV:rotationOfVFig1}{0.3}{orientedAreas.nb}

In complex number theory the complex exponential \( e^{i\theta} \) can be used as a rotation operator.
Geometric algebra puts this rotation operator into the vector algebra toolbox, by utilizing
Euler's formula
\index{Euler's formula}
\begin{equation}\label{eqn:2dRotations:1140}
e^{i\theta} = \cos\theta + i \sin\theta,
\end{equation}
which holds for this pseudoscalar imaginary representation too (\cref{problem:2dRotations:Euler}).
\index{complex exponential}
By writing \( \Be_2 = \Be_1 \Be_1 \Be_2 \),
a complex exponential can be factored directly out of the polar vector representation \cref{eqn:2dRotations:280}
\begin{equation}\label{eqn:2dRotations:940}
\begin{aligned}
\Bx
&= \rho \lr{ \Be_1 \cos\theta + \Be_2 \sin\theta } \\
&= \rho \lr{ \Be_1 \cos\theta + (\Be_1 \Be_1) \Be_2 \sin\theta } \\
&= \rho \Be_1 \lr{ \cos\theta + \Be_1 \Be_2 \sin\theta } \\
&= \rho \Be_1 \lr{ \cos\theta + i \sin\theta } \\
&= \rho \Be_1 e^{i\theta}.
\end{aligned}
\end{equation}

We end up with a complex exponential multivector factor on the right.
Alternatively, since \( \Be_2 = \Be_2 \Be_1 \Be_1 \), a complex exponential can be factored out on the left
\begin{equation}\label{eqn:2dRotations:960}
\begin{aligned}
\Bx &= \rho \lr{ \Be_1 \cos\theta + \Be_2 \sin\theta } \\
&= \rho \lr{ \Be_1 \cos\theta + \Be_2 (\Be_1 \Be_1) \sin\theta } \\
&= \rho \lr{ \cos\theta - \Be_1 \Be_2 \sin\theta } \Be_1 \\
&= \rho \lr{ \cos\theta - i \sin\theta } \Be_1 \\
&= \rho e^{-i\theta} \Be_1.
\end{aligned}
\end{equation}

Left and right exponential expressions have now been found for the polar representation
\begin{equation}\label{eqn:2dRotations:1120}
\rho \lr{ \Be_1 \cos\theta + \Be_2 \sin\theta }
= \rho e^{-i\theta} \Be_1 = \rho \Be_1 e^{i\theta}.
\end{equation}

This is essentially a recipe for rotation of a vector in the x-y plane.
Such rotations are
illustrated in \cref{fig:rotationOfX:rotationOfXFig1}.
\mathImageFigure{../figures/GAelectrodynamics/rotationOfXFig1}{Rotation in a plane.}{fig:rotationOfX:rotationOfXFig1}{0.3}{orientedAreas.nb}

This generalizes to rotations of \R{N} vectors constrained to a plane.
Given orthonormal vectors \( \Bu, \Bv \) and any vector in the plane of these two vectors (\( \Bx \in \Span\setlr{\Bu,\Bv} \)), this vector is rotated \( \theta \) radians in the direction of rotation that takes \( \Bu \) to \( \Bv \) by
\begin{equation}\label{eqn:2dRotations:1160}
\Bx' = \Bx e^{ \Bu \Bv \theta } = e^{-\Bu \Bv \theta} \Bx.
\end{equation}

The sense of rotation for the rotation \( e^{ \Bu \Bv \theta} \) is opposite that of \( e^{\Bv \Bu \theta} \), which provides a first hint that bivectors can be characterized as having an orientation, somewhat akin to thinking of a vector as having a head and a tail.

\makeexample{Velocity and acceleration in polar coordinates.}{example:2dRotations:1180}{
Complex exponential representations of rotations work very nicely for describing vectors in polar coordinates.
A radial vector can be written as
\begin{equation}\label{eqn:2dRotations:1200}
\Br = r \rcap,
\end{equation}
as illustrated in \cref{fig:radialVectorCylindrical:radialVectorCylindricalFig1}.
The polar representation of the radial and azimuthal unit vector are simply
\mathImageFigure{../figures/GAelectrodynamics/radialVectorCylindricalFig1}{Radial vector in polar coordinates.}{fig:radialVectorCylindrical:radialVectorCylindricalFig1}{0.3}{radialVectorCylindricalFig1.nb}
\begin{equation}\label{eqn:2dRotations:1220}
\begin{aligned}
\rcap &= \Be_1 e^{i\theta} =
\Be_1 \lr{ \cos\theta + \Be_1 \Be_2 \sin\theta } = \Be_1 \cos\theta + \Be_2 \sin\theta \\
\thetacap &= \Be_2 e^{i\theta} =
\Be_2 \lr{ \cos\theta + \Be_1 \Be_2 \sin\theta } = \Be_2 \cos\theta - \Be_1 \sin\theta,
\end{aligned}
\end{equation}
where \( i = \Be_{12} \) is the unit bivector for the x-y plane.  We can easily show that these unit vectors are orthogonal
\begin{equation}\label{eqn:2dRotations:1340}
\begin{aligned}
\rcap \thetacap
&= \lr{ \Be_1 e^{i \theta}} \lr{ e^{-i\theta} \Be_2} \\
&= \Be_1 \cancel{e^{i \theta} e^{-i\theta}} \Be_2 \\
&= \Be_1 \Be_2.
\end{aligned}
\end{equation}
By \cref{thm:multiplication:anticommutationNormal}, since the product of \( \rcap \thetacap \) is a bivector,
\( \rcap \) is orthogonal to \( \thetacap \).

We can find the
velocity and acceleration by taking time derivatives
\begin{equation}\label{eqn:2dRotations:1240}
\begin{aligned}
\Bv &= r' \rcap + r \rcap' \\
\Ba &= r'' \rcap + 2 r' \rcap' + r \rcap'',
\end{aligned}
\end{equation}
but to make these more meaningful want to evaluate the \( \rcap, \thetacap \) derivatives explicitly.  Those are
\begin{equation}\label{eqn:2dRotations:1260}
\begin{aligned}
\rcap' &= \lr{ \Be_1 e^{i \theta} }' =
\mathLabelBox[ labelstyle={yshift=1.2em}, linestyle={} ]
{
\Be_1 i
}
{
\(\Be_1 (\Be_1 \Be_2) = (\Be_1 \Be_1) \Be_2 \)
}
 e^{i\theta} \theta' = \Be_2 e^{i\theta} \theta' = \thetacap \omega \\
\thetacap' &= \lr{ \Be_2 e^{i \theta} }' =
\mathLabelBox[ labelstyle={below of=m\themathLableNode, below of=m\themathLableNode} ]
{
\Be_2 i
}
{
\(\Be_2 \Be_1 \Be_2
=
(-\Be_1 \Be_2) \Be_2\)
}
 e^{i\theta} \theta' = -\Be_1 e^{i\theta} \theta' = -\rcap \omega,
\end{aligned}
\end{equation}
where \( \omega = d\theta/dt \), and primes denote time derivatives.  The velocity and acceleration vectors can now be written explicitly in terms of radial and azimuthal components.  The velocity is
\begin{equation}\label{eqn:2dRotations:1280}
\Bv = r' \rcap + r \omega \thetacap,
\end{equation}
and the acceleration is
\begin{equation}\label{eqn:2dRotations:1300}
\begin{aligned}
\Ba
&= r'' \rcap + 2 r' \omega \thetacap + r (\omega \thetacap)' \\
&= r'' \rcap + 2 r' \omega \thetacap + r \omega' \thetacap - r \omega^2 \rcap,
\end{aligned}
\end{equation}
or
\begin{equation}\label{eqn:2dRotations:1320}
\Ba
= \rcap \lr{ r'' - r \omega^2 }
+ \inv{r} \thetacap \lr{ r^2 \omega }'.
\end{equation}

Using \cref{eqn:2dRotations:1220}, we also have the option of factoring out the rotation operation from the position vector or any of its derivatives
\begin{equation}\label{eqn:2dRotations:1360}
\begin{aligned}
\Br &= \lr{ r \Be_1 } e^{i \theta } \\
\Bv &= \lr{ r' \Be_1 + r \omega \Be_2 } e^{i \theta } \\
\Ba &= \lr{ \lr{ r'' - r \omega^2 } \Be_1 + \inv{r} \lr{ r^2 \omega }' \Be_2 } e^{i\theta}.
\end{aligned}
\end{equation}

In particular,
for uniform circular motion, each of the position, velocity and acceleration vectors can be represented by a vector that is fixed in space, subsequently rotated by an angle \( \theta \).
} % example
\makeproblem{\R{2} rotations.}{problem:2dRotations:1}{
Using familiar methods, such as rotation matrices, show that the counterclockwise and clockwise rotations of
\cref{eqn:2dRotations:280} are given by
\cref{eqn:2dRotations:300} and
\cref{eqn:2dRotations:3} respectively.
} % problem
\makeanswer{problem:2dRotations:1}{
The 2D rotation matrix is
\begin{equation}\label{eqn:2dRotations:1380}
R_\theta =
\begin{bmatrix}
   \cos\theta & -\sin\theta \\
   \sin\theta & \cos\theta
\end{bmatrix},
\end{equation}
so to rotate coordinates by \(\pm\pi/2\), we multiply by
\begin{equation}\label{eqn:2dRotations:1400}
R_{\pm\pi/2} = \pm
\begin{bmatrix}
   0 & -1 \\
   1 & 0
\end{bmatrix}.
\end{equation}
In particular
\begin{equation}\label{eqn:2dRotations:1420}
   R_{\pm\pi/2}
\begin{bmatrix}
   \rho \cos\theta \\
   \rho \sin\theta
\end{bmatrix}
   = \pm \pi/2
\begin{bmatrix}
   0 & -1 \\
   1 & 0
\end{bmatrix}
\begin{bmatrix}
   \rho \cos\theta \\
   \rho \sin\theta
\end{bmatrix}
=
\pm
\rho
\begin{bmatrix}
-\sin\theta \\
\cos\theta
\end{bmatrix},
\end{equation}
consistent with the results observed from left and right multiplication with the plane pseudoscalar \( \Be_1 \Be_2 \).
}
\makeproblem{Multivector Euler's formula and trig relations.}{problem:2dRotations:Euler}{
For a multivector \( x \) assume an infinite series representation of the exponential, sine and cosine functions and their hyperbolic analogues
\begin{equation*}
\begin{aligned}
e^x &= \sum_{k = 0}^\infty \frac{x^k}{k!} \\
\cos x &= \sum_{k = 0}^\infty (-1)^k \frac{x^{2k}}{(2k)!} \qquad \sin x = \sum_{k = 0}^\infty (-1)^k \frac{x^{2k+1}}{(2k+1)!} \\
\cosh x &= \sum_{k = 0}^\infty \frac{x^{2k}}{(2k)!} \qquad \sinh x = \sum_{k = 0}^\infty \frac{x^{2k+1}}{(2k+1)!} \\
\end{aligned}
\end{equation*}
\makesubproblem{}{problem:2dRotations:Euler:a}
Show that for scalar \( \theta \), and any multivectors \( J \) that satisfies \( J^2 = -1 \), and \( K^2 = 1 \), then
hold for multivectors \( J, K \) satisfying \( J^2 = -1 \) and \( K^2 = 1 \) respectively.
\begin{equation*}
\begin{aligned}
\cosh (J \theta ) &= \cos \theta, \quad \cosh (K \theta ) = \cosh \theta \\
\sinh (J \theta ) &= J \sin \theta, \quad \sinh (K \theta ) = K \sinh \theta.
\end{aligned}
\end{equation*}
\makesubproblem{}{problem:2dRotations:Euler:c}
Show that the trigonometric and hyperbolic Euler formulas
\begin{equation*}
\begin{aligned}
e^{ J \theta } &= \cos \theta + J \sin \theta \\
e^{ K \theta } &= \cosh \theta + K \sinh \theta,
\end{aligned}
\end{equation*}
hold for multivectors \( J, K \) satisfying \( J^2 = -1 \) and \( K^2 = 1 \) respectively.
\makesubproblem{}{problem:2dRotations:Euler:b}
Given multivectors \( X, Y \), show that \( e^{ X + Y } = e^{ X } e^{ Y } \) if \( X, Y \) commute.  That is \( X Y = Y X \).
} % problem
\makeanswer{problem:2dRotations:Euler}{
\makesubanswer{}{problem:2dRotations:Euler:a}
Let \( \chi \) be a multivector that squares to \( \pm 1 \).  Series expansion of \( \cosh(\chi \theta) \), for scalar \(theta\) yields
\begin{equation}\label{eqn:2dRotations:1440}
\cosh(\chi\theta)
= \sum_{k = 0}^\infty \frac{\lr{ \chi \theta }^{2k}}{(2k)!}
= \sum_{k = 0}^\infty \frac{ \chi^{2k} \theta^{2k} }{(2k)!}.
\end{equation}
In particular, for \( \chi = J, K \) respectively, we have
\begin{equation}\label{eqn:2dRotations:1460}
\begin{aligned}
\cosh\lr{J\theta} &= \sum_{k = 0}^\infty \frac{ \lr{-1}^k \theta^{2k} }{(2k)!} = \cos \theta \\
\cosh\lr{K\theta} &= \sum_{k = 0}^\infty \frac{ \lr{+1}^k \theta^{2k} }{(2k)!} = \cosh \theta.
\end{aligned}
\end{equation}
Similarly,
\begin{equation}\label{eqn:2dRotations:1480}
\sinh(\chi\theta)
= \sum_{k = 0}^\infty \frac{\lr{ \chi \theta }^{2k+1}}{(2k+1)!}
= \chi \sum_{k = 0}^\infty \frac{ \chi^{2k} \theta^{2k+1} }{(2k+1)!}.
\end{equation}
So, for \( \chi = J, K \) respectively, we have
\begin{equation}\label{eqn:2dRotations:1500}
\begin{aligned}
\sinh\lr{J\theta} &= J \sum_{k = 0}^\infty \frac{ \lr{-1}^k \theta^{2k+1} }{(2k +1)!} = J \sin \theta \\
\sinh\lr{K\theta} &= K \sum_{k = 0}^\infty \frac{ \lr{+1}^k \theta^{2k+1} }{(2k +1)!} = K \sinh \theta.
\end{aligned}
\end{equation}
\makesubanswer{}{problem:2dRotations:Euler:c}
Series expanding again, we may split the exponential into even and odd parts, for any multivector \( x \)
\begin{equation}\label{eqn:2dRotations:1520}
\begin{aligned}
e^x
&=
\sum_{k = 0}^\infty \frac{x^k}{k!} \\
&=
\sum_{k = 0}^\infty \frac{x^{2k}}{(2k)!}
+
\sum_{k = 0}^\infty \frac{x^{2k+1}}{(2k+1)!} \\
&=
\cosh( x ) + \sinh(x).
\end{aligned}
\end{equation}
There is nothing in such a series expansion that cares about the type of \( x \), only that we can take repeated powers.  The remainder of the problem follows from our results above after substitution of \( x = J \theta \) and \( x = K \theta \) respectively.
\makesubanswer{}{problem:2dRotations:Euler:b}
The exponential of a sum, such as \( X + Y \), regardless of the types or characteristics of \( X \) and \( Y \) is
\begin{equation}\label{eqn:2dRotations:1540}
e^{X + Y}
= \sum_{k = 0}^\infty \frac{\lr{X + Y}^k}{k!}.
\end{equation}
Let's look at the powers of such a sum.  For the square and cube we have
\begin{equation}\label{eqn:2dRotations:1560}
\lr{ X + Y}^2 = X^2 + X Y + Y X + Y^2,
\end{equation}
\begin{equation}\label{eqn:2dRotations:1580}
\lr{ X + Y}^3 = X^3 + X^2 Y + X Y X + Y X^2 + Y^2 X + Y X Y + X Y^2 + Y^3.
\end{equation}
Observe that the conventional binomial series form for these powers is only possible if \( X \) and \( Y \) commute.  If we have such commutation, then the exponential takes the form
\begin{equation}\label{eqn:2dRotations:1600}
\begin{aligned}
e^{X + Y}
&= \sum_{k = 0}^\infty \sum_{j = 0}^k \binom{k}{j} \frac{X^j Y^{k-j}}{k!} \\
%&= \sum_{k = 0}^\infty \sum_{j = 0}^k \frac{k!}{j!(k - j)!} \frac{X^j Y^k}{k!} \\
&= \sum_{k = 0}^\infty \sum_{j = 0}^k \frac{X^j Y^{k-j}}{j!(k - j)!}.
\end{aligned}
\end{equation}
This is a sum over all points in a trianglular region of the first quadrant of indexes on the \( k, j \) axes.
%in \cref{fig:exponentialOfSum:exponentialOfSumFig1}
We can, however, sum over all the diagonals \( s = k - j = \textrm{constant} \), and index our position on each of those diagonals by \( u = j \), to find
\begin{equation}\label{eqn:2dRotations:1620}
\begin{aligned}
e^{X + Y}
&= \sum_{s = 0}^\infty \sum_{u = 0}^\infty \frac{X^u Y^{s}}{u!s!} \\
&= \sum_{u = 0}^\infty \frac{X^u}{u!}
   \sum_{s = 0}^\infty \frac{Y^s}{s!} \\
&= e^X e^Y.
\end{aligned}
\end{equation}
%\imageFigure{../figures/GAelectrodynamics/exponentialOfSumFig1}{Exponential of sums, index points and change of variables.}{fig:exponentialOfSum:exponentialOfSumFig1}{0.3}
We see that commutation of variables is required for an exponential of a sum to equal the product of the exponentials.  This is worth understanding since it shows us that we can factor exponentials of sums such as \( Z = 1 + \Be_1 \Be_2 \), \( Z = \Be_1 \Be_2 + \Be_3 \Be_4 \), \( Z = \Be_1 + \Be_1 \Be_2 \Be_3 \), into the product of the exponentials of the summands of those multivectors, but cannot do so with multivectors like \( Z = \Be_1 + \Be_2 \), \( Z = \Be_1 + \Be_1 \Be_2 \), or \( Z = \Be_1 \Be_2 + \Be_2 \Be_3 \).
}
