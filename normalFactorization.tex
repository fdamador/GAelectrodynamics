%
% Copyright © 2017 Peeter Joot.  All Rights Reserved.
% Licenced as described in the file LICENSE under the root directory of this GIT repository.
%
A general bivector has the form
\begin{equation}\label{eqn:normalFactorization:1800}
B = \sum_{i \ne j} a_{ij} \Be_{ij},
\end{equation}
which is not necessarily a blade.
%For example the bivector \( \Be_1 \Be_2 + \Be_3 \Be_4 \) cannot be factored into any product of normal vectors.
On the other hand, a wedge product is always a blade
\footnote{In \R{3} any bivector is also a blade \citep{ablamowicz2004lectures:chapter1}}

\index{wedge factorization}
\maketheorem{Wedge product normal factorization}{thm:SimpleProducts2:wnormalfactorize}{
The wedge product of any two non-colinear vectors \( \Ba, \Bb \) always has a orthogonal (2-blade) factorization
\begin{equation*}
\Ba \wedge \Bb = \Bu \Bv, \quad \Bu \cdot \Bv = 0.
\end{equation*}
} % theorem

This can be proven by construction.
Pick \( \Bu = \Ba \) and \( \Bv = \Rej{\Ba}{\Bb} \), then
\begin{equation}\label{eqn:normalFactorization:1840}
\begin{aligned}
\Ba \Rej{\Ba}{\Bb}
&= \cancel{\Ba \cdot \Rej{\Ba}{\Bb}} + \Ba \wedge \Rej{\Ba}{\Bb} \\
&= \Ba \wedge \lr{ \Bb - \frac{\Bb \cdot \Ba}{\Norm{\Ba}^2} \Ba } \\
&= \Ba \wedge \Bb,
\end{aligned}
\end{equation}
since \( \Ba \wedge (\alpha \Ba) = 0 \) for any scalar \( \alpha \).

The significance of \cref{thm:SimpleProducts2:wnormalfactorize} is that the square of any wedge product is negative
\begin{dmath}\label{eqn:normalFactorization:1820}
(\Bu \Bv)^2
=
(\Bu \Bv) (-\Bv \Bu)
=
-\Bu (\Bv^2) \Bu
=
- \Abs{\Bu}^2 \Abs{\Bv}^2,
\end{dmath}
which in turn means that exponentials with wedge product arguments can be used as rotation operators.

\makeproblem{\R{3} bivector factorization.}{problem:normalFactorization:1}{
Find some orthogonal factorizations for the \R{3} bivector \( \Be_{12} + \Be_{23} + \Be_{31} \).
} % problem

\makeanswer{problem:normalFactorization:1}{
\begin{equation*}
\begin{aligned}
\Be_{12} + \Be_{23} + \Be_{31}
&= \lr{ \Be_1 + \Be_2 - 2 \Be_3 } \frac{ \Be_2 - \Be_1 }{2} \\
&= \frac{ \Be_3 - \Be_2 }{2} \lr{ 2 \Be_1 - \Be_2 - \Be_3 }.
\end{aligned}
\end{equation*}
Each set of factors above can be interpretted as the edges of two different rectangular representations of the bivector, for which the total area is fixed.  The span of either set of factors describes the plane that the bivector represents.
%The respective parallelogram representations of these bivector factorizations are illustrated in
%\cref{fig:bivectorFactorization:bivectorFactorizationFig1}, showing the bivectors face on, and from a side view that shows they are coplanar.
%
%% \imageTwoFigures{path1}{path2}{fancy plots}{fig:blah}{scale=0.3}
%\imageTwoFigures
%{../figures/GAelectrodynamics/bivectorFactorizationFig2}
%{../figures/GAelectrodynamics/bivectorFactorizationFig1}
%{Two equivalent bivector factorizations.}
%{fig:bivectorFactorization:bivectorFactorizationFig1}{scale=0.3}
} % answer
