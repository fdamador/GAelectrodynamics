%
% Copyright � 2018 Peeter Joot.  All Rights Reserved.
% Licenced as described in the file LICENSE under the root directory of this GIT repository.
%
%{
%\input{../latex/blogpost.tex}
%\renewcommand{\basename}{gaugeTransformation}
%%\renewcommand{\dirname}{notes/phy1520/}
%\renewcommand{\dirname}{notes/ece1228-electromagnetic-theory/}
%%\newcommand{\dateintitle}{}
%%\newcommand{\keywords}{}
%
%\input{../latex/peeter_prologue_print2.tex}
%
%\usepackage{peeters_layout_exercise}
%\usepackage{peeters_braket}
%\usepackage{peeters_figures}
%\usepackage{siunitx}
%%\usepackage{mhchem} % \ce{}
%%\usepackage{macros_bm} % \bcM
%\usepackage{macros_qed} % \qedmarker
%%\usepackage{txfonts} % \ointclockwise
%
%%\newcommand{\dAlembertian}[0]{\Box}
%\newcommand{\dAlembertian}[0]{\square}
%
%\newcommand{\stgrad}[0]{\lr{ \spacegrad + \inv{c} \PD{t}{}}}
%\newcommand{\conjstgrad}[0]{\lr{ \spacegrad - \inv{c} \PD{t}{}}}
%
%\beginArtNoToc
%
%\generatetitle{Multivector potentials.}
%%\chapter{Multivector potentials.}
\label{chap:gaugeTransformation}

Conventional electromagnetism utilizes scalar and vector potentials, so it is reasonable to expect that
the desired multivector representation of the potential is a grade \((0,1)\)-multivector.
A potential representation with grades 2,3 works for (fictitious) magnetic sources, so we may generally
allow a multivector potential to have any grades.  Such a potential is related to the field as follows.

\index{\(A\)}
\makedefinition{Multivector potential.}{thm:generalPotential:80}{
The electromagnetic field strength \( F \) for a \textit{multivector potential} \( A \) is
\begin{equation*}
F = \gpgrade{\conjstgrad A}{1,2}.
\end{equation*}
} % definition

Before unpacking \( \conjstgrad A \), we want to label the
different grades of the multivector potential, and do so in a way that is consistent with the conventional
potential representation of the electric and magnetic fields.
\index{\(\phi\)}
\index{\(\BA\)}
\index{\(\phi_m\)}
\index{\(\BF\)}
\makedefinition{Multivector potential representation.}{dfn:unpackStaticPotential:80}{
Let
%\label{eqn:gaugeTransformation:1111}
\begin{equation*}
A =
      - \phi
      + c \BA
      + \eta I \lr{ -\phi_m + c \BF },
\end{equation*}
where
\begin{enumerate}
\item \( \phi \) is the scalar potential \si{V} (Volts).
\item \( \BA \) is the vector potential \si{W/m} (Webers/meter).
\item \( \phi_m \) is the scalar potential for (fictitious) magnetic sources \si{A} (Amperes).
\item \( \BF \) is the vector potential for (fictitious) magnetic sources \si{C} (Coulombs).
\end{enumerate}
} % definition
This specific breakdown of \( A \) into scalar and vector potentials, and dual (pseudoscalar and bivector) potentials has been chosen to match SI conventions, specifically those of \citep{balanis2005antenna} (which includes fictitious magnetic sources.)

We can now express the fields in terms of the potentials.
\input{Theorem_fields_and_potential_wave_equations.tex}
\begin{proof}
To prove \cref{thm:generalPotential:40} we start by expanding \( (\spacegrad - (1/c)\partial_t) A \) using
\cref{dfn:unpackStaticPotential:80} and then group by grade to find
\begin{dmath}\label{eqn:gaugeTransformation:1111}
\begin{aligned}
\conjstgrad A
&=
\conjstgrad \lr{  - \phi
      + c \BA
      + \eta I \lr{ -\phi_m + c \BF } } \\
&=
- \spacegrad \phi + c \spacegrad \cdot \BA + c \spacegrad \wedge \BA + \inv{c} \PD{t}{\phi} - \PD{t}{\BA} \\
&\quad + I \eta
\lr{
- \spacegrad \phi_\txtm + c \spacegrad \cdot \BF + c \spacegrad \wedge \BF + \inv{c} \PD{t}{\phi_\txtm} - \PD{t}{\BF}
} \\
&=
c \spacegrad \cdot \BA
+ \inv{c} \PD{t}{\phi}
\\
&
+
\mathLabelBox[ labelstyle={below of=m\themathLableNode, below of=m\themathLableNode} ]
{
   - \spacegrad \phi
   - \PD{t}{\BA}
   - \inv{\epsilon} \spacegrad \cross \BF
}
{
\(\BE\)
}
+
\mathLabelBox[ labelstyle={below of=m\themathLableNode, below of=m\themathLableNode} ]
{
   I \eta
   \lr{
      - \spacegrad \phi_\txtm
      - \PD{t}{\BF}
      + \inv{\mu} \spacegrad \cross \BA
   }
}
{\(I \eta \BH\)
} \\
&
+ I \eta\lr{
  c \spacegrad \cdot \BF
+ \inv{c} \PD{t}{\phi_\txtm}
},
\end{aligned}
\end{dmath}
which shows the claimed field split.

In terms of the potentials Maxwell's equation \( \stgrad F = J \) is
\begin{dmath}\label{eqn:gaugeTransformation:20}
\stgrad \gpgrade{\conjstgrad A}{1,2} = J,
\end{dmath}
or
\begin{dmath}\label{eqn:gaugeTransformation:40}
\dAlembertian A = J + \stgrad \gpgrade{\conjstgrad A}{0,3}.
\end{dmath}
This is almost a wave equation.
Inserting \cref{eqn:gaugeTransformation:1111} into \cref{eqn:gaugeTransformation:40} and selecting each grade gives four almost-wave equations
\begin{equation*}
\begin{aligned}
-
\dAlembertian
\phi &= \frac{\rho}{\epsilon} + \inv{c} \PD{t}{} \lr{ c \spacegrad \cdot \BA + \inv{c} \PD{t}{\phi} } \\
c
\dAlembertian
\BA &= -\eta \BJ + \spacegrad \lr{ c \spacegrad \cdot \BA + \inv{c} \PD{t}{\phi} } \\
\eta c I
\dAlembertian
\BF &= - I \BM + \spacegrad \cdot \lr{ I \eta\lr{ c \spacegrad \cdot \BF + \inv{c} \PD{t}{\phi_\txtm} } } \\
-I \eta
\dAlembertian
\phi_\txtm &= I c \rho_\txtm + \inv{c} \PD{t}{} I \eta\lr{ c \spacegrad \cdot \BF + \inv{c} \PD{t}{\phi_\txtm} }
\end{aligned}
\end{equation*}
Using \( \eta = \mu c, \eta c \epsilon = 1 \), and
\( \spacegrad \cdot (I \psi) = I \spacegrad \psi \) for scalar \(\psi\), a bit
of rearrangement completes the proof.
\end{proof}

\subsection{Gauge transformations.}
Clearly it is desirable if potentials can be found for which \( \spacegrad \cdot \BA + (1/c^2) \partial_t \phi = \spacegrad \cdot \BF + (1/c^2) \partial_t \phi_\txtm = 0 \).
Finding such potentials relies on the fact that the potential representation is not unique.
In particular,
we have the freedom to add any spacetime gradient of any scalar or pseudoscalar potential without changing the field.
\index{gauge transformation}
\input{Theorem_gauge_invariance.tex}
\begin{proof}
To prove \cref{thm:gaugeTransformation:60} let
\begin{dmath}\label{eqn:gaugeTransformation:100}
A' = A + \stgrad (\psi + I \phi),
\end{dmath}
where \( \psi \) and \( \phi \) are scalar functions.
The field for potential \( A' \) is
\begin{dmath}\label{eqn:gaugeTransformation:120}
F'
= \gpgrade{\conjstgrad A'}{1,2}
= \gpgrade{\conjstgrad \lr{A + \stgrad (\psi + I \phi)} }{1,2}
= \gpgrade{\conjstgrad A}{1,2} + \gpgrade{ \conjstgrad \stgrad (\psi + I \phi)} {1,2}
= F + \gpgrade{ \dAlembertian (\psi + I \phi)} {1,2},
\end{dmath}
which is just \( F \) since the d'Alembertian operator \( \dAlembertian \) is a scalar operator and \( \psi + I \phi \) has no vector nor bivector grades.
\end{proof}

We say that we are working in the Lorenz gauge, if the 0,3 grades of \( \conjstgrad A \) are zero, or a transformation that kills those grades is made.
\index{Lorenz gauge}
\input{Theorem_lorentz_gauge_transformation.tex}
\begin{proof}
To prove \cref{thm:gaugeTransformation:140}, let
\begin{dmath}\label{eqn:gaugeTransformation:200}
A = A' + \stgrad \Psi,
\end{dmath}
so Maxwell's equation becomes
\begin{dmath}\label{eqn:gaugeTransformation:220}
J
= \stgrad \gpgrade{ \conjstgrad A }{1,2}
= \dAlembertian A - \stgrad \gpgrade{ \conjstgrad A }{0,3}
= \dAlembertian A' + \dAlembertian \stgrad \Psi - \stgrad \gpgrade{ \conjstgrad A }{0,3}
= \dAlembertian A' + \stgrad \lr{ \dAlembertian \Psi - \gpgrade{ \conjstgrad A }{0,3} }.
\end{dmath}
Requiring
\begin{dmath}\label{eqn:gaugeTransformation:240}
\dAlembertian \Psi = \gpgrade{ \conjstgrad A }{0,3},
\end{dmath}
completes the proof.
\end{proof}
Observe that \( \Psi \) has only grades 0,3 as required of a gauge function.

Such a transformation completely decouples Maxwell's equation, providing one scalar wave equation for each grade of \( \dAlembertian A' = J \), relating each grade of the potential \(A'\) to exactly one grade of the source multivector current \( J \).
We are free to immediately solve for \( A' \) using the (causal) Green's function for the d'Alembertian
\begin{dmath}\label{eqn:gaugeTransformation:160}
A'(\Bx, t)
= -\int dV' dt' \frac{\delta(\Abs{\Bx - \Bx'} - c(t - t')}{4 \pi \Norm{\Bx - \Bx'} } J(\Bx', t')
= -\inv{4\pi} \int dV' \frac{J(\Bx', t - \inv{c} \Norm{\Bx - \Bx'})}{\Norm{\Bx - \Bx'}},
\end{dmath}
which is the sum of all the current contributions relative to the point \( \Bx \) at the retarded time \( t_\txtr = t - (1/c) \Norm{\Bx - \Bx'}\).
The field follows immediately by differentiation and grade selection
\begin{dmath}\label{eqn:gaugeTransformation:180}
F = \gpgrade{ \conjstgrad A' }{1,2}.
\end{dmath}

Again, using the Green's function for the d'Alembertian, the explicit form of the gauge function \( \Psi \) is
\begin{dmath}\label{eqn:gaugeTransformation:260}
\Psi = -\inv{4\pi} \int dV' \frac{\gpgrade{ \conjstgrad A(\Bx', t_\txtr) }{0,3}}{\Norm{\Bx - \Bx'}},
\end{dmath}
however, we don't actually need to compute this.
Instead, we only have to know we are free to construct a field from any solution \( A' \) of \( \dAlembertian A' = J \) using \cref{eqn:gaugeTransformation:180}.

%}
%\EndArticle
