%
% Copyright © 2017 Peeter Joot.  All Rights Reserved.
% Licenced as described in the file LICENSE under the root directory of this GIT repository.
%

%Vectors have many generalizations in mathematics, where
%a number of disparate mathematical objects
%%, such as
%%directed ``arrows'', tuples of real or complex numbers, matrices, functions, polynomials, and quantum states
%can all be considered vectors.
%A vector space is an enumeration of the properties and operations that are common to a set of
%vector-like objects, allowing them to be treated in a unified fashion, regardless of their representation and application.
%%The definition of a vector space and some other basic ideas from linear algebra are all reviewed here.
%%This review will set the stage for the definition of a \boldTextAndIndex{multivector space}, the GA analogue of a vector space.

Two representation specific methods of vector addition and multiplication have been described.
Addition can be performed graphically, connecting vectors heads to tails, or by adding the respective coordinates.
Multiplication can be performed by changing the length of a vector represented by an arrow, or by multiplying each coordinate algebraically.
These rules can be formalized and abstracted by introducing the concept of vector space, which describes both vector addition and multiplication in a representation agnostic fashion.
\index{vector space}
\makedefinition{Vector space.}{def:prerequisites:vectorspace}{
A vector space is a set \( V = \setlr{\Bx, \By, \Bz, \cdots} \), the elements of which are called vectors, which has an addition operation designated \( + \) and a scalar multiplication operation designated by juxtaposition, where the following axioms are satisfied for all
for all vectors \( \Bx, \By, \Bz \in V \) and scalars \( a, b \in \bbR \).
\begin{tablebox}[tabularx={X|Y}]%{Vector space axioms.}
    V is closed under addition & \( \Bx + \By \in V \) \\ \hline
    V is closed under scalar multiplication & \( a \Bx \in V \) \\ \hline
    Addition is associative & \( (\Bx + \By) + \Bz = \Bx + (\By + \Bz) \) \\ \hline
    Addition is commutative & \( \By + \Bx = \Bx + \By \) \\ \hline
    There exists a zero element \( \Bzero \in V \)  & \( \Bx + \Bzero = \Bx \) \\ \hline
    For any \( \Bx \in V \) there exists a negative additive inverse \( -\Bx \in V \) & \( \Bx + (-\Bx) = \Bzero \) \\ \hline
    Scalar multiplication is distributive  & \( a( \Bx + \By ) = a \Bx + a \By \), \( (a + b)\Bx = a \Bx + b\Bx \) \\ \hline
    Scalar multiplication is associative & \( (a b) \Bx = a ( b \Bx ) \) \\ \hline
    There exists a multiplicative identity & \( 1 \Bx = \Bx \) \\ \hline
\end{tablebox}
}

One may define finite or infinite dimensional vector spaces with matrix, polynomial, complex tuple, or many other types of elements.
Some examples of general vector spaces are given in the problems below, and many more can be found in any introductory book on linear algebra.
The applications of geometric algebra to electromagnetism found in this book require only real vector spaces with dimension no greater than three.
\Cref{def:prerequisites:vectorspace} serves as a reminder, as the concept of vector space will be built upon and generalized shortly.
