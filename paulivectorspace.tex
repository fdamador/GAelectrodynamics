%
% Copyright © 2017 Peeter Joot.  All Rights Reserved.
% Licenced as described in the file LICENSE under the root directory of this GIT repository.
%
\makeproblem{Pauli matrices.}{problem:prerequisites:20}{
The Pauli matrices are defined as

\begin{equation}\label{eqn:prerequisites:160}
   \sigma_1 = \PauliX,\quad
   \sigma_2 = \PauliY,\quad
   \sigma_3 = \PauliZ.
\end{equation}

Given any scalars \( a, b, c \in \bbR \), show that the set \( V = \setlr{ a \sigma_1 + b \sigma_2 + c \sigma_3 } \) is a vector space with respect to matrix addition.  Determine the form of the zero and identity elements.
Given a vector \( \Bx =  x_1 \sigma_1 + x_2 \sigma_2 + x_3 \sigma_3 \), show that the coordinates \( x_i \) can be extracted by evaluating the matrix trace of the matrix product \( \sigma_i \Bx \).
% FIXME: make this a problem at at location where it makes sense:
%\makesubproblem{}{problem:prerequisites:20:b}
%
%Show that \( \sigma_k^2 = I \), where \( I \) is the 2x2 identity matrix, and that \( \sigma_k \sigma_j = -\sigma_k \sigma_j \) for all \( k \ne j \).
%
%\makesubproblem{}{problem:prerequisites:20:c}
%
%Using the results of \partref{problem:prerequisites:20:b}, show that
%\( \lr{ a \sigma_x + b \sigma_y + c \sigma_z }^2 = (a^2 + b^2 + c^2) I \), where \( I \) is the 2x2 identity matrix.
%%This shows that the Pauli matrices are an example \R{3} basis for which the contraction axiom is built right into the representation.
} % problem
\makeanswer{problem:prerequisites:20}{
The reader can check that with zero element \( 0 = \begin{bmatrix}0 & 0 \\ 0 & 0 \end{bmatrix} \), and a scalar multiplicative identity \( 1 \), all the vector space properties are satisified.

For the coordinates observe that \( \Bx =
\begin{bmatrix}
   c & a - i b \\
   a + i b & -c
\end{bmatrix} \), and
\begin{equation}\label{eqn:paulivectorspace:180}
\begin{aligned}
   \traceB{\sigma_1 \Bx} &=
%   \traceB{ \PauliX
%\begin{bmatrix}
%   c & a - i b \\
%   a + i b & -c
%\end{bmatrix}
%}
%=
\trace{\begin{bmatrix}
   a + i b & -c \\
   c & a - i b
\end{bmatrix}
}
=
2 a \\
\traceB{\sigma_2 \Bx} &=
%   \traceB{ \PauliY
%\begin{bmatrix}
%   c & a - i b \\
%   a + i b & -c
%\end{bmatrix}
%}
%=
\trace{\begin{bmatrix}
   -i a + b & i c \\
   i c & i a + b
\end{bmatrix}
}
=
2 b \\
\traceB{\sigma_3 \Bx} &=
%   \traceB{ \PauliZ
%\begin{bmatrix}
%   c & a - i b \\
%   a + i b & -c
%\end{bmatrix}
%}
%=
\trace{\begin{bmatrix}
      c & a - i b \\
      a + i b & c
\end{bmatrix}
}
=
2 c,
\end{aligned}
\end{equation}
so \( a = \traceB{\sigma_1 \Bx}/2 \), \( b = \traceB{\sigma_2 \Bx}/2 \), \( c = \traceB{\sigma_3 \Bx}/2 \).
}
