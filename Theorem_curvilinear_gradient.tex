%
% Copyright � 2018 Peeter Joot.  All Rights Reserved.
% Licenced as described in the file LICENSE under the root directory of this GIT repository.
%
\maketheorem{Curvilinear representation of the gradient}{thm:curvilinearGradient:2}{
Given an N-parameter vector parameterization
\( \Bx = \Bx(u_1, u_2, \cdots, u_N) \)
of \R{N},
with curvilinear basis elements \( \Bx_i = \PDi{u_i}{\Bx} \), the \textit{gradient} is
\begin{equation*}
\spacegrad = \sum_i \Bx^i \PD{u_i}{}.
\end{equation*}
It is convenient to define \( \partial_i \equiv \PDi{u_i}{} \), so that the gradient can be expressed in mixed index representation
\begin{equation*}
\spacegrad = \sum_i \Bx^i \partial_i.
\end{equation*}
%or the same with sums over mixed indexes implied.
} % theorem
