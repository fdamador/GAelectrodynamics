%
% Copyright © 2016 Peeter Joot.  All Rights Reserved.
% Licenced as described in the file LICENSE under the root directory of this GIT repository.
%
\makeproblem{\R{4} wedge of a non-blade with itself.}{problem:gradeselection:r4nonzerobivectorwedgewithself}{
While the wedge product of a blade with itself is always zero, this is not generally true of the wedge products of arbitrary k-vectors in higher dimensional spaces.
To demonstrate this, show that the wedge of the bivector
\( B = \Be_1 \Be_2 + \Be_3 \Be_4 \) with itself is non-zero.
Why is this bivector not a blade?
%, show that \( B \wedge B \ne 0 \).
} % problem
\makeanswer{problem:gradeselection:r4nonzerobivectorwedgewithself}{
\begin{equation}\label{eqn:r4nonzerobivectorwedgewithself:20}
\begin{aligned}
\lr{ \Be_{12} + \Be_{34} } \wedge \lr{ \Be_{12} + \Be_{34} }
&=
\Be_{1234} + \Be_{3412}
&=
2 \Be_{1234}.
\end{aligned}
\end{equation}
A blade is the wedge product of two vectors, or the geometric product of two orthogonal vectors.
The grade-2 multivector \( \Be_{12} + \Be_{34} \) is not a blade, since there is no common factor between \( \Be_{12} \) and \( \Be_{34} \).  It is not possible to factor this multivector into two orthogonal products.
} % answer
