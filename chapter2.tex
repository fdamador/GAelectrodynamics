%
% Copyright © 2018 Peeter Joot.  All Rights Reserved.
% Licenced as described in the file LICENSE under the root directory of this GIT repository.
%
%{
\mychapter{Multivector calculus.}
   \section{Reciprocal frames.}
      \subsection{Motivation and definition.}
         %
% Copyright � 2016 Peeter Joot.  All Rights Reserved.
% Licenced as described in the file LICENSE under the root directory of this GIT repository.
%
%{
%\label{chap:reciprocal}

The end goal of this chapter is to be able to integrate multivector functions along curves and surfaces, known collectively as manifolds.
For our purposes, a manifold is defined by a parameterization, such as the vector valued function \( \Bx(a,b) \) where \( a, b\) are scalar parameters.  With one parameter the vector traces out a curve, with two a surface, three a volume, and so forth.
The respective partial derivatives of such a parameterized vector define a local basis for the surface at the point at which the partials are evaluated.
The span of such a basis is called the tangent space, and the partials that constitute it are not necessarily orthonormal, or even orthogonal.

Unfortunately, in order to work with the curvilinear non-orthonormal bases that will be encountered in general integration theory, some
additional tools are required.

\begin{itemize}
\item
We introduce a reciprocal frame which partially generalizes the notion of orthogonality to non-orthonormal bases.
\item
We will borrow the upper and lower index (tensor) notation from relativistic physics that is useful for the intrinsically non-orthonormal spaces encountered in that study, as this notation works well to define the reciprocal frame.
\end{itemize}
\index{reciprocal frame}
\index{\(\Bx^i\)}
\index{\({\delta_i}^j\)}
%
% Copyright � 2018 Peeter Joot.  All Rights Reserved.
% Licenced as described in the file LICENSE under the root directory of this GIT repository.
%
\makedefinition{Reciprocal frame}{dfn:reciprocal:frame}{
Given a subspace basis \( \beta = \setlr{ \Bx_1, \Bx_2, \cdots \Bx_m } \), not
necessarily orthonormal, the \textit{reciprocal frame} is defined as the set of vectors \( \setlr{ \Bx^1, \Bx^2, \cdots \Bx^m } \in \Span \beta \) satisfying
\begin{dmath*}
\Bx_i \cdot \Bx^j = {\delta_i}^j,
\end{dmath*}
where the vector \( \Bx^j \) is not the j-th power of \( \Bx \), but is a superscript index, the conventional way of denoting a reciprocal frame vector, and \( {\delta_i}^j \) is the Kronecker delta.
} % definition

With an orthogonal basis, every basis vector is orthogonal to every other vector, so any orthonormal basis is also its reciprocal basis.
In general, if the original basis is not-orthogonal, every reciprocal basis vector is orthogonal to all but one of the original basis vectors, but may not be orthogonal to any other reciprocal basis vector.

If the curvilinear basis vectors \( \setlr{ \Bx_i } \) are orthogonal, then the reciprocal vectors \( \Bx^i \) are easy to compute.  Since they must each satisfy \( \Bx^i \cdot \Bx_i = 1 \), let \( \Bx^i = \alpha \Bx_i \), then
\begin{equation}\label{eqn:reciprocal:401}
\begin{aligned}
1
&= \Bx^i \cdot \Bx_i \\
&= \lr{ \alpha \Bx_i } \cdot \Bx_i \\
&= \alpha \lr{ \Bx_i \cdot \Bx_i },
\end{aligned}
\end{equation}
or
\begin{equation}\label{eqn:reciprocal:402}
\Bx^i = \inv{ \Bx_i \cdot \Bx_i } \Bx_i = \inv{\Bx_i}.
\end{equation}
More general techniques for computation of reciprocal bases will be required for non-orthoganal bases.

Mixed index variables have been introduced for the first time in this text, which may be unfamiliar.  These are most often used in tensor algebra, where any expression that has pairs of upper and lower indexes implies a sum, and is called the summation (or Einstein) convention.  For example:
\begin{equation}\label{eqn:reciprocal:400}
\begin{aligned}
a_i b^i &\equiv \sum_i a_i b^i \\
{A^{i}}_j B_i C^j &\equiv \sum_{i,j} {A^{i}}_j B_i C^j.
\end{aligned}
\end{equation}

Summation convention will not be used explicitly in this text, as it deviates from normal practises in electrical engineering\footnote{Generally, when summation convention is used, explicit summation is only used explicitly when upper and lower indexes are not perfectly matched, but summation is still implied.  Readers of texts that use summation convention can check for proper matching of upper and lower indexes to ensure that the expressions make sense.  Such matching is the reason a mixed index Kronecker delta has been used in the definition of the reciprocal frame.}.

\subsubsection{Vector coordinates.}
The most important application of a reciprocal frame is for the computation of the coordinates of a vector with respect to a non-orthonormal frame.
Let a vector \( \Ba \) have coordinates \( a^i \) with respect to a basis \( \setlr{ \Bx_i } \)
\begin{equation}\label{eqn:reciprocal:20}
\Ba = \sum_j a^j \Bx_j,
\end{equation}
where \( j \) in \( a^j \) is an index not a power\footnote{In tensor algebra, any index that is found in matched upper and lower index pairs, is known as a dummy summation index, whereas an index that is unmatched is known as a free index.  For example, in \( a^j b_{ij} \) (summation implied) \( j \) is a summation index, and \( i \) is a free index.  We are free to make a change of variables of any summation index, so for the same example we can write
\( a^k b_{ik} \).  These index tracking conventions are obvious when summation symbols are included explicitly, as we will do.}.

Dotting with the reciprocal frame vectors \( \Bx^i \) provides these coordinates \( a^i \)
\begin{equation}\label{eqn:reciprocal:40}
\begin{aligned}
\Ba \cdot \Bx^i
&= \lr{\sum_j a^j \Bx_j} \cdot \Bx^i \\
&= \sum_j a^j {\delta_j}^i \\
&= a^i.
\end{aligned}
\end{equation}

Alternatively, coordinates can be computed with respect to the reciprocal frame.  Let those coordinates be \( a_i \), so that
\begin{equation}\label{eqn:reciprocal:60}
\Ba = \sum_i a_i \Bx^i.
\end{equation}

Dotting with the basis vectors \( \Bx_i \) provides the reciprocal frame relative coordinates \( a_i \)
\begin{equation}\label{eqn:reciprocal:80}
\begin{aligned}
\Ba \cdot \Bx_i
&= \lr{\sum_j a_j \Bx^j} \cdot \Bx_i \\
&= \sum_j a_j {\delta^j}_i \\
&= a_i.
\end{aligned}
\end{equation}

We can summarize \cref{eqn:reciprocal:40} and \cref{eqn:reciprocal:80} by stating that a vector can be expressed in terms of coordinates relative to either the original or reciprocal basis as follows
\begin{equation}\label{eqn:reciprocal:420}
\Ba
= \sum_j \lr{ \Ba \cdot \Bx^j } \Bx_j
= \sum_j \lr{ \Ba \cdot \Bx_j } \Bx^j.
\end{equation}

In tensor algebra the basis is generally implied\footnote{
In tensor algebra, a vector, identified by the coordinates \( a^i \) is called a contravariant vector.
When that vector is identified by the coordinates \( a_i \) it is called a covariant vector.  These labels relate to how the coordinates transform with respect to norm preserving transformations.
We have no need of this nomenclature, since we never transform coordinates in isolation, but will always transform the coordinates along with their associated basis vectors.}.

%When doing tensor algebra manipulations, you'll generally have the freedom to swap any pairs of upper and lower indexes as done above.

An example of a 2D oblique Euclidean basis and a corresponding reciprocal basis is plotted in \cref{fig:obliqueReciprocal:obliqueReciprocalFig2}.
Also plotted are the superposition of the projections required to arrive at point \( (4,2) \) along the \( \Be_1, \Be_2 \) directions or the \( \Be^1, \Be^2 \) directions.
In this plot, neither of the reciprocal frame vectors \( \Be^i \) are orthogonal to the corresponding basis vectors \( \Be_i \).
When one of \( \Be_i \) is increased(decreased) in magnitude, there will be a corresponding decrease(increase) in the magnitude of \( \Be^i \), but if the orientation is remained fixed, the corresponding direction of the reciprocal frame vector stays the same.

\mathImageFigure{../figures/GAelectrodynamics/obliqueReciprocalFig2}{Oblique and reciprocal bases.}{fig:obliqueReciprocal:obliqueReciprocalFig2}{0.5}{obliqueReciprocal.nb}


         %
% Copyright © 2018 Peeter Joot.  All Rights Reserved.
% Licenced as described in the file LICENSE under the root directory of this GIT repository.
%
%{
%%%Any multivector can be expressed in terms of the curvilinear basis \( \setlr{ \Bx_{u_1}, \Bx_{u_2}, \cdots, \Bx_k} \), but computation of the curvilinear coordinates requires the reciprocal basis.
%%%
%%%For example, a vector \( \Bf \) constrained to the tangent space admits a representation
%%%
%%%\begin{dmath}\label{eqn:curvilinearDefinedCoordinates:380}
%%%\Bf = \sum_i a_i \Bx_{u_i}.
%%%\end{dmath}
%%%
%%%Dotting with \( \Bx^{u_j} \) gives
%%%
%%%\begin{dmath}\label{eqn:curvilinearDefinedCoordinates:280}
%%%\Bf \cdot \Bx^{u_j}
%%%= \sum_i a_i \Bx_{u_i} \cdot \Bx^{u_j}
%%%= \sum_i a_i {\delta^i}_j
%%%= a_j,
%%%\end{dmath}
%%%
%%%so
%%%\begin{dmath}\label{eqn:curvilinearDefinedCoordinates:300}
%%%\Bf = \sum_i \lr{ \Bf \cdot \Bx^{u_i} } \Bx_{u_i}.
%%%\end{dmath}
%%%
\subsubsection{Bivector coordinates.}
Higher grade multivector objects may also be represented in curvilinear coordinates.  Illustrating by example, we will calculate the coordinates of a
bivector constrained to a three parameter manifold \( \Span \setlr{ \Bx_1, \Bx_2, \Bx_3 } \) which can be represented as
\begin{equation}\label{eqn:curvilinearDefinedCoordinates:320}
B
= \inv{2} \sum_{i, j} B^{ij} \Bx_{i} \wedge \Bx_{j}
= \sum_{i < j} B^{ij} \Bx_{i} \wedge \Bx_{j}.
\end{equation}

The coordinates \( B^{ij} \) can be determined by dotting \( B \) with \( \Bx^{j} \wedge \Bx^{i} \), where \( i \ne j \), yielding
\begin{equation}\label{eqn:curvilinearDefinedCoordinates:340}
\begin{aligned}
B \cdot \lr{ \Bx^{j} \wedge \Bx^{i} }
&= \inv{2} \sum_{r, s} B^{rs} \lr{ \Bx_{r} \wedge \Bx_{s} } \cdot \lr{ \Bx^{j} \wedge \Bx^{i} } \\
&= \inv{2} \sum_{r, s} B^{rs} \lr{ \lr{ \Bx_{r} \wedge \Bx_{s} } \cdot \Bx^{j} } \cdot \Bx^{i} \\
&= \inv{2} \sum_{r, s} B^{rs} \lr{ \Bx_{r} {\delta_s}^j - \Bx_{s} {\delta_r}^j } \cdot \Bx^{i} \\
&= \inv{2} \sum_{r, s} B^{rs} \lr{ {\delta_r}^i {\delta_s}^j - {\delta_s}^i {\delta_r}^j } \\
&= \inv{2} \lr{ B^{i j} - B^{j i} }.
\end{aligned}
\end{equation}

We see that the coordinates of a bivector, even with respect to a non-orthonormal basis, are antisymmetric, so
\cref{eqn:curvilinearDefinedCoordinates:340} is just \( B^{ij} \) as claimed.  That is
\begin{equation}\label{eqn:curvilinearDefinedCoordinates:401}
B^{ij} = B \cdot \lr{ \Bx^{j} \wedge \Bx^{i} }.
\end{equation}
Just as the reciprocal frame was instrumental for computation of the coordinates of a vector with respect to an arbitrary (i.e. non-orthonormal frame), we use the reciprocal frame to calculate the coordinates of a bivector, and could do the same for higher grade k-vectors as well.
%}

      \subsection{\R{2} reciprocal frame.}
         %
% Copyright © 2018 Peeter Joot.  All Rights Reserved.
% Licenced as described in the file LICENSE under the root directory of this GIT repository.
%
%{
How are the reciprocal frame vectors computed?  While these vectors have a natural GA representation, this is not intrinsically a GA problem, and can be solved with standard linear algebra, using a matrix inversion.
For example, given a 2D basis \( \setlr{ \Bx_1, \Bx_2 } \), the reciprocal basis can be assumed to have a coordinate representation in the original basis
\begin{equation}\label{eqn:reciprocal_R2:100}
\begin{aligned}
\Bx^1 &= a \Bx_1 + b \Bx_2 \\
\Bx^2 &= c \Bx_1 + d \Bx_2.
\end{aligned}
\end{equation}

Imposing the constraints of \cref{dfn:reciprocal:frame} leads to a pair of 2x2 linear systems that are easily solved to find
\begin{equation}\label{eqn:reciprocal_R2:120}
\begin{aligned}
\Bx^1 &= \inv{ (\Bx_1)^2 (\Bx_2)^2 - \lr{ \Bx_1 \cdot \Bx_2}^2 } \lr{ (\Bx_2)^2 \Bx_1 - \lr{ \Bx_1 \cdot \Bx_2 } \Bx_2 } \\
\Bx^2 &= \inv{ (\Bx_1)^2 (\Bx_2)^2 - \lr{ \Bx_1 \cdot \Bx_2}^2 } \lr{ (\Bx_1)^2 \Bx_2 - \lr{ \Bx_1 \cdot \Bx_2 } \Bx_1 }.
\end{aligned}
\end{equation}

The reader may notice that for \R{3} the denominator is related to the norm of the cross product \( \Bx_1 \cross \Bx_2 \).
More generally, this can be expressed as the square of the bivector \( \Bx_1 \wedge \Bx_2 \)
\begin{equation}\label{eqn:reciprocal_R2:140}
\begin{aligned}
-\lr{\Bx_1 \wedge \Bx_2 }^2
&= -\lr{\Bx_1 \wedge \Bx_2 } \cdot \lr{\Bx_1 \wedge \Bx_2 } \\
&= -\lr{ \lr{\Bx_1 \wedge \Bx_2 } \cdot \Bx_1 } \cdot \Bx_2 \\
&= (\Bx_1)^2 (\Bx_2)^2 - \lr{\Bx_1 \cdot \Bx_2}^2.
\end{aligned}
\end{equation}

Additionally, the numerators are each dot products of \( \Bx_1, \Bx_2 \) with that same bivector
\begin{equation}\label{eqn:reciprocal_R2:160}
\begin{aligned}
\Bx^1 &= \frac{\Bx_2 \cdot \lr{ \Bx_1 \wedge \Bx_2 } }{ \lr{ \Bx_1 \wedge \Bx_2}^2 } \\
\Bx^2 &= \frac{\Bx_1 \cdot \lr{ \Bx_2 \wedge \Bx_1 } }{ \lr{ \Bx_1 \wedge \Bx_2}^2 },
\end{aligned}
\end{equation}
or
\boxedEquation{eqn:reciprocal_R2:180}{
\begin{aligned}
\Bx^1 &= \Bx_2 \cdot \inv{ \Bx_1 \wedge \Bx_2 } \\
\Bx^2 &= \Bx_1 \cdot \inv{ \Bx_2 \wedge \Bx_1 }.
\end{aligned}
}

Recall that dotting with the unit bivector of a plane (or its inverse) rotates a vector in that plane by \( \pi/2 \).
In a plane subspace, such a rotation is exactly the transformation to ensure that \( \Bx_1 \cdot \Bx^2 = \Bx_2 \cdot \Bx^1 = 0 \).
This shows that the reciprocal frame for the basis of a two dimensional subspace is found by a duality transformation of each of the curvilinear coordinates, plus a subsequent scaling operation.
As \( \Bx_1 \wedge \Bx_2 \), the pseudoscalar for the subspace spanned by \( \setlr{ \Bx_1, \Bx_2 } \), is not generally a unit bivector, the dot product with its inverse also has a scaling effect.

%\makeexample{Numerical example}{example:reciprocal_R2:320}{
\paragraph{Numerical example:}
\iftoggle{kindle-version}{It is left to the reader to calculate the reciprocal frame depicted in \cref{fig:obliqueReciprocal:obliqueReciprocalFig2}.
One should find that the defining property \( \Bx_i \cdot \Bx^j = {\delta_i}^j \) of the reciprocal frame vectors is satisfied.
}
{
Here is a Mathematica calculation of the reciprocal frame depicted in \cref{fig:obliqueReciprocal:obliqueReciprocalFig2}

\begin{mmaCell}[moredefined={x1, x2, inverse, e, x12, OuterProduct, GeometricProduct, x12inverse, s1, InnerProduct, s2, dots},morepattern={a_, a, b_, b}]{Input}
  ClearAll[x1, x2, inverse]
  x1 = e[1] + e[2]; x2 = e[1] + 2 e[2];
  x12 = OuterProduct[x1, x2];
  inverse[a_] := a / GeometricProduct[a, a] ;
  x12inverse = inverse[x12];
  s1 = InnerProduct[x2, x12inverse];
  s2 = InnerProduct[x1, -x12inverse];
  s1
  s2
  dots[a_,b_] := \{a , "\(\pmb{\cdot}\)", b, " = ",
                   InnerProduct[a // ReleaseHold, b // ReleaseHold]\};
  MapThread[dots, \{\{x1 // HoldForm, x2 // HoldForm,
                      x1 // HoldForm, x2 // HoldForm\},
                  \{s1 // HoldForm, s1 // HoldForm,
                    s2 // HoldForm, s2 // HoldForm\}\}] // Grid
\end{mmaCell}
\begin{mmaCell}{Output}
  2 e[1] - e[2]
\end{mmaCell}
\begin{mmaCell}{Output}
  -e[1] + e[2]
\end{mmaCell}
\begin{mmaCell}{Output}
  x1	\(\cdot\)	s1	 = 	1
  x2	\(\cdot\)	s1	 = 	0
  x1	\(\cdot\)	s2	 = 	0
  x2	\(\cdot\)	s2	 = 	1
\end{mmaCell}

This shows the reciprocal vector calculations using \cref{eqn:reciprocal_R2:180} and that the
%This gives \( \Bx^1 = 2 \Be_1 - \Be_2, \Bx^2 = \Be_2 - \Be_1 \), which
%satisfies the
defining property
\( \Bx_i \cdot \Bx^j = {\delta_i}^j \)
of the reciprocal frame vectors
is satisfied.
}
%} % example

\paragraph{Example: \R{2}:}
Given a pair of arbitrary oriented vectors in \R{2}, \( \Bx_1 = a_1 \Be_1 + a_2 \Be_2, \Bx_2 = b_1 \Be_1 + b_2 \Be_2 \), the pseudoscalar associated with the basis \( \setlr{ \Bx_1, \Bx_2} \) is
\begin{equation}\label{eqn:reciprocal_R2:260}
\begin{aligned}
\Bx_1 \wedge \Bx_2
&= \lr{ a_1 \Be_1 + a_2 \Be_2 } \wedge \lr{ b_1 \Be_1 + b_2 \Be_2 } \\
&= \lr{ a_1 b_2 - a_2 b_1 } \Be_{12}.
\end{aligned}
\end{equation}

The inverse of this pseudoscalar is
\begin{equation}\label{eqn:reciprocal_R2:280}
\inv{\Bx_1 \wedge \Bx_2 }
=
\inv{ a_1 b_2 - a_2 b_1 } \Be_{21}.
\end{equation}

So for this fixed oblique \R{2} basis, the reciprocal frame is just
\begin{equation}\label{eqn:reciprocal_R2:300}
\begin{aligned}
\Bx^1 &= \Bx_2 \frac{\Be_{21}}{ a_1 b_2 - a_2 b_1 } \\
\Bx^2 &= \Bx_1 \frac{\Be_{12}}{ a_1 b_2 - a_2 b_1 }.
\end{aligned}
\end{equation}

The vector \( \Bx^1 \) is obtained by rotating \( \Bx_2 \) by \( -\pi/2 \), and rescaling it by the area of the parallelogram spanned by \( \Bx_1, \Bx_2 \).
The vector \( \Bx^2 \) is obtained with the same scaling plus a rotation of \( \Bx_1 \) by \( \pi/2 \).

%}

      \subsection{\R{3} reciprocal frame.}
         %
% Copyright © 2018 Peeter Joot.  All Rights Reserved.
% Licenced as described in the file LICENSE under the root directory of this GIT repository.
%
%{

In this section we generalize \cref{eqn:reciprocal_R2:180} to \R{3} vectors, which will illustrate the general case by example.

Given
a subspace spanned by a three vector basis \( \setlr{ \Bx_1, \Bx_2, \Bx_3 } \) the reciprocal frame vectors can be written as dot products
\begin{dmath}\label{eqn:reciprocal_R3:320}
\begin{aligned}
\Bx^1 &= \lr{ \Bx_2 \wedge \Bx_3 } \cdot \lr{ \Bx^3 \wedge \Bx^2 \wedge \Bx^1 } \\
\Bx^2 &= \lr{ \Bx_3 \wedge \Bx_1 } \cdot \lr{ \Bx^1 \wedge \Bx^3 \wedge \Bx^2 } \\
\Bx^3 &= \lr{ \Bx_1 \wedge \Bx_2 } \cdot \lr{ \Bx^2 \wedge \Bx^1 \wedge \Bx^3 }.
\end{aligned}
\end{dmath}

Each of those trivector terms equals \( - \Bx^1 \wedge \Bx^2 \wedge \Bx^3 \) and can be related to the (known) pseudoscalar \( \Bx_1 \wedge \Bx_2 \wedge \Bx_3 \) by observing that
\begin{equation}\label{eqn:reciprocal_R3:340}
\begin{aligned}
\lr{ \Bx^1 \wedge \Bx^2 \wedge \Bx^3 } \cdot \lr{ \Bx_3 \wedge \Bx_2 \wedge \Bx_1 }
&= \Bx^1 \cdot \lr{ \Bx^2 \cdot \lr{ \Bx^3 \cdot \lr{ \Bx_3 \wedge \Bx_2 \wedge \Bx_1 } }} \\
&= \Bx^1 \cdot \lr{ \Bx^2 \cdot \lr{ \Bx_2 \wedge \Bx_1 } } \\
&= \Bx^1 \cdot \Bx_1 \\
&= 1,
\end{aligned}
\end{equation}
which means that
\begin{dmath}\label{eqn:reciprocal_R3:360}
-\Bx^1 \wedge \Bx^2 \wedge \Bx^3
= -\inv{ \Bx_3 \wedge \Bx_2 \wedge \Bx_1 }
= \inv{ \Bx_1 \wedge \Bx_2 \wedge \Bx_3 },
\end{dmath}
and
\boxedEquation{eqn:reciprocal:380}{
\begin{aligned}
\Bx^1 &= \lr{ \Bx_2 \wedge \Bx_3 } \cdot \inv{ \Bx_1 \wedge \Bx_2 \wedge \Bx_3 } \\
\Bx^2 &= \lr{ \Bx_3 \wedge \Bx_1 } \cdot \inv{ \Bx_1 \wedge \Bx_2 \wedge \Bx_3 } \\
\Bx^3 &= \lr{ \Bx_1 \wedge \Bx_2 } \cdot \inv{ \Bx_1 \wedge \Bx_2 \wedge \Bx_3 }
\end{aligned}
}

Geometrically, dotting with this trivector is a duality transformation within the subspace spanned by the three vectors \( \Bx_1, \Bx_2, \Bx_3 \), also scaling the result so that the \( \Bx_i \cdot \Bx^j = {\delta_i}^j \) condition is satisfied.  The scaling factor is the volume of the parallelepiped spanned by \( \Bx_1, \Bx_2, \Bx_3 \).
%}

      \subsection{Problems.}
         %
% Copyright © 2016 Peeter Joot.  All Rights Reserved.
% Licenced as described in the file LICENSE under the root directory of this GIT repository.
%
\index{reciprocal frame}
\makeproblem{Reciprocal frame for two dimensional subspace.}{problem:reciprocal:2dsubspaceRecip}{
Prove \cref{eqn:reciprocal_R2:120}.
} % problem

\makeanswer{problem:reciprocal:2dsubspaceRecip}{

Assuming the representation of \cref{eqn:reciprocal_R2:100}, the dot products are
\begin{dmath}\label{eqn:2dreciprocalMatrixCalculation:200}
\begin{aligned}
1 &= \Bx_1 \cdot \Bx^1 = a \Bx_1^2 + b \Bx_1 \cdot \Bx_2 \\
0 &= \Bx_2 \cdot \Bx^1 = a \Bx_2 \cdot \Bx_1 + b \Bx_2^2 \\
0 &= \Bx_1 \cdot \Bx^2 = c \Bx_1^2 + d \Bx_1 \cdot \Bx_2 \\
1 &= \Bx_2 \cdot \Bx^2 = c \Bx_2 \cdot \Bx_1 + d \Bx_2^2.
\end{aligned}
\end{dmath}

This can be written out as a pair of matrix equations
\begin{dmath}\label{eqn:2dreciprocalMatrixCalculation:220}
\begin{aligned}
\begin{bmatrix}
1 \\
0
\end{bmatrix}
&=
\begin{bmatrix}
\Bx_1^2 & \Bx_1 \cdot \Bx_2 \\
\Bx_2 \cdot \Bx_1 & \Bx_2^2 \\
\end{bmatrix}
\begin{bmatrix}
a \\
b
\end{bmatrix} \\
\begin{bmatrix}
0 \\
1
\end{bmatrix}
&=
\begin{bmatrix}
\Bx_1^2 & \Bx_1 \cdot \Bx_2 \\
\Bx_2 \cdot \Bx_1 & \Bx_2^2 \\
\end{bmatrix}
\begin{bmatrix}
c \\
d
\end{bmatrix}.
\end{aligned}
\end{dmath}

The matrix inverse is
\begin{equation}\label{eqn:2dreciprocalMatrixCalculation:240}
{
\begin{bmatrix}
\Bx_1^2 & \Bx_1 \cdot \Bx_2 \\
\Bx_2 \cdot \Bx_1 & \Bx_2^2 \\
\end{bmatrix}
}^{-1}
=
\inv{ \Bx_1^2 \Bx_2^2 - \lr{\Bx_1 \cdot \Bx_2}^2 }
\begin{bmatrix}
\Bx_2^2 & -\Bx_1 \cdot \Bx_2 \\
-\Bx_2 \cdot \Bx_1 & \Bx_1^2 \\
\end{bmatrix},
\end{equation}
and multiplying by the \( (1,0) \), and \( (0,1) \) vectors picks out the respective columns, giving \cref{eqn:reciprocal_R2:120}.
} % answer

         %
% Copyright © 2016 Peeter Joot.  All Rights Reserved.
% Licenced as described in the file LICENSE under the root directory of this GIT repository.
%

\index{reciprocal frame}
\makeproblem{Two vector reciprocal frame}{problem:2subspaceR3reciprocalExample:2subspaceR3reciprocalExample}{
Calculate the reciprocal frame for the \R{3} subspace spanned by \( \setlr{ \Bx_1, \Bx_2 } \) where
\begin{dmath}\label{eqn:2subspaceR3reciprocalExample:20}
\begin{aligned}
\Bx_1 &= \Be_1 + 2 \Be_2 \\
\Bx_2 &= \Be_2 - \Be_3.
\end{aligned}
\end{dmath}
} % problem

\makeanswer{problem:2subspaceR3reciprocalExample:2subspaceR3reciprocalExample}{
The bivector for the plane spanned by this basis is
\begin{equation}\label{eqn:2subspaceR3reciprocalExample:40}
\begin{aligned}
\Bx_1 \wedge \Bx_2
&= \lr{ \Be_1 + 2 \Be_2 } \wedge \lr{ \Be_2 - \Be_3 } \\
&= \Be_{12} - \Be_{13} - 2 \Be_{23} \\
&= \Be_{12} + \Be_{31} + 2 \Be_{32}.
\end{aligned}
\end{equation}

This has the square
\begin{equation}\label{eqn:2subspaceR3reciprocalExample:60}
\begin{aligned}
\lr{ \Bx_1 \wedge \Bx_2 }^2
&=
\lr{ \Be_{12} + \Be_{31} + 2 \Be_{32} }
\cdot
\lr{ \Be_{12} + \Be_{31} + 2 \Be_{32} } \\
&= -1 -1 -4 \\
&= -6.
\end{aligned}
\end{equation}

Dotting \( -\Bx_1 \) with the bivector is
\begin{equation}\label{eqn:2subspaceR3reciprocalExample:80}
\begin{aligned}
\Bx_1 \cdot \lr{ \Bx_2 \wedge \Bx_1 }
&= -\lr{ \Be_1 + 2 \Be_2 } \cdot \lr{\Be_{12} + \Be_{31} + 2 \Be_{32} } \\
&= -\lr{ \Be_2 - \Be_3 - 2 \Be_1 - 4 \Be_3 } \\
&= 2 \Be_1 - \Be_2 + 5 \Be_3.
\end{aligned}
\end{equation}

For \( \Bx_2 \) the dot product with the bivector is
\begin{equation}\label{eqn:2subspaceR3reciprocalExample:100}
\begin{aligned}
\Bx_2 \cdot \lr{ \Bx_1 \wedge \Bx_2 }
&= \lr{ \Be_2 - \Be_3 } \cdot \lr{\Be_{12} + \Be_{31} + 2 \Be_{32} } \\
&= - \Be_1 - 2 \Be_3 - \Be_1 - 2 \Be_2 \\
&= - 2 \Be_1 - 2 \Be_2 - 2 \Be_3,
\end{aligned}
\end{equation}
so
\begin{equation}\label{eqn:2subspaceR3reciprocalExample:120}
\begin{aligned}
\Bx^1 &= \inv{3} \lr{ \Be_1 + \Be_2 + \Be_3 } \\
\Bx^2 &= \inv{6} \lr{ -2 \Be_1 + \Be_2 - 5 \Be_3 }.
\end{aligned}
\end{equation}
It is easy to verify that this has the desired semantics.
} % answer

   \section{Curvilinear coordinates.}
      \subsection{Two parameters.}
         %
% Copyright © 2017 Peeter Joot.  All Rights Reserved.
% Licenced as described in the file LICENSE under the root directory of this GIT repository.
%
\index{curvilinear coordinates}
Curvilinear coordinates can be defined for any subspace spanned by a parameterized vector into that space.
%Consider a continuous subspace parameterized by a two parameter vector function \( \Bx = \Bx(u_1, u_2) \) that is differentiable with respect to either parameter
As an example, consider a two parameter planar subspace parameterized by the following continuous vector function
%\begin{equation}\label{eqn:curvilinearDefined:480}
%\Bx(u_1, u_2) = u_1 \Be_1 \frac{\sqrt{3}}{2} \cosh\lr{ \Atanh(1/2) + \Be_{12} u_2 },
%\end{equation}
\begin{equation}\label{eqn:curvilinearDefined:480}
\Bx(u_1, u_2) = u_1 \lr{ \Be_1 \cos u_2 + \inv{2} \Be_2 \sin u_2 },
\end{equation}
where \( u_1 \in [0,1] \) and \( u_2 \in [0, \pi/2] \).
This parameterization spans the first quadrant of the ellipse with semi-major axis length 1, and semi-minor axis length \( 1/2 \).
%\footnote{
A parameterization of an elliptic area may or may not be of much use in electrodynamics, but it happens to provide a non-trivial, yet simple, example of a non-orthonormal parameterization.
%}
Contours for this parameterization are plotted in \cref{fig:ellipticalContours:ellipticalContoursFig1}.
The radial contours are for fixed values of \( u_2 \) and the elliptical contours fix the value of \( u_1 \), and depict a set of elliptic curves
with a semi-major/major axis ratio of \( 1/2 \).
\pmathImageFigure{../figures/GAelectrodynamics/}{ellipticalContoursFig1}{Contours for an elliptical region.}{fig:ellipticalContours:ellipticalContoursFig1}{0.2}{ellipticalContoursFigures.nb}

We define a curvilinear basis associated with each point in the region by the partials
\begin{equation}\label{eqn:curvilinearDefined:80}
\begin{aligned}
\Bx_{1} &= \PD{u_1}{\Bx} \\
\Bx_{2} &= \PD{u_2}{\Bx}.
\end{aligned}
\end{equation}

For \cref{eqn:curvilinearDefined:480} our curvilinear basis elements are
\begin{equation}\label{eqn:curvilinearDefined:520}
%\begin{aligned}
%\Bx_{1} &= \Be_1 \frac{\sqrt{3}}{2} \cosh\lr{ \Atanh(1/2) + \Be_{12} u_2 } \\
%\Bx_{2} &= u_1 \Be_2 \frac{\sqrt{3}}{2} \sinh\lr{ \Atanh(1/2) + \Be_{12} u_2 }.
%\end{aligned}
\begin{aligned}
\Bx_1 &= \Be_1 \cos u_2 + \inv{2} \Be_2 \sin u_2 \\
\Bx_2 &= u_1 \lr{ - \Be_1 \sin u_2 + \inv{2} \Be_2 \cos u_2 },
\end{aligned}
\end{equation}

We form vector valued differentials for each parameter
\begin{equation}\label{eqn:curvilinearDefined:500}
\begin{aligned}
d\Bx_{1} &= \Bx_1 du_1 \\
d\Bx_{2} &= \Bx_2 du_2.
\end{aligned}
\end{equation}

For \cref{eqn:curvilinearDefined:480},
the values of these differentials \( d\Bx_1, d\Bx_2 \) with \( du_1 = du_2 = 0.1 \) are plotted
in
\cref{fig:ellipticalContours:ellipticalContoursFig2}
for the points
\begin{equation}\label{eqn:curvilinearDefined:600}
(u_1, u_2) = (0.7, 5 \pi/20), (0.9, 3 \pi/20), (1.0, 5 \pi/20)
\end{equation}
in
(dark-thick) red, blue and purple respectively.
\pmathImageFigure{../figures/GAelectrodynamics/\subfigdir/}{ellipticalContoursFig2}{Differentials for an elliptical parameterization.}{fig:ellipticalContours:ellipticalContoursFig2}{0.3}{ellipticalContoursFigures.nb}

In this case and in general there is no reason to presume that there is any orthonormality constraint on the basis \( \setlr{ \Bx_{1}, \Bx_{2} } \) for a given two parameter subspace.

Should we wish to calculate the reciprocal frame
for \cref{eqn:curvilinearDefined:480}, we would find
%(\cref{problem:ellipticproblem:10}) that
(\cref{problem:ellipseCurvilinear:200}) that
\begin{equation}\label{eqn:curvilinearDefined:540}
\begin{aligned}
\Bx^{1} &= \Be_1 \cos u_2 + 2 \Be_2 \sin u_2 \\
\Bx^{2} &= \inv{u_1} \lr{ - \Be_1 \sin u_2 + 2 \Be_2 \cos u_2 }
%\Bx^{1} &= \Be_1 \sqrt{3} \sinh\lr{ \Atanh(1/2) + \Be_{12} u_2 } \\
%\Bx^{2} &= \frac{\Be_2}{u_1} \sqrt{3} \cosh\lr{ \Atanh(1/2) + \Be_{12} u_2 }.
\end{aligned}
\end{equation}

These are plotted (scaled by \( da = 0.1 \) so they fit in the image nicely) in \cref{fig:ellipticalContours:ellipticalContoursFig2} using thin light arrows.

When evaluating surface integrals, we will form
oriented (bivector) area elements from the wedge product of the differentials
\begin{equation}\label{eqn:curvilinearDefined:60}
d^2 \Bx \equiv d\Bx_{1} \wedge d\Bx_{2}.
\end{equation}

This absolute value of this area element \( \sqrt{-(d^2 \Bx)^2} \) is the area of the parallelogram spanned by \( d\Bx_1, d\Bx_2 \).
In this example, all such area elements lie in the \( x-y \) plane, but that need not be the case.

Also note that we will only perform integrals for those parametrizations for which the area element \( d^2 \Bx \) is non-zero.

%If the spacing between the contours is made small enough, the boundaries of each partition will define a planar region at the point of evaluation.
%All points in the interior will be accessible by a combination of the vectors formed from the partials of \( \Bx \) at that point.
%
% Copyright � 2023 Peeter Joot.  All Rights Reserved.
% Licenced as described in the file LICENSE under the root directory of this GIT repository.
%
%{
%\input{../latex/blogpost.tex}
%\renewcommand{\basename}{ellipseCurvilinear}
%%\renewcommand{\dirname}{notes/phy1520/}
%\renewcommand{\dirname}{notes/ece1228-electromagnetic-theory/}
%%\newcommand{\dateintitle}{}
%%\newcommand{\keywords}{}
%
%\input{../latex/peeter_prologue_print2.tex}
%
%\usepackage{peeters_layout_exercise}
%\usepackage{peeters_braket}
%\usepackage{peeters_figures}
%\usepackage{siunitx}
%\usepackage{verbatim}
%%\usepackage{mhchem} % \ce{}
%%\usepackage{macros_bm} % \bcM
%%\usepackage{macros_qed} % \qedmarker
%%\usepackage{txfonts} % \ointclockwise
%
%\beginArtNoToc
%
%\generatetitle{XXX}
%%\chapter{XXX}
%%\label{chap:ellipseCurvilinear}
%
\makeproblem{Elliptic parameterization.}{problem:ellipseCurvilinear:200}{
An elliptical area can be parameterized as
\begin{equation}\label{eqn:ellipseCurvilinear:20}
   \Bx(u_1, u_2) = u_1 \lr{ \Be_1 \cos u_2 + \beta \Be_2 \sin u_2 },
\end{equation}
where \( \beta = \sqrt{1 - \epsilon^2} \), and \( \epsilon \) is the eccentricity of the ellipse.
\makesubproblem{}{problem:ellipseCurvilinear:200:a}
Compute the curvilinear vectors
\begin{equation}\label{eqn:ellipseCurvilinear:200}
\begin{aligned}
   \Bx_1 &= \PDi{u_1}{\Bx} \\
   \Bx_2 &= \PDi{u_2}{\Bx}.
\end{aligned}
\end{equation}
\makesubproblem{}{problem:ellipseCurvilinear:200:b}
Compute the reciprocal frame vectors
\begin{equation}\label{eqn:ellipseCurvilinear:220}
\begin{aligned}
\Bx^1 &= \Bx_2 \cdot \inv{ \Bx_1 \wedge \Bx_2 } \\
\Bx^2 &= -\Bx_1 \cdot \inv{ \Bx_1 \wedge \Bx_2 }.
\end{aligned}
\end{equation}
\makesubproblem{}{problem:ellipseCurvilinear:200:c}
Verify that \( \Bx_i \cdot \Bx^j = {\delta_i}^j \).
} % problem

\makeanswer{problem:ellipseCurvilinear:200}{
\makesubanswer{}{problem:ellipseCurvilinear:200:a}
The curvilinear basis associated with this parameterization can be computed by inspection
\begin{equation}\label{eqn:ellipseCurvilinear:40}
\begin{aligned}
   \Bx_1 &= \Be_1 \cos u_2 + \beta \Be_2 \sin u_2 \\
   \Bx_2 &= u_1 \lr{ -\Be_1 \sin u_2 + \beta \Be_2 \cos u_2 }.
\end{aligned}
\end{equation}
\makesubanswer{}{problem:ellipseCurvilinear:200:b}

We need to compute the area element first
\begin{equation}\label{eqn:ellipseCurvilinear:60}
\begin{aligned}
\Bx_1 \wedge \Bx_2
&= \lr{ \Be_1 \cos u_2 + \beta \Be_2 \sin u_2 } \wedge u_1 \lr{ -\Be_1 \sin u_2 + \beta \Be_2 \cos u_2 } \\
&= u_1 \gpgradetwo{ \lr{ \Be_1 \cos u_2 + \beta \Be_2 \sin u_2 } \lr{ -\Be_1 \sin u_2 + \beta \Be_2 \cos u_2 } } \\
&= u_1 \lr{
\beta \Be_{12} \cos^2 u_2 - \beta \Be_{21} \sin^2 u_2
} \\
&= u_1 \beta i,
\end{aligned}
\end{equation}
where \( i = \Be_{12} \).

The reciprocal frame vectors are given by
\begin{equation}\label{eqn:ellipseCurvilinear:80}
\begin{aligned}
\Bx^1
   &= \Bx_2 \cdot \inv{ \Bx_1 \wedge \Bx_2 } \\
   &= u_1 \lr{ -\Be_1 \sin u_2 + \beta \Be_2 \cos u_2 } \inv{ u_1 \beta i } \\
   %&= \lr{ - \inv{\beta} \Be_1 \sin u_2 + \Be_2 \cos u_2 } (-i)
   &= \inv{\beta} \Be_2 \sin u_2 + \Be_1 \cos u_2,
\end{aligned}
\end{equation}
\begin{equation}\label{eqn:ellipseCurvilinear:100}
\begin{aligned}
   \Bx^2
   &= -\Bx_1 \cdot \inv{ \Bx_1 \wedge \Bx_2 } \\
   &= - \lr{ \Be_1 \cos u_2 + \beta \Be_2 \sin u_2 } \inv{ u_1 \beta i } \\
   %&= \lr{ \Be_1 \cos u_2 + \beta \Be_2 \sin u_2 } \frac{i}{ u_1 \beta }
   &= \inv{u_1} \lr{ \inv{\beta} \Be_2 \cos u_2 - \Be_1 \sin u_2 }.
\end{aligned}
\end{equation}

\makesubanswer{}{problem:ellipseCurvilinear:200:c}
To verify that \( \Bx_i \cdot \Bx^j = {\delta_i}^j \) we can compute each of the dot products
\begin{equation}\label{eqn:ellipseCurvilinear:120}
\begin{aligned}
\Bx^1 \cdot \Bx_1
&= \gpgradezero{
\lr{ \Be_1 \cos u_2 + \beta \Be_2 \sin u_2 } \lr{ \inv{\beta} \Be_2 \sin u_2 + \Be_1 \cos u_2 }
} \\
&=
\cos^2 u_2 + \sin^2 u_2 \\
&= 1,
\end{aligned}
\end{equation}
\begin{equation}\label{eqn:ellipseCurvilinear:140}
\begin{aligned}
\Bx^2 \cdot \Bx_2
&= \gpgradezero{
u_1 \lr{ -\Be_1 \sin u_2 + \beta \Be_2 \cos u_2 } \inv{u_1} \lr{ \inv{\beta} \Be_2 \cos u_2 - \Be_1 \sin u_2 }
} \\
&=
% \lr{ -\Be_1 \sin u_2 + \beta \Be_2 \cos u_2 } \lr{ \inv{\beta} \Be_2 \cos u_2 - \Be_1 \sin u_2 }
\sin^2 u_2 + \cos^2 u_2 \\
&= 1,
\end{aligned}
\end{equation}
\begin{equation}\label{eqn:ellipseCurvilinear:160}
\begin{aligned}
\Bx^1 \cdot \Bx_2
&= \gpgradezero{
\lr{ \inv{\beta} \Be_2 \sin u_2 + \Be_1 \cos u_2 } u_1 \lr{ -\Be_1 \sin u_2 + \beta \Be_2 \cos u_2 }
} \\
&=
u_1 \sin u_2 \cos u_2 - u_1 \cos u_2 \sin u_2 \\
&= 0.
\end{aligned}
\end{equation}
\begin{equation}\label{eqn:ellipseCurvilinear:180}
\begin{aligned}
\Bx^2 \cdot \Bx_1
&= \gpgradezero{
\inv{u_1} \lr{ \inv{\beta} \Be_2 \cos u_2 - \Be_1 \sin u_2 } \lr{ \Be_1 \cos u_2 + \beta \Be_2 \sin u_2 }
} \\
&=
\inv{u_1} \lr{
\cos u_2 \sin u_2 - \sin u_2 \cos u_2
}
\\
&= 0.
\end{aligned}
\end{equation}
} % answer
%}
%\EndNoBibArticle

%
% Copyright © 2023 Peeter Joot.  All Rights Reserved.
% Licenced as described in the file LICENSE under the root directory of this GIT repository.
%
%{
%\input{../latex/blogpost.tex}
%\renewcommand{\basename}{ellipticproblem}
%\renewcommand{\dirname}{notes/phy1520/}
%\renewcommand{\dirname}{notes/ece1228-electromagnetic-theory/}
%\newcommand{\dateintitle}{}
%\newcommand{\keywords}{}
%
%\input{../latex/peeter_prologue_print2.tex}
%
%\usepackage{peeters_layout_exercise}
%\usepackage{peeters_braket}
%\usepackage{peeters_figures}
%\usepackage{amsthm}
%\usepackage{siunitx}
%\usepackage{verbatim}
%\usepackage{mhchem} % \ce{}
%\usepackage{macros_bm} % \bcM
%\usepackage{macros_qed} % \qedmarker
%\usepackage{txfonts} % \ointclockwise
%
%\beginArtNoToc
%
%\generatetitle{XXX}
%\chapter{XXX}
%\label{chap:ellipticproblem}
%
\makeproblem{Hyperbolic identities.}{lemma:ellipticproblem:280}{
Show that
\begin{equation}\label{eqn:ellipticproblem:300}
2 \cosh\lr{ \mu - i \theta } \sinh\lr{ \mu + i \theta } = \sinh(2 \mu) + i \sin(2 \theta).
\end{equation}
\begin{equation}\label{eqn:ellipticproblem:420}
2 \cosh\lr{ \mu } \sinh\lr{ \mu } = \sinh(2 \mu).
\end{equation}
\begin{equation}\label{eqn:ellipticproblem:440}
\cosh\lr{ \mu + i \theta } = \cosh\mu \cos\theta + i \sinh\mu \sin\theta.
\end{equation}
} % problem
\makeanswer{lemma:ellipticproblem:280}{
\begin{equation}\label{eqn:ellipticproblem:320}
\begin{aligned}
\cosh\lr{ \mu - i \theta } \sinh\lr{ \mu + i \theta }
&=
\frac{1}{4}
   \lr{ e^{\mu - i\theta} - e^{-\mu + i \theta } } \lr{ e^{\mu + i\theta} - e^{-\mu - i \theta } }
\\
&=
\frac{1}{4} \lr{
   e^{2 \mu} - e^{-2\mu} + e^{2 i \theta} - e^{-2 i \theta}
} \\
&=
\inv{2} \lr{  \sinh(2 \mu) + i \sin(2 \theta) }.
\end{aligned}
\end{equation}
The second identity follows from the first, setting \( \theta = 0 \).
Finally, for the third expanding the \( \cosh \) in terms of exponentials, we find
\begin{equation}\label{eqn:ellipticproblem:50}
\begin{aligned}
\cosh\lr{ \mu + i \theta }
&=
\frac{1}{2} \lr{ e^{\mu + i \theta} + e^{-\mu - i\theta} } \\
&=
\frac{e^\mu}{2} \lr{ \cos\theta + i \sin\theta }
+
\frac{e^{-\mu}}{2} \lr{ \cos\theta - i \sin\theta } \\
&=
\frac{ e^\mu + e^{-\mu} }{2} \cos\theta
+ i \frac{ e^\mu - e^{-\mu} }{2} \sin\theta \\
&=
\cosh\mu \cos\theta + i \sinh\mu \sin\theta.
\end{aligned}
\end{equation}
}

\makeproblem{Elliptic curvilinear and reciprocal basis.}{problem:ellipticproblem:10}{
\makesubproblem{}{problem:ellipticproblem:10:a}
Show that an ellipse can be parameterized by
\begin{equation}\label{eqn:ellipticproblem:20}
   \Bx = u_1 \Be_1 \cosh\lr{ \mu + i u_2 },
\end{equation}
where \( i = \Be_{12} \), and find the values of the semi-major and semi-minor axes.
\makesubproblem{}{problem:ellipticproblem:10:b}
Determine how \( \mu \) and the eccentricity \( \epsilon = \sqrt{1 - b^2/a^2} \) are related.
\makesubproblem{}{problem:ellipticproblem:10:c}
Compute the curvilinear and reciprocal frame vectors for the parameterization \( \Bx(u_1, u_2) \) above.
%, and use this to verify
%\cref{eqn:curvilinearDefined:520} and \cref{eqn:curvilinearDefined:540} respectively.
\makesubproblem{}{problem:ellipticproblem:10:d}
Check that \( \Bx^i \cdot \Bx_j = {\delta^i}_j \).
} % problem
\makeanswer{problem:ellipticproblem:10}{
\makesubanswer{}{problem:ellipticproblem:10:a}
Using the multiple angle \( \cosh \) expansion, we find
\begin{equation}\label{eqn:ellipticproblem:40}
\begin{aligned}
\Be_1 \cosh\lr{ \mu + i u_2 }
&=
\Be_1 \lr{
\cosh\mu \cos u_2 + i \sinh\mu \sin u_2
} \\
&=
\Be_1 \cosh\mu \cos u_2 + \Be_2 \sinh\mu \sin u_2,
\end{aligned}
\end{equation}
so
\begin{equation}\label{eqn:ellipticproblem:60}
\Bx = u_1 \Be_1 \cosh\lr{ \mu + i u_2 } = \Be_1 a \cos u_2 + \Be_2 b \sin u_2,
\end{equation}
where
\begin{equation}\label{eqn:ellipticproblem:80}
\begin{aligned}
   a &= u_1 \cosh\mu \\
   b &= u_1 \sinh\mu,
\end{aligned}
\end{equation}
are the semi-major and semi-minor axis values.
\makesubanswer{}{problem:ellipticproblem:10:b}
The eccentricity (squared) is
\begin{equation}\label{eqn:ellipticproblem:100}
\begin{aligned}
   \epsilon^2
   &= 1 - \tanh^2\mu \\
   &= \frac{\cosh^2\mu - \sinh^2\mu}{\cosh^2\mu} \\
   &= \inv{\cosh^2\mu},
\end{aligned}
\end{equation}
so the eccentricity is
\begin{equation}\label{eqn:ellipticproblem:120}
   \epsilon = \inv{\cosh\mu}.
\end{equation}
\makesubanswer{}{problem:ellipticproblem:10:c}
Our curvilinear basis vectors are
\begin{equation}\label{eqn:ellipticproblem:140}
\begin{aligned}
\Bx_1 &= \Be_1 \cosh\lr{ \mu + i u_2 } \\
\Bx_2 &= \Be_2 u_1 \sinh\lr{ \mu + i u_2 } \\
\end{aligned}
\end{equation}

To compute the reciprocals we need the area element
\begin{equation}\label{eqn:ellipticproblem:160}
\begin{aligned}
\Bx_1 \wedge \Bx_2
&=
\gpgradetwo{
   \Be_1 \cosh\lr{ \mu + i u_2 } \Be_2 u_1 \sinh\lr{ \mu + i u_2 }
} \\
&=
u_1 \gpgradetwo{
   i \cosh\lr{ \mu - i u_2 } \sinh\lr{ \mu + i u_2 }
} \\
&=
\frac{u_1}{2} \gpgradetwo{
   i \lr{ \sinh(2 \mu) + i \sin(2 u_2) }
} \\
&=
u_1 i \cosh \mu \sinh \mu.
\end{aligned}
\end{equation}

Our recipocal basis vectors are
\begin{equation}\label{eqn:ellipticproblem:240}
\begin{aligned}
   \Bx^1
   &= \Bx_2 \inv{ \Bx_1 \wedge \Bx_2 } \\
   &= \Be_2 u_1 \sinh\lr{ \mu + i u_2 } \inv{u_1 i \cosh \mu \sinh \mu} \\
   &= \Be_1 \frac{\sinh\lr{ \mu + i u_2 }}{\cosh \mu \sinh \mu},
\end{aligned}
\end{equation}
and
\begin{equation}\label{eqn:ellipticproblem:180}
\begin{aligned}
\Bx^2
   &= -\Bx_1 \inv{ \Bx_1 \wedge \Bx_2 } \\
   &= -\lr{ \Be_1 \cosh\lr{ \mu + i u_2 } } \inv{ u_1 i \cosh \mu \sinh \mu} \\
   &= \frac{ \Be_2 \cosh\lr{ \mu + i u_2 } }{ u_1 \cosh \mu \sinh \mu}.
\end{aligned}
\end{equation}
\makesubanswer{}{problem:ellipticproblem:10:d}
\begin{equation}\label{eqn:ellipticproblem:260}
\begin{aligned}
\Bx_1 \cdot \Bx^1
   &= \gpgradezero{
\Be_1 \cosh\lr{ \mu + i u_2 } \Be_1 \frac{\sinh\lr{ \mu + i u_2 }}{\cosh \mu \sinh \mu}
} \\
   &=
\inv{\cosh \mu \sinh \mu}
\gpgradezero{
\Be_1^2 \cosh\lr{ \mu - i u_2 } \lr{\sinh\lr{ \mu + i u_2 }}
} \\
   &=
\inv{\cosh \mu \sinh \mu}
\gpgradezero{
   \inv{2} \lr{ \sinh(2 \mu) + i \sin(2 u_2) }
} \\
&=
\frac{\sinh(2 \mu) }{2 \cosh \mu \sinh \mu} \\
&= 1.
\end{aligned}
\end{equation}
\begin{equation}\label{eqn:ellipticproblem:360}
\begin{aligned}
\Bx_2 \cdot \Bx^2
&=
\gpgradezero{
\Be_2 u_1 \sinh\lr{ \mu + i u_2 }
\frac{ \Be_2 \cosh\lr{ \mu + i u_2 } }{ u_1 \cosh \mu \sinh \mu}
} \\
&=
\inv{ \cosh \mu \sinh \mu}
\gpgradezero{
\sinh\lr{ \mu + i u_2 }
\Be_2^2 \cosh\lr{ \mu - i u_2 }
} \\
&= 1.
\end{aligned}
\end{equation}
\begin{equation}\label{eqn:ellipticproblem:380}
\begin{aligned}
\Bx_1 \cdot \Bx^2
&=
\gpgradezero{
\Be_1 \cosh\lr{ \mu + i u_2 }
\frac{ \Be_2 \cosh\lr{ \mu + i u_2 } }{ u_1 \cosh \mu \sinh \mu}
} \\
&=
\inv{ u_1 \cosh \mu \sinh \mu}
\gpgradezero{
   \Be_{12} \cosh\lr{ \mu - i u_2 } \cosh\lr{ \mu + i u_2 }
} \\
&=
\frac{ \Abs{ \cosh\lr{ \mu + i u_2 } }^2 }
{ u_1 \cosh \mu \sinh \mu}
\gpgradezero{
   \Be_{12}
} \\
&= 0.
\end{aligned}
\end{equation}
\begin{equation}\label{eqn:ellipticproblem:400}
\begin{aligned}
\Bx^1 \cdot \Bx_2
&=
\gpgradezero{
\Be_1 \frac{\sinh\lr{ \mu + i u_2 }}{\cosh \mu \sinh \mu}
\Be_2 u_1 \sinh\lr{ \mu + i u_2 }
} \\
&=
\frac{ u_1 \Abs{\sinh\lr{ \mu + i u_2 }}^2 }{\cosh \mu \sinh \mu} \gpgradezero{ \Be_{12} } \\
&= 0.
\end{aligned}
\end{equation}
} % answer
%}
%\EndNoBibArticle

\todo{Don't introduce the idea of tangent space until a 3D example.
Remove the \R{3} reference above, and keep this first example planar.}

At the point of evaluation, the span of these differentials is called the tangent space.
In this particular case the tangent space at all points in the region is the entire x-y plane.
These partials locally span the tangent space at a given point on the surface.
\subsubsection{Curved two parameter surfaces.}
Continuing to illustrate by example, let's now consider a non-planar two parameter surface
\begin{equation}\label{eqn:curvilinearDefined:560}
\Bx(u_1, u_2) =
(u_1-u_2)^2
\Be_1
+ (1-(u_2)^2 ) \Be_2
+ u_1 u_2 \Be_3.
\end{equation}

The curvilinear basis elements, and the area element, are
\begin{equation}\label{eqn:curvilinearDefined:580}
\begin{aligned}
\Bx_1 &= 2 (u_1 - u_2) \Be_1 + u_2 \Be_3 \\
\Bx_2 &= 2 (u_2 - u_1) \Be_1 - 2 u_2 \Be_2 + u_1 \Be_3 \\
\Bx_1 \wedge \Bx_2 &=
-4 u_2
\left(u_1-u_2\right)
\Be_{12}
+2 u_2^2 \Be_{23}
+ 2
\lr{u_1^2 - u_2^2}
\Be_{13}.
\end{aligned}
\end{equation}
Two examples of these vectors and the associated area element (rescaled to fit) is plotted in
\cref{fig:2dmanifold:2dmanifoldFig1}.
This plane is called the tangent space at the point in question, and has been evaluated at \( (u_1, u_2) = (0.5,0.5), (0.35, 0.75) \).
\pmathImageFigure{../figures/GAelectrodynamics/}{2dmanifoldFig1}{Two parameter manifold.}{fig:2dmanifold:2dmanifoldFig1}{0.3}{2dmanifoldPlot.nb}
%\mathImageFigure{../figures/GAelectrodynamics/twoParameterDifferentialFieldFig1}{Curvilinear coordinates along a two parameter surface.}{fig:twoParameterDifferentialField:twoParameterDifferentialFieldFig1}{0.3}{twoParameterDifferentialFieldFig1.nb}
The results of \cref{eqn:curvilinearDefined:580} can be calculated easily by hand for this particular parameterization, but also submit to symbolic calculation software.
%\iftoggle{kindle-version}{}{
Here's a complete example using CliffordBasic

\begin{mmaCell}[moredefined={$SetSignature}]{Input}
  << CliffordBasic`;
  $SetSignature=\{3,0\};
\end{mmaCell}

\begin{mmaCell}[moredefined={e, OuterProduct, GFactor},morepattern={u_, v_}]{Input}
  ClearAll[xp, x, x1, x2]
  (* Use dummy parameter values for the derivatives,
     and then switch them to function parameter values. *)
  xp :=  (a - b)^2  e[1] + (1 - b^2) e[2] + b a e[3];
  x[u_, v_] := xp /. \{a \(\pmb{\to}\) \mmaPat{u}, b\(\pmb{\to}\)\mmaPat{v}\};
  x1[u_, v_] := D[xp, a] /. \{a \(\pmb{\to}\) \mmaPat{u}, b\(\pmb{\to}\)\mmaPat{v}\};
  x2[u_, v_] := D[xp, b] /. \{a \(\pmb{\to}\) \mmaPat{u}, b\(\pmb{\to}\)\mmaPat{v}\};

  x1[u,v]
  x2[u,v]
  OuterProduct[x1[u, v], x2[u, v]] // GFactor
\end{mmaCell}
%  (* CliffordBasic display of wedge doesn't currently group by the wedge basis bivectors. *)

\begin{mmaCell}{Output}
  2 (u-v) e[1] + v e[3]
\end{mmaCell}

\begin{mmaCell}{Output}
  -2 (u - v) e[1] - 2 v e[2] + u e[3]
\end{mmaCell}

\begin{mmaCell}{Output}
  (-4 u v + 4 \mmaSup{v}{2}) e[1,2] + (2 \mmaSup{u}{2} - 2 \mmaSup{v}{2}) e[1,3] + 2 \mmaSup{v}{2} e[2,3]
\end{mmaCell}
%}

      \subsection{Three (or more) parameters.}
         %
% Copyright © 2018 Peeter Joot.  All Rights Reserved.
% Licenced as described in the file LICENSE under the root directory of this GIT repository.
%
%{
We can extend the previous two parameter subspace ideas to higher dimensional (or one dimensional) subspaces associated with a parameterization
\index{tangent space}
\index{curvilinear bases}
\index{oriented volume element}
\index{volume element}
\index{\(d^k \Bx\)}
\index{\(\Bx_i\)}
\makedefinition{Curvilinear bases and volume element}{dfn:curvilinearThree:280}{
Given a parameterization \( \Bx(u_1, u_2, \cdots, u_k) \) with \( k \) degrees of freedom, we define the curvilinear basis elements \( \Bx_i \) by the partials
\begin{equation*}
\Bx_{i} = \PD{u_i}{\Bx}.
\end{equation*}
The span of \( \setlr{ \Bx_{i} } \) at the point of evaluation is called the tangent space.
A subspace associated with a parameterization of this sort is also called a manifold.
The volume element for the subspace is
\begin{equation*}
d^k \Bx = du_1 du_2 \cdots du_k\,
\Bx_{1} \wedge
\Bx_{2} \wedge \cdots \wedge
\Bx_{k}.
\end{equation*}
Such a volume element is a k-vector.  The volume of the (hyper-) parallelepiped bounded by \( \setlr{ \Bx_{i} } \)  is \( \sqrt{\Abs{(d^k \Bx)^2}} \).
} % definition

We will assume that the parameterization is non-generate.
This means that the
volume element \( d^k \Bx \) is non-zero in the region of interest.
Note that a zero volume element implies a linear dependency in the curvilinear basis elements \( \Bx_i \).

Given a parameterization \( \Bx = \Bx(u,v,\cdots, w) \), we may write
\( \Bx_u, \Bx_v, \cdots, \Bx_w \) for the curvilinear basis elements, and
\( \Bx^u, \Bx^v, \cdots, \Bx^w \) for the reciprocal frame elements.
When doing so, sums over numeric indexes like \( \sum_i \Bx^i \Bx_i \) should be interpreted as a sum over all the parameter labels, i.e. \( \Bx^u \Bx_u + \Bx^v \Bx_v + \cdots \).
%}

      \subsection{Gradient.}
         %
% Copyright � 2018 Peeter Joot.  All Rights Reserved.
% Licenced as described in the file LICENSE under the root directory of this GIT repository.
%
%{
%%%\input{../latex/blogpost.tex}
%%%\renewcommand{\basename}{gradient}
%%%%\renewcommand{\dirname}{notes/phy1520/}
%%%\renewcommand{\dirname}{notes/ece1228-electromagnetic-theory/}
%%%%\newcommand{\dateintitle}{}
%%%%\newcommand{\keywords}{}
%%%
%%%\input{../latex/peeter_prologue_print2.tex}
%%%
%%%\usepackage{peeters_layout_exercise}
%%%\usepackage{peeters_braket}
%%%\usepackage{peeters_figures}
%%%\usepackage{siunitx}
%%%%\usepackage{mhchem} % \ce{}
%%%%\usepackage{macros_bm} % \bcM
%%%%\usepackage{macros_qed} % \qedmarker
%%%%\usepackage{txfonts} % \ointclockwise
%%%
%%%\beginArtNoToc
%%%
%%%\generatetitle{Gradient and vector derivative.}
%%%%\chapter{Gradient.}
%%%\label{chap:gradient}
%%%
%%%\paragraph{definition}
%%%%
% Copyright © 2018 Peeter Joot.  All Rights Reserved.
% Licenced as described in the file LICENSE under the root directory of this GIT repository.
%
%{
We can extend the previous two parameter subspace ideas to higher dimensional (or one dimensional) subspaces associated with a parameterization
\index{tangent space}
\index{curvilinear bases}
\index{oriented volume element}
\index{volume element}
\index{\(d^k \Bx\)}
\index{\(\Bx_i\)}
\makedefinition{Curvilinear bases and volume element}{dfn:curvilinearThree:280}{
Given a parameterization \( \Bx(u_1, u_2, \cdots, u_k) \) with \( k \) degrees of freedom, we define the curvilinear basis elements \( \Bx_i \) by the partials
\begin{equation*}
\Bx_{i} = \PD{u_i}{\Bx}.
\end{equation*}
The span of \( \setlr{ \Bx_{i} } \) at the point of evaluation is called the tangent space.
A subspace associated with a parameterization of this sort is also called a manifold.
The volume element for the subspace is
\begin{equation*}
d^k \Bx = du_1 du_2 \cdots du_k\,
\Bx_{1} \wedge
\Bx_{2} \wedge \cdots \wedge
\Bx_{k}.
\end{equation*}
Such a volume element is a k-vector.  The volume of the (hyper-) parallelepiped bounded by \( \setlr{ \Bx_{i} } \)  is \( \sqrt{\Abs{(d^k \Bx)^2}} \).
} % definition

We will assume that the parameterization is non-generate.
This means that the
volume element \( d^k \Bx \) is non-zero in the region of interest.
Note that a zero volume element implies a linear dependency in the curvilinear basis elements \( \Bx_i \).

Given a parameterization \( \Bx = \Bx(u,v,\cdots, w) \), we may write
\( \Bx_u, \Bx_v, \cdots, \Bx_w \) for the curvilinear basis elements, and
\( \Bx^u, \Bx^v, \cdots, \Bx^w \) for the reciprocal frame elements.
When doing so, sums over numeric indexes like \( \sum_i \Bx^i \Bx_i \) should be interpreted as a sum over all the parameter labels, i.e. \( \Bx^u \Bx_u + \Bx^v \Bx_v + \cdots \).
%}

%%%
%%%\paragraph{Gradient.}
%%%
With the introduction of the ideas of reciprocal frame and curvilinear coordinates, we are getting closer to be able to formulate the geometric algebra generalizations of vector calculus.

The next step in the required mathematical preliminaries for geometric calculus is to determine the form of the gradient with respect to curvilinear coordinates and the
parameters associated with those coordinates.

Suppose we have a vector parameterization of \R{N}
\begin{dmath}\label{eqn:gradient:60}
\Bx = \Bx(u_1, u_2, \cdots, u_N).
\end{dmath}

We can employ the chain rule to express the gradient in terms of derivatives with respect to \( u_i \)
\begin{dmath}\label{eqn:gradient:80}
\spacegrad
=
\sum_i \Be_i \PD{x_i}{}
=
\sum_{i,j} \Be_i
\PD{x_i}{u_j}
\PD{u_j}{}
=
\sum_j \lr{ \sum_i \Be_i \PD{x_i}{u_j} } \PD{u_j}{}
=
\sum_j \lr{ \spacegrad u_j } \PD{u_j}{}.
\end{dmath}

It turns out that the gradients of the parameters are in fact the reciprocal frame vectors

%
% Copyright � 2018 Peeter Joot.  All Rights Reserved.
% Licenced as described in the file LICENSE under the root directory of this GIT repository.
%
\maketheorem{Reciprocal frame vectors}{thm:curvilinearGradient:1}{
Given a curvilinear basis with elements \( \Bx_i = \PDi{u_i}{\Bx} \), the \textit{reciprocal frame} vectors are given by
\begin{equation*}
\Bx^i = \spacegrad u_i.
\end{equation*}
} % theorem

\begin{proof}
This can be proven by direct computation
\begin{equation}\label{eqn:gradient:20}
\begin{aligned}
\Bx^i \cdot \Bx_j
&= (\spacegrad u_i) \cdot \PD{u_j}{\Bx} \\
&= \sum_{r,s=1}^n \lr{ \Be_r \PD{x_r}{u_i} } \cdot \lr{ \Be_s \PD{u_j}{x_s} } \\
&= \sum_{r,s=1}^n (\Be_r \cdot \Be_s) \PD{x_r}{u_i} \PD{u_j}{x_s} \\
&= \sum_{r,s=1}^n \delta_{rs} \PD{x_r}{u_i} \PD{u_j}{x_s} \\
&= \sum_{r=1}^n \PD{x_r}{u_i} \PD{u_j}{x_r} \\
&= \PD{u_i}{u_j} \\
&= \delta_{ij}.
\end{aligned}
\end{equation}
This shows that \( \Bx^i = \spacegrad u_i \) has the properties required of the reciprocal frame, proving the theorem.
\end{proof}

We are now able to define the gradient with respect to an arbitrary set of parameters
\index{\(\spacegrad\)}
\index{\(\partial_i\)}
%
% Copyright � 2018 Peeter Joot.  All Rights Reserved.
% Licenced as described in the file LICENSE under the root directory of this GIT repository.
%
\maketheorem{Curvilinear representation of the gradient}{thm:curvilinearGradient:2}{
Given an N-parameter vector parameterization
\( \Bx = \Bx(u_1, u_2, \cdots, u_N) \)
of \R{N},
with curvilinear basis elements \( \Bx_i = \PDi{u_i}{\Bx} \), the \textit{gradient} is
\begin{equation*}
\spacegrad = \sum_i \Bx^i \PD{u_i}{}.
\end{equation*}
It is convenient to define \( \partial_i \equiv \PDi{u_i}{} \), so that the gradient can be expressed in mixed index representation
\begin{equation*}
\spacegrad = \sum_i \Bx^i \partial_i.
\end{equation*}
%or the same with sums over mixed indexes implied.
} % theorem


%}
%\EndArticle

      \subsection{Vector derivative.}
         %
% Copyright © 2018 Peeter Joot.  All Rights Reserved.
% Licenced as described in the file LICENSE under the root directory of this GIT repository.
%
%{

Given curvilinear coordinates defined on a subspace \cref{dfn:curvilinearThree:280}, we don't have enough parameters to define the gradient.  For calculus on the k-dimensional subspace, we define the vector derivative

\index{vector derivative}
\index{\(\boldpartial\)}
%
% Copyright � 2018 Peeter Joot.  All Rights Reserved.
% Licenced as described in the file LICENSE under the root directory of this GIT repository.
%
\makedefinition{Vector derivative}{dfn:gradient:100}{
Given a k-parameter vector parameterization
\( \Bx = \Bx(u_1, u_2, \cdots, u_k) \) of \R{N} with \( k \le N \),
and curvilinear basis elements \( \Bx_i = \PDi{u_i}{\Bx} \), the \textit{vector derivative} \( \boldpartial \) is defined as
\begin{equation*}
\boldpartial = \sum_{i=1}^k \Bx^i \partial_i.
\end{equation*}
} % theorem


When the dimension of the subspace (number of parameters) equals the dimension of the underlying vector space, the vector derivative equals the gradient.  Otherwise we can write
\begin{equation}\label{eqn:vectorDerivative:101}
\spacegrad = \boldpartial + \spacegrad_\perp,
\end{equation}
and can think of the vector derivative as the projection of the gradient onto the tangent space at the point of evaluation.

Please see \citep{aMacdonaldVAGC} for an excellent introduction of the reciprocal frame, the gradient, and the vector derivative, and for
details about the connectivity of the manifold ignored here.

%}

      \subsection{Examples.}
         \input{curvilinearExamples.tex}
      \subsection{Problems.}
         %
% Copyright � CCYY Peeter Joot.  All Rights Reserved.
% Licenced as described in the file LICENSE under the root directory of this GIT repository.
%
\makeproblem{Spherical coordinate basis orthogonality.}{problem:sphericaldot:1}{
\index{spherical coordinates}
Using scalar selection, show that the spherical curvilinear basis of \cref{eqn:curvilinearspherical:80} are all mutually orthogonal.
} % problem

\makeanswer{problem:sphericaldot:1}{
Computing the various dot products is made easier by noting that \( \Be_3 \) and \( e^{i \phi } \) commute, whereas \( e^{j\theta } \Be_3 = \Be_3 e^{-j\theta}, \Be_1 e^{i\phi} = e^{-i\phi} \Be_1, \Be_2 e^{i\phi} = e^{-i\phi} \Be_2 \) (since \( \Be_3 j \), \( \Be_1 i \) and \( \Be_2 i \) all anticommute)

\begin{subequations}
\label{eqn:sphericaldot:160}
\begin{equation}\label{eqn:sphericaldot:180}
\begin{aligned}
\Bx_r \cdot \Bx_\theta
&=
\gpgradezero{
\Be_3 e^{j \theta} \Be_1 e^{i\phi} e^{j \theta}
} \\
&=
\gpgradezero{
e^{j \theta} \Be_3 e^{j \theta} \Be_1 e^{i\phi}
} \\
&=
\gpgradezero{
\Be_3 e^{-j \theta} e^{j \theta} \Be_1 e^{i\phi}
} \\
&=
\gpgradezero{
\Be_3 \Be_1 e^{i\phi}
} \\
&= 0
\end{aligned}
\end{equation}
\begin{equation}\label{eqn:sphericaldot:200}
\begin{aligned}
\Bx_r \cdot \Bx_\phi
&=
\gpgradezero{
\Be_3 e^{j \theta} r \sin\theta \Be_2 e^{i \phi}
} \\
&=
r \sin\theta
\gpgradezero{
\Be_3 \lr{ \cos\theta + \Be_{31} \sin\theta e^{i\phi} } \Be_2 e^{i \phi}
} \\
&=
r \sin^2\theta
\gpgradezero{
\Be_{1} e^{i\phi} \Be_2 e^{i \phi}
} \\
&=
r \sin^2\theta
\gpgradezero{
\Be_{1} \Be_2
} \\
&=
0
\end{aligned}
\end{equation}
\begin{equation}\label{eqn:sphericaldot:220}
\begin{aligned}
\Bx_\theta \cdot \Bx_\phi
&=
r \sin\theta
\gpgradezero{
\Be_1 e^{i\phi} e^{j \theta}
\Be_2 e^{i \phi}
} \\
&=
r \sin\theta
\gpgradezero{
\Be_2 \Be_1 e^{j \theta}
} \\
&=
r \sin\theta
\gpgradezero{
\Be_2 \Be_1 \lr{ \cos\theta + \Be_{31} \sin\theta e^{i \phi} }
} \\
&=
r \sin^2\theta
\gpgradezero{
\Be_{32} e^{i \phi}
} \\
&=
0.
\end{aligned}
\end{equation}
\end{subequations}

} % answer

         %
% Copyright � CCYY Peeter Joot.  All Rights Reserved.
% Licenced as described in the file LICENSE under the root directory of this GIT repository.
%
\makeproblem{Spherical volume element pseudoscalar.}{problem:volumeselection:1}{
Using geometric algebra, perform the reduction of the grade three selection made in the final step of \cref{eqn:curvilinearspherical:300}.
} % problem
\makeanswer{problem:volumeselection:1}{
\begin{equation}\label{eqn:volumeselection:n}
\begin{aligned}
\Be_{31} e^{i\phi} \Be_2 e{i\phi}
&=
\Be_{31} e^{i\phi} e^{-i\phi} \Be_2 \\
&=
\Be_{312} \\
&= I.
\end{aligned}
\end{equation}
} % answer

         %
% Copyright © 2018 Peeter Joot.  All Rights Reserved.
% Licenced as described in the file LICENSE under the root directory of this GIT repository.
%
%{
\makeproblem{Spherical volume Jacobian.}{problem:curvilinearspherical:1}{
Without software, expand and simplify the determinant of \cref{eqn:curvilinearspherical:320}.
} % problem
\makeanswer{problem:curvilinearspherical:1}{
A bit of shorthand is useful.  We can write our Jacobian as
\begin{equation}\label{eqn:curvilinearsphericalProblem:20}
J
=
\begin{vmatrix}
S_\theta C_\phi & S_\theta S_\phi & C_\theta \\
r C_\theta C_\phi & r C_\theta S_\phi & - r S_\theta \\
- r S_\theta S_\phi & r S_\theta C_\phi & 0
\end{vmatrix}
=
r^2 \begin{vmatrix}
S_\theta C_\phi & S_\theta S_\phi & C_\theta \\
 C_\theta C_\phi &  C_\theta S_\phi & -  S_\theta \\
-  S_\theta S_\phi &  S_\theta C_\phi & 0
\end{vmatrix},
\end{equation}
where the common factor of the two last rows has been factored out.  Expanding the cofactors along the bottom row we have
\begin{equation}\label{eqn:curvilinearsphericalProblem:40}
\begin{aligned}
J
&=
-  r^2 S_\theta S_\phi
\begin{vmatrix}
S_\theta S_\phi & C_\theta \\
C_\theta S_\phi & -  S_\theta \\
\end{vmatrix}
-  r^2 S_\theta C_\phi
\begin{vmatrix}
S_\theta C_\phi & C_\theta \\
 C_\theta C_\phi &  -  S_\theta \\
\end{vmatrix} \\
&=
- r^2 S_\theta S_\phi^2
\begin{vmatrix}
S_\theta & C_\theta \\
C_\theta & -  S_\theta \\
\end{vmatrix}
-  r^2 S_\theta C_\phi^2
\begin{vmatrix}
S_\theta C_\theta \\
 C_\theta &  -  S_\theta \\
\end{vmatrix} \\
&=
- r^2 S_\theta S_\phi^2 \lr{ - S_\theta^2 - C_\theta^2 }
-  r^2 S_\theta C_\phi^2 \lr{ -S_\theta^2 - C_\theta^2 } \\
&=
r^2 S_\theta \lr{ S_\phi^2 + C_\phi^2 } \\
&=
r^2 S_\theta.
\end{aligned}
\end{equation}
Had we done a traditional first row or column determinant expansion, things would have been considerably messier.
} % problem
%}

         %
% Copyright © 2023 Peeter Joot.  All Rights Reserved.
% Licenced as described in the file LICENSE under the root directory of this GIT repository.
%
%{
\makeproblem{Curvilinear basis for toroidal parameterization.}{problem:toriodalProblem:1}{
Prove \cref{eqn:torusCenterOfMassParameterization:3}.
} % problem

\makeanswer{problem:toriodalProblem:1}{
We'll only compute \( \Bx_\theta \) here explicitly, as the other two vectors can be computed by inspection.

We start with a plain old chain rule expansion, with the cavaet that we must b e careful not to commute \( j \) with anything but the \( e^{\pm j \theta/2} \) terms.
\begin{equation}\label{eqn:toriodal:21}
\begin{aligned}
\Bx_\theta &= \PD{\theta}{\Bx} \\
&=
-\frac{j}{2}
e^{-j\theta/2} \left( \rho \Be_1 e^{ i \phi } + R \Be_3 \right) e^{j \theta/2} 
+
e^{-j\theta/2} \left( \rho \Be_1 e^{ i \phi } + R \Be_3 \right) e^{j \theta/2}
\frac{j}{2}
\end{aligned}
\end{equation}
Note that the bivector \( j \) commutes with \( \Be_1 \), and then proceed to compute the \( \rho \) dependent part of \( \Bx_\theta \)
\begin{equation}\label{eqn:toriodal:41}
\begin{aligned}
& \frac{\rho}{2} e^{-j \theta/2} \Be_1 \lr{ -j e^{i \phi}  + e^{i \phi} j } e^{ j\theta/2 } \\
&=\frac{\rho}{2} e^{-j \theta/2} \Be_1 \lr{ - \Be_{32} \lr{ \cos\phi + \Be_{13} \sin\phi} + \lr{ \cos\phi + \Be_{13} \sin\phi} \Be_{32} } e^{ j\theta/2 } \\
&=\frac{\rho}{2} e^{-j \theta/2} \Be_1 \lr{ - \Be_{3213}  \sin\phi + \Be_{1332} \sin\phi } e^{ j\theta/2 } \\
&=\frac{\rho}{2} e^{-j \theta/2} \Be_1 \lr{ - \Be_{21}  \sin\phi + \Be_{12} \sin\phi } e^{ j\theta/2 } \\
&=      \rho e^{-j \theta/2} \Be_{112} \sin\phi e^{ j\theta/2 } \\
&=      \rho e^{-j \theta/2} \Be_{2} \sin\phi e^{ j\theta/2 }.
%&=      \rho \sin\phi \Be_2 e^{j \theta/2} e^{ j\theta/2 } \\
%&=      \rho \sin\phi \Be_2 e^{j \theta}.
\end{aligned}
\end{equation}
Similarly, the \( R \) dependent contribution is
\begin{equation}\label{eqn:toriodal:61}
\begin{aligned}
& \frac{R}{2} e^{-j \theta/2} \lr{ -j \Be_3  + \Be_3 j } e^{ j\theta/2 } \\
& \frac{R}{2} e^{-j \theta/2} \lr{ -\Be_{323}  + \Be_{332} } e^{ j\theta/2 } \\
& \frac{R}{2} e^{-j \theta/2} \lr{ \Be_{2}  + \Be_{2} } e^{ j\theta/2 } \\
&       R     e^{-j \theta/2} \Be_{2}  e^{ j\theta/2 }.
\end{aligned}
\end{equation}
Putting the pieces together, we have
\begin{equation}\label{eqn:toriodal:81}
\Bx_\theta = e^{-j \theta/2} \lr{ R + \rho \sin\phi } \Be_{2}  e^{ j\theta/2 },
\end{equation}
as claimed.
} % answer
%}

   \section{Integration theory.}
      \subsection{Line integral.}
         %
% Copyright � 2018 Peeter Joot.  All Rights Reserved.
% Licenced as described in the file LICENSE under the root directory of this GIT repository.
%
%{
\index{differential form}
In geometric algebra, the integrand of a multivector line integral contains a product of multivector(s) and a single parameter differential.
\makedefinition{Multivector line integral.}{dfn:lineintegraldef:multivectorlineintegral}{
Given a continuous and differentiable curve described by a vector function \( \Bx(a) \), parameterized by single value \( a \) with differential
\begin{equation*}
d^1 \Bx \equiv d\Bx_a = \PD{a}{\Bx} da = \Bx_a da,
\end{equation*}
and multivector functions \( F, G \), the integral
\begin{equation*}
\int F d^1 \Bx G
\end{equation*}
is called a multivector line integral.
} % definition

An illustration of a single parameter curve and its
differential with respect to that parameter, is given in
\cref{fig:oneParameterDifferential:oneParameterDifferentialFig1}.
Observe that the differential is tangent to the curve at all points.
Possible physical realizations of the parameter describing the curve include
time, arclength, or angle.
\pmathImageFigure{../figures/GAelectrodynamics/}{oneParameterDifferentialFig1}{One parameter manifold.}{fig:oneParameterDifferential:oneParameterDifferentialFig1}{0.2}{oneParameterDifferentialFig.nb}

Suppose that \( \Bf(\Bx(a)) \) is a vector valued function defined along the curve.
The conventional line integral from vector calculus, a dot product of a differential and the function \( \Bf \)
may be obtained by the sum of two multivector line integrals one with \( F,G = \Bf/2,1 \), and the other with \( F,G = 1,\Bf/2 \)
\begin{equation}\label{eqn:lineintegraldef:20}
\int d \Bx \frac{\Bf}{2}
+\int
\frac{\Bf}{2}
d \Bx
=
\int d \Bx \cdot \Bf.
\end{equation}
Unlike the conventional dot product line integral, the multivector line integral of a vector function such as \( \int d \Bx \Bf \) is generally multivector valued, with both a scalar and a bivector component.  Let's consider some examples of multivector line integrals.

\paragraph{Example: Circular path.}
Let \( f(t) = \Ba t + \Bb t^2 \), where \( \Ba, \Bb \) are constant vectors, \( t \) is a scalar parameter, and the integration path is circular \( \Bx(t) = \Be_1 e^{i t} \), where \( i = \Be_1 \Be_2 \).
The line integral of \( \Bf\, d\Bx \) is
\begin{equation}\label{eqn:lineintegraldef:40}
\begin{aligned}
\int \Bf(t)\, d\Bx
&= \Ba \Be_2 \int t e^{i t} dt + \Bb \Be_2 \int t^2 e^{i t} dt \\
&= \lr{\Ba \Be_2 \lr{ 1 - i t } + \Bb \Be_2 \lr{ 2 i + 2 t - i t^2 } } e^{i t} \\
&= \lr{ \Ba + 2 \Bb t} \Be_2 e^{i t} + \lr{ \Ba t - 2 \Bb  + \Bb t^2 } \Be_1 e^{i t},
\end{aligned}
\end{equation}
and the
line integral of \( d\Bx\, \Bf \) is
\begin{equation}\label{eqn:lineintegraldef:60}
\begin{aligned}
\int d\Bx\, \Bf
&= \Be_2 \int t e^{i t} dt \Ba + \Be_2 \int t^2 e^{i t} dt \Bb \\
&= \Be_2 e^{i t} \lr{ \lr{ 1 - i t } \Ba + \lr{ 2 i + 2 t - i t^2 } \Bb } \\
&= \Be_2 e^{i t} \lr{ \Ba + 2 \Bb t} + \Be_1 e^{i t} \lr{ \Ba t - 2 \Bb  + \Bb t^2 }.
\end{aligned}
\end{equation}
Unless the vector constants \( \Ba, \Bb \) have only components along the z-axis, \cref{eqn:lineintegraldef:40} and \cref{eqn:lineintegraldef:60} are not generally equal.

\paragraph{Example: Circular bivector.}
Given a bivector valued function \( F(t) = \Be_2 \wedge \lr{ \Be_3 e^{it} } \), where \( i = \Be_{3} \Be_1 \), and a curve
\( \Bx(t) = \Be_3 + \Be_2 t + \Be_1 t^2/2 \), we can compute the line integral with the differential on the right
\begin{equation}\label{eqn:lineintegraldef:80}
\begin{aligned}
\int F\, d\Bx
&= \Be_{23} \int e^{i t} \lr{ \Be_2 + \Be_1 t} dt \\
&= -\Be_3 \int e^{it} dt + \Be_{123} \int t e^{-i t} dt \\
&= \Be_{1} e^{it} + \Be_{123} \lr{ 1 + i t} e^{-it} \\
&= \Be_1 e^{it} - \Be_2 t e^{-it} + \Be_{123} e^{-it},
\end{aligned}
\end{equation}
or the line integral with the differential on the left
\begin{equation}\label{eqn:lineintegraldef:100}
\begin{aligned}
\int d\Bx\, F
&= \int \lr{ \Be_2 + \Be_1 t} \Be_{23} e^{it} dt \\
&= \Be_{3} \int e^{it} dt + \Be_{123} \int t e^{it} \\
&= -\Be_{1} e^{it} + \Be_{123} \lr{ 1 - i t } e^{it} \\
&= -\Be_{1} e^{it} + \Be_{123} e^{it} + \Be_{2} t e^{it}.
\end{aligned}
\end{equation}
In both \cref{eqn:lineintegraldef:80} and \cref{eqn:lineintegraldef:100} the end result has both vector and trivector grades.  While both integrals are equal (zero) when the angular velocity parameter \( t \) is a multiple of \( 2 \pi \), this shows that the order of the products in the integrand makes a difference once again.

\paragraph{Example: Function with only scalar and pseudoscalar grades.}
In \R{3}, given any function with only scalar and pseudoscalar grades, say \( F(t) = f(t) + I g(t) \), where \( f, g \) are both scalar functions, then the order of the products in a line integrand do not matter.  For any such function we have
\begin{equation}\label{eqn:lineintegraldef:120}
\int F d\Bx = \int d\Bx F,
\end{equation}
since both the scalar and pseudoscalar grades commute with any vector differential.

%}

      \subsection{Surface integral.}
         %
% Copyright � 2018 Peeter Joot.  All Rights Reserved.
% Licenced as described in the file LICENSE under the root directory of this GIT repository.
%
%{
\index{area element}
\index{differential form}
\makedefinition{Multivector surface integral.}{dfn:lineintegraldef:multivectorsurfaceintegral}{
Given a continuous and differentiable surface described by a vector function \( \Bx(a, b) \), parameterized by two scalars \( a, b \) with differential
\begin{equation*}
d^2 \Bx \equiv d\Bx_a \wedge d\Bx_b =
\PD{a}{\Bx} \wedge \PD{b}{\Bx}
da db = {\Bx_a \wedge \Bx_b } da db,
\end{equation*}
and multivector functions \( F, G \), the integral
\begin{equation*}
\int F d^2 \Bx G
\end{equation*}
is called a multivector surface integral.
} % definition

%Unless \( F, G \) are both scalars, such a surface integral is not generally bivector valued like the area element.
An example of a two parameter surface, and the corresponding differentials with respect to those parameters, is illustrated in
\cref{fig:twoParameterDifferential:twoParameterDifferentialFig1}.

\mathImageFigure{../figures/GAelectrodynamics/\subfigdir/twoParameterDifferentialFig1}{Two parameter manifold differentials.}{fig:twoParameterDifferential:twoParameterDifferentialFig1}{0.4}{twoParameterDifferentialFig.nb}

In \R{3} it will often be convenient to utilize a dual representation of the area element \( d^2 \Bx = I \ncap dA \), where \( dA \) is a scalar area element, and \( \ncap \) is a normal vector to the surface.  With such an area element representation we will call \( I \int dA\, F \ncap G \) a surface integral.

\paragraph{Example: Spherical surface integral.}

From \cref{eqn:curvilinearspherical:300}, we know that
\begin{equation}\label{eqn:surfaceintegraldef:140}
\Bx_r \Bx_\theta \Bx_\phi = I r^2 \sin\theta,
\end{equation}
so
\begin{equation}\label{eqn:surfaceintegraldef:160}
\begin{aligned}
\Bx_\theta \wedge \Bx_\phi
&= \Bx_\theta \Bx_\phi \\
&= \Bx_r I r^2 \sin\theta,
\end{aligned}
\end{equation}
so the (bivector-valued) area element for a spherical surface is
\begin{equation}\label{eqn:surfaceintegraldef:180}
d^2 \Bx = I \Bx_r r^2 \sin\theta d\theta d\phi.
\end{equation}

Suppose we integrate a vector valued function \( F(\theta, \phi) = \alpha \Bx^r + \beta \Bx^\theta + \gamma \Bx^\phi \), where \( \alpha, \beta, \gamma\) are constants, over the surface of a sphere of radius \( r \), then the surface integral (with the area element on the right) is
\begin{dmath}\label{eqn:surfaceintegraldef:200}
\int F d^2\Bx
=
\alpha I r^2 \int \Bx^r \Bx_r \sin\theta d\theta d\phi
+
\beta I r^2 \int \Bx^\theta \Bx_r \sin\theta d\theta d\phi
+
\gamma I r^2 \int \Bx^\phi \Bx_r \sin\theta d\theta d\phi.
\end{dmath}
This can be simplified using \( \rcap \thetacap \phicap = I \), and \cref{eqn:curvilinearspherical:260}, to find
\begin{equation}\label{eqn:surfaceintegraldef:220}
\begin{aligned}
\Bx^r \Bx_r &= 1 \\
I \Bx^\theta \Bx_r &= \inv{r} I \thetacap \rcap = \inv{r} \phicap \\
I \Bx^\phi \Bx_r &= \inv{r \sin\theta} I \phicap \rcap = -\inv{r \sin\theta} \thetacap,
\end{aligned}
\end{equation}
so
\begin{dmath}\label{eqn:surfaceintegraldef:240}
\int F d^2\Bx =
\alpha I 4 \pi r^2
+
\beta r \int \phicap \sin\theta d\theta d\phi
-
\gamma r \int \thetacap d\theta d\phi
=
\alpha I 4 \pi r^2,
\end{dmath}
where the integrands containing \( \thetacap, \phicap \) are killed by the integral over \( \phi \in [0, 2\pi] \).  If integrated over a subset of the spherical surface, where such perfect cancellation does not occur, this surface integral may have both vector and trivector components.

\paragraph{Example: Bivector function.}
Given a bivector valued function \( F(a,b) = (a + b) \Be_2 \Be_1 + 2 (a \Be_1 - b \Be_2) \Be_3 \) defined over the unit square \( a,b \in [0, 1] \), and a surface \( \Bx(a,b) = a \Be_1 + b \Be_2 \), the multivector surface integral (with the area element on the right) is
\begin{equation}\label{eqn:surfaceintegraldef:260}
\begin{aligned}
\int F d^2 \Bx
&= \int_0^1\int_0^1 (a + b) \,da db + 2 \int_0^1\int_0^1 (a \Be_1 - b \Be_2) \Be_3 \Be_1 \Be_2\, da db \\
&= 1+ I \int_0^1 \evalrange{a^2}{0}{1} \Be_1 db - I \int_0^1 \evalrange{b^2}{0}{1} \Be_2 da \\
&= 1+ I \lr{ \Be_1 - \Be_2 } \\
&= 1+ \lr{ \Be_{1} + \Be_{2} } \Be_3.
\end{aligned}
\end{equation}
In this example, the integral of a bivector valued function over a (bivector-valued) surface area element results in a multivector with a scalar and bivector grade.  In higher dimensional spaces, such an integral may also have grade-4 components.
%}

      \subsection{Volume integral.}
         %
% Copyright � 2018 Peeter Joot.  All Rights Reserved.
% Licenced as described in the file LICENSE under the root directory of this GIT repository.
%
%{
\makedefinition{Multivector volume integral.}{dfn:volumeintegraldef:multivectorvolumeintegral}{
Given a continuous and differentiable volume described by a vector function \( \Bx(a, b,c) \), parameterized by scalars \( a, b, c \) with volume element
\begin{equation*}
d^3 \Bx \equiv
d\Bx_a
\wedge
d\Bx_b
\wedge
d\Bx_c
=
\PD{a}{\Bx}
\wedge
\PD{b}{\Bx}
\wedge
\PD{c}{\Bx}
\,da db dc = \Bx_a \Bx_b \Bx_c \, da db dc,
\end{equation*}
and multivector functions \( F, G \), the integral
\begin{equation*}
\int F d^3 \Bx G
\end{equation*}
is called a multivector volume integral.
} % definition

In \R{3} the volume element is always a pseudoscalar, which commutes with all grades, so we are free to write \( \int F d^3 \Bx G = \int d^3 \Bx F G \) for any multivectors \( F, G \).  It will often be useful to make the pseudoscalar nature of the volume element explicit, writing \( d^3 \Bx = I dV \), where \( dV \) is a scalar volume element.

As an example, let \( F(\Bx) = r(\Bx) + \Bs(\Bx) + I \Bt(\Bx) + I u(\Bx) \) be an arbitrary multivector function in \R{3}, where \( r, u \) are scalar functions and \( \Bs, \Bt \) are vector functions.
Integrating over a unit cube in rectangular coordinates \( d^3 \Bx = I dx dy dz = I dV \),
the volume integral of such a multivector function is
\begin{equation}\label{eqn:volumeintegraldef:20}
\begin{aligned}
\int F d^3 \Bx
&= \int \lr{ r(\Bx) + \Bs(\Bx) + I \Bt(\Bx) + I u(\Bx) } I dV \\
&= \int \lr{ I r(\Bx) + I \Bs(\Bx) - \Bt(\Bx) - u(\Bx) } dV.
\end{aligned}
\end{equation}
The result still has all grades, but each of the original grade components is mapped onto its dual space.
%}

      \subsection{Bidirectional derivative operators.}
         %
% Copyright � 2016 Peeter Joot.  All Rights Reserved.
% Licenced as described in the file LICENSE under the root directory of this GIT repository.
%
%{
%\input{../blogpost.tex}
%\renewcommand{\basename}{fundamentalTheoremOfCalculus}
%\renewcommand{\dirname}{notes/phy1520/}
%%\newcommand{\dateintitle}{}
%%\newcommand{\keywords}{}
%
%\input{../peeter_prologue_print2.tex}
%
%\usepackage{peeters_layout_exercise}
%\usepackage{peeters_braket}
%\usepackage{peeters_figures}
%\usepackage{siunitx}
%
%\beginArtNoToc
%
%\generatetitle{Fundamental theorem of geometric calculus}
%\label{chap:fundamentalTheoremOfCalculus}

Having generalized line, surface, and volume integrals to multivector functions, we wish to state the form of the integrand that is perfectly integrable.  That statement requires bidirectional integration operators, denoted using left, right, or left-right overarrows, as follows.

\index{\(\rboldpartial\)}
\index{\(\lboldpartial\)}
\index{\(\lrboldpartial\)}
%why: after:
%Because integral operators like the vector derivative and gradient do not necessarily commute with multivector functions nor hyper-volume elements, it is useful to introduce

\makedefinition{Bidirectional vector derivative operators.}{dfn:fundamentalTheoremOfCalculus:240}{
Given a hypervolume parameterized by \( k \) parameters, k-volume volume element \( d^k \Bx \), and
multivector functions \( F, G \), let
\begin{equation*}
\lroverarrow{L} = \sum_i \Bx^i \partial_i,
\end{equation*}
designate a linear differential operator (i.e. the gradient or vector derivative), where the partials act on multivector functions to the left or right (but not the reciprocal frame vectors \( \Bx^i \)).

To express unidirection action of the operator only to the left or right, we use arrows to designate the scope of the derivatives, writing respectively
\begin{equation*}
\begin{aligned}
\int_V F d^k\Bx \loverarrow{L} G &= \sum_i \int_V \lr{ \partial_i F } d^k\Bx \Bx^i G \\
\int_V F d^k\Bx \roverarrow{L} G &= \sum_i \int_V F d^k\Bx \Bx^i \lr{ \partial_i G },
\end{aligned}
\end{equation*}
and designate bidirectional action as
\begin{equation*}
\int_V F d^k\Bx \lroverarrow{L} G
\equiv
\int_V \lr{ F d^k\Bx \loverarrow{L}} G
+
\int_V F d^k\Bx \lr{ \roverarrow{L} G }.
\end{equation*}
In all such cases \( L \) operates on \( F \) and \( G \), but not the volume element \( d^k \Bx \), which may also be a function of the implied parameterization.
%%The explicit meaning of these directional acting derivative operations is given by the chain rule coordinate expansion
%%\begin{dmath*}
%%F d^k \Bx \lroverarrow{L} G
%%=
%%F d^k \Bx \lr{ \sum_i \Bx^i {\stackrel{ \leftrightarrow }{\partial_i}} } G
%%=
%%(\partial_i F) d^k \Bx \sum_i \Bx^i G
%%+
%%F d^k \Bx \sum_i \Bx^i (\partial_i G)
%%\equiv
%%(F d^k \Bx \loverarrow{L}) G
%%+
%%F d^k \Bx (\roverarrow{L} G),
%%\end{dmath*}
} % definition

The vector derivative
% (and gradient)
may not commute with \( F, G \) nor the volume element \( d^k \Bx \), so we are forced to use some notation to indicate what the vector derivative (or gradient) acts on.
In conventional right acting cases, where there is no ambiguity, arrows will usually be omitted, but braces may also be used to indicate the scope of derivative operators.
This bidirectional notation will also be used for the gradient, especially for volume integrals in \R{3} where the vector derivative is identical to the gradient.

Some authors use the Hestenes dot notation, with overdots or primes to indicating the exact scope of multivector derivative operators, as in
\begin{equation}\label{eqn:fundamentalTheoremOfCalculus:260}
\dot{F} d^k \Bx \dot{\boldpartial} \dot{G} =
\dot{F} d^k \Bx \dot{\boldpartial} G
+
F d^k \Bx \dot{\boldpartial} \dot{G}.
\end{equation}
The dot notation has the advantage of emphasizing that the action of the vector derivative (or gradient) is on the functions \( F, G \), and not on the hypervolume element \( d^k \Bx \).
However, in this book, where primed operators such as \( \spacegrad' \) are used to indicate that derivatives are taken with respect to primed \( \Bx' \) variables, a mix of dots and ticks would have been confusing.
%Over arrows also have the advantage of being visually conspicuous.


      \subsection{Fundamental theorem.}
         %
% Copyright © 2018 Peeter Joot.  All Rights Reserved.
% Licenced as described in the file LICENSE under the root directory of this GIT repository.
%
%{
\index{fundamental theorem of geometric calculus}

The fundamental theorem of geometric calculus is a generalization of many conventional scalar and vector integral theorems, and relates a hypervolume integral to its boundary.
This is a powerful theorem, which we will use with Green's functions to solve Maxwell's equation, but also to derive the geometric algebra form of Stokes' theorem, from which most of the familiar integral calculus results follow.
\input{Theorem_fundamental_theorem_of_gc.tex}
The geometry of the hypersurface element \( d^{k-1} \Bx \) will be made more clear when we
consider the specific cases of \( k = 1, 2, 3 \), representing generalized line, surface, and volume integrals respectively.
Instead of terrorizing the reader with a general proof
\cref{thm:fundamentalTheoremOfCalculus:1},
which requires some unpleasant index gymnastics,
this book
will separately state and prove the fundamental theorem of calculus
for each of the \( k = 1, 2, 3 \) cases that are of interest for problems in \R{2} and \R{3}.
For the interested reader, a sketch of the general proof
of \cref{thm:fundamentalTheoremOfCalculus:1}
is available in \cref{chap:gagcProof}.

Before moving on to the line, surface, and volume integral cases, we will state and prove the
general Stokes' theorem in its geometric algebra form.

%}

      \subsection{Stokes' theorem.}
         %
% Copyright © 2013 Peeter Joot.  All Rights Reserved.
% Licenced as described in the file LICENSE under the root directory of this GIT repository.
%
An important consequence of the fundamental theorem of geometric calculus is the
geometric algebra generalization of Stokes' theorem.
The Stokes' theorem that we know from conventional vector calculus relates
\R{3} surface integrals to the line integral around a bounding surface.
The geometric algebra form of Stokes' theorem is equivalent to Stokes' theorem from the theory of differential forms, which relates
hypervolume integrals of blades\footnote{Blades are isomorphic to the k-forms found in the theory of differential forms.} to the integrals over their hypersurface boundaries, a much more general result.

\input{Theorem_stokes_theorem_general.tex}

We will see that most of the well known scalar and vector integral theorems can easily be derived as direct consequences of \cref{thm:stokesTheoremGeometricAlgebra:1740}, itself a special case of \cref{thm:fundamentalTheoremOfCalculus:1}.

\begin{proof}
We can prove Stokes' theorem
from \cref{thm:fundamentalTheoremOfCalculus:1}
by setting \( F = 1 \), and requiring that \( G \)
is an s-blade, with grade \( s < k \).
The proof follows by selecting the \( k-(s+1) \) grade, the lowest grade of \( d^k \Bx (\boldpartial \wedge G) \), from both sides of \cref{thm:fundamentalTheoremOfCalculus:1}.

For the grade selection of the hypervolume integral we have
\begin{equation}\label{eqn:stokesTheoremTheStatement:100}
\gpgrade{ \int_V d^k \Bx \boldpartial G }{k-(s+1)}
=
\gpgrade{
\int_V d^k \Bx (\boldpartial \cdot G )
+
\int_V d^k \Bx (\boldpartial \wedge G )
}{k-(s-1)},
\end{equation}
however, the lowest grade of \( d^k \Bx (\boldpartial \cdot G ) \) is \( k -(s-1) = k - s + 1 > k - (s+1) \), so the divergence integral is zero.  As \( d^{k-1} \Bx \) is a \( k - 1 \) blade
\begin{equation}\label{eqn:stokesTheoremTheStatement:110}
\begin{aligned}
\int_V d^k \Bx \cdot (\boldpartial \wedge G )
&= \int_{\partial V} \gpgrade{d^{k-1} \Bx G}{k-(s+1)} \\
&= \int_{\partial V} d^{k-1} \Bx \cdot G.
\end{aligned}
\end{equation}
\end{proof}

      \subsection{Fundamental theorem for Line integral.}
         %
% Copyright � 2018 Peeter Joot.  All Rights Reserved.
% Licenced as described in the file LICENSE under the root directory of this GIT repository.
%
%{
%%%\input{../latex/blogpost.tex}
%%%\renewcommand{\basename}{lineintegral}
%%%%\renewcommand{\dirname}{notes/phy1520/}
%%%\renewcommand{\dirname}{notes/ece1228-electromagnetic-theory/}
%%%%\newcommand{\dateintitle}{}
%%%%\newcommand{\keywords}{}
%%%
%%%\input{../latex/peeter_prologue_print2.tex}
%%%
%%%\usepackage{peeters_layout_exercise}
%%%\usepackage{peeters_braket}
%%%\usepackage{peeters_figures}
%%%\usepackage{siunitx}
%%%%\usepackage{mhchem} % \ce{}
%%%%\usepackage{macros_bm} % \bcM
%%%%\usepackage{macros_qed} % \qedmarker
%%%%\usepackage{txfonts} % \ointclockwise
%%%
%%%\beginArtNoToc
%%%
%%%\generatetitle{Multivector line integral.}
%\section{Line integral.}
\label{chap:lineintegral}
The line integral specialization of \cref{thm:fundamentalTheoremOfCalculus:1} is
\input{Theorem_multivector_line_integral.tex}
The differential form \( d\Bx = d^1 \Bx = du\, \Bx_u = du\, \PDi{u}{\Bx} \) varies over the curve,
and the vector derivative is just \( \boldpartial = \Bx^u \partial_u \) (no sum).
\begin{proof}
The proof follows by expansion.
For
\begin{equation}\label{eqn:lineintegral:120}
\begin{aligned}
\int_C F d\Bx\, \lrboldpartial G
&= \int_C \lr{ F d\Bx \lboldpartial} G + \int_C F d\Bx \lr{ \rboldpartial G } \\
&= \int_C \PD{u}{F} du\, \Bx_u \Bx^u G + \int_C F du\, \Bx_u \Bx^u \PD{u}{G} \\
&= \int_C du\, \PD{u}{F} G + \int_C du\, F \PD{u}{G} \\
&= \int_C du\, \PD{u}{} \lr{ F G } \\
&= F(u_1) G(u_1) - F(u_0) G(u_0),
\end{aligned}
\end{equation}
%where the boundaries of the parameterization have been assumed to be
%\( C: u \in [u_0, u_1] \).
We have a perfect cancellation of the reciprocal frame \( \Bx^u \) with the vector \( \Bx_u \) that lies along the curve, since \( \Bx^u \Bx_u = 1 \).  This leaves a perfect derivative of the product of \( F G \), which can be integrated over the length of the curve, yielding the difference of the product with respect to the parameterization of the end points of the curve.
\end{proof}

For a single parameter subspace
the reciprocal frame vector \( \Bx^u \)
is trivial to calculate, as it is just the inverse of \( \Bx_u \), that is \( \Bx^u = \Bx_u/\Norm{\Bx_u}^2 \).
Observe that we did not actually have to calculate it, but instead only require that the vector is invertible.

An important (and familiar) special case of \cref{thm:lineintegral:100} is the fundamental theorem of calculus for line integrals, which can be obtained by using a
single scalar function \( f \)

\input{Theorem_line_integral_scalar_function.tex}
\begin{proof}
\Cref{thm:lineintegral:180} is no doubt familiar in its gradient form.  Our proof starts with
\cref{thm:lineintegral:100} setting \( F = 1, G = f(\Bx(u)) \)
\begin{equation}\label{eqn:lineintegral:140}
\int_C d\Bx\, \boldpartial f = \evalbar{f}{\Delta u},
\end{equation}
which is a multivector equation with scalar and bivector grades on the left hand side, but only scalar grades on the right.  Equating grades yields two equations
\begin{subequations}
\label{eqn:lineintegral:180}
\begin{equation}\label{eqn:lineintegral:160}
\int_C d\Bx \cdot \boldpartial f = \evalbar{f}{\Delta u}
\end{equation}
\begin{equation}\label{eqn:lineintegral:200}
\int_C d\Bx \wedge \boldpartial f = 0.
\end{equation}
\end{subequations}
\Cref{eqn:lineintegral:160}, the scalar grade of \cref{eqn:lineintegral:140}, proves part of \cref{thm:lineintegral:180}.
To complete the proof, consider the specific case of \R{3} which is representitive.
Suppose, that we have an \R{3} volume parameterization \( \Bx(u, v, w) \) sharing an edge with the curve \( C = \Bx(u,0,0) \).
The curvilinear representation of the \R{3} gradient is
\begin{equation}\label{eqn:lineintegral:220}
\spacegrad = \Bx^u \partial_u + \Bx^v \partial_v + \Bx^w \partial_w = \boldpartial + \Bx^v \partial_v + \Bx^w \partial_w,
\end{equation}
Over the curve \( C \)
\begin{equation}\label{eqn:lineintegral:240}
d\Bx \cdot \spacegrad
=
du\, \Bx_u \cdot \lr{ \boldpartial + \Bx^v \partial_v + \Bx^w \partial_w },
\end{equation}
but \( \Bx_u \cdot \Bx^v = \Bx_u \cdot \Bx^w = 0 \), so \( d\Bx \cdot \spacegrad = d\Bx \cdot \boldpartial \) over the curve.
%, completing the proof.
\end{proof}
%}
%\EndNoBibArticle

      \subsection{Fundamental theorem for Surface integral.}
         %
% Copyright � 2018 Peeter Joot.  All Rights Reserved.
% Licenced as described in the file LICENSE under the root directory of this GIT repository.
%
%{
%%%\input{../latex/blogpost.tex}
%%%\renewcommand{\basename}{surfaceintegral}
%%%%\renewcommand{\dirname}{notes/phy1520/}
%%%\renewcommand{\dirname}{notes/ece1228-electromagnetic-theory/}
%%%%\newcommand{\dateintitle}{}
%%%%\newcommand{\keywords}{}
%%%
%%%\input{../latex/peeter_prologue_print2.tex}
%%%
%%%\usepackage{peeters_layout_exercise}
%%%\usepackage{peeters_braket}
%%%\usepackage{peeters_figures}
%%%\usepackage{siunitx}
%%%%\usepackage{mhchem} % \ce{}
%%%%\usepackage{macros_bm} % \bcM
%%%%\usepackage{macros_qed} % \qedmarker
%%%\usepackage{txfonts} % \ointclockwise
%%%
%%%\beginArtNoToc
%%%
%%%\generatetitle{Multivector surface integral.}
%\section{Surface integral.}
%\label{chap:surfaceintegral}

%%As mentioned in a line integral context,
%%multivectors may not commute with the vector derivative or the differential, so we allow the vector derivative to act bidirectionally using the chain rule.
%%The scope of the action of the vector derivative when acting only to the left or right is indicated using braces above.
%%Should we wish to only integrate single functions, we can set either of the other to \( 1 \), yielding integrals of the form
%%\( \int_S F d^2\Bx \lboldpartial, \) or \( \int_S d^2\Bx \boldpartial G \).

The surface integral specialization of \cref{thm:fundamentalTheoremOfCalculus:1} is

\input{Theorem_multivector_surface_integral.tex}
\begin{proof}
To prove \cref{thm:surfaceintegral:100} we start by expanding the multivector product \( d^2\Bx\, \boldpartial \) in curvilinear coordinates, where we discover
that this product has only a vector grade.
The vector derivative, the projection of the gradient onto the surface at the point of integration (also called the tangent space), now has two components
\begin{equation}\label{eqn:surfaceintegral:200}
\begin{aligned}
\boldpartial
&= \sum_i \Bx^i (\Bx_i \cdot \spacegrad) \\
&=
\Bx^u \PD{u}{}
+
\Bx^v \PD{v}{}
\equiv
\Bx^u \partial_u
+
\Bx^v \partial_v.
\end{aligned}
\end{equation}
To see why the product of the area elements and the vector derivative
\begin{equation}\label{eqn:surfaceintegral:300}
d^2\Bx\, \boldpartial
=
du dv\, \lr{ \Bx_u \wedge \Bx_v } \lr{ \Bx^u \partial_u + \Bx^v \partial_v },
\end{equation}
has only a vector grade, observe that \( \Bx^u \in \Span \setlr{ \Bx_u, \Bx_v } \), so
\begin{equation}\label{eqn:surfaceintegral:320}
\begin{aligned}
\lr{ \Bx_u \wedge \Bx_v } \Bx^u
&= \lr{ \Bx_u \wedge \Bx_v } \cdot \Bx^u + \cancel{ \lr{ \Bx_u \wedge \Bx_v } \wedge \Bx^u } \\
&= \lr{ \Bx_u \wedge \Bx_v } \cdot \Bx^u \\
&= \Bx_u \lr{ \Bx_v \cdot \Bx^u } - \Bx_v \lr{ \Bx_u \cdot \Bx^u } \\
&= -\Bx_v.
\end{aligned}
\end{equation}
Similarly
\begin{equation}\label{eqn:surfaceintegral:340}
\begin{aligned}
\lr{ \Bx_u \wedge \Bx_v } \Bx^v
&= \lr{ \Bx_u \wedge \Bx_v } \cdot \Bx^v + \cancel{ \lr{ \Bx_u \wedge \Bx_v } \wedge \Bx^v } \\
&= \lr{ \Bx_u \wedge \Bx_v } \cdot \Bx^v \\
&= \Bx_u \lr{ \Bx_v \cdot \Bx^v } - \Bx_v \lr{ \Bx_u \cdot \Bx^v } \\
&= \Bx_u.
\end{aligned}
\end{equation}
Not only does \cref{eqn:surfaceintegral:300} have only a vector grade, that product reduces to just
\begin{equation}\label{eqn:surfaceintegral:360}
d^2\Bx\, \boldpartial
=
\Bx_u \partial_v
-\Bx_v \partial_u.
\end{equation}
Inserting \cref{eqn:surfaceintegral:360} into the surface integral, we find
\begin{equation}\label{eqn:surfaceintegral:380}
\begin{aligned}
\int_S F d^2\Bx\, \boldpartial G
&= \int_S \lr{ F d^2\Bx\, \lboldpartial} G + \int_S F d^2\Bx \lr{ \rboldpartial G } \\
&= \int_S du dv\, \lr{ \partial_v F \Bx_u - \partial_u F \Bx_v } G + \int_S du dv\, F \lr{ \Bx_u \partial_v G - \Bx_v \partial_u G } \\
&= \int_S du dv\, \lr{ \PD{v}{F} \PD{u}{\Bx} - \PD{u}{F} \PD{v}{\Bx} } G + \int_S du dv\, F \lr{ \PD{u}{\Bx} \PD{v}{G} - \PD{v}{\Bx} \PD{u}{G} } \\
&= \int_S du dv\, \PD{v}{} \lr{ F \PD{u}{\Bx} G } - \int_S du dv\, \PD{u}{} \lr{ F \PD{v}{\Bx} G }  \\
&\quad - \int_S du dv\, F \lr{ \PD{v}{} \PD{u}{\Bx} - \PD{u}{} \PD{v}{\Bx} } G \\
&= \int_S du dv\, \PD{v}{} \lr{ F \PD{u}{\Bx} G } - \int_S du dv\, \PD{u}{} \lr{ F \PD{v}{\Bx} G }.
\end{aligned}
\end{equation}
This leaves two perfect differentials, which can both be integrated separately
\begin{equation}\label{eqn:surfaceintegral:400}
\begin{aligned}
\int_S F d^2\Bx\, \boldpartial G
&= \int_{\Delta u} du\, \evalbar{\lr{ F \PD{u}{\Bx} G }}{\Delta v} - \int_{\Delta v} dv\, \evalbar{\lr{ F \PD{v}{\Bx} G }}{\Delta u} \\
&= \int_{\Delta u} \evalbar{\lr{ F d\Bx_u G }}{\Delta v} - \int_{\Delta v} \evalbar{\lr{ F d\Bx_v G }}{\Delta u}.
\end{aligned}
\end{equation}
\Cref{eqn:surfaceintegral:400} is an explicit algebraic expression of the boundary integral of \cref{thm:surfaceintegral:100}.
To complete the proof, we are left with the task of geometrically interpretting this integrand.
Suppose we are integrating over the unit parameter volume space \( [u, v] \in [0,1] \otimes [0,1] \) as illustrated in
\cref{fig:twoParameterDifferentialBoundary:twoParameterDifferentialBoundaryFig2}.
\pmathImageFigure
%\imageTwoFigures
{../figures/GAelectrodynamics/}{twoParameterDifferentialBoundaryFig2}
%{../figures/GAelectrodynamics/}{twoParameterDifferentialBoundaryEditedFig3}
%{Contour for two parameter surface boundary, and subdivision into finer loop mesh.}
{Contour for two parameter surface boundary.}
%{Contour for two parameter surface boundary, and subdivision into finer loop mesh.}
{fig:twoParameterDifferentialBoundary:twoParameterDifferentialBoundaryFig2}
{0.4}{twoParameterDifferentialFig.nb}
%{scale=0.4}
Comparing to the figure we see that we've ended up with a clockwise line integral around the boundary of the surface.
For a given subset of the surface, the bivector area element can be chosen small enough that it lies in the tangent space
to the surface at the point of integration.
In that case, a larger bounding loop can be conceptualized as the sum of a number of smaller ones, as sketched
in \cref{fig:loopIntegralInfinitesimalSum:loopIntegralInfinitesimalSumFig2},
in which case the
contributions of the interior loop paths (red and blue) cancel out, leaving only the exterior loop contributions (green.)  When that subdivision is made small enough (assuming that the surface is continuous and differentiable along each of the parameterization paths) then each area element approximates the tangent space at the point of evaluation.
\end{proof}
\pmathImageFigure{../figures/GAelectrodynamics/}{twoParameterDifferentialBoundaryEditedFig3}{Sum of infinitesimal loops.}{fig:loopIntegralInfinitesimalSum:loopIntegralInfinitesimalSumFig2}{0.35}{twoParameterDifferentialFig.nb}
\subsubsection{Two parameter Stokes' theorem.}
Two
special cases of \cref{thm:surfaceintegral:100}, both variations of Stokes' theorem, result by considering scalar and vector functions.  For the scalar case we have
\input{Theorem_surface_integral_of_scalar_stokes.tex}
\begin{proof}
To show the first part, we can split the (multivector) surface integral into vector and trivector grades
\begin{equation}\label{eqn:surfaceintegral:440}
\int_S d^2\Bx\, \boldpartial f
=
\int_S d^2\Bx \cdot \boldpartial f
+
\int_S d^2\Bx \wedge \boldpartial f.
\end{equation}

Since \( \Bx^u, \Bx^v \) both lie in the span of \( \setlr{ \Bx_u, \Bx_v } \),
\( d^2\Bx \wedge \boldpartial = 0 \), killing the second integral in \cref{eqn:surfaceintegral:440}.
If the gradient is decomposed into its projection along the tangent
space (the vector derivative) and its perpendicular components, only the vector derivative components of the
gradient contribute to its dot product with the area element.  That is
\begin{equation}\label{eqn:surfaceintegral:460}
\begin{aligned}
d^2 \Bx \cdot \spacegrad
&= d^2 \Bx \cdot \lr{ \Bx^u \partial_u + \Bx^v \partial_v + \cdots } \\
&= d^2 \Bx \cdot \lr{ \Bx^u \partial_u + \Bx^v \partial_v } \\
&= d^2 \Bx \cdot \boldpartial.
\end{aligned}
\end{equation}

This means that for a scalar function
\begin{equation}\label{eqn:surfaceintegral:480}
\int_S d^2\Bx\, \boldpartial f
=
\int_S d^2\Bx \cdot \spacegrad f.
\end{equation}

The second part of the theorem follows by grade selection, and application of a duality transformation for the area element
\begin{equation}\label{eqn:surfaceintegral:500}
\begin{aligned}
d^2 \Bx \cdot \spacegrad f
&= \gpgradeone{ d^2 \Bx \spacegrad f } \\
&= dA\, \gpgradeone{ I \ncap \spacegrad f } \\
&= dA\, \gpgradeone{ I \lr{ \ncap \cdot \spacegrad f + I \ncap \cross \spacegrad f} } \\
&= -dA\, \ncap \cross \spacegrad f.
\end{aligned}
\end{equation}

back substitution of \cref{eqn:surfaceintegral:500} completes the proof.
\end{proof}
% of \cref{thm:surfaceintegral:420}.

For vector functions we have
\input{Theorem_surface_integral_of_vector_stokes.tex}

%%This follows by setting \( F = 1, G = \Bf \) in \cref{thm:surfaceintegral:100} and selecting the scalar grade.  In particular we may form the
%%scalar selection of \( d^2 \Bx \boldpartial \Bf \) in two different ways.  The first is
%%\begin{equation}\label{eqn:surfaceintegral:520}
%%\gpgradezero{ d^2 \Bx \boldpartial \Bf }
%%=
%%\gpgradezero{ (d^2 \Bx \cdot \boldpartial + d^2 \Bx \wedge \boldpartial ) \Bf }.
%%\end{equation}
%%
%%The \( d^2 \Bx \wedge \boldpartial \) product with \( \Bf \) has only bivector and quad-vector components (the latter is zero in \R{3}), so its scalar grade selection is zero, and we are left with
%%\begin{equation}\label{eqn:surfaceintegral:540}
%%\gpgradezero{ d^2 \Bx \boldpartial \Bf }
%%=
%%(d^2 \Bx \cdot \boldpartial) \cdot \Bf
%%=
%%(d^2 \Bx \cdot \spacegrad) \cdot \Bf,
%%\end{equation}
%%where we have used \cref{eqn:surfaceintegral:460} again.  This product can also be written as
%%\begin{equation}\label{eqn:surfaceintegral:560}
%%(d^2 \Bx \cdot \spacegrad) \cdot \Bf
%%=
%%\gpgradezero{ (d^2 \Bx \cdot \spacegrad) \Bf }
%%=
%%\gpgradezero{ (d^2 \Bx \spacegrad - d^2 \Bx \wedge \spacegrad) \Bf }
%%=
%%\gpgradezero{ d^2 \Bx \spacegrad \Bf }
%%=
%%\gpgradezero{ d^2 \Bx \lr{ \cancel{ \spacegrad \cdot \Bf } + \spacegrad \wedge \Bf } }
%%=
%%d^2 \Bx \cdot \lr{ \spacegrad \wedge \Bf }.
%%\end{equation}
%%
%%\begin{equation}\label{eqn:surfaceintegral:580}
%%\ointclockwise_{\partial S} d\Bx \cdot \Bf
%%=
%%\gpgradezero{ \int_S d^2\Bx \boldpartial \Bf }
%%=
%%\int_S \lr{ d^2\Bx \cdot \spacegrad } \cdot \Bf
%%=
%%\int_S d^2\Bx \cdot \lr{ \spacegrad \wedge \Bf },
%%\end{equation}
%%as claimed.  In particular in \R{3}, we have
%%\begin{equation}\label{eqn:surfaceintegral:600}
%%d^2\Bx \cdot \lr{ \spacegrad \wedge \Bf }
%%=
%%dA \gpgradezero{ I \ncap I \lr{ \spacegrad \cross \Bf } }
%%=
%%-dA \ncap \cdot \lr{ \spacegrad \cross \Bf }.
%%\end{equation}
%%
%%Substitution into \cref{eqn:surfaceintegral:580} proves the last part of \cref{thm:surfaceintegral:500}.
%%
\subsubsection{Green's theorem.}

\Cref{thm:surfaceintegral:500}, when stated in terms of coordinates, is another well known result.
\input{Theorem_greens.tex}
\todo{Add an example (lots to pick from in any 3rd term calc text).}
\begin{proof}
The first equality in \cref{thm:surfaceintegral:620} holds in \R{N} for vectors expressed in terms of an arbitrary curvilinear basis.
Only the (curvilinear) coordinates of the vector \( \Bf \) contribute to this integral, and only those that lie in the tangent space.
The reciprocal basis vectors \( \Bx^i \) are also nowhere to be seen.  This is because they are either obliterated in dot products with \( \Bx_j \), or cancel due to mixed partial equality.

To see how this occurs let's look at the
area integrand of \cref{thm:surfaceintegral:500}
\begin{equation}\label{eqn:surfaceintegral:660}
\begin{aligned}
d^2 \Bx \cdot \lr{ \spacegrad \wedge \Bf }
&= du_1 du_2\, \lr{ \Bx_1 \wedge \Bx_2 } \cdot \lr{ \sum_{ij} \lr{ \Bx^i \partial_i } \wedge \lr{ f_j \Bx^j } } \\
&= du_1 du_2\, \sum_{ij} \lr{ \lr{ \Bx_1 \wedge \Bx_2 } \cdot \Bx^i } \cdot \lr{ \partial_i (f_j \Bx^j) } \\
&= du_1 du_2\, \sum_{ij} \lr{ \lr{ \Bx_1 \wedge \Bx_2 } \cdot \Bx^i } \cdot \Bx^j \partial_i f_j \\
&\quad + du_1 du_2\, \sum_{ij} f_j \lr{ \lr{ \Bx_1 \wedge \Bx_2 } \cdot \Bx^i } \cdot (\partial_i \Bx^j).
\end{aligned}
\end{equation}

With a bit of trouble, we will see that the second integrand is zero.  On the other hand, the first integrand
simplifies
without too much trouble
\begin{equation}\label{eqn:surfaceintegral:680}
\begin{aligned}
\sum_{ij} \lr{ \lr{ \Bx_1 \wedge \Bx_2 } \cdot \Bx^i } \cdot \Bx^j \partial_i f_j
&= \sum_{ij} \lr{ \Bx_1 \delta_{2i} - \Bx_2 \delta_{1i} } \cdot \Bx^j \partial_i f_j \\
&= \sum_{j} \Bx_1 \cdot \Bx^j \partial_2 f_j -\Bx_2 \cdot \Bx^j \partial_1 f_j \\
&= \partial_2 f_1 - \partial_1 f_2.
\end{aligned}
\end{equation}

For the second integrand, we have
\begin{equation}\label{eqn:surfaceintegral:700}
\begin{aligned}
\sum_{ij} &f_j \lr{ \lr{ \Bx_1 \wedge \Bx_2 } \cdot \Bx^i } \cdot (\partial_i \Bx^j) \\
&=
\sum_{j} f_j \sum_i \lr{ \Bx_1 \delta_{2i} - \Bx_2 \delta_{1i} } \cdot (\partial_i \Bx_j) \\
&=
\sum_{j} f_j
\lr{
\Bx_1 \cdot (\partial_2 \Bx^j)
-
\Bx_2 \cdot (\partial_1 \Bx^j)
}
\end{aligned}
\end{equation}

We can apply the chain rule (backwards) to the portion in brackets to find
\begin{equation}\label{eqn:surfaceintegral:720}
\begin{aligned}
\Bx_1 \cdot (\partial_2 \Bx^j) - \Bx_2 \cdot (\partial_1 \Bx^j)
&= \cancel{\partial_2 \lr{ \Bx_1 \cdot \Bx^j }} - (\partial_2 \Bx_1) \cdot \Bx^j - \cancel{\partial_1 \lr{ \Bx_2 \cdot \Bx^j }} + (\partial_1 \Bx_2) \cdot \Bx^j \\
&= \Bx_j \cdot \lr{ \partial_1 \Bx_2 - \partial_2 \Bx_1 } \\
&= \Bx_j \cdot \lr{ \PD{u_1}{} \PD{u_2}{\Bx} - \PD{u_2}{} \PD{u_1}{\Bx} } \\
&= 0.
\end{aligned}
\end{equation}
In this reduction the derivatives of \( \Bx_i \cdot \Bx^j = {\delta_i}^j \) were killed since those are constants (either zero or one).  The final step relies on the fact that we assume our vector parameterization is well behaved enough that the mixed partials are zero.

Substituting these results into \cref{thm:surfaceintegral:500} we find
\begin{equation}\label{eqn:surfaceintegral:740}
\begin{aligned}
\ointclockwise_{\partial S} d\Bx \cdot \Bf
&= \ointclockwise_{\partial S} \lr{ du_1 \Bx_1 + du_2 \Bx_2 } \cdot \lr{ \sum_i f_i \Bx^i } \\
&= \ointclockwise_{\partial S} du_1\, f_1 + du_2\, f_2 \\
&= \int_S du_1 du_2\, \lr{ \partial_2 f_1 - \partial_1 f_2 }.
\end{aligned}
\end{equation}
\end{proof}
%}
%%%\EndArticle

      \subsection{Fundamental theorem for Volume integral.}
         %
% Copyright � 2018 Peeter Joot.  All Rights Reserved.
% Licenced as described in the file LICENSE under the root directory of this GIT repository.
%
%{
%%%\input{../latex/blogpost.tex}
%%%\renewcommand{\basename}{volumeintegral}
%%%%\renewcommand{\dirname}{notes/phy1520/}
%%%\renewcommand{\dirname}{notes/ece1228-electromagnetic-theory/}
%%%%\newcommand{\dateintitle}{}
%%%%\newcommand{\keywords}{}
%%%
%%%\input{../latex/peeter_prologue_print2.tex}
%%%
%%%\usepackage{peeters_layout_exercise}
%%%\usepackage{peeters_braket}
%%%\usepackage{peeters_figures}
%%%\usepackage{siunitx}
%%%%\usepackage{mhchem} % \ce{}
%%%%\usepackage{macros_bm} % \bcM
%%%%\usepackage{macros_qed} % \qedmarker
%%%\usepackage{txfonts} % \ointclockwise
%%%
%%%\beginArtNoToc
%%%
%%%\generatetitle{Volume integral.}
%%%%\chapter{Volume integral.}
\label{chap:volumeintegral}

%\subsection{Volume integral.}
\index{volume parameterization}
\index{volume element}
\index{differential form}
The volume integral specialization of \cref{thm:fundamentalTheoremOfCalculus:1} follows.

\input{Theorem_multivector_volume_integral.tex}
Before diving into the proof of \cref{thm:volumeintegral:100}, let's consider the geometry of the volume element briefly.

For uniformity, let \( u = u_1, v = u_2, w = u_3 \), so that the differentials along each of the parameterization directions are
\begin{dmath}\label{eqn:volumeintegral:100}
\begin{aligned}
d\Bx_1 &= \PD{u_1}{\Bx} du_1 = \Bx_1 du_1 \\
d\Bx_2 &= \PD{u_2}{\Bx} du_2 = \Bx_2 du_2 \\
d\Bx_3 &= \PD{u_3}{\Bx} du_3 = \Bx_3 du_3.
\end{aligned}
\end{dmath}
The trivector valued volume element for this parameterization is
\begin{equation}\label{eqn:volumeintegral:120}
d^3 \Bx
=
d\Bx_1 \wedge
d\Bx_2 \wedge
d\Bx_3
=
d^3 u\, (\Bx_1 \wedge \Bx_2 \wedge \Bx_3),
\end{equation}
where \( d^3 u = du_1 du_2 du_3 \).
A volume and its corresponding differentials with respect to three parameters is sketched in
\cref{fig:normalsOnVolumeAreaElement:normalsOnVolumeAreaElementFig11}.

\imageFigure{../figures/gabook/normalsOnVolumeAreaElementFig11}{Three parameter volume element.}{fig:normalsOnVolumeAreaElement:normalsOnVolumeAreaElementFig11}{0.4}

In \R{3} the vector derivative for a volume parameterization and the gradient are identical.  In higher dimensional spaces
the projection of the gradient onto the volume at the point of integration (also called the tangent space), has three components
\begin{dmath}\label{eqn:volumeintegral:200}
\boldpartial
=
\sum_i \Bx^i (\Bx_i \cdot \spacegrad)
=
\Bx^1 \PD{u_1}{}
+
\Bx^2 \PD{u_2}{}
+
\Bx^3 \PD{u_3}{}
\equiv
\Bx^1 \partial_1
+
\Bx^2 \partial_2
+
\Bx^3 \partial_3.
\end{dmath}
\todo{Have I not spelled out the volume element and the vector derivative for three parameters before?  Check for this and remove redudancy.}

With the volume element and the vector derivative spelled out explicitly, we can proceed with a proof of \cref{thm:volumeintegral:100}.  
\begin{proof}
The first step, is the 
reduction of the product of the volume element and the vector derivative, which we will see is a bivector.
\begin{equation}\label{eqn:volumeintegral:300}
d^3\Bx \boldpartial
=
d^3 u\, \lr{ \Bx_1 \wedge \Bx_2 \wedge \Bx_3 } \lr{ \Bx^1 \partial_1 + \Bx^2 \partial_2 + \Bx^3 \partial_3 }.
\end{equation}
Since all \( \Bx^i \) lie within \( \Span \setlr{ \Bx_1, \Bx_2, \Bx_3 } \), this multivector product has only a vector grade.  That is
\begin{equation}\label{eqn:volumeintegral:320}
\lr{ \Bx_1 \wedge \Bx_2 \wedge \Bx_3 } \Bx^i
=
\lr{ \Bx_1 \wedge \Bx_2 \wedge \Bx_3 } \cdot \Bx^i
+
\cancel{ \lr{ \Bx_1 \wedge \Bx_2 \wedge \Bx_3 } \wedge \Bx^i },
\end{equation}
for all \( \Bx^i \).  These products reduces to
\begin{dmath}\label{eqn:volumeintegral:1621}
\begin{aligned}
\lr{ \Bx_2 \wedge \Bx_3 \wedge \Bx_1 } \Bx^1 &= \Bx_2 \wedge \Bx_3 \\
\lr{ \Bx_3 \wedge \Bx_1 \wedge \Bx_2 } \Bx^2 &= \Bx_3 \wedge \Bx_1 \\
\lr{ \Bx_1 \wedge \Bx_2 \wedge \Bx_3 } \Bx^3 &= \Bx_1 \wedge \Bx_2.
\end{aligned}
\end{dmath}

Inserting \cref{eqn:volumeintegral:1621} into the volume integral, we find
\begin{equation}\label{eqn:volumeintegral:380}
\begin{aligned}
\int_V &F d^3\Bx \boldpartial G \\
&=
\int_V \lr{ F d^3\Bx \lboldpartial} G
+
\int_V F d^3\Bx \lr{ \rboldpartial G } \\
&=
\int_V d^3 u\, \lr{
   (\partial_1 F) \Bx_2 \wedge \Bx_3 G
   +
   (\partial_2 F) \Bx_3 \wedge \Bx_1 G
   +
   (\partial_3 F) \Bx_1 \wedge \Bx_2 G
} \\
&\quad +
\int_V d^3 u\, \lr{
   F \Bx_2 \wedge \Bx_3 (\partial_1 G)
   +
   F \Bx_3 \wedge \Bx_1 (\partial_2 G)
   +
   F \Bx_1 \wedge \Bx_2 (\partial_3 G)
} \\
&=
\int_V d^3 u\, \lr{
   \partial_1 (F \Bx_2 \wedge \Bx_3 G)
   +
   \partial_2 (F \Bx_3 \wedge \Bx_1 G)
   +
   \partial_3 (F \Bx_1 \wedge \Bx_2 G)
} \\
&\quad -
\int_V d^3 u\, \lr{
   F (\partial_1 (\Bx_2 \wedge \Bx_3)) G
   +
   F (\partial_2 (\Bx_3 \wedge \Bx_1)) G
   +
   F (\partial_3 (\Bx_1 \wedge \Bx_2)) G
} \\
&=
\int_V d^3 u\, \lr{
   \partial_1 (F \Bx_2 \wedge \Bx_3 G)
   +
   \partial_2 (F \Bx_3 \wedge \Bx_1 G)
   +
   \partial_3 (F \Bx_1 \wedge \Bx_2 G)
} \\
&\quad -
\int_V d^3 u\, F
\lr{
   \partial_1 (\Bx_2 \wedge \Bx_3)
   +
   \partial_2 (\Bx_3 \wedge \Bx_1)
   +
   \partial_3 (\Bx_1 \wedge \Bx_2)
}
G
.
\end{aligned}
\end{equation}

The sum within the second integral is
\begin{equation}\label{eqn:volumeintegral:1111}
\begin{aligned}
&\sum_{i = 1}^3 \partial_i \lr{ I_k \cdot \Bx^i } \\
&=
\partial_3 \lr{ (\Bx_1 \wedge \Bx_2 \wedge \Bx_3) \cdot \Bx^3 }  \\
&\quad +
\partial_1 \lr{ (\Bx_2 \wedge \Bx_3 \wedge \Bx_1) \cdot \Bx^1 }  \\
&\quad +
\partial_2 \lr{ (\Bx_3 \wedge \Bx_1 \wedge \Bx_2) \cdot \Bx^2 } \\
&=
\partial_3 \lr{ \Bx_1 \wedge \Bx_2 }
+
\partial_1 \lr{ \Bx_2 \wedge \Bx_3 }
+
\partial_2 \lr{ \Bx_3 \wedge \Bx_1 } \\
&=
         (\partial_3 \Bx_1) \wedge \Bx_2 + \Bx_1 \wedge (\partial_3 \Bx_2) \\
&\quad + (\partial_1 \Bx_2) \wedge \Bx_3 + \Bx_2 \wedge (\partial_1 \Bx_3) \\
&\quad + (\partial_2 \Bx_3) \wedge \Bx_1 + \Bx_3 \wedge (\partial_2 \Bx_1) \\
&=
\Bx_2 \wedge \lr{ - \partial_3 \Bx_1 + \partial_1 \Bx_3 }
+
\Bx_3 \wedge \lr{ - \partial_1 \Bx_2 + \partial_2 \Bx_1 }
+
\Bx_1 \wedge \lr{ - \partial_2 \Bx_3 + \partial_3 \Bx_2 } \\
&=
\Bx_2 \wedge \lr{ - \frac{\partial^2 \Bx}{\partial_3 \partial_1} + \frac{\partial^2 \Bx}{\partial_1 \partial_3} }
+
\Bx_3 \wedge \lr{ - \frac{\partial^2 \Bx}{\partial_1 \partial_2} + \frac{\partial^2 \Bx}{\partial_2 \partial_1} }
+
\Bx_1 \wedge \lr{ - \frac{\partial^2 \Bx}{\partial_2 \partial_3} + \frac{\partial^2 \Bx}{\partial_3 \partial_2} },
\end{aligned}
\end{equation}
which is zero by equality of mixed partials.
This leaves three perfect differentials, which can integrated separately, giving
\begin{dmath}\label{eqn:volumeintegral:400}
\int_V F d^3\Bx \boldpartial G
=
\int du_2 du_3
\evalbar{ \lr{ F \Bx_2 \wedge \Bx_3 G } }{\Delta u_1}
+
\int du_3 du_1
\evalbar{ \lr{ F \Bx_3 \wedge \Bx_1 G } }{\Delta u_2}
+
\int du_1 du_2
\evalbar{ \lr{ F \Bx_1 \wedge \Bx_2 G } }{\Delta u_3}
=
\int
\evalbar{ \lr{ F d\Bx_2 \wedge d\Bx_3 G } }{\Delta u_1}
+
\int
\evalbar{ \lr{ F d\Bx_3 \wedge d\Bx_1 G } }{\Delta u_2}
+
\int
\evalbar{ \lr{ F d\Bx_1 \wedge d\Bx_2 G } }{\Delta u_3}.
\end{dmath}

This proves the theorem from an algebraic point of view.
\end{proof}
With the aid of a geometrical model, such as that of \cref{fig:differentialVolume:differentialVolumeFig}, if
assuming that \( d\Bx_1, d\Bx_2, d\Bx_3 \) is a right handed triple).
it is possible to convince oneself that the two parameter integrands describe an integral over a counterclockwise oriented surface.
\imageTwoFigures{../figures/GAelectrodynamics/differentialVolumeFig1}{../figures/GAelectrodynamics/differentialVolumeFig2}{Differential surface of a volume.}{fig:differentialVolume:differentialVolumeFig}{scale=0.05}

We obtain the RHS of \cref{thm:volumeintegral:100} if we
introduce a mnemonic for the bounding oriented surface of the volume
\begin{dmath}\label{eqn:volumeintegral:1641}
d^2 \Bx \equiv d\Bx_1 \wedge d\Bx_2 + d\Bx_2 \wedge d\Bx_3 + d\Bx_3 \wedge d\Bx_1,
\end{dmath}
where it is implied that each component of this area element and anything that it is multiplied with is evaluated on the boundaries of the integration volume (for the parameter omitted) as detailed explicitly in
\cref{eqn:volumeintegral:400}.

\subsubsection{Three parameter Stokes' theorem.}

Three special cases of \cref{thm:volumeintegral:100} can be obtained by integrating scalar, vector or bivector functions over the volume, as follows

\maketheorem{Volume integral of scalar function (Stokes').}{thm:volumeintegral:420}{
Given a scalar function \( f(\Bx) \) its volume integral is given by
\begin{equation*}
\int_V d^3 \Bx \cdot \boldpartial f =
\int_V d^3 \Bx \cdot \spacegrad f = \ointctrclockwise_{\partial V} d^2\Bx f.
\end{equation*}
In \R{3}, this can be written as
\begin{equation*}
\int_V dV\, \spacegrad f = \int_{\partial V} dA\, \ncap f
\end{equation*}
where \( \ncap \) is the outwards normal specified by \( d^2 \Bx = I \ncap dA, \) and \( d^3 \Bx = I dV \).
} % theorem

\input{Theorem_volume_integral_of_vector_stokes.tex}

\input{Theorem_volume_integral_of_bivector_stokes_divergence.tex}

\subsubsection{Divergence theorem.}

Observe that for \R{3} we there are dot product relations in each of
\cref{thm:volumeintegral:420},
\cref{thm:volumeintegral:1661} and
\cref{thm:volumeintegral:1681} which can be summarized as
\index{divergence theorem}
\input{Theorem_divergence.tex}

%}
%\EndNoBibArticle

   \section{Multivector Fourier transform and phasors.}
      \input{fourier.tex}
   \section{Green's functions.}
      %
% Copyright � 2016 Peeter Joot.  All Rights Reserved.
% Licenced as described in the file LICENSE under the root directory of this GIT repository.
%
%{
\label{chap:GreensFunctions}
\index{Green's function}

\subsection{Motivation.}

%We will now introduce Green's functions, which provide a general method of solving many of the linear differential equations that will be encountered in electromagnetism.

%
% Copyright � 2018 Peeter Joot.  All Rights Reserved.
% Licenced as described in the file LICENSE under the root directory of this GIT repository.
%
%{
%%\input{../latex/blogpost.tex}
%%\renewcommand{\basename}{greensIntro}
%%%\renewcommand{\dirname}{notes/phy1520/}
%%\renewcommand{\dirname}{notes/ece1228-electromagnetic-theory/}
%%%\newcommand{\dateintitle}{}
%%%\newcommand{\keywords}{}
%%
%%\input{../latex/peeter_prologue_print2.tex}
%%
%%\usepackage{peeters_layout_exercise}
%%\usepackage{peeters_braket}
%%\usepackage{peeters_figures}
%%\usepackage{siunitx}
%%%\usepackage{mhchem} % \ce{}
%%%\usepackage{macros_bm} % \bcM
%%%\usepackage{macros_qed} % \qedmarker
%%\usepackage{macros_cal} % \LL
%%%\usepackage{txfonts} % \ointclockwise
%%
%%\beginArtNoToc
%%
%%\generatetitle{Green's functions.}
%%%\chapter{Green's functions.}
\label{chap:greensIntro}

Every engineer's toolbox includes Laplace and Fourier transform tricks for transforming differential equations to the frequency domain.
Here the space and time domain equivalent of the frequency and time domain linear system response function, called the Green's function, is introduced.

Everybody's favorite differential equation, the harmonic oscillator, can be used as an illustrative example
\begin{dmath}\label{eqn:greensIntro:300}
x'' + k x' + (\omega_0)^2 x = f(t).
\end{dmath}

Here derivatives are with respect to time, \( \omega_0 \) is the intrinsic angular frequency of the oscillator, \( k \) is a damping factor, and \( f(t) \) is a forcing function.
If the oscillator represents a child on a swing at the park (a pendulum system), then \( k \) represents the friction in the swing pivot and retardation due to wind, and the forcing function represents the father pushing the swing.  The forcing function \( f(t) \) could include an initial impulse to get the child up to speed, or could have a periodic aspect, such as the
father running underdogs\footnote{The underdog is a non-passive swing pushing technique, where you run behind and under the swing and child, giving a push as you go.
Before my kids learned to ``pump their legs'', and even afterwards, this was their favorite way of being pushed on the swing.
With two kids the Dad-forcing-function tires quickly, as it is applied repeatedly to both oscillating children.}
 as the child gleefully shouts ``Again, again, again!''

The full solution of this problem is
\( x(t) = x_s(t) + x_0(t) \),
where \( x_s(t) \) is a solution of \cref{eqn:greensIntro:300} and
\( x_0 \) is any solution of the homogeneous equation \( x_0'' + k x_0' + (\omega_0)^2 x_0 = 0 \), picked to satisfy the boundary value constraints of the problem.

Let's attack this problem initially with Fourier transforms (\cref{dfn:greensFunctionOverview:280})

We can assume zero position and velocity for the non-homogeneous problem, since we can adjust the boundary conditions with the homogeneous solution \( x_0(t) \).  With zero boundary conditions on \( x, x' \), the transform of \cref{eqn:greensIntro:300} is
\begin{dmath}\label{eqn:greensIntro:320}
((-j \omega)^2 + j \omega + (\omega_0)^2) X(\omega) = F(\omega),
\end{dmath}
so the system is solved in the frequency domain by the system response function \( G(\omega) \)
\begin{dmath}\label{eqn:greensIntro:340}
X(\omega) = G(\omega) F(\omega),
\end{dmath}
where
\begin{dmath}\label{eqn:greensIntro:360}
G(\omega) = -\inv{\omega^2 - j \omega - (\omega_0)^2}.
\end{dmath}

We can apply the inverse transformation to find the time domain solution for the forced oscillator problem.
\begin{dmath}\label{eqn:greensIntro:380}
x(t)
= \int d\omega G(\omega) F(\omega) e^{j \omega t}
= \int d\omega G(\omega) \lr{ \inv{2 \pi} \int dt' f(t') e^{-j \omega t'} dt' } e^{j \omega t}
= \int dt' f(t')
\lr{ \inv{2 \pi} \int d\omega G(\omega) e^{j \omega (t-t')} }.
\end{dmath}

The frequency domain integral is the Green's function.
We'll write this as
\begin{dmath}\label{eqn:greensIntro:400}
G(t,t') = \inv{2 \pi} \int d\omega G(\omega) e^{j \omega (t-t')}.
\end{dmath}

If we can evaluate this integral, then the system can be considered solved, where a solution is given by the convolution integral
\begin{dmath}\label{eqn:greensIntro:420}
x(t) = \int dt' f(t') G(t,t') + x_0(t).
\end{dmath}
The Green's function is the weighting factor that determines how much of \( f(t') \) for each time \( t' \) contributes to the motion of the system that is explicitly due to the forcing function.  Green's functions for physical problems are causal, so only forcing events in the past contribute to the current state of the system (i.e. if you were to depend on only future pushes of the swing, you would have a very bored child.)

An alternate way of viewing a linear systems problem is to assume that a convolution solution of the form
\cref{eqn:greensIntro:420} must exist.
Since the equation is a linear, it is reasonable to assume that a linear weighted sum of all the forcing function values must contribute to the solution.
If such a solution is assumed, then we can operate on that solution with the differential operator for the problem.
For our harmonic oscillator problem that operator is
\begin{dmath}\label{eqn:greensIntro:440}
\LL = \PDSq{t}{} + k \PD{t}{} + (\omega_0)^2.
\end{dmath}
We find
\begin{equation}\label{eqn:greensIntro:460}
\begin{aligned}
f(t)
&= \lr{ \PDSq{t}{} + k \PD{t}{} + (\omega_0)^2 } x(t) \\
&= \lr{ \PDSq{t}{} + k \PD{t}{} + (\omega_0)^2 } \int dt' f(t') G(t,t') \\
&= \int dt' f(t') \lr{ \PDSq{t}{} + k \PD{t}{} + (\omega_0)^2 } G(t,t'),
\end{aligned}
\end{equation}
and see that the Green's function, when acted on by the differential operator, must have the characteristics of a delta function
\begin{equation}\label{eqn:greensIntro:480}
\lr{ \PDSq{t}{} + k \PD{t}{} + (\omega_0)^2 }
G(t,t')
=
\delta( t - t').
\end{equation}

The problem of determining the Green's function, implicitly determining the solution of any forced system, can be viewed as seeking
the solution of distribution equations of the form
\boxedEquation{eqn:greensIntro:500}{
\LL G(t,t') = \delta( t - t').
}
Framing the problem this way is independent of whatever techniques (transform or other) that we may choose to use to determine the structure of the Green's function itself.  Observe that the Green's function itself is not unique.  In particular, we may add any solution of the homogeneous problem \( \LL G_0(t,t') = 0 \) to the Green's function, just as we can do so for the forced system itself.

We will see that Green's functions provide a general method of solving many of the linear differential equations that will be encountered in electromagnetism.
%}
%\EndNoBibArticle

%
% Copyright © 2023 Peeter Joot.  All Rights Reserved.
% Licenced as described in the file LICENSE under the root directory of this GIT repository.
%
%{
%\input{../latex/blogpost.tex}
%\renewcommand{\basename}{greensIntroProblem}
%%\renewcommand{\dirname}{notes/phy1520/}
%\renewcommand{\dirname}{notes/ece1228-electromagnetic-theory/}
%%\newcommand{\dateintitle}{}
%%\newcommand{\keywords}{}
%
%\input{../latex/peeter_prologue_print2.tex}
%
%\usepackage{peeters_layout_exercise}
%\usepackage{peeters_braket}
%\usepackage{peeters_figures}
%\usepackage{siunitx}
%\usepackage{verbatim}
%%\usepackage{mhchem} % \ce{}
%%\usepackage{macros_bm} % \bcM
%%\usepackage{macros_qed} % \qedmarker
%%\usepackage{txfonts} % \ointclockwise
%
%\beginArtNoToc
%
%\generatetitle{XXX}
%%\chapter{XXX}
%%\label{chap:greensIntroProblem}
%
\makeproblem{Harmonic oscillator Green's function.}{problem:greensIntroProblem:20}{
Evaluate the integral
\begin{equation}\label{eqn:greensIntroProblem:20}
G(\tau) = \frac{-1}{2 \pi} \int_{-\infty}^\infty \inv{ \omega^2 - 2 j \omega k - \omega_0^2 } e^{j \omega \tau} d\omega,
\end{equation}
using the semicircular infinite contours depicted in \cref{fig:greensFunctionContours:greensFunctionContoursFigure}.
\pimageFigure{../figures/GAelectrodynamics/}{greensFunctionContoursFigure}{Contours for harmonic oscillator Green's function.}{fig:greensFunctionContours:greensFunctionContoursFigure}{0.2}
} % problem
\makeanswer{problem:greensIntroProblem:20}{
With \( \alpha = \sqrt{ \omega_0^2 - k^2 } \), we may factor the denominator
\begin{equation}\label{eqn:greensIntroProblem:40}
\omega^2 - 2 j \omega k - \omega_0^2 = \lr{ \omega - \lr{ j k + \alpha } } \lr{ \omega - \lr{ j k - \alpha } },
\end{equation}
showing that we have poles in the upper half plane at \( j k \pm \alpha \).

It's important to understand the behaviour of the integral on the infinite semi-circular contours, which we can parameterize as \( \omega = R e^{i\theta} \).  The denominator is \( O(1/R^2) \), but the exponential has the form
\begin{equation}\label{eqn:greensIntroProblem:60}
\begin{aligned}
e^{j \omega \tau}
&= e^{j \tau R\lr{ \cos\theta + j \sin\theta } } \\
&= e^{j \tau R \cos\theta } e^{ - R \tau \sin\theta }.
\end{aligned}
\end{equation}
We see that the integral diverges on the upper half contour for \( \tau < 0 \), and diverges on the lower half contour for \( \tau > 0 \).  There's a theorem (who's name I forget) that shows that the upper half contour integral evaluated for \( \tau > 0 \) will be zero on the infinite semicircle, as will the lower semicircular contour for \( \tau < 0 \), so if we compute the residues for the complete contours we find the value of the integral along the \( [-\infty,\infty] \) horizontal.

We find
\begin{equation}\label{eqn:greensIntroProblem:80}
\begin{aligned}
G(\tau < 0) &= 0 \\
G(\tau > 0) &=
\frac{-1}{2 \pi} 2 \pi j \lr{
   \evalbar{
   \frac{e^{j \omega \tau}}{\omega - \lr{ j k + \alpha } }
   }
   {
      \omega = j k - \alpha
   }
   +
   \evalbar{
   \frac{e^{j \omega \tau}}{\omega - \lr{ j k - \alpha } }
   }
   {
      \omega = j k + \alpha
   }
} \\
&=
\inv{j} \lr{
   \frac{e^{j \tau \lr{ j k - \alpha } }}{-2\alpha}
   +
   \frac{e^{j \tau \lr{ j k + \alpha } }}{ 2\alpha}
} \\
&=
\inv{\alpha} e^{ - k \tau } \sin\lr{ \alpha \tau },
\end{aligned}
\end{equation}
or
\begin{equation}\label{eqn:greensIntroProblem:100}
G(\tau) = \Theta(\tau) e^{ - k \tau } \frac{ \sin\lr{ \alpha \tau } }{\alpha}.
\end{equation}
Rather amusingly, when the system is supplied with an impulse function \( f(t) = \delta(t) \), we see that the response to that infinite push on the swing is
\begin{equation}\label{eqn:greensIntroProblem:120}
\begin{aligned}
x(t)
&= \int_{-\infty}^\infty \Theta(t-t') e^{ - k (t-t') } \frac{ \sin\lr{ \alpha (t-t') } }{\alpha} \delta(t') dt' \\
&= \Theta(t) e^{-k t} \frac{ \sin\lr{ \alpha t } }{\alpha},
\end{aligned}
\end{equation}
which describes an oscillation that starts at the point of the push, but decreases in amplitude steadily after that due to the damping term.  Even with an infinite strength initial push, the child will eventually be exhorting the dad to supply another underdog, ``Again, again, again!''.
} % answer
%}
%\EndNoBibArticle


\subsubsection{Time domain problems in electromagnetism}

Examples of the PDEs that we can apply Green's function techniques to include

\begin{subequations}
\label{eqn:greensFunctionOverview:200}
\begin{equation}\label{eqn:greensFunctionOverview:220}
\stgrad F(\Bx, t) = J(\Bx, t)
\end{equation}
\begin{equation}\label{eqn:greensFunctionOverview:162}
\lr{ \spacegrad^2 - \inv{c^2} \PDSq{t}{} } F(\Bx, t) =
\conjstgrad
\stgrad F(\Bx, t) =
B(\Bx, t).
\end{equation}
\end{subequations}

The reader is no doubt familiar with the wave equation (\cref{eqn:greensFunctionOverview:162}), where \( F \) is the waving function, and \( B \) is the forcing function.
Scalar and vector valued wave equations are
encountered in scalar and vector forms in conventional electromagnetism.
We will see multivector variations of the wave equation, so it should be assumed that \( F \) and \( B \) are multivector valued.

\Cref{eqn:greensFunctionOverview:220} is actually the geometric algebra form of Maxwell's equation (singular),
where \( F \) is a multivector with grades 1 and 2, and \( J \) is a multivector containing all the charge and current density contributions.
We will call the operator in \cref{eqn:greensFunctionOverview:220} the spacetime gradient\footnote{A slightly different operator is also called the spacetime gradient in STA (Space Time Algebra) \citep{doran2003gap}, which employs a non-Euclidean basis to generate a four dimensional relativistic geometric algebra.
Our spacetime gradient is related to the STA spacetime gradient by a constant factor.}.


Armed with Fourier transform or phasor representations, the frequency domain representations of
\cref{eqn:greensFunctionOverview:200} are found to be

\index{Helmholtz operator}
\begin{subequations}
\label{eqn:greensFunctionOverview:320}
\begin{equation}\label{eqn:greensFunctionOverview:240}
\lr{ \spacegrad + j k } F(\Bx) = J(\Bx)
\end{equation}
\begin{equation}\label{eqn:greensFunctionOverview:260}
\lr{ \spacegrad^2 + k^2 } F(\Bx)
=
\lr{ \spacegrad - j k } \lr{ \spacegrad + j k } F(\Bx)
= B(\Bx),
\end{equation}
\end{subequations}

where \( k = \omega/c \), and any explicit frequency dependence in our transform pairs has been suppressed.
We will call these equations the first and second order Helmholtz equations respectively.
The first order equation applies a multivector differential operator to a multivector field, which must equal the multivector forcing function (the sources).

For statics problems (\( k = 0 \)), we may work with real fields and sources, dispensing with any need to take real parts.

%}

      \subsection{Green's function solutions.}
         %
% Copyright © 2018 Peeter Joot.  All Rights Reserved.
% Licenced as described in the file LICENSE under the root directory of this GIT repository.
%
%{
\subsubsection{Unbounded.}

The operators in \cref{eqn:greensFunctionOverview:200}, and \cref{eqn:greensFunctionOverview:320} all have a similar linear structure.
Abstracting that structure, all these problems have the form
\begin{equation}\label{eqn:greensFunctionSolutions:340}
\LL F(\Bx) = J(\Bx),
\end{equation}
where \( \LL \) is an operator formed from a linear combination of linear operators \( 1, \spacegrad, \spacegrad^2, \partial_t, \partial_{tt} \).

Given the linear structure of the PDE that we wish to solve, it makes sense to assume that the solutions also have a linear structure.
The most general such solution we can assume has the form

\index{Green's function}
\begin{equation}\label{eqn:greensFunctionSolutions:360}
F(\Bx, t) = \int G(\Bx, \Bx' ; t, t') J(\Bx', t') dV' dt' + F_0(\Bx, t),
\end{equation}
where \( F_0(\Bx, t) \) is any solution to the equivalent homogeneous equation \( \LL F_0 = 0 \), and \( G(\Bx, \Bx' ; t, t') \) is the Green's function (to be determined) associated with \cref{eqn:greensFunctionSolutions:340}.
Operating on the presumed solution
\cref{eqn:greensFunctionSolutions:360} with \( \LL \) yields
\begin{equation}\label{eqn:greensFunctionSolutions:380}
\begin{aligned}
J(\Bx, t)
&= \LL F(\Bx, t) = \LL\lr{ \int G(\Bx, \Bx' ; t, t') J(\Bx', t') dV' dt' + F_0(\Bx, t) } \\
&= \int \lr{ \LL G(\Bx, \Bx'; t, t') } J(\Bx', t') dV' dt',
\end{aligned}
\end{equation}
which shows that we require the Green's function to have delta function semantics satisfying
\begin{equation}\label{eqn:greensFunctionSolutions:400}
\LL G(\Bx, \Bx' ; t, t') = \delta(\Bx - \Bx') \delta(t - t').
\end{equation}

The scalar valued Green's functions for the Laplacian and the (2nd order) Helmholtz equations are well known.
The Green's functions for the spacetime gradient and the 1st order Helmholtz equation (which is just the gradient when \( k = 0 \)) are multivector valued and will be derived here.

%%%For multivector functions, it can be helpful to assume that the assumed solution
%%%\cref{eqn:greensFunctionSolutions:360} includes a grade selection operation.  In particular, for the electromagnetic field, which has only grades 1,2, we may start by demanding that our solution is of the form
%%%
%%%\begin{equation}\label{eqn:greensFunctionSolutions:360}
%%%F(\Bx) = \gpgrade{ \int G(\Bx, \Bx') J(\Bx') dV'}{1,2} + F_0(\Bx),
%%%\end{equation}
%%%
\subsubsection{Green's theorem.}

When the presumed solution is a superposition of only states in a bounded region
then life gets a bit more interesting.  For instance, consider a problem for which the differential operator is a function of space only, with a presumed solution such as
\begin{equation}\label{eqn:greensFunctionSolutions:200}
F(\Bx) = \int_V dV' B(\Bx') G(\Bx, \Bx') + F_0(\Bx),
\end{equation}
then life gets a bit more interesting.
This sort of problem requires different treatment for operators that are first and second order in the gradient.

For the second order problems, we require Green's theorem, which must be generalized slightly for use with multivector fields.

The basic idea is that we can relate the Laplacian of the Green's function and the field
\( F(\Bx') \lr{ (\spacegrad')^2 G(\Bx, \Bx') } = G(\Bx, \Bx') \lr{ (\spacegrad')^2 F(\Bx')} + \cdots \).
That relationship can be expressed as the integral of an antisymmetric sandwich of the two functions

\maketheorem{Green's theorem}{thm:gradientGreensFunctionEuclidean:220}{
Given a multivector function \( F \) and a scalar function \( G \)
\begin{equation*}
\int_V \lr{ F \spacegrad^2 G - G \spacegrad^2 F } dV = \int_{\partial V} \lr{ F \ncap \cdot \spacegrad G - G \ncap \cdot \spacegrad F },
\end{equation*}
where \( \partial V \) is the boundary of the volume \( V \).
} % theorem
\begin{proof}
A straightforward, but perhaps inelegant way of proving this theorem is to expand the antisymmetric product in coordinates
\begin{equation}\label{eqn:greensFunctionSolutions:260}
\begin{aligned}
F \spacegrad^2 G - G \spacegrad^2 F
&= \sum_k F \partial_k \partial_k G - G \partial_k \partial_k F \\
&= \sum_k \partial_k \lr{ F \partial_k G - G \partial_k F } - (\partial_k F)(\partial_k G) + (\partial_k G)(\partial_k F).
\end{aligned}
\end{equation}

Since \( G \) is a scalar, the last two terms cancel, and we can integrate
\begin{equation}\label{eqn:greensFunctionSolutions:280}
\int_V \lr{ F \spacegrad^2 G - G \spacegrad^2 F } dV
=
\sum_k \int_V \partial_k \lr{ F \partial_k G - G \partial_k F }.
\end{equation}

Each integral above involves one component of the gradient.
From
%the fundamental theorem of geometric calculus
\cref{thm:fundamentalTheoremOfCalculus:1}
we know that
\begin{equation}\label{eqn:greensFunctionSolutions:300}
\int_V \spacegrad Q dV = \int_{\partial V} \ncap Q dA,
\end{equation}
for any multivector \( Q \).
Equating components gives
\begin{equation}\label{eqn:greensFunctionSolutions:460}
\int_V \partial_k Q dV = \int_{\partial V} \ncap \cdot \Be_k Q dA,
\end{equation}
which can be substituted into \cref{eqn:greensFunctionSolutions:280} to find
\begin{equation}\label{eqn:greensFunctionSolutions:480}
\begin{aligned}
\int_V \lr{ F \spacegrad^2 G - G \spacegrad^2 F } dV
&= \sum_k \int_{\partial V} \ncap \cdot \Be_k \lr{ F \partial_k G - G \partial_k F } dA \\
&= \int_{\partial V} \lr{ F (\ncap \cdot \spacegrad) G - G (\ncap \cdot \spacegrad) F } dA.
\end{aligned}
\end{equation}
\end{proof}

\subsubsection{Bounded solutions to first order problems.}

For first order problems we will need an intermediate result similar to Green's theorem.

\index{\(\spacegrad'\)}
\index{\(\rspacegrad'\)}
\index{\(\lspacegrad'\)}
\makelemma{Normal relations for a gradient sandwich.}{lemma:greensFunctionOverview:420}{
Given multivector functions \( F(\Bx'), G(\Bx, \Bx') \), and a gradient \( \spacegrad' \) acting bidirectionally on functions of \( \Bx' \), we have
\begin{equation*}
\begin{aligned}
- \int_V \lr{ G(\Bx, \Bx') \lspacegrad' } F(\Bx') dV'
&=
\int_V G(\Bx, \Bx') \lr{ \rspacegrad' F(\Bx') } dV' \\
&\qquad -
\int_{\partial V} G(\Bx, \Bx') \ncap' F(\Bx') dA'.
\end{aligned}
\end{equation*}
} % lemma
\begin{proof}
% caused mysterious latex error in the .ind file:
%\index{\(\lrspacegrad'\)}
\index{\(\lroverarrow{\spacegrad}'\)}
This follows directly from \cref{thm:fundamentalTheoremOfCalculus:1}
\begin{equation}\label{eqn:greensFunctionSolutions:440}
\begin{aligned}
   \int_{\partial V} &G(\Bx, \Bx') \ncap' F(\Bx') dA'
= \int_V G(\Bx, \Bx') \lrspacegrad' F(\Bx') dV' \\
&\quad = \int_V \lr{ G(\Bx, \Bx') \lspacegrad' } F(\Bx') dV' + \int_V G(\Bx, \Bx') \lr{ \rspacegrad' F(\Bx') } dV',
\end{aligned}
\end{equation}
which can be rearranged to prove \cref{lemma:greensFunctionOverview:420}.
\end{proof}
%}

      \subsection{Helmholtz equation.}
         %
% Copyright © 2018 Peeter Joot.  All Rights Reserved.
% Licenced as described in the file LICENSE under the root directory of this GIT repository.
%
%{
\subsubsection{Unbounded superposition solutions for the Helmholtz equation.}

The specialization of \cref{eqn:greensFunctionSolutions:400} to the Helmholtz equation \cref{eqn:greensFunctionOverview:260} is
\begin{equation}\label{eqn:greensFunctionHelmholtz:420}
\lr{ \spacegrad^2 + k^2 } G(\Bx, \Bx') = \delta(\Bx - \Bx').
\end{equation}

While it is possible \citep{schwinger1998classical} to derive the Green's function using Fourier transform techniques, we will state the result instead, which is well known

\index{Helmholtz!Green's function}
\index{Green's function!Helmholtz}
\input{Theorem_greens_function_helmholtz.tex}

We will use the advancing (causal) Green's function, and refer to this function as \( G(\Bx, \Bx') \) without any subscript.
Because it may not be obvious that these
Green's function representations are valid in a multivector context, a demonstration of this fact can be found in \cref{chap:helmholtzGreens}.

\index{Laplacian!Green's function}
\index{Green's function!Laplacian}
Observe that as a special case, the Helmholtz Green's function reduces to the Green's function for the Laplacian when \( k = 0 \)
\begin{equation}\label{eqn:greensFunctionHelmholtz:80}
G(\Bx, \Bx') = -\inv{ 4 \pi \Norm{\Bx - \Bx'}}.
\end{equation}

\subsubsection{Bounded superposition solutions for the Helmholtz equation.}

For our application of
\cref{thm:gradientGreensFunctionEuclidean:3} to the Helmholtz problem, we
are actually interested in a antisymmetric sandwich of the Helmholtz operator by the function \( F \) and the scalar (Green's) function \( G \), but
that reduces to an asymmetric sandwich of our functions around the Laplacian
\begin{equation}\label{eqn:greensFunctionHelmholtz:240}
\begin{aligned}
F &\lr{ \spacegrad^2 + k^2 } G - G \lr{ \spacegrad^2 + k^2 } F \\
&=
F \spacegrad^2 G + \cancel{F k^2 G} - G \spacegrad^2 F - \cancel{G k^2 F} \\
&=
F \spacegrad^2 G - G \spacegrad^2 F,
\end{aligned}
\end{equation}
so
\begin{equation}\label{eqn:greensFunctionHelmholtz:380}
\begin{aligned}
\int_V &F(\Bx') \lr{ (\spacegrad')^2 + k^2 } G(\Bx, \Bx') \\
&= \int_V G(\Bx, \Bx') \lr{ (\spacegrad')^2 + k^2} F(\Bx') dV'  \\
&\quad + \int_{\partial V} \lr{ F(\Bx') (\ncap' \cdot \spacegrad') G(\Bx, \Bx') - G(\Bx, \Bx') (\ncap' \cdot \spacegrad') F(\Bx') } dA'.
\end{aligned}
\end{equation}

This shows that if we assume the Green's function satisfies
the delta function condition
\cref{eqn:greensFunctionHelmholtz:420}
, then the general solution of
\cref{eqn:greensFunctionOverview:260} is
formed from a bounded superposition of sources is
\boxedEquation{eqn:gradientGreensFunctionEuclidean:400}{
\begin{aligned}
F(\Bx) &=
\int_V G(\Bx, \Bx') B( \Bx' ) dV' \\
&+
\int_{\partial V} \lr{
 G(\Bx, \Bx') (\ncap' \cdot \spacegrad') F(\Bx')
-F(\Bx') (\ncap' \cdot \spacegrad') G(\Bx, \Bx')
} dA'.
\end{aligned}
}

We are also free to add in any specific solution \( F_0(\Bx) \) that satisfies the
homogeneous Helmholtz equation.
There is also freedom to add any solution of the homogeneous Helmholtz equation to the Green's function itself, so it is not unique.
For a bounded superposition we generally desire that the solution \( F \) and its normal derivative, or the Green's function \( G \) (and it's normal derivative) or an appropriate combination of the two are zero on the boundary, so that the surface integral is killed.

%}

      \subsection{First order Helmholtz equation.}
         %
% Copyright © 2018 Peeter Joot.  All Rights Reserved.
% Licenced as described in the file LICENSE under the root directory of this GIT repository.
%
%{

The specialization of \cref{eqn:greensFunctionSolutions:400} to the first order Helmholtz equation \cref{eqn:greensFunctionOverview:240} is
\begin{equation}\label{eqn:greensFunctionFirstOrderHelmholtz:700}
\lr{ \spacegrad + j k } G(\Bx, \Bx')  = \delta(\Bx - \Bx').
\end{equation}

This Green's function is multivector valued
\input{Theorem_greens_function_helmholtz_first_order.tex}
A special but important case is the \( k = 0 \) condition, which provides the
Green's function for the gradient, which is vector valued
\begin{equation}\label{eqn:greensFunctionFirstOrderHelmholtz:900}
G(\Bx, \Bx' ; k = 0) = \inv{4 \pi} \frac{\rcap}{r^2}.
\end{equation}
\begin{proof}
If we denote the (advanced) Green's function for the 2nd order Helmholtz operator
\cref{thm:gradientGreensFunctionEuclidean:3}
as \( \phi(\Bx, \Bx') \), we must have
\begin{equation}\label{eqn:greensFunctionFirstOrderHelmholtz:740}
\lr{ \rspacegrad + j k } G(\Bx, \Bx') = \delta(\Bx - \Bx') =
\lr{ \rspacegrad + j k } \lr{ \rspacegrad - j k } \phi(\Bx, \Bx'),
\end{equation}
we see that the Green's function is given by
\begin{equation}\label{eqn:greensFunctionFirstOrderHelmholtz:760}
G(\Bx, \Bx')
=
\lr{ \rspacegrad - j k } \phi(\Bx, \Bx').
\end{equation}

This can be computed directly
\begin{equation}\label{eqn:greensFunctionFirstOrderHelmholtz:780}
\begin{aligned}
G(\Bx, \Bx')
&= \lr{ \rspacegrad - j k } \lr{ -\frac{e^{-j k r}}{4 \pi r} } \\
&= \lr{ \rcap \PD{r}{} -j k } \lr{ -\frac{e^{-j k r}}{4 \pi r} } \\
&= \frac{-e^{-j k r}}{4 \pi} \lr{ \rcap \lr{ -\frac{j k}{r} - \inv{ r^2 } } - \frac{j k}{r} } \\
&= \frac{e^{-j k r}}{4 \pi} \lr{ j k \lr{ 1 + \rcap } + \frac{\rcap}{r} },
\end{aligned}
\end{equation}
as claimed.
\end{proof}
Observe that since \( \phi \) is scalar valued, we can also rewrite
\cref{eqn:greensFunctionFirstOrderHelmholtz:760} in terms of a right acting operator
\begin{equation}\label{eqn:greensFunctionFirstOrderHelmholtz:800}
\begin{aligned}
G(\Bx, \Bx')
&= \phi(\Bx, \Bx') \lr{ \lspacegrad - j k } \\
&= \phi(\Bx, \Bx') \lr{ -\lspacegrad' - j k },
\end{aligned}
\end{equation}
so
\begin{equation}\label{eqn:greensFunctionFirstOrderHelmholtz:820}
G(\Bx, \Bx') \lr{ -\lspacegrad' + j k } =
\phi(\Bx, \Bx') \lr{ (\lspacegrad')^2 + k^2 }
=
\delta(\Bx - \Bx').
\end{equation}

This is relevant for bounded superposition states, which we will discuss next now that the proof of
\cref{thm:gradientGreensFunctionEuclidean:720} is complete.
In particular addition of
\( \int_V G(\Bx, \Bx') j k F(\Bx') dV' \) to both sides of \cref{lemma:greensFunctionOverview:420} gives
\begin{equation}\label{eqn:greensFunctionFirstOrderHelmholtz:860}
\begin{aligned}
\int_V \lr{ G(\Bx, \Bx') \lr{ -\lspacegrad' + j k } } F(\Bx') dV'
&=
\int_V G(\Bx, \Bx') \lr{ \lr{ \rspacegrad' + j k } F(\Bx') } dV' \\
&-
\int_{\partial V} G(\Bx, \Bx') \ncap' F(\Bx') dA'.
\end{aligned}
\end{equation}

Utilizing \cref{thm:gradientGreensFunctionEuclidean:720}, and substituting \( J(\Bx') \)
from \cref{eqn:greensFunctionOverview:240},
we find that one solution to the first order Helmholtz equation is
\begin{equation}\label{eqn:greensFunctionFirstOrderHelmholtz:880}
F(\Bx)
=
\int_V G(\Bx, \Bx') J(\Bx') dV'
-
\int_{\partial V} G(\Bx, \Bx') \ncap' F(\Bx') dA'.
\end{equation}

We are free to
add any specific solution \( F_0 \) that satisfies the homogeneous equation \( \lr{ \spacegrad + j k } F_0 = 0 \).
%}

      \subsection{Spacetime gradient.}
         %
% Copyright � 2018 Peeter Joot.  All Rights Reserved.
% Licenced as described in the file LICENSE under the root directory of this GIT repository.
%
%{
%%%\input{../latex/blogpost.tex}
%%%\renewcommand{\basename}{greensFunctionSpacetimeGradient}
%%%%\renewcommand{\dirname}{notes/phy1520/}
%%%\renewcommand{\dirname}{notes/ece1228-electromagnetic-theory/}
%%%%\newcommand{\dateintitle}{}
%%%%\newcommand{\keywords}{}
%%%
%%%\input{../latex/peeter_prologue_print2.tex}
%%%
%%%\usepackage{peeters_layout_exercise}
%%%\usepackage{peeters_braket}
%%%\usepackage{peeters_figures}
%%%\usepackage{siunitx}
%%%%\usepackage{mhchem} % \ce{}
%%%%\usepackage{macros_bm} % \bcM
%%%%\usepackage{macros_qed} % \qedmarker
%%%%\usepackage{txfonts} % \ointclockwise
%%%
%%%\beginArtNoToc
%%%
%%%\generatetitle{Green's function for the spacetime gradient}
%%%%\chapter{Green's function for the spacetime gradient}
%%%%\label{chap:greensFunctionSpacetimeGradient}
%%%% \citep{griffiths1999introduction}
We want to find the Green's function that solves spacetime gradient equations of the form \cref{eqn:greensFunctionOverview:220}.
For the wave equation operator, it is helpful to introduce a d'Alembertian operator, defined as follows.
\index{\(\dAlembertian\)}
%
% Copyright � 2018 Peeter Joot.  All Rights Reserved.
% Licenced as described in the file LICENSE under the root directory of this GIT repository.
%
\makedefinition{d'Alembertian (wave equation) operator.}{dfn:continuity:120}{
The symbol \( \dAlembertian \) is used to represent the \textit{d'Alembertian (wave equation) operator}, with a positive sign on the Laplacian term
\begin{equation*}
\dAlembertian =
\conjstgrad
\stgrad
=
\spacegrad^2 - \inv{c^2} \PDSq{t}{}.
\end{equation*}
} % definition


We will be able to derive the Green's function for the spacetime gradient from the Green's function for the d'Alembertian.  The Green's function for the spacetime gradient is multivector valued and given by the following.
%
% Copyright � 2018 Peeter Joot.  All Rights Reserved.
% Licenced as described in the file LICENSE under the root directory of this GIT repository.
%
\maketheorem{Green's function for the spacetime gradient.}{thm:greensFunctionSpacetimeGradient:120}{
The \textit{Green's function for the spacetime gradient} \( \spacegrad + (1/c) \partial_t \) satisfies
\begin{equation*}
\stgrad G(\Bx - \Bx', t - t') = \delta(\Bx - \Bx') \delta(t - t'),
\end{equation*}
and has the value
\begin{equation*}
G(\Bx - \Bx', t - t')
=
\inv{4\pi} \lr{
- \frac{\rcap}{r^2} \PD{r}{}
+ \frac{\rcap}{r^2}
+ \inv{c r} \PD{t}{}
}
\delta( -r/c + t - t' ),
\end{equation*}
where \( \Br = \Bx - \Bx', r = \Norm{\Br} \) and \( \rcap = \Br/r \).
} % theorem


With the help of \cref{eqn:derivativeOfDeltaFunction:140}
it is possible to further evaluate the delta function derivatives, however, we will defer doing so until we are ready to apply this Green's
function in a convolution integral to solve Maxwell's equation.
\begin{proof}
To prove this result, let \( \phi(\Bx - \Bx', t - t') \) be the retarded time (causal)
Green's function for the wave equation, satisfying
\begin{equation}\label{eqn:greensFunctionSpacetimeGradient:40}
\begin{aligned}
\dAlembertian \phi(\Bx - \Bx', t - t')
&= \stgrad \conjstgrad \phi(\Bx - \Bx', t - t') \\
&= \delta(\Bx - \Bx') \delta(t - t').
\end{aligned}
\end{equation}

This function has the value
\begin{equation}\label{eqn:greensFunctionSpacetimeGradient:60}
\phi(\Br, t - t')
=
-\inv{4 \pi r} \delta( -r/c + t - t' ),
\end{equation}
where \( \Br = \Bx - \Bx', r = \Norm{\Br} \).  Derivations of this Green's function, and it's acausal advanced time friend, can be found in
\citep{schwinger1998classical}, \citep{jackson1975cew}, and use the usual Fourier transform and contour integration tricks.

Comparing \cref{eqn:greensFunctionSpacetimeGradient:40} to the defining statement of \cref{thm:greensFunctionSpacetimeGradient:120}, we see that the spacetime gradient Green's function is given by
\begin{equation}\label{eqn:greensFunctionSpacetimeGradient:80}
\begin{aligned}
G(\Bx - \Bx', t - t')
&= \conjstgrad \phi(\Br, t - t') \\
&= \lr{ \rcap \PD{r}{} - \inv{c} \PD{t}{} } \phi(\Br, t - t'),
\end{aligned}
\end{equation}
where \( \rcap = \Br/r \).  Evaluating the derivatives gives
\begin{equation}\label{eqn:greensFunctionSpacetimeGradient:100}
\begin{aligned}
G&(\Br, t - t')
= -\inv{4\pi} \lr{ \rcap \PD{r}{} - \inv{c} \PD{t}{} } \frac{ \delta( -r/c + t - t' ) }{r} \\
&= -\inv{4\pi} \lr{
\frac{\rcap}{r} \PD{r}{} \delta( -r/c + t - t' )
- \frac{\rcap}{r^2} \delta( -r/c + t - t' )
- \inv{c r} \PD{t}{} \delta( -r/c + t - t' )
},
\end{aligned}
\end{equation}
which completes the proof after some sign cancellation and minor rearrangement.
\end{proof}
%}
%%%\EndArticle
%%%%\EndNoBibArticle

   \section{Helmholtz theorem.}
      \input{helmholtzDerviationMultivectorStatement.tex}
      %
% Copyright © 2016 Peeter Joot.  All Rights Reserved.
% Licenced as described in the file LICENSE under the root directory of this GIT repository.
%
%{
The conventional proof of Helmholtz's theorem uses the Green's function for the (second order) Helmholtz operator.
Armed with a vector valued Green's function for the gradient, a first order proof is also possible.
As illustrations of the geometric integration theory developed in this chapter, both
strategies will be applied here to this problem.

In either case, we start by forming an even grade multivector (gradient) equation containing both the dot and cross product contributions
\begin{equation}\label{eqn:helmholtzDerviationMultivectorSolution:60}
\spacegrad \BM
= \spacegrad \cdot \BM + I \spacegrad \cross \BM
= s + I \BC.
\end{equation}

\paragraph{First order proof.}

For the first order case, we
perform a grade one selection of \cref{lemma:greensFunctionOverview:420}, setting
\( F = \BM \) where \( G \) is the Green's function for the gradient given by
\cref{eqn:greensFunctionFirstOrderHelmholtz:900}.  The proof follows directly

\begin{equation}\label{eqn:helmholtzDerviationMultivectorSolution:820}
\begin{aligned}
M(\Bx)
&= - \int_V \lr{ G(\Bx, \Bx') \lspacegrad' } \BM(\Bx') dV' \\
&= \int_V \gpgradeone{G(\Bx, \Bx') \lr{ \rspacegrad' \BM(\Bx') }} dV'
-
\int_{\partial V} \gpgradeone{ G(\Bx, \Bx') \ncap' \BM(\Bx') } dA' \\
&=
\int_V
\inv{4 \pi \Norm{\Bx - \Bx'}^3 }
\gpgradeone{ (\Bx - \Bx') \lr{ s(\Bx') + I \BC(\Bx') }} dV' \\
&\quad -
\int_{\partial V}
\inv{4 \pi \Norm{\Bx - \Bx'}^3 }
\gpgradeone{ (\Bx - \Bx') \ncap' \BM(\Bx') } dA' \\
&=
\int_V
\inv{4 \pi \Norm{\Bx - \Bx'}^3 }
\lr{ (\Bx - \Bx') s(\Bx') - (\Bx - \Bx') \cross \BC(\Bx') } dV' \\
&\quad -
\int_{\partial V}
\inv{4 \pi \Norm{\Bx - \Bx'}^3 }
\gpgradeone{ (\Bx - \Bx') \ncap' \BM(\Bx') } dA'.
\end{aligned}
\end{equation}
If \( \BM \) is well behaved enough that the boundary integral vanishes on an infinite surface, we see that \( \BM \) is completely specified by the divergence and the curl.
In general, the divergence and the curl, must also be supplemented by the value of vector valued function on the boundary.

Observe that the boundary integral has a particularly simple form for a spherical surface or radius \( R \) centered on \( \Bx' \).
Switching to spherical coordinates \( \Br = \Bx' - \Bx = R\, \rcap(\theta, \phi) \) where \( \rcap = (\Bx' - \Bx)/\Norm{\Bx' - \Bx} \) is the outwards normal, we have
\begin{equation}\label{eqn:helmholtzDerviationMultivectorSolution:840}
\begin{aligned}
-
\int_{\partial V} &
\inv{4 \pi \Norm{\Bx - \Bx'}^3 }
\gpgradeone{ (\Bx - \Bx') \ncap' \BM(\Bx') } dA' \\
&= \int_{\partial V} \frac{\BM(\Bx')}{4 \pi \Norm{\Bx - \Bx'}^2 } dA' \\
&= \inv{4\pi} \int_{\theta = 0}^\pi \int_{\phi = 0}^{2 \pi} \BM(R, \theta, \phi) \sin\theta d\theta d\phi.
\end{aligned}
\end{equation}
This is an average of \( \BM \) over the surface of the radius-\(R\) sphere surrounding the point \( \Bx \) where the field \( \BM \) is evaluated.

\paragraph{Second order proof.}

%Observe that the Laplacian of \( \BM \) is vector valued
%
%\begin{equation}\label{eqn:helmholtzDerviationMultivectorSolution:760}
%\spacegrad^2 \BM = \spacegrad s + I \spacegrad \BC.
%\end{equation}
%
%This means that \( \spacegrad \BC \) must be a bivector \( \spacegrad \BC = \spacegrad \wedge \BC \), or that \( \BC \) has zero divergence
%
%\begin{equation}\label{eqn:helmholtzDerviationMultivectorSolution:780}
%\spacegrad \cdot \BC = 0.
%\end{equation}

Again, we use \cref{eqn:helmholtzDerviationMultivectorSolution:60}
to discover the relation between the vector \( \BM \) and its divergence and curl.
\index{delta function}
The vector \( \BM \) can be expressed at the point of interest as a convolution with the delta function at all other points in space
\index{convolution}
\begin{equation}\label{eqn:helmholtzDerviationMultivectorSolution:80}
\BM(\Bx) = \int_V dV'\, \delta(\Bx - \Bx') \BM(\Bx').
\end{equation}

\index{Laplacian}
The Laplacian representation of the delta function in \R{3} is
\begin{equation}\label{eqn:helmholtzDerviationMultivectorSolution:100}
\delta(\Bx - \Bx') = -\inv{4\pi} \spacegrad^2 \inv{\Norm{\Bx - \Bx'}},
\end{equation}
so \( \BM \) can be represented as the following convolution
\begin{equation}\label{eqn:helmholtzDerviationMultivectorSolution:120}
\BM(\Bx) = -\inv{4\pi} \int_V dV'\, \spacegrad^2 \inv{\Norm{\Bx - \Bx'}} \BM(\Bx').
\end{equation}

%As noted in \cref{eqn:helmholtzDerviationMultivector:460} the Laplacian of a vector can be factored as
%
%\begin{dmath}\label{eqn:helmholtzDerviationMultivectorSolution:140}
%\spacegrad^2 \Ba
%=
%\spacegrad (\spacegrad \cdot \Ba)
%-
%\spacegrad \cross (\spacegrad \cross \Ba).
%\end{dmath}
%
%Note that the last term can be written in cross product notation using \( \Bc \cdot (\Ba \wedge \Bb) = -\Bc \cross (\Ba \cross \Bb) \) if desired.

Using this relation and proceeding with a few applications of the chain rule, plus the fact that \( \spacegrad 1/\Norm{\Bx - \Bx'} = -\spacegrad' 1/\Norm{\Bx - \Bx'} \), we find
%
%I previously posted a Geometric Algebra attack on the Helmholtz theorem.  Here is
%
%Here's a third way of deriving the Helmholtz theorem inversion relation.  This is a refinement of the traditional vector algebra solution that led to \cref{eqn:helmholtzDerviationMultivector:200}, that uses a factorization of the Laplacian directly, deferring any expansion in terms of dot and cross (or wedge) products until the very end.
%
%Starting from the first line of \cref{eqn:helmholtzDerviationMultivector:160}, we have
\begin{equation}\label{eqn:helmholtzDerviationMultivectorSolution:720}
\begin{aligned}
-4 \pi \BM(\Bx)
&= \int_V dV'\, \spacegrad^2 \inv{\Norm{\Bx - \Bx'}} \BM(\Bx') \\
&= \gpgradeone{\int_V dV'\, \spacegrad^2 \inv{\Norm{\Bx - \Bx'}} \BM(\Bx')} \\
&= -\gpgradeone{\int_V dV'\, \spacegrad \lr{ \spacegrad' \inv{\Norm{\Bx - \Bx'}}} \BM(\Bx')} \\
&= -\gpgradeone{\spacegrad \int_V dV' \lr{
   \spacegrad' \frac{\BM(\Bx')}{\Norm{\Bx - \Bx'}}
   -\frac{\spacegrad' \BM(\Bx')}{\Norm{\Bx - \Bx'}}
   } } \\
&= -\gpgradeone{\spacegrad \int_{\partial V} dA'\,
   \ncap \frac{\BM(\Bx')}{\Norm{\Bx - \Bx'}}
    }
   +\gpgradeone{\spacegrad \int_V dV'
   \frac{s(\Bx') + I\BC(\Bx')}{\Norm{\Bx - \Bx'}}
    } \\
&= -\gpgradeone{\spacegrad \int_{\partial V} dA'\,
   \ncap \frac{\BM(\Bx')}{\Norm{\Bx - \Bx'}}
    }
   +\spacegrad \int_V dV'\,
   \frac{s(\Bx')}{\Norm{\Bx - \Bx'}}
   +\spacegrad \cdot \int_V dV'
   \frac{I\BC(\Bx')}{\Norm{\Bx - \Bx'}}.
\end{aligned}
\end{equation}

By inserting a no-op grade selection operation in the second step, the trivector terms that would show up in subsequent steps are automatically filtered out.
%the troublesome trivector term that shows up in my first purely Geometric Algebra
%attempt is eliminated.
This leaves us with a boundary term dependent on the surface and the normal and tangential components of \( \BM \).
Added to that is a pair of volume integrals that provide the unique dependence of \( \BM \) on its divergence and curl.
When the surface is taken to infinity, which requires \( \Norm{\BM}/\Norm{\Bx - \Bx'} \rightarrow 0 \), then the dependence of \( \BM \) on its divergence and curl is unique.

In order to express final result in traditional vector algebra form, a couple transformations are required.
The first is that
\begin{equation}\label{eqn:helmholtzDerviationMultivectorSolution:800}
\gpgradeone{ \Ba I \Bb } = I^2 \Ba \cross \Bb = -\Ba \cross \Bb.
\end{equation}

For the grade selection in the boundary integral, note that
\begin{equation}\label{eqn:helmholtzDerviationMultivectorSolution:740}
\begin{aligned}
\gpgradeone{ \spacegrad \ncap \BX }
&= \gpgradeone{ \spacegrad (\ncap \cdot \BX) } + \gpgradeone{ \spacegrad (\ncap \wedge \BX) } \\
&= \spacegrad (\ncap \cdot \BX) + \gpgradeone{ \spacegrad I (\ncap \cross \BX) } \\
&= \spacegrad (\ncap \cdot \BX) - \spacegrad \cross (\ncap \cross \BX).
\end{aligned}
\end{equation}

These give
%\begin{dmath}\label{eqn:helmholtzDerviationMultivectorSolution:721}
\boxedEquation{eqn:helmholtzDerviationMultivectorSolution:721}{
\begin{aligned}
\BM(\Bx)
&=
\spacegrad \inv{4\pi} \int_{\partial V} dA'\, \ncap \cdot \frac{\BM(\Bx')}{\Norm{\Bx - \Bx'}}
-
\spacegrad \cross \inv{4\pi} \int_{\partial V} dA'\, \ncap \cross \frac{\BM(\Bx')}{\Norm{\Bx - \Bx'}} \\
&-\spacegrad \inv{4\pi} \int_V dV'
\frac{s(\Bx')}{\Norm{\Bx - \Bx'}}
+\spacegrad \cross \inv{4\pi} \int_V dV'
\frac{\BC(\Bx')}{\Norm{\Bx - \Bx'}}.
\end{aligned}
}
%\end{dmath}
%}

   \section{Problem solutions.}
      \shipoutAnswer
%}
