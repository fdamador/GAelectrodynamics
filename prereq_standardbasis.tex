%
% Copyright © 2017 Peeter Joot.  All Rights Reserved.
% Licenced as described in the file LICENSE under the root directory of this GIT repository.
%
\index{dot product}
\makedefinition{Dot product.}{dfn:prerequisites:dotproduct}{
Let \( \Bx, \By \) be vectors from a vector space \( V \).
A dot product \( \Bx \cdot \By \) is a mapping \( V \cross V \rightarrow \bbR \)
with the following properties.

\begin{tablebox}[tabularx={X|Y}]%{Dot product properties.}
    Symmetric & \( \Bx \cdot \By = \By \cdot \Bx \) \\ \hline
    Bilinear & \( (a \Bx + b \By) \cdot \Bz = a \Bx \cdot \Bz + b \By \cdot \Bz,\quad
\Bx \cdot (a \By + b \Bz) = a \Bx \cdot \By + b \Bx \cdot \Bz \)
\\ \hline
    Positive length & \( \Bx \cdot \Bx > 0, \Bx \ne 0 \) \\ \hline
\end{tablebox}
} % definition

\index{length}
\makedefinition{Length}{dfn:prerequisites:norm}{
%   The squared length of a vector \( \Bx \) is
%\begin{equation*}
%   \Norm{\Bx}^2 = \Bx \cdot \Bx,
%\end{equation*}
%%a quantity that need not be positive.
The length of a vector \( \Bx \in V \) is defined as
\begin{equation*}
\Norm{\Bx} =
%\sqrt{\Abs{ \Bx \cdot \Bx }}.
\sqrt{ \Bx \cdot \Bx }.
\end{equation*}
}

For example, \( \Bx = \Be_1 + \Be_2 \) has length \( \Norm{\Bx} = \sqrt{2} \), and \( \Bx = x \Be_1 + y \Be_2 + z \Be_3 \) has length \( \Norm{\Bx} = \sqrt{}\lr{ x^2 + y^2 + z^2} \).

%A vector space with an associated norm based length is called a normed vector space.
%Any dot product space is also a normed vector space.

\index{unit vector}
\makedefinition{Unit vector}{dfn:prerequisites:unitvector}{
   A vector \( \Bx \) is called a unit vector if the dot product with itself is unity (\( \Bx \cdot \Bx = 1 \)).
} % definition

Examples of unit vectors include \( \Be_1, (\Be_1 + \Be_3)/\sqrt{3}, (2 \Be_1 - \Be_2 - \Be_3)/\sqrt{6}\), and any vector \( \Bx = \alpha \Be_1 + \beta \Be_2 + \gamma \Be_3 \), where \( \alpha, \beta, \gamma \) are direction cosines satisfying \( \alpha^2 + \beta^2 + \gamma^2 = 1 \).

%A unit vector \( \xcap \) may be generated from any vector \( \Bx \) that has a non-zero squared norm by computing
%
%\begin{equation}\label{eqn:prereq_standardbasis:220}
%\xcap = \frac{\Bx}{\sqrt{\Abs{\Norm{\Bx}^2}}}.
%\end{equation}
%
\index{orthogonal}
\makedefinition{Orthogonal}{dfn:prerequisites:normal}{
   Two vectors \( \Bx, \By \in V \) are orthogonal if their dot product is zero, \( \Bx \cdot \By = 0 \).
}

Examples of orthogonal vectors
include \( \Bx, \By \) where
\begin{equation}\label{eqn:prereq_standardbasis:300}
\begin{aligned}
\Bx &= \Be_1 + \Be_2 \\
\By &= \Be_1 - \Be_2,
\end{aligned}
\end{equation}
and \( \Bx, \By, \Bz \) where
\begin{equation}\label{eqn:prereq_standardbasis:320}
\begin{aligned}
\Bx &= \Be_1 + \Be_2 + \Be_3 \\
\By &= 2 \Be_1 - \Be_2 - \Be_3 \\
\Bz &= \Be_3 - \Be_2.
\end{aligned}
\end{equation}

\index{orthonormal}
\makedefinition{Orthonormal}{dfn:prerequisites:orthonormal}{
   Two vectors \( \Bx, \By \in V \) are orthonormal if they are both unit vectors and orthogonal to each other.
% (\( \Bx \cdot \By = 0 \), \( \Bx \cdot \Bx = \By \cdot \By = 1 \)).
   A set of vectors \( \setlr{ \Bx, \By, \cdots, \Bz } \) is an orthonormal set if all pairs of vectors in that set are orthonormal.
}

Examples of orthonormal vectors
include \( \Bx, \By \) where
\begin{equation}\label{eqn:prereq_standardbasis:340}
\begin{aligned}
\Bx &= \inv{\sqrt{2}}\lr{\Be_1 + \Be_2} \\
\By &= \inv{\sqrt{2}}\lr{\Be_1 - \Be_2},
\end{aligned}
\end{equation}
and \( \Bx, \By, \Bz \) where
\begin{equation}\label{eqn:prereq_standardbasis:360}
\begin{aligned}
\Bx &= \inv{\sqrt{3}}\lr{\Be_1 + \Be_2 + \Be_3} \\
\By &= \inv{\sqrt{6}}\lr{2 \Be_1 - \Be_2 - \Be_3} \\
\Bz &= \inv{\sqrt{2}}\lr{\Be_3 - \Be_2}.
\end{aligned}
\end{equation}

\index{standard basis}
\index{\(\Be_1, \Be_2, \cdots\)}
\makedefinition{Standard basis.}{dfn:prerequisites:standardbasis}{
   A basis
\( \setlr{ \Be_1, \Be_2, \cdots, \Be_N} \) is called a standard basis if that set is orthonormal.
} % definition

Any number of possible standard bases are possible, each differing by combinations of rotations and reflections.  For example, given a standard basis \( \setlr{ \Be_1, \Be_2, \Be_3 } \), the set \( \setlr{\Bx, \By, \Bz} \) from \cref{eqn:prereq_standardbasis:360} is also a standard basis.

\makedefinition{Metric.}{dfn:prereq_standardbasis:380}{
Given a basis \( B = \setlr{ \Bx_1, \Bx_2, \cdots \Bx_N } \), the metric of the space with respect to \( B \) is the (symmetric) matrix \( G \) with elements \( g_{ij} = \Bx_i \cdot \Bx_j \).
} % definition

For example, with a basis \( B = \setlr{\Bx_1, \Bx_2} \) where \( \Bx_1 = \Be_1 + \Be_2, \Bx_2 = 2 \Be_1 - \Be_2 \), the metric is
\begin{equation}\label{eqn:prereq_standardbasis:400}
G =
\begin{bmatrix}
2 & 1 \\
1 & 5
\end{bmatrix}.
\end{equation}
The metric with respect to a standard basis is just the identity matrix.

In relativisitic geometric algebra, the positive definite property of \cref{dfn:prerequisites:dotproduct} is considered optional.
In this case, the definition of length must be modified, and one would say the length of a vector \( \Bx \) is \( \sqrt{\Abs{\Bx \cdot \Bx}} \), and that \( \Bx \) is a unit vector if \( \Bx \cdot \Bx = \pm 1 \).
Such relativisitic dot products will not be used in this book, but they are ubiquitous in the geometric algebra literature, so
it is worth knowing that the geometric algebra literature may use a weaker defition of dot product than typical.
The metric for a relativistic vector space is not a positive definite matrix.  In particular, the metric with respect to a relativistic standard basis is zero off diagonal, and has
diagonals valued \( (1, -1, -1, -1) \) or \( (-1, 1, 1, 1) \).
A space is called Euclidean, when the metric with respect to a standard basis is the identity matrix, that is
\( \Be_i \cdot \Be_j = \delta_{ij} \) for all standard basis elements \( \Be_i, \Be_j \), and
called non-Euclidean if \( \Be_i \cdot \Be_i = -1 \) for at least one standard basis vector \( \Be_i \).

%%%For example,
%%%given a basis \( \setlr{ \Bx_1, \Bx_2, \cdots, \Bx_N} \), and two vectors
%%%\begin{equation}\label{eqn:prereq_standardbasis:240}
%%%   \Ba = \sum_{i = 1}^N a_i \Bx_i, \qquad
%%%   \Bb = \sum_{i = 1}^N b_i \Bx_i,
%%%\end{equation}
%%%the dot product of the two is
%%%\begin{equation}\label{eqn:prereq_standardbasis:260}
%%%\Ba \cdot \Bb
%%%=
%%%   \lr{ \sum_{i = 1}^N a_i \Bx_i } \cdot
%%%   \lr{ \sum_{j = 1}^N b_j \Bx_j }
%%%=
%%%   \sum_{i,j = 1}^N a_i b_j \lr{ \Bx_i \cdot \Bx_j }.
%%%\end{equation}
%%%Such an expansion in coordinates can be written in matrix form as
%%%\begin{dmath}\label{eqn:prereq_standardbasis:280}
%%%\Ba \cdot \Bb
%%%=
%%%\Ba^\T G \Bb,
%%%\end{dmath}
%%%where \( G \) is the symmetric matrix with elements \( g_{ij} = \Bx_i \cdot \Bx_j \).
%%%This matrix \( G \), or its elements \( g_{ij} \) is also called the metric for the space, with respect to the basis \( \setlr{ \Bx_1, \Bx_2, \cdots, \Bx_N} \).
%%%
%%%%For Euclidean spaces, which are the primary focus of this book, the metric is not only diagonal, but is the identity matrix.
%%%%Slightly more general metrics are of interest in electrodynamics.  In particular, a four dimensional (relativistic) vector space, where the metric
%%%% allows for the construction of a geometric algebra that allows Maxwell's equations to take their very simplest form.  Such a metric does not have the (Euclidean) positive definite property \( \Bx^\T G \Bx > 0, \Bx \ne 0 \).
%%%%
%%%
