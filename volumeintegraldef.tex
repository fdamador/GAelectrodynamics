%
% Copyright � 2018 Peeter Joot.  All Rights Reserved.
% Licenced as described in the file LICENSE under the root directory of this GIT repository.
%
%{
\makedefinition{Multivector volume integral.}{dfn:volumeintegraldef:multivectorvolumeintegral}{
Given a continuous and differentiable volume described by a vector function \( \Bx(a, b,c) \), parameterized by scalars \( a, b, c \) with volume element
\begin{equation*}
d^3 \Bx \equiv
d\Bx_a
\wedge
d\Bx_b
\wedge
d\Bx_c
=
\PD{a}{\Bx}
\wedge
\PD{b}{\Bx}
\wedge
\PD{c}{\Bx}
\,da db dc = \Bx_a \Bx_b \Bx_c \, da db dc,
\end{equation*}
and multivector functions \( F, G \), the integral
\begin{equation*}
\int F d^3 \Bx G
\end{equation*}
is called a multivector volume integral.
} % definition

In \R{3} the volume element is always a pseudoscalar, which commutes with all grades, so we are free to write \( \int F d^3 \Bx G = \int d^3 \Bx F G \) for any multivectors \( F, G \).  It will often be useful to make the pseudoscalar nature of the volume element explicit, writing \( d^3 \Bx = I dV \), where \( dV \) is a scalar volume element.

As an example, let \( F(\Bx) = r(\Bx) + \Bs(\Bx) + I \Bt(\Bx) + I u(\Bx) \) be an arbitrary multivector function in \R{3}, where \( r, u \) are scalar functions and \( \Bs, \Bt \) are vector functions.
Integrating over a unit cube in rectangular coordinates \( d^3 \Bx = I dx dy dz = I dV \),
the volume integral of such a multivector function is
\begin{equation}\label{eqn:volumeintegraldef:20}
\begin{aligned}
\int F d^3 \Bx
&= \int \lr{ r(\Bx) + \Bs(\Bx) + I \Bt(\Bx) + I u(\Bx) } I dV \\
&= \int \lr{ I r(\Bx) + I \Bs(\Bx) - \Bt(\Bx) - u(\Bx) } dV.
\end{aligned}
\end{equation}
The result still has all grades, but each of the original grade components is mapped onto its dual space.
%}
