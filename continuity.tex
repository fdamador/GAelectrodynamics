%
% Copyright � 2018 Peeter Joot.  All Rights Reserved.
% Licenced as described in the file LICENSE under the root directory of this GIT repository.
%
%{
\label{chap:continuity}
Some would argue that the conventional form \cref{eqn:freespace:3399} of Maxwell's equations have built in redundancy since continuity equations on the charge and current densities couple some of these equations.
We will take an opposing view, and show that such continuity equations are necessary consequences of Maxwell's equation in its wave equation form, and derive those conditions.
This amounts to a statement that the multivector current \( J \) is not completely unconstrained.

%It's helpful to introduce a d'Alembertian operator.  The sign convention that we will use for this operator is given by the following.
%\index{\(\dAlembertian\)}
%\makedefinition{d'Alembertian (wave equation) operator.}{dfn:continuity:190}{
%Let
%\begin{equation*}
%\dAlembertian =
%\conjstgrad
%\stgrad
%=
%\spacegrad^2 - \inv{c^2} \PDSq{t}{}.
%\end{equation*}
%} % definition
%
\input{Theorem_continuity_wave.tex}
\begin{proof}
To prove, we operate on \cref{dfn:isotropicMaxwells:680} with \( \spacegrad - (1/c) \partial_t \), one of the factors, along with the
spacetime gradient, of the
d'Alembertian (wave equation) operator, which gives
\begin{dmath}\label{eqn:continuity:110}
%\lr{ \spacegrad^2 - \inv{c^2} \PDSq{t}{} }
\dAlembertian
F = \conjstgrad J.
\end{dmath}
Since the left hand side has only grades 1 and 2, \cref{eqn:continuity:110} splits naturally into two equations, one for grades 1,2 and one for grades 0,3
\begin{equation}\label{eqn:continuity:130}
\begin{aligned}
%\lr{ \spacegrad^2 - \inv{c^2} \PDSq{t}{} }
\dAlembertian
F &= \gpgrade{ \conjstgrad J }{1,2} \\
                                           0 &= \gpgrade{ \conjstgrad J }{0,3}.
\end{aligned}
\end{equation}
Unpacking these further, we find that there is information
carried in the requirement that the grade 0,3 selection of \cref{eqn:continuity:130} is zero.
In particular, grade 0 selection gives
\begin{dmath}\label{eqn:continuity:40}
0
=
\gpgradezero{ ( \spacegrad - (1/c) \partial_t ) J }
=
\gpgradezero{
\Biglr{ \spacegrad - \inv{c} \PD{t}{} }
\biglr{
   \eta \lr{ c \rho - \BJ } + I \lr{ c \rho_\txtm - \BM }
}
}
=
-\eta
\lr{ \spacegrad \cdot \BJ + \PD{t}{\rho} }
,
\end{dmath}
which demonstrates the continuity condition on the electric sources.
Similarly, grade three selection gives
\begin{dmath}\label{eqn:continuity:60}
0
=
\gpgradethree{  (\spacegrad - (1/c) \partial_t ) J }
=
\gpgradethree{
\Biglr{ \spacegrad - \inv{c} \PD{t}{} }
\lr{
   \eta \lr{ c \rho - \BJ } + I \lr{ c \rho_\txtm - \BM }
}
}
=
-I \lr{
   \spacegrad \cdot \BM + \PD{t}{\rho_\txtm}
},
\end{dmath}
which demonstrates the continuity condition on the (fictitious) magnetic sources if included in the current.

For the non-homogeneous wave equation of \cref{thm:continuity:600}, the current derivatives may be expanded explicitly.
For the wave equation for the electric field, this is
\begin{dmath}\label{eqn:continuity:150}
%\lr{ \spacegrad^2 - \inv{c^2} \PDSq{t}{} } \BE
\dAlembertian \BE
=
\gpgradeone{\conjstgrad J}
=
\gpgradeone{
   \conjstgrad
   \lr{
      \frac{\rho}{\epsilon} - \eta \BJ + I \lr{ c \rho_\txtm - \BM }
   }
}
=
\inv{\epsilon} \spacegrad \rho -I \lr{ \spacegrad \wedge \BM } + \inv{c} \eta \PD{t}{\BJ}
= \gpgrade{\conjstgrad J}{1} = \inv{\epsilon} \spacegrad \rho + \mu \PD{t}{\BJ} + \spacegrad \cross \BM,
\end{dmath}
as claimed.
The forced magnetic field equation is
\begin{dmath}\label{eqn:continuity:170}
%\lr{ \spacegrad^2 - \inv{c^2} \PDSq{t}{} } \BH
\dAlembertian \BH
=
\inv{\eta I}
\gpgradetwo{\conjstgrad J}
=
\inv{\eta I}
\gpgradetwo{
   \conjstgrad
   \lr{
      \frac{\rho}{\epsilon} - \eta \BJ + I \lr{ c \rho_\txtm - \BM }
   }
}
=
\inv{\eta I}
\lr{
   -\spacegrad \wedge \BJ + I c \spacegrad \rho_\txtm + \frac{I}{c} \PD{t}{\BM}
}
=
\inv{I}
\lr{
   -I \lr{ \spacegrad \cross \BJ} + I \inv{\mu} \spacegrad \rho_\txtm + I \epsilon \PD{t}{\BM}
}
= \inv{\mu} \spacegrad \rho_\txtm + \epsilon \PD{t}{\BM} - \spacegrad \cross \BJ.
\end{dmath}
\end{proof}
%}
