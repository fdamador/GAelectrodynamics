%
% Copyright � 2019 Peeter Joot.  All Rights Reserved.
% Licenced as described in the file LICENSE under the root directory of this GIT repository.
%
%{
%%\input{../latex/blogpost.tex}
%%\renewcommand{\basename}{formsVsGA}
%%%\renewcommand{\dirname}{notes/phy1520/}
%%\renewcommand{\dirname}{notes/ece1228-electromagnetic-theory/}
%%%\newcommand{\dateintitle}{}
%%%\newcommand{\keywords}{}
%%
%%\input{../latex/peeter_prologue_print2.tex}
%%
%%\usepackage{peeters_layout_exercise}
%%\usepackage{peeters_braket}
%%\usepackage{peeters_figures}
%%\usepackage{siunitx}
%%\usepackage{verbatim}
%%%\usepackage{mhchem} % \ce{}
%%%\usepackage{macros_bm} % \bcM
%%%\usepackage{macros_qed} % \qedmarker
%%%\usepackage{txfonts} % \ointclockwise
%%
%%\beginArtNoToc
%%
%\generatetitle{Differential forms vs geometric calculus}

It is likely that a student of electromagnetism will encounter differential forms in their studies.  As with geometric algebra, Maxwell's equations also have a compact representation in differential forms.  That formalism requires recasting the scalars or vectors of Maxwell's equations as 1-forms (differentials), 2-forms, or 3-forms
\begin{dmath}\label{eqn:formsVsGA:220}
\begin{aligned}
\BE &\rightarrow E_x \,dx c dt + E_y \,dy c dt + E_z \,dz c dt, \\
\BB &\rightarrow B_x \,dy dz + B_y \,dz dx + B_z \,dx dy, \\
\BH &\rightarrow -H_x \,dx c dt - H_y \,dy c dt - H_z \,dz c dt, \\
\BJ &\rightarrow J_x \,dy dz c dt + J_y \,dz dx c dt + J_z \,dx dy c dt, \\
\rho &\rightarrow -\rho \, dx dy dz.
\end{aligned}
\end{dmath}
This appendix is not intended to teach differential forms, nor electrodynamics using differential forms\footnote{The interested reader is referred to \citep{flanders1989dfa} for an introduction to both differential forms, and an introduction to their application to electrodynamics.}.  Instead, this appendix assumes some passing familiarity with differential forms, and provides an example that
illustrates how differential forms and geometric calculus can be related.

The key to relating the two formalisms is the introduction of a
parameterization.
To consider these relations, consider a vector surface those span is controlled by two parameters
\begin{dmath}\label{eqn:formsVsGA:20}
\Bx = \Bx(a , b).
\end{dmath}
In geometric calculus we introduce differentials that span the tangent plane at the point of evaluation
\begin{dmath}\label{eqn:formsVsGA:40}
\begin{aligned}
   dx_a &= \PD{a}{\Bx}\, da \\
   dx_b &= \PD{b}{\Bx}\, db,
\end{aligned}
\end{dmath}
so the area element for this parameterization is
\begin{dmath}\label{eqn:formsVsGA:60}
\begin{aligned}
   d^2 \Bx &= dx_a \wedge dx_b \\
           &= \PD{a}{\Bx} \wedge \PD{b}{\Bx}\, da db.
\end{aligned}
\end{dmath}
To relate this to differential forms, introduce an
orthonormal basis \( \Be_k \cdot \Be_j = 0, \Be_k^2 = 1\).  In this basis, the coordinate expansion (summation implied) of the vector \( \Bx \) is
\begin{dmath}\label{eqn:formsVsGA:80}
   \Bx = \Be_k x_k.
\end{dmath}
The coordinate expansion of the geometric area element is
\begin{dmath}\label{eqn:formsVsGA:100}
\begin{aligned}
   d^2 \Bx &=
   \PD{a}{x_k} \PD{b}{x_j} \Be_k \wedge \Be_j\, da db \\
           &=
   \sum_{\mu < \nu}
   \lr{
      \PD{a}{x_k} \PD{b}{x_j} -
      \PD{a}{x_j} \PD{b}{x_k}
   }
      \Be_k \wedge \Be_j\, da db \\
           &=
   \sum_{\mu < \nu}
      \Be_k \Be_j
\begin{vmatrix}
   \PD{a}{x_k} & \PD{a}{x_j} \\
   \PD{b}{x_k} & \PD{b}{x_j}
\end{vmatrix}
      \, da db \\
           &=
   \sum_{\mu < \nu}
      \Be_k \Be_j
      \PD{(a,b)}{(x_k, x_j)}
      \, da db.
\end{aligned}
\end{dmath}
Each element of this sum includes a product of a pseudoscalar, a Jacobian determinant, and a scalar two parameter differential.

Now consider a two parameter differential for the same vector.  Recall that a differential (1-form) of a scalar function, again assuming two parameters, has the characteristic
\begin{dmath}\label{eqn:formsVsGA:120}
   df  =
   \PD{a}{f} \, da +
   \PD{b}{f} \, db.
\end{dmath}
In particular, we may compute the differentials of the coordinate functions
\begin{dmath}\label{eqn:formsVsGA:140}
\begin{aligned}
   dx_k &= \PD{a}{x_k} \, da + \PD{b}{x_k} \, db \\
   dx_j &= \PD{a}{x_j} \, da + \PD{b}{x_j} \, db,
\end{aligned}
\end{dmath}
from which we can compute a 2-form
\begin{dmath}\label{eqn:formsVsGA:160}
\begin{aligned}
   dx_k \wedge dx_j
   &= \lr{ \PD{a}{x_k} \, da + \PD{b}{x_k} \, db } \wedge \lr{ \PD{a}{x_j} \, da + \PD{b}{x_j} \, db } \\
   &= \PD{a}{x_k} \PD{b}{x_j} \, da \wedge db + \PD{b}{x_k} \PD{a}{x_j} \, db \wedge da \\
   &=
\begin{vmatrix}
   \PD{a}{x_k} & \PD{a}{x_j} \\
   \PD{b}{x_k} & \PD{b}{x_j}
\end{vmatrix}
   \, da \wedge db \\
   &=
      \PD{(a,b)}{(x_k, x_j)}
   \, da \wedge db.
\end{aligned}
\end{dmath}
We have almost the same structure as with geometric algebra, however, in differential forms, the antisymmetry of the surface area element is encoded in the 2-form \( da \wedge db \) whereas in geometric calculus the required antisymmetry is encoded in a unit bivector.

Should we restrict our attention to a strictly planar subspace, the mapping between the two formalisms becomes more striking.  We now have
\begin{dmath}\label{eqn:formsVsGA:180}
\begin{aligned}
   d^2 \Bx &= \Be_1 \Be_2 \PD{(a,b)}{(x_1, x_2)} \, da db \\
   dx_1 \wedge dx_2 &= \PD{(a,b)}{(x_1, x_2)} \, da \wedge db.
\end{aligned}
\end{dmath}
That is, we can relate the formalisms by the mapping
\begin{dmath}\label{eqn:formsVsGA:200}
   \Be_1 \Be_2 \, da db \leftrightarrow da \wedge db.
\end{dmath}
The 1-form has an intrinsic vectorial nature, the 2-form has a bivector nature, and a 3-form has a trivector nature.
%When using geometric calculus, we need not express our vector quantities (\(\BE, \BH, \BJ, \cdots \)) as differentials, but can continue to use unit vectors instead of differentials as our basis elements.

%}
%\EndArticle
%\EndNoBibArticle
